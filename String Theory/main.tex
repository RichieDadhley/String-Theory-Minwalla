\documentclass[11pt,oneside]{book}
\usepackage[margin=1.2in]{geometry}
\usepackage[toc,page]{appendix}
\usepackage{graphicx}
\usepackage{natbib}
\usepackage{lipsum}
\usepackage{caption}
\usepackage[T1]{fontenc}
\usepackage{titlesec, blindtext, color}
\usepackage{xcolor,tikz}
\usetikzlibrary{patterns}
\usetikzlibrary{decorations.markings}
\usepackage{amsmath,stix,amssymb,amsthm,mathrsfs,amsfonts,xfrac,pifont,bbold,physics,slashed} % Stix is the thing that changed the font for colons.

\usepackage[utf8]{inputenc}
\usepackage{amsthm}
\usepackage[breakable, theorems, skins]{tcolorbox}
\usepackage[colorlinks = true,
            linkcolor = red,
            urlcolor  = blue,
            citecolor = red,
            anchorcolor = red]{hyperref}
\usepackage{cleveref}


\usepackage{soul}
\usepackage{frcursive}

% -------------------------------------------------------------------
% Theorem Styles
% -------------------------------------------------------------------

\theoremstyle{definition} % Define theorem styles here based on the definition style (used for definitions and examples)
\newtheorem*{definition}{Definition}

\theoremstyle{plain} % Define theorem styles here based on the plain style (used for theorems, lemmas, propositions)
\newtheorem{theorem}{Theorem}[section]
\newtheorem{axiom}{Axiom}
\newtheorem{corollary}[theorem]{Corollary}
\newtheorem{lemma}[theorem]{Lemma}
\newtheorem{proposition}[theorem]{Proposition}

\theoremstyle{remark} % Define theorem styles here based on the remark style (used for remarks and notes)
\newtheorem*{notation}{Notation}
\newtheorem*{solution}{Solution}

\newtheoremstyle{underline}% name
{}        % Space above, empty = `usual value'
{}              % Space below
{}              % Body font
{}    % Indent amount (empty = no indent, \parindent = para indent)
{}              % Thm head font
{.}             % Punctuation after thm head
{1.5mm}         % Space after thm head: \newline = linebreak
{{\underline{\textit{\thmname{#1}\thmnumber{ #2}}~\thmnote{(#3)}\unskip}}}  % Thm head spec

\theoremstyle{underline}

\newtheorem{remark}[theorem]{Remark}
\newtheorem{example}[theorem]{Example}
\newtheorem{claim}[theorem]{Claim}
% -------------------------------------------------------------------
% Chapter Headings
% -------------------------------------------------------------------

\setcounter{chapter}{-1}

\makeatletter
\renewcommand{\@chapapp}{Lecture}
\makeatother
\definecolor{lightergray}{rgb}{0.9,0.9,0.9}

\usepackage{titlesec}
\titleformat{\section}{\large\bfseries\raggedright}{}{0em}{\colorsection}[\titlerule]
\titleformat{name=\section,numberless}{\large\scshape\bfseries\raggedright}{}{0em}{\colorsectionnonumber}[\titlerule]

\titleformat{\subsection}{\bfseries\raggedright}{}{0em}{\colorsubsection}
\titleformat{name=\subsection,numberless}{\bfseries\raggedright}{}{0em}{\colorsubsectionnonumber}

\newcommand{\colorsection}[1]{%
    \colorbox{lightergray}{\parbox{\dimexpr\textwidth-2\fboxsep}{\thesection\ \ #1}}}
\newcommand{\colorsectionnonumber}[1]{%
    \colorbox{lightergray}{\parbox{\dimexpr\textwidth-2\fboxsep}{#1}}}
    
\newcommand{\colorsubsection}[1]{%
    \colorbox{lightergray}{\parbox{\dimexpr\textwidth-2\fboxsep}{\thesubsection\ #1}}}
\newcommand{\colorsubsectionnonumber}[1]{%
    \colorbox{lightergray}{\parbox{\dimexpr\textwidth-2\fboxsep}{#1}}}
    
\definecolor{gray75}{gray}{0.75}
\newcommand{\hsp}{\hspace{20pt}}
\titleformat{\chapter}[hang]{\Huge\bfseries}{\thechapter\hsp\textcolor{gray75}{|}\hsp}{0pt}{\Huge\bfseries}

\input{Shortcuts.tex}

\begin{document}

\captionsetup[figure]{margin=1.5cm,font=small,labelfont={bf},name={Figure},labelsep=colon,textfont={it}}
\captionsetup[table]{margin=1.5cm,font=small,labelfont={bf},name={Table},labelsep=colon,textfont={it}}
\setlipsumdefault{1}

\frontmatter

\begin{titlepage}
	\centering
	    \scshape % Use small caps for all text on the title page
        \vspace*{\baselineskip} % White space at the top of the page
        
	    \rule{\textwidth}{1.6pt}\vspace*{-\baselineskip}\vspace*{2pt} % Thick horizontal rule
	    \rule{\textwidth}{0.4pt} % Thin horizontal rule
	    
	    \vspace{0.75\baselineskip} % Whitespace above the title
	    
	    {\LARGE String Theory} % Title
	    
	    \vspace{0.75\baselineskip} % Whitespace below the title
	    
	    \rule{\textwidth}{0.4pt}\vspace*{-\baselineskip}\vspace{3.2pt} % Thin horizontal rule
	    \rule{\textwidth}{1.6pt} % Thick horizontal rule
	    
        \vspace{5\baselineskip} % Whitespace after the title block

	    Course delivered in 2018-19 by 
	
	    \vspace{0.5\baselineskip} % Whitespace before the editors
	
	    {\scshape\Large Shiraz Minwalla} % Lecturer Name
	
	    \vspace{0.5\baselineskip} % Whitespace below the editor list
	
	    \textit{Tata Institute of Fundamental Research (TIFR) \\ Mumbai, India } % Lecturer Institution
	    
	    \vspace{2\baselineskip} % Whitespace after the title block
	    
	    \includegraphics[width=8cm]{images/TataLogo.png}\\[1cm] % Logo 

	    Notes taken by 
	
	    \vspace{0.5\baselineskip} % Whitespace before my name
	
	    {\scshape\Large Richie Dadhley} % Lecturer Name
	   
	    \vspace{0.5\baselineskip} % Whitespace below my name
	    \textit{richie@dadhley.com} % Email
	
	    \vfill % Whitespace between editor names and publisher logo
\end{titlepage}



% -------------------------------------------------------------------
% Acknowledgements
% -------------------------------------------------------------------

\newpage
\section*{Acknowledgements}

This set of notes is intended to accompany the lecture course on String Theory taught by Dr. Shiraz Minwalla from August 2018 to March 2019. The videos for the course are available on YouTube via the following link

\begin{center}
    \href{https://www.youtube.com/watch?v=96RaoTmBsTw&list=PL3PVFGnaPl_sCp2A87NVD8GT5Z8Oqw3Yr&index=1}{Dr. Shiraz Minwalla’s String Theory Course}
\end{center}

I have tried to correct any typos and/or mistakes I think I have noticed over the course. I have also tried to include additional information that I think supports the taught material well, which sometimes has resulted in modifying the order the material was taught. Obviously, any mistakes made because of either of these points are entirely mine and should not reflect on the taught material in any way. \\

One of the main sources I have used to obtain the additional information is Dr. David Tong’s String Theory notes, available via the following link

\begin{center}
    \href{http://www.damtp.cam.ac.uk/user/tong/string.html}{http://www.damtp.cam.ac.uk/user/tong/string.html}
\end{center}

I would like to extend a message of thanks to both Dr. Minwalla and Dr. Tong for making their courses available and for their great explanations of a topic that is highly complicated. \\

If you have any comments and/or questions please feel free to contact me via the email provided on the title page. \\

These notes are currently a work in progress, so for updated notes, as well as a list of other notes/works I have available, visit my blog site

\begin{center}
    \href{https://richie291.wixsite.com/theoreticalphysics}{https://richie291.wixsite.com/theoreticalphysics}
\end{center}

These notes are not endorsed by Dr. Minwalla or TIFR.

\vspace{1cm}

\begin{flushright}
    \Huge{{\cursive\setul{0.1ex}{}\ul{~~Richie Dadhley~~}}}
\end{flushright}


% -------------------------------------------------------------------
% Contents
% -------------------------------------------------------------------

\tableofcontents

% -------------------------------------------------------------------
% Main sections (as required)
% -------------------------------------------------------------------

\mainmatter

\chapter{Introduction}

I shall update the introduction once I have completed more of the course and have a better understanding of what to write here... For now I shall leave you with a nice quote from Dr. David Tong's String Theory course:

\begin{center}
    \textit{
        "Our current understanding of physics, embodied in the standard model, is valid up to energy scales of 103 GeV. This is 15 orders of magnitude away from the Planck scale. Why do we think the time is now ripe to tackle quantum gravity? Surely we are like the ancient Greeks arguing about atomism. Why on earth do we believe that we've developed the right tools to even address the question?\\
        The honest answer, I think, is hubris."
    }
\end{center}

% Lecture 10 at 1:38:00 ish for comment on why he teaches worldsheet string theory

\textcolor{red}{Disclaimer: It turns out Polchinski uses the convention $d^2\sig = \frac{1}{2}d^2z$. I have missed the fraction in some places. This obviously doesn't effect any of the physics but is worth noting. In fact in general the factors of $2$ and $\pi$ are likely to be wrong here and there. I shall try fix this at some point, but it will involve rereading the entire set of notes with a keen eye so might take some time. If anyone notices any, I would appreciate them being pointed out.}

% Maybe put a comment about not stressing when stuff is on the worldsheet enough. The worldsheet is not physical so just because something is on there, doesn't mean its on the spacetime (target space). When we quantise worldsheet we get spacetime objects. 
\chapter{Actions}

String theory studies the motion of a relativistic \textit{string} propagating through spacetime, and seeks to quantize this theory. For this reason it is standard procedure to review some of the basic notions introduced in general relativity for the relativistic motion of a point particle in spacetime. So this is where our journey begins. 

Before moving on to do that, we would like to just highlight a point here. As is standard when approaching a quantum field theory, we approach the problem by first studying it classically. So the results that follow (until specified) are all classical methods. We just mention this here because we need to remember certain things that are equal classically do not necessarily have quantised equivalence. The most obvious example of this is $xp-px =0$ classically, but once we promote the position and momentum to quantum operators, we arrive at their commutation relation $[x,p]=i\hbar$.

We shall use capital $X$ to indicate the spacetime coordinates of the background in these notes, as this appears to be a standard notation for string theory.

\section{The Relativistic Point Particle}

First some quick notes on notation/convention. As is standard we shall use Greek indices to label spacetime coordinates (e.g. $\mu = 0,1,2,...$) and Latin indices to label spatial coordinates (e.g. $i=1,2,...$). There will be occasionally be exceptions to this (for example we shall often use $\a=1,2$ when discussing the coordinates on the string's worldsheet), however we shall try to be as explicit as possible. We will of course use Einstein's summation convention. 

As you are probably aware, string theory is a high dimensional theory, and so in preparation for this, we shall work in a $d$ dimensional theory\footnote{Note this means we have $d-1$ spatial dimensions} with Minkowski metric with signature\footnote{Note this is the opposite convention then is used in quantum field theory.} 

\bse 
    \eta_{\mu\nu} = \diag(-1,+1,+1,...,+1).
\ese 

Ok so that's that settled. Now recall that the action of a point particle of mass $m$ (in a fixed frame) is 
\be
\label{eqn:PointActionSpatial}
    S = -m\int dt \sqrt{1-\dot{X}_i\dot{X}^i},
\ee 
where the dot indicates the time derivative. The relativist would (or at least should) cause a fuss here; the action \Cref{eqn:PointActionSpatial} does not appear to be Lorentz invariant! The easiest way to see this is by the fact that the spatial coordinates are dynamical degrees of freedom, whereas the temporal coordinate is merely a label for the position. If a Lorentz transformation is meant to (at least potentially) mix space and time, we run into an immediate problem; you can't exchange a degree of freedom for a label. 

With this in mind, the relativist sets off trying to find a way to rewrite the action such that it is manifestly Lorentz invariant. There are two obvious ways to do this, we could either `reduce' the spatial components to labels (which is what field theorists do) or we could `promote' the temporal coordinate to a degree of freedom, which we shall do here.

However, as Dr. Tong explains in his notes, this seems like an absurd thing to do; you can't just go about willy nilly promoting labels to degrees of freedom! As the name suggest, a degree of \textit{freedom} gives the particle freedom to move how it likes. This is fine for the spatial degrees of freedom (the particle can move left, right, up, down, round in circles, sit still, etc.), but this is clearly not true for time. You can't just decide you don't want to age anymore and so `sit still in time'. 

So what we really need to do is promote time to a `fake' degree of freedom, which is done using a gauge symmetry. Let our new action be given by 

\be 
\label{eqn:PointActionLorentz}
    S = -m \int d\tau \sqrt{-\dot{X}^{\mu}\dot{X}_{\mu}},
\ee 
where $\tau$ is some new parameter and
\bse 
    \dot{X}_{\mu} := \frac{d X_{\mu}}{d\tau}.
\ese 

Geometrically, what we are looking at here is the \textit{worldline} of the particle, and $\tau$ parameterises the motion along this worldline:

\begin{center}
    \btik
        \draw[ultra thick] (-0.5,0) -- (3,0);
        \node at (3,-0.2) {$\Vec{x}$};
        \draw[ultra thick] (0,-0.5) -- (0,5);
        \node at (-0.2,5) {$x^0$};
        \draw[thick] (0.5,0.5) .. controls (3,3) and (1,3) .. (2,4.5);
        \draw[rotate around={35:(0.6,0.6)}] (0.6,0.4) -- (0.6,0.8);
        \draw[rotate around={38:(1,1)}] (1,0.8) -- (1,1.2);
        \draw[rotate around={40:(1.4,1.5)}] (1.4,1.3) -- (1.4,1.7);
        \draw[rotate around={45:(1.65,2)}] (1.65,1.8) -- (1.65,2.2);
        \draw[rotate around={70:(1.8,2.6)}] (1.8,2.4) -- (1.8,2.8);
        \draw[rotate around={100:(1.75,3.3)}] (1.75,3.1) -- (1.75,3.5);
        \draw[rotate around={70:(1.75,4)}] (1.75,3.8) -- (1.75,4.2);
    \etik
\end{center}
where the notches give increasing $\tau$ values. We see, then, that the action \Cref{eqn:PointActionLorentz} measures the length of this worldline, often known as the \textit{proper time}. It should be clear, therefore, that we expect this action to be invariant under reparameterisation $\tau \to \widetilde{\tau}(\tau)$. That is, it doesn't matter how `quickly' we move along the line, the length of it is the length of it. Mathematically this is seen via 
\bse 
    d\tau \to \frac{d\tau}{d\widetilde{\tau}} d\widetilde{\tau}, \qquad \frac{dX_{\mu}}{d\tau} \to \frac{dX_{\mu}}{d\widetilde{\tau}}\cdot  \frac{d\widetilde{\tau}}{d\tau},
\ese 
which gives the invariance result once plugged into \Cref{eqn:PointActionLorentz}. 

This might seem like a strange result to show here, but it is exactly what we need as it allows us to remove the physical meaning form the temporal degree of freedom. In other words, we can use this invariance to give us a coordinate system in which $\tau=x^0(\tau)$, and seeing as the former has no physical meaning, neither does the latter. Note, in this case we actually return to \Cref{eqn:PointActionSpatial}, which is seen from direct calculation.

The invariance used above is an example of a \textit{gauge symmetry}. It has allowed us to express our action in a way that is manifestly Lorentz invariant, while retaining the number of physical degrees of freedom. This is a really important concept, and we shall return to it soon! 

\br 
    It is important to note that, although we say gauge \textit{symmetry}, this is not a true symmetry (in relation to Noether's theorem). A true symmetry takes you from one physical state to another (for example a Lorentz boost), whereas a gauge symmetry does nothing to the system, i.e. it is a redundancy in our description of the state. Put another way, a real symmetry will take you from one element of the phase space of solutions to another element, whereas a gauge symmetry will not.
\er 

\section{The Einbein}

Before moving on to discuss the action for the string, we want to make one final comment about the point-particle. Although we have managed to make our action Lorentz invariant, it still contains a square root, which is ugly. It would be a much nicer expression to use if it did not have this square root, and this is exactly where we introduce an \textit{einbein}, $e$, giving the action 

\be 
\label{eqn:PointActionEinbein}
    S = \frac{1}{2}\int d\tau \big( e^{-1} \dot{X}^{\mu}\dot{X}_{\mu} - em^2\big)
\ee 

Now if you're like me, before this you have never heard of an einbein and so have no idea where the above action came from. Not to worry, we are saved by David Zaslavsky via the following link: \href{http://www.ellipsix.net/blog/2010/08/the-origin-of-the-einbein.html}{http://www.ellipsix.net/blog/2010/08/the-origin-of-the-einbein.html}. It would be completely pointless for me to just rewrite what David has written there, so I simply refer you to his explanation. 

This new action not only has the advantage of removing the square root, but it also gives meaning to studying massless particles (which, as David explains, \Cref{eqn:PointActionLorentz} did not). We will not discuss this action in any further detail here,\footnote{Mainly because I am not familiar with it} however we shall just note that \Cref{eqn:PointActionEinbein} can be interpreted as a theory which couples our worldline to 1d gravity.\footnote{See Dr. Tong's String Theory notes for a slightly longer explanation.} We simply introduce the idea that this is possible here, as we shall do a similar thing in a moment with the worldsheet of the string. 

\section{The Nambu-Goto Action}

In light of everything we just did, we can begin to think of starting points for the action of our relativistic string. The most immediate thought that comes to mind is that, just as the action for the point particle gave the length of the worldline, we would like the action for the string to give the \textit{area} of the worldsheet. Clearly in order to do this, we will need two labels (i.e. two quantities that allow us to parameterise the sheet, just as $\tau$ allowed us to parameterise the worldline). We shall call this $\tau$ (for `time') and $\sig$ (for `space'). 
\begin{center}
    \btik
        \draw[ultra thick] (-0.5,0) -- (4,0);
        \node at (4,-0.2) {$\Vec{x}$};
        \draw[ultra thick] (0,-0.5) -- (0,5);
        \node at (-0.2,5) {$x^0$};
        \draw[thick] (0.8,0.5) .. controls (2.3,3) and (0.3,3) .. (0.8,4.5);
        \draw[thick] (2.3,0.5) .. controls (4.8,2.5) and (0.8,3) .. (2.3,4.5);
        \draw[thick] (1.55,4.5) ellipse (0.74 and 0.18);
        \draw[thick] (0.8,0.5) arc (180:360:0.74 and 0.18);
        \draw[thick, dashed] (0.8,0.5) arc (180:360:0.74 and -0.18);
        \draw[thick, dashed] (2.22,2) ellipse (0.87 and 0.2);
        \draw[->, thick] (3.3,2) .. controls (3.1,2.5) .. (2.6,3);
        \node at (3.3,2.5) {$\tau$};
        \draw[->, thick] (1.5,1.7) .. controls (2.25,1.5) .. (2.9,1.7);
        \node at (2.25, 1.4) {$\sig$};
    \etik
\end{center}

We wish to consider closed strings, and we choose to parameterise $\sig$ such that $\sig \in [0,2\pi)$ with periodicity, i.e. $\sig(x+2\pi)=\sig(x)$.

\bnn 
    It is now that we introduce our first exception to our index convention. We, rather confusingly, choose to define $\sig^{\alpha} = (\sig,\tau)$. That is $\sig^1=\sig$ and $\sig^2=\tau$.
\enn 

\br 
    It is important to note that we have two sets of coordinates here. We have the $\{X^{\mu}\}$ set which is the coordinate system for the background spacetime, and the $\{\sig\}$ which is the coordinate system on the worldsheet. We shall stick to this convention throughout these notes. We highlight this here, because as we shall shortly see, we are going to have multiple different kinds of indices, contracted using different metrics (one for the spacetime and one for the worldsheet). 
\er 

Referring to the point particle's action \Cref{eqn:PointActionLorentz}, we see that we expect our action to be some kind of double integral (over $\tau$ and $\sig$) containing derivative terms 
\bse 
    \p_{1}X^{\mu} := \frac{\p X^{\mu}}{\p \sig}, \qquad \text{and} \qquad  \p_{2}X^{\mu} :=\frac{\p X^{\mu}}{\p \tau}.
\ese 

However, we should not be so quick as to just plug this in and claim something like 
\bse 
    S = A\int d\tau d\sig \sum_{\alpha,\beta=1}^2\sqrt{- \p_{\alpha}X^{\mu}\p_{\beta}X^{\mu}},
\ese 
for some constant $A$, is what we want. In fact, this does not look good at all --- it contains a sum in it! 

In order to get a better idea for what we want to write, consider Euclidean 2-space. Given some surface, we find its area by breaking it up into little `tiles', whose area we calculate, and then integrate over the whole surface. If $\Vec{d\ell_1}$ and $\Vec{d\ell_2}$ are our two infinitesimal vectors for a tile, we calculate the area of that title via 
\bse 
    dA = |\Vec{d\ell_1}||\Vec{d\ell_2}|\sin\theta,
\ese
where $\theta$ is the angle between the two vectors. 

\begin{center}
    \btik
        \draw[thick, ->] (0,0) -- (1,2);
        \node at (0,1.2) {$\Vec{d\ell_1}$};
        \draw[thick, ->] (0,0) -- (2,0);
        \node at (1,-0.5) {$\Vec{d\ell_2}$};
        \draw[thick] (0.5,0) arc (0:55:0.60 and 0.60);
        \node at (0.6,0.4) {$\theta$};
        \draw[thick, dashed] (2,0) -- (3,2);
        \draw[thick, dashed] (1,2) -- (3,2);
    \etik
\end{center}
However, $\sin$ is not a nice function to have, it doesn't occur nicely in differential geometry, whereas $\cos$ on the other hand does --- from the dot product! After some rearranging we arrive at 
\bse 
    dA = \sqrt{|\Vec{d\ell_1}|^2|\Vec{d\ell_2}|^2 - (\Vec{d\ell_1}\cdot \Vec{d\ell_2})^2}.
\ese 
This still looks a little messy, however we can make it look nicer by defining a matrix 
\bse 
    M := \begin{pmatrix}
    |\Vec{d\ell_1}|^2 & \Vec{d\ell_1}\cdot \Vec{d\ell_2} \\
    \Vec{d\ell_2}\cdot \Vec{d\ell_1} & |\Vec{d\ell_2}|^2
    \end{pmatrix},
\ese 
giving 
\bse 
    dA = \sqrt{\det M}.
\ese 

Translating this into a problem in Minkowski spacetime, and using the fact that the tangent vectors\footnote{To anyone unfamiliar with why this is the case, I refer you to any differential geometry textbook. The quick explanation is that these derivative terms lie tangent to the surface of the worldsheet, and so when you zoom in close enough, they make up small tiles on the surface.}
\bse 
    \Vec{d\ell_1} = \frac{\p x}{\p \sig}, \qquad \text{and} \qquad \Vec{d\ell_2} = \frac{\p x}{\p \tau},
\ese 
where $X=(X^{\mu})$ are the spacetime coordinates, we arrive at our first reasonable action 
\be 
\label{eqn:StringActionDet}
    S = -A \int d\sig d\tau \sqrt{-\det\big(\p_{\alpha}X^{\mu} \p_{\beta}X_{\mu}\big)}.
\ee 

\br 
Note here we do \textit{not} need a summation, the $\alpha,\beta$ indices label the entries of the matrix!
\er 

If we now introduce the notation $\dot{X}^{\mu} := \partial_1X^{\mu}$, ${X^{\mu}}' := \partial_2X^{\mu}$, $(\dot{X})^2 := \partial_1X^{\mu}\partial_1X_{\mu}$ and similarly for $(X')^2$ and $(\dot{X}\cdot X')$, we arrive at the so-called \textit{Nambu-Goto action}:

\mybox{
\bd 
    The Nambu-Goto action is
    \be
    \label{eqn:NambuGotoAction}
        S = -T \int d\sig d\tau \sqrt{ -(\dot{X})^2(X')^2 + (\dot{X}\cdot X')^2 }.
    \ee 
\ed 
}

\br 
    Note we have relabelled $A\to T$ in the above definition. The reason for this shall become clear soon.
\er 

\br 
    Note that, just as \Cref{eqn:PointActionLorentz} was reparameterisation invariant, so is \Cref{eqn:NambuGotoAction} in both $\tau$ and $\sig$. Again this is a required fact to keep our physical degrees of freedom correct. These are both gauge symmetries, and reflect the fact that the coordinates used to describe the worldsheet have no physical meaning (just as $\tau$ has no physical meaning for the point-particle). Note also that the Nambu-Goto action is manifestly Lorentz invariant (which we had in mind when trying to derive it!). It is also interesting to note that this invariance holds without the requirement of inserting the root of the metric into the action (as is the case with Einstein's Field equations).
\er 

\subsection{The Pullback/Induced Metric}

Before moving on to massage \Cref{eqn:NambuGotoAction} into a nicer form (i.e. remove the square root etc), we first wish to make a note of a different, although equivalent, route to the Nambu-Goto action. We do this for two reasons
\ben 
    \item I personally prefer it as an argument,
    \item It will give us nice insight in whats to come. 
\een 

We are trying to find a way to calculate the area of the worldsheet traced out by our string as it propagates through spacetime. As anyone who has taken a course in differential geometry knows, the key thing to consider is a metric (which allows us to measure lengths). But we already have a metric, the one the spacetime is equipped with! So the obvious thing to ask is `can we use this metric in order to induce a metric on the worldsheet?' The answer is yes.\footnote{Otherwise we wouldn't be having this discussion.} This worldsheet metric is known as the \textit{induced metric} and is given by the pullback of the flat Minkowski metric. 

Let us be a bit more rigorous. Consider a manifold equipped with a metric $(\cM,\cO,\cA,g)$ where $\cO$ and $\cA$ are the topology and atlas respectively.\footnote{See, e.g., Dr. Schuller's International Winter school for Gravity and Light course.} Now consider some submanifold $\cN\ss\cM$, where $\cN$ is described using coordinates $\xi^a$, $a=1,...,\dim \cN$. 

We can embed $\cN$ into $\cM$ by defining the functions $X^{\mu}(\xi^a)$, $\mu=1,...,\dim\cM$, that tell us how $\cN$ sits in $\cM$. We can then induce a metric, $\g$, on $\cN$ from $g$ given by the following expression 
\be 
\label{eqn:PullbackMetric}
    \g_{ab} = \p_a X^{\mu} \p_b X^{\nu} g_{\mu\nu}.
\ee 

\br 
    The previous equation makes sense. We take the metric on our larger manifold and multiply it factors that ask the question `how do the embedding functions change as we move along with sheet coordinates?' Clearly if we are not on the sheet (i.e. somewhere in $\cM\sm\cN$) then both of these terms vanish and so the induced metric vanishes. They also take into account the potential intrinsic curvature of the embedded surface, which we clearly want. In other words, we have basically just restricted the metric on $\cM$ to $\cN$ and then accounted for the potential curvature. 
\er 


For us the induced metric simply comes out as
\be 
    \g_{\a \beta} = \p_{\a}X^{\mu} \p_{\beta} X^{\mu} \eta_{\mu\nu},
\ee 
and the action, which is proportional to the area of the worldsheet, is given by 
\be 
\label{eqn:StringActionDetMetric}
    S = -T\int d\sig d\tau \sqrt{-\det\g}. 
\ee 

This is the Nambu-Goto action, as is easily seen by writing the metric as a matrix
\bse 
    \g = \begin{pmatrix}
    (\dot{X})^2 & \dot{X}\cdot X' \\
    X'\cdot \dot{X} & (X')^2
    \end{pmatrix}.
\ese 

The useful piece of information, which we shall use shortly, comes when we consider the equations of motion that emerge from \Cref{eqn:StringActionDetMetric}. Recalling the result from general relativity,\footnote{Or looking it up if you're not familiar.} we arrive at 
\be 
\label{eqn:EOMInducedMetric}
    \p_{\a}\Big(\sqrt{-\det\g} \g^{\a\beta}\p_{\beta}X^{\mu}\Big) = 0.
\ee 
We shall return to this in a mo. 

\subsection{The Tension}

Everything above seems fine, but we still need to work out what the constant $T$ is. As the title of this subsection suggests, it is related to the tension of the string. The argument we provide here follows (almost exactly) the argument given by Dr. Tong in his notes. 

Let us chose our gauge such that $t:= x^0=R\tau$, for some dimensionful $R$. Now consider a snapshot of the string such that its instantaneous kinetic energy vanishes, 
\bse 
    \frac{\p\Vec{x}}{\p\tau} = 0.
\ese 
Then, from the Nambu-Goto action, it follows that 
\bse
    S = - T \int d\sig d\tau R \sqrt{\bigg(\frac{\p \Vec{x}}{\p \sig}\bigg)^2} = -T \int dt L,
\ese 
where $L$ is the spatial length of the string. But the action has units [energy][time], and so it follows that $T$ has units [energy][length]$^{-1}$, which is Newtons. We see, therefore, the $T$ at least has the dimensions of a tension, and so we interpret it as so. The explicit form is given by

\bse 
    T = \frac{1}{2\pi \a'}.
\ese 

If we work in units $\hbar = c = 1$ (as is standard in field theory) we see that $[T]=2$\footnote{See Dr. Tong's Quantum field theory notes for what this means if you are unsure.}, or equivalently $[\a']=-2$. Then recalling that [length$]=1$, it follows that we can define a length $\ell_s$ such that 
\be
\label{eqn:StringScale}
    \ell_s^2 := \a'.
\ee  
This is known as the \textit{string scale}, and it appears as the natural length scale in string theory. 

\section{The Polyakov Action}

Now recall that with the point particle action, we (somewhat hand-wavingly) argued that we could remove the square root present in \Cref{eqn:PointActionLorentz} by coupling it to gravity. We would now like to do a similar thing for the Nambu-Goto action. This will be particularly useful, as in the end we would like to quantise the action, and quantising square roots is a pain.\footnote{Those who have read my notes on Dr. Schuller's Quantum Mechanics course will have an insight into the sorts of troubles that can arise!}

\mybox{
\bd 
    The \textit{Polyakov action} is 
    \be 
    \label{eqn:PolyakovAction}
        S = -\frac{1}{4\pi\a'} \int d\sig d\tau \sqrt{-g} g^{\a\beta} \p_{\a} X^{\mu} \p_{\beta} X_{\mu},
    \ee 
    where $g:=\det g$.
\ed 
}

\bp 
\label{prop:PolyakovNambuGoto}
    The Polyakov action is classically equivalent, up to a gauge, to the Nambu-Goto action. 
\ep 

Before proving this proposition it is instructive to make some remarks and introduce a definition.

\br 
    The equations of motion one obtains from the Polyakov action are 
    \be 
    \label{eqn:EOMDynamicMetric}
        \p_{\a}\Big(\sqrt{-g} g^{\a\beta}\p_{\beta}X^{\mu}\Big) = 0,
    \ee 
    which is of exactly the same form as \Cref{eqn:EOMInducedMetric} with $\g \to g$. However, we should not be so quick to conclude that $g$ is therefore the induced metric. Recall, $\g$ was obtain from the flat Minkowski metric $\eta$, whereas $g$ is a independent variable, with its own equations of motion. These equations of motion will, necessarily,\footnote{Otherwise we would be introducing more degrees of freedom into our system.} fix the values of $g$, but we see that it is not the same thing as the induced metric. We can call $g$ the \textit{dynamical metric} on the worldsheet. 
\er 

\bd 
    A \textit{Weyl transformation} is a local rescaling of a metric, 
    \bse 
        g \to e^{2\phi(x)}g.
    \ese 
    It maps a metric to another metric of the same \textit{conformal class}, i.e. angles are preserved but lengths can vary locally.
\ed 

\br 
    Note, using 
    \bse 
        g^{\a\beta} \to e^{\phi(x)}g^{\a\beta} 
    \ese 
    under Weyl transformation, we see that the Polyakov action is Weyl invariant. This tells us that any variation of the action wrt the dynamic metric that corresponds to a Weyl transformation must lead to a trivial result (i.e. 0=0). We will see that this is the case in the proof. 
\er 

\bq (\Cref{prop:PolyakovNambuGoto}) 
    Let us vary the Polyakov action wrt $g^{\a\beta}$. Again, using the results from general relativity, we have 
    \be 
    \label{eqn:PolyakovVariation}
        0 = \del S = \del g^{\a\beta} \Big( \sqrt{-g} \p_{\a} X^{\mu} \p_{\beta} X_{\mu} - \frac{1}{2}g_{\a\beta} \sqrt{-g} g^{\rho\g} \p_{\rho}X^{\mu} \p_{\g} X_{\mu}\Big) 
    \ee 
    Firstly lets show that the Weyl transformation leads to a trivial result, 
    \bse 
    \begin{split}
         0 & = e^{\phi(x)}g^{\a\beta} \Big( \sqrt{-g} \p_{\a} X^{\mu} \p_{\beta} X_{\mu} - \frac{1}{2}g_{\a\beta} \sqrt{-g} g^{\rho\g} \p_{\rho}X^{\mu} \p_{\g} X_{\mu}\Big) \\
         & = e^{\phi(x)} \sqrt{-g} \big( g^{\a\beta}\p_{\a}X^{\mu} \p_{\beta}X_{\mu} - \frac{1}{2}g^{\a\beta}g_{\a\beta} g^{\rho\g} \p_{\rho}X^{\mu} \p_{\g} X_{\mu}\big) \\
         & = e^{\phi(x)} \sqrt{-g} \big( g^{\a\beta}\p_{\a}X^{\mu} \p_{\beta}X_{\mu} - \frac{2}{2} g^{\a\beta}\p_{\a}X^{\mu} \p_{\beta} X_{\mu}\big)  \\
         & = 0.
    \end{split} 
    \ese 
    Next dividing by $\del g^{\a\beta}\sqrt{-g}$ in \Cref{eqn:PolyakovVariation}, taking the determinant,  multiplying by $-1$, and then taking the square root, we have 
    \bse
    \begin{split}
        \sqrt{-\det\big(\p_{\a}X^{\mu}\p_{\beta}X_{\mu}\big)} & = \sqrt{-\det\bigg( \frac{1}{2} \big(g^{\rho\gamma}\p_{\rho}X^{\mu} \p_{\gamma} X_{\mu}\big) \cdot g_{\a\beta}\bigg)} \\
        & = \sqrt{-g} \cdot \frac{1}{2}g^{\a\beta}\p_{\a}X^{\mu} \p_{\beta} X_{\mu},
    \end{split}
    \ese 
    where we have used the fact that $\det aM = a^2\det M$ for any scalar multiple $a$ and $2\times 2$ matrix $M$. Finally plugging either side into the respective actions (i.e. plugging the left-hand side into the Nambu-Goto action, or the right-hand side into the Polyakov action) gives the other. 
    
    Finally we note that the two are only equivalent if we take the equivalence class 
    \bse
        g \sim \widetilde{g},
    \ese 
    where $\widetilde{g}$ is some Weyl transformation of $g$. It is easy to see why we need this, if we didn't the phase space of solutions for the Polyakov action would be infinitely times bigger then the phase space of the Nambu-Goto action. In other words, the equations of motion \Cref{eqn:EOMInducedMetric} and \Cref{eqn:EOMDynamicMetric} only have the same number of solutions if we take this equivalence class.
    
    This equivalence class corresponds to a (huge!) redundancy in the solutions and is a gauge symmetry (as promised by the proposition).
\eq 

\br 
    We should make a important remark here. The Weyl invariance only holds because we are considering a 2-dimensional worldsheet. If we were considering a $d$ dimensional sheet we would get
    \bse 
        S \to e^{\big(\frac{d}{2}-1\big)\phi(x)} S.
    \ese 
\er 

\br 
    Just as the Einbein told us that we have coupled the scalar fields of a point particle's worldline to a 1d gravity, the Polyakov action tells us that we have coupled the scalar fields of the string to a 2d gravity. 
\er 

\subsection{Repercussions of Weyl Invariance}

As we have just shown the Poylakov action is invariant under Weyl transformations. This is, obviously, not something general actions have, and so if we wish to maintain this invariance we need to be careful about how we modify \Cref{eqn:PolyakovAction}. For example, should we want to include some potential $V(x)$ into the action, we could not simply include it as 
\bse 
    \int d\sig d\tau \sqrt{-g} V(x),
\ese 
as this would not be Weyl invariant. In other words, any additional terms must either not contain anything to do with the metric $g$ or must come in invariant products (such as $\sqrt{-g}g^{\a\beta}$). 

\subsection{Using the Gauge}

As we have mentioned, the Polyakov action enjoys three gauge symmetries\footnote{Again, recall that the Lorentz symmetries are not gauge symmetries.}
\ben 
    \item Diffeomorphism (i.e. reparameterisation) in $\sig$,
    \item Diffeomorphism (i.e. reparameterisation) in $\tau$,
    \item Weyl invariance. 
\een 

Now the worldsheet metric has three independent components. So we can use these gauges to fix them. 
\bcl
    We can use the gauge symmetries to set $g_{\a\beta}$ such that it is the flat metric, i.e. 
    \be
    \label{eqn:FlatMetricPolyakov}
        ds^2 = -d\tau^2 + d\sig^2.
    \ee
\ecl 

We shall show why this is true in more detail later, however as a rough argument now: we use the diffeomorphisms to give us something that is locally conformal to the flat metric,\footnote{Note here the $\sig=(\tau,\sig)$ notation is being used.} 
\bse 
    g_{\a\beta} = e^{\Phi(\sig)}\eta_{\a\beta}. 
\ese 
We then use the Weyl invariance to remove the prefactor (i.e. set $\Phi(\sig)=0$), giving the flat metric. 

As an alternative way to see this at this stage, consider the problem geometrically. We are considering the surface traced out in spacetime from a propagating string. The shape of the string can vary as it propagates, so we are essentially left with a tube-like structure where we allow the surface to contains `lumps'. To clarify, we do not require the worldsheet surface to be a perfect hollow cylinder, but could contain intrinsic curvature given by the fact that $\sig$ varies. 

However, we can choose our parameterisations such that $\tau$ and $\sig$ always make the same angle with each other (namely perpendicular). We are therefore in a situation where we have a tube-like shape where the intrinsic curvature comes solely from the fact that length scales can vary locally; but this is just a conformal transformation of the hollow cylinder! We therefore use our Weyl invariance to remove this behaviour, giving us the metric on the surface of a cylinder, which we know from general relativity courses is just the flat metric. See \Cref{fig:WeylMetricPolyakov}.

\begin{figure}
    \begin{center}
        \btik 
            \draw[ultra thick] (0,0) -- (0,5);
            \draw[ultra thick] (2.5,0) -- (2.5,5);
            \draw[dashed, blue] (0,0.5) -- (2.5,0.5);
            \draw[dashed, blue] (0,0.7) -- (2.5,0.7);
             \draw[dashed, blue] (0,0.9) -- (2.5,0.9);
            \draw[dashed, blue] (0,1.5) -- (2.5,1.5);
            \draw[dashed, blue] (0,2.5) -- (2.5,2.5);
            \draw[dashed, blue] (0,3) -- (2.5,3);
            \draw[dashed, blue] (0,3.2) -- (2.5,3.2);
            \draw[dashed, blue] (0,4) -- (2.5,4);
            \draw[dashed, blue] (0,4.7) -- (2.5,4.7);
            \draw[dashed, red] (0.2,0) -- (0.2,5);
            \draw[dashed, red] (0.5,0) -- (0.5,5);
            \draw[dashed, red] (0.6,0) -- (0.6,5);
            \draw[dashed, red] (1,0) -- (1,5);
            \draw[dashed, red] (1.6,0) -- (1.6,5);
            \draw[dashed, red] (1.8,0) -- (1.8,5);
            \draw[dashed, red] (2.2,0) -- (2.2,5);
            %%
            \draw[ultra thick] (6,0) -- (6,5);
            \draw[ultra thick] (8.5,0) -- (8.5,5);
            \draw[dashed, blue] (6,0.5) -- (8.5,0.5);
            \draw[dashed, blue] (6,1.5) -- (8.5,1.5);
            \draw[dashed, blue] (6,2.5) -- (8.5,2.5);
            \draw[dashed, blue] (6,3.5) -- (8.5,3.5);
            \draw[dashed, blue] (6,4.5) -- (8.5,4.5);
            \draw[dashed, red] (6.2,0) -- (6.2,5);
            \draw[dashed, red] (6.7,0) -- (6.7,5);
            \draw[dashed, red] (7.2,0) -- (7.2,5);
            \draw[dashed, red] (7.7,0) -- (7.7,5);
            \draw[dashed, red] (8.2,0) -- (8.2,5);
        \etik 
    \end{center}
    \caption{A side on view of the worldsheet. The dashed lines are equidistant steps in $\textcolor{red}{\sig}$ and $\textcolor{blue}{\tau}$. The left-hand side image is intrinsically curved (the length scale varies as you move along), whereas the right-hand side is flat. The two are linked by a Weyl transformation on the metrics.}
    \label{fig:WeylMetricPolyakov}
\end{figure}

Now, using \Cref{eqn:FlatMetricPolyakov} the Polyakov action becomes
\be 
\label{eqn:PolyakovActionFlat}
    S = -\frac{1}{4\pi\a'} \int d\sig d\tau \p^{\a} X^{\mu} \p_a X_{\mu}. 
\ee 

We now need to ask an important question: 

\begin{center}
    Have we used up \textit{all} of our gauge freedom by setting the worldsheet metric to the flat metric? 
\end{center}

In other words, does making this specific choice of form for $g_{\a\beta}$ completely exhaust our gauge choice. In other (other) words, in choosing this particular gauge, do we cross all the gauge orbits\footnote{For those not familiar with a gauge orbit, it can be thought of as a set of lines, along which every element represents the same physical state, all being joined by a gauge transformation.} only once, or do we `run along' the some of the orbits, leaving us with some still further gauge freedom. 

It is important that this question is understood, and so we word in yet another way: does \textit{every} combination of diffeomorphism and Weyl transformation acting on the flat metric \Cref{eqn:FlatMetricPolyakov} change it (i.e. make it non-flat), or is there \textit{some} combination in which still leaves it as a flat metric (resulting in a gauge freedom)? In, yet other, words, is there any diffeomorphisms that change the metric by a Weyl factor? If so then we do have some gauge freedom left.

To answer this question, consider the change of coordinates into the worldsheet light-cone coordinates: 
\bd 
    The worldsheet light-cone coordinates are given by:
    \be 
    \label{eqn:WorldsheetLightConeCoord}
        \sig_{\pm} := \tau\pm\sig.
    \ee 
\ed 

Direct calculation tells us that 
\be 
\label{eqn:dSigPMdSig}
    d\tau^2 =  d\sig_+^2 + d\sig_-^2 + 2d\sig_+d\sig_-, \qquad  d\sig^2 =  d\sig_+^2 + d\sig_-^2 - 2d\sig_+d\sig_-,
\ee 
and so the metric \Cref{eqn:FlatMetricPolyakov} becomes 
\be 
\label{eqn:FlatMetricPolyakovLightcone}
    ds^2 = -4d\sig_+d\sig_-.
\ee 

Now, recall one way of wording our question was `does there exist diffeomorphisms such that the metric is only changed by a Weyl transformation?' i.e. it is just multiplied by some factor. From \Cref{eqn:FlatMetricPolyakovLightcone}, the answer becomes clear straight away; yes! Simply let 
\bse 
    \sig_+ = f(\widetilde{\sig}_+), \qquad \sig_- = g(\widetilde{\sig}_-),
\ese 
for some functions $f,g$, and where $\widetilde{\sig}$ is some diffeomorphism of the coordinates used to define $\sig_{\pm}$. It follows (from the definition of the exterior derivative, $d$) that 
\bse 
    ds^2 = f'(\widetilde{\sig}_+) g'(\widetilde{\sig}_-)d\widetilde{\sig}_+ d\widetilde{\sig}_-,
\ese 
where the prime indicates the derivatives. This is just a local conformal factor and so we can remove it using a Weyl transformation. So we do have some left over gauge freedom, which we will make use of soon. 

\br 
    A analogous result follows if we use $f(\widetilde{\sig}_-)$ and $g(\widetilde{\sig}_+)$. This is known as \textit{orientation reversal}. However we \textit{cannot} let $f$ or $g$ be a function of \textit{both} $\widetilde{\sig}_{\pm}$, as this would result in $d\sig_{\pm}^2$ terms in $ds^2$. Equally, we cannot have $f$ and $g$ being a function of the same $\widetilde{\sig}_{\pm}$.
\er 

\br 
    Note, that although we have an infinite amount of gauge invariance left (i.e. we can write $f$ and $g$ is all kinds of ways), this remaining gauge invariance is negligible compared to the amount of gauge invariance we started off with. The easiest way to see this is to consider the form the invariances came in. The three initial gauge invariances (the two diffeomorphisms and the Weyl transformation) came as functions of two variables; we used \textit{both} $\sig$ and $\tau$. However this reduced number of gauge invariances depend only on one variable each; $f$ and $g$ depend only on $\sig_+$ \textit{or} $\sig_-$ each. Now simply from the taking the Taylor expansion of a two variable function about the second variable shows that the number of one variable functions is infinitesimal compared to the number of two variable functions; only the leading order term in the expansion is a one variable function, but every other term in the expansion is two variables. 
\er 

From \Cref{eqn:WorldsheetLightConeCoord}, it follows by direct substitution that the Polyakov action can be written as 
\be
\label{eqn:PolyakovActionLightcone}
    S = \frac{1}{\pi\a'}\int d\sig d\tau \p_+X^{\mu}\p_-X_{\mu}.
\ee 
However, we just remarked that the solutions to the equations of motion of this action will be infinitely redundant (one redundancy for each $f,g$ choice). Therefore we must mod out these conformal diffeomorphisms from our phase space. We shall do this next lecture.

Finally note that in light-cone coordinates we have
\bse 
    g_{++} = 0 = g_{--}, \qquad g_{+-} = -1 = g_{-+},
\ese 
and so \Cref{eqn:PolyakovVariation} tells us 
\be 
\label{eqn:PolyakovConstrainsLightcone}
    (\p_+X)^2  = 0, \qquad \text{and} \qquad (\p_-X)^2  = 0,
\ee 
which we must apply as constraints to our solutions. 

\br 
    Note \Cref{eqn:PolyakovVariation} in the light-cone coordinates also gives us our required trivial solution ($0=0$). We expect this term to follow from the $g_{+-}/g_{-+}$ terms (as these are what the only non-vanishing components), which it does --- try it via direct calculation!
\er 

To summarise, what we have done so far is to start off with the daunting task of trying to find the solutions to the equations of motion from the Nambu-Goto action, but through some clever tricks we have managed to reduce the problem to finding the solutions to the equations of motion of \Cref{eqn:PolyakovActionLightcone} subject to the constraints \Cref{eqn:PolyakovConstrainsLightcone}, modulo the conformal diffeomorphisms. Although the latter might sound complicated, it is in-fact a much, much easier problem to solve.\footnote{Indeed, whenever we include extra conditions to a problem it almost always makes the problem \textit{sound} more and more complicated, however in-fact we are normally making it easier to solve; we have extra conditions and so `force' the problem into an easier problem. If you don't understand what I mean, consider trying to solve a set of 3 simultaneous variables for 3 variables vs trying to solve for the same 3 variables but now I give you 10 simultaneous equations.} 
\chapter{Mode Expansions and Quantisation of Bosonic String}

So far we have we have fixed the gauge of our problem such that the worldsheet metric is the flat metric. However, we also showed that this gauge fixing is not complete, that is we still have a gauge associated to conformal diffeomorphisms. We can express this condition as an equivalence class 
\be 
\label{eqn:ConformalDiffeoEquivClass}
   x^{\mu}\big( f_1(\sig_-), g_1(\sig_+)\big) \sim x^{\mu}(\sig_+, \sig_-) \sim x^{\mu}\big( f_2(\sig_+), g_2(\sig_-)\big),
\ee 
where we have included a subscript on the $f/g$ to indicate that it need they need not be the same functions. 

As mentioned in the previous lecture, we shall use this remaining gauge invariance in a crucial way in a moment. First, though, let's consider the equations of motion arising from varying \Cref{eqn:PolyakovActionLightcone} wrt $x^{\mu}$. Direct calculation gives us 
\be 
\label{eqn:EOMLightcone}
    \p_+\p_-x^{\mu} = 0.
\ee 
The most general solution to this problem is 
\be 
\label{eqn:EOMLightconeGeneralSolution}
    x^{\mu}(\tau,\sig) = x^{\mu}_L(\sig_+) + x^{\mu}_R(\sig_-),
\ee 
where the $L/R$ subscripts denote the left/right moving waves respectively. We can expand this out as Fourier modes, 
\be 
\label{eqn:FourierModesLightcone}
    \begin{split}
        x^{\mu}_L(\sig_+) & = a^{\mu} + b^{\mu} \sig_+ + \sum_{n\neq0} c^{\mu}_n e^{-in\sig_+}, \\
        x^{\mu}_R(\sig_-) & = \widetilde{a}^{\mu} + \widetilde{b}^{\mu} \sig_- + \sum_{n\neq0} \widetilde{c}^{\mu}_n e^{-in\sig_-},
    \end{split}
\ee 
for $n\in\Z$.

Now recall that we imposed the periodic condition 
\bse 
    x^{\mu}(\tau,\sig+2\pi) = x^{\mu}(\tau,\sig).
\ese 
In the Fourier series above the exponential terms are fine, but if we want the $b^{\mu}/\widetilde{b}^{\mu}$ terms to give an overall period term we require 
\bse 
    b^{\mu} = \widetilde{b}^{\mu},
\ese 
as then this term in $x^{\mu}(\tau,\sig)$ just becomes 
\bse 
    b^{\mu} (\sig_+ + \sig_-) = 2b^{\mu}\tau,
\ese
which clearly is unaffected by the periodic condition. 

\section{Lightcone Gauge}

As it stands $x^{\mu}$ has $d$ degrees of freedom (one for each $\mu$), however this clearly cannot be right. To understand why, consider a real string tied at both ends. Now ask the question about the how the string can move/oscillate. Clearly the answer is it can move along either of the transverse modes (i.e. it can move left-right or up-down). The longitudinal mode (along the string) is not accessible, and it sounds absurd to talk about it oscillating in the time dimension! So the number of degrees of freedom we have $4-2=2$. It is easy to convince yourself that this argument extends to some $d$-dimensional spacetime, in which case we expect $d-2$ degrees of freedom. 

So how do we reconcile this? The answer, of course, is use the gauge freedom and the constraints we place on the system. Consider first the gauge freedom. We are free to redefine $\sig_{\pm}$ as any one variable function of $\sig_{\pm}$. We can, therefore, choose to fix this gauge such that we isolate one of our coordinates. We can do this in such a way as to identify (up to a factor) this coordinate with $\sig_{\pm}$ itself (in exactly the same way that we did for the point like particle and the temporal coordinate). In other words, we can choose a gauge such that the exponential terms in \Cref{eqn:FourierModesLightcone} for a \textit{single}, given, $\mu$ vanish. 

We shall, in-fact, not do it for any of our current space-time coordinates, but instead do it for some rather strange choice of combination. 

\bd 
We define 
\be 
\label{eqn:xpm}
    x^{\pm} := t \pm x^{d-1}.
\ee 
\ed 

\br 
Note in making this choice we have broken Lorentz invariance --- we have singled out two of your coordinates and made them `more important' then the others. Nevertheless, we continue.
\er 

We now fix our gauge such that 
\be 
\label{eqn:xplus}
    x^+ = \a'p^+\tau = \frac{1}{2}\a'p^+ (\sig_+ +\sig_-),
\ee 
for some constant $p^+$. In other words we have 
\be 
\label{eqn:xplusLR}
    x^+_L = \frac{\a'p^+}{2} \sig_+, \qquad \text{and} \qquad x^+_R = \frac{\a'p^+}{2}\sig_-.
\ee 

\br 
To see why we have used the notation $p^+$, consider the following argument. The above scalar field has no $\sig$ dependence, and so it follows from \Cref{eqn:PolyakovActionFlat} that 
\bse 
    S = \frac{1}{2\a'} \int d\tau (\dot{x}^+)^2.
\ese
But this is the action for a relativistic particle of mass $m=1/\a'$. Then recall the conjugate momenta is simply $p=m\dot{x}$, which using the definition of $x^+$ gives
\bse 
    p = m \a' p^+ = \frac{1}{\a'} \a' p^+ = p^+.
\ese 
We shall see a more convincing argument for this when we move on to the quantisation. 
\er 

Ok, so we've managed to remove one of the degrees of freedom through this gauge fixing, leaving us with $d-1$ degrees of freedom. How do we remove the other? As mentioned above, we use the constraint we placed on the system at the end of the last lecture, 
\bse 
    (\p_+x)^2 := \p_+x^{\mu}\p_+x_{\mu} = 0,
\ese 
and similarly for $\p_-$. Expanding this out gives\footnote{We are using $x^0=t$, as we did in defining $x^{\pm}$.}
\bse 
    \begin{split}
        0 & = -\p_+t\p_+t + \sum_{i=1}^{d-1} \p_+x^i\p_+x_i  \\
        & = -\p_+t\p_+t + \p_+x^{d-1}_L\p_+x_{d-1,L} + \sum_{j=1}^{d-2} \p_+x^j_L\p_+x_{j,L} \\
        & = -\p_+x^+_L\p_+x^-_L + \sum_{j=1}^{d-2} \p_+x^j_L\p_+x_{j,L}, 
    \end{split}
\ese 
where we've used the fact that only the left-wave part depends on $\sig_+$. Then, using our fixture of gauge for $x^+$, gives a condition for $x^-_L$, namely 
\be 
\label{eqn:xmLCondition}
    \frac{\a'p^+}{2} \p_+x^-_L = \sum_{j=1}^{d-2} \p_+x^j_L\p_+x_{j,L}.
\ee 
The same method for the $(\p_-x)^2=0$ equation gives us the condition for $x^-_R$,
\be 
\label{eqn:xmRCondition}
    \frac{\a'p^+}{2} \p_+x^-_R = \sum_{j=1}^{d-2} \p_+x^j_R\p_+x_{j,R}.
\ee 
So, apart from the constant factor (the $a/\widetilde{a}$ in \Cref{eqn:FourierModesLightcone}), we can determine $x^-$ given that we know the remaining $d-2$ scalar field solutions. So we have successfully reduced the problem to one with only $d-2$ degrees of freedom, which represent the different transverse modes of the string.

\br 
Note the fact that we have no longitudinal modes makes perfect sense when we remember we have a gauge freedom associated to spatial diffeomorphisms. The easiest way to think of this would be to imagine having beads\footnote{Be careful to not take this analogy too far and think about the string being made up of particles. The idea of string theory is that the string is the fundamental building block of nature!} on our string. A longitudinal mode would move these beads along the direction of the string, but that is just equivalent to a spatial diffeomorphism --- the string is still the string.
\er 

\subsection{Mode Expansion Coeffients} 

In light of what is to come, we want to renormalise \Cref{eqn:FourierModesLightcone} to be 
\be 
\label{eqn:FourierModesNormalised}
    \begin{split}
        x^{\mu}_L (\sig_+) & = \frac{1}{2}x^{\mu}_0 + \frac{1}{2}\a'p^{\mu} \sig_+ + i\sqrt{\frac{\a'}{2}}\sum_{n\neq0} \frac{\a^{\mu}_n}{n} e^{-in\sig_+} \\
        x^{\mu}_R (\sig_-) & = \frac{1}{2}x^{\mu}_0 + \frac{1}{2}\a'p^{\mu} \sig_- + i\sqrt{\frac{\a'}{2}}\sum_{n\neq0} \frac{\widetilde{\a}^{\mu}_n}{n} e^{-in\sig_-},
    \end{split}
\ee 
where we have labelled the zero-order term $x^{\mu}_0$, because, as we shall see, it corresponds to the position of the string's centre of mass. We have also included the rooted fraction before the summations. This is done in order for our Poisson bracket (and so the commutation relations) to come out in a nice way. It is for the same reason that we have divided by $n$ inside the summations.

Now, if we are to interpret $x^{\mu}_0$ as the position of the centre of mass of the string, we also need it present in the definitions of $x^{\pm}$, i.e. we need 
\be 
    x^+ = x^+_0 + \a' p^+ \tau.
\ee 
Note, the constraints we have placed on $x^-$ will not give us the value of $x^-_0$, it is an integration constant. 

\br 
It worth remarking the following condition. The $x^{\mu}_{L/R}$ are obviously real, and so it follows that 
\be 
\label{eqn:FourierAlphaConjugate}
    \a^{\mu}_n = \big(\a^{\mu}_{-n}\big)^*, \qquad \text{and} \qquad \widetilde{\a}^{\mu}_n = \big(\widetilde{a}^{\mu}_{-n}\big)^*.
\ee 
This will become important once we quantize the system and start to related $\a/\widetilde{\a}$ to creation/annihilation operators.
\er 

\section{Quantisation}

We now wish to quantise our classical theory of a string to obtain a quantum theory. Before moving on to do this, let's first make a comment on the potential ways we could do this. 

There are in-fact two way we could go about quantising our string. The method we are going to use (as we've already begun doing it...) is to first impose the constraints onto the classical system (as we did above to remove two degrees of freedom) and then seek to quantise this constrained system. However, this was not the only option. We could have decided to first quantise the system (i.e. way back before we'd even defined the lightcone gauge, $x^{\pm}$) and then placed our constraints onto the quantised system. This procedure is known as \textit{covariant quantisation}. We shall not discuss the covariant quantisation approach here\footnote{If anyone thinks it would be particularly beneficial to add it to these notes, feel free to contact me and I shall try add a short section.} as problems arise from it (namely we get so-called \textit{ghost} states --- states with negative norm!). The interested reader, however, can read the short explanation given in Section 2.1 of Dr. Tong's notes.

\subsection{Poisson Brackets and Sympletic Forms}

A Poisson bracket is an important structure in classical mechanics. It is the classical equivalent to the quantum mechanical commutation relations, and it is through the Poisson bracket that we seek our commutation relations. Mathematically: 

\bd[Poisson Bracket] 
A Poisson bracket is a bilinear operator 
\bse 
    \{ \cdot, \cdot \} : C^{\infty}(\cF) \times  C^{\infty}(\cF) \to  C^{\infty}(\cF),
\ese 
where $C^{\infty}(\cF)$ is the space of smooth functions on phase space, $\cF$, of the system. It obeys the following properties: 
\ben 
\item Anticommutivity: $\{f,g\} = -\{g,f\}$, 
\item Binlineatity: $\{af+g,bh+\ell\} = ab\{f,h\} + a\{f,\ell\} + b\{g,h\} + \{g,\ell\}$,
\item Leibniz: $\{fg,h\} = f\{g,h\} + \{f,h\}g$,
\item Jaccobi: $\{f,\{g,h\}\} + \{g,\{h,f\}\} + \{h,\{f,g\}\} =0$.
\een 
\ed 

We can find a set of canonical coordinates\footnote{That is a set of coordinates on the phase space that can be used to define the physical state of the system --- i.e. the `position' and `momentum' of the system.} such that the Poisson brackets take a particularly neat form:
\be 
\label{eqn:PoissonCanonical}
    \{q^i,p^j\} = \del_{ij}, \qquad \{p^i,p^j\} = 0 = \{q^i,q^j\},
\ee 
where $p^i/q^i$ are the $i$-th canonical position/momentum, respectively. 

\bd[Sympletic Form]
Let $\cM$ be a $2n$-dimensional manifold. A 2-form $\omega$ on $\cM$ that is globally nondegenerate,
\bse 
    \omega(X,Y) = 0 \quad \forall Y \implies X = 0,
\ese 
closed
\bse 
    d\omega = 0,
\ese 
is called a \textit{symplectic form} (and $\cM$ is a symplectic manifold). 
\ed 

\bcl
The phase space of a system is a symplectic manifold. 
\ecl
We shall not prove this here as the details are not too important.\footnote{The argument comes from the fact that the phase space can be viewed as the cotangent bundle to the manifold describing the set of all possible configurations, along with the fact that in classical mechanics cotangent bundles are symplectic manifolds.}

\bt[Darboux] 
A symplectic $2n$-dimensional manifold $(\cM,\omega)$ is locally isomorphic to $(\R^{2n},\omega)$. In other words, we can find local coordinates $\{q^i,p^i\}$ for $i=1,...,n$, such that 
\bse
    \omega = dq^i \wedge dp_i.
\ese 
\et 

\br 
Note the above summation only works for the canonical coordinates such that \Cref{eqn:PoissonCanonical} hold. See \Cref{prop:PoissonSymplecticForm} for more clarity.
\er 

\bp 
In canonical coordinates, the Poisson bracket on the phase space can be written 
\bse 
    \{f,g\} = \omega(X_f,X_g) = \cL_{X_g} f,
\ese 
where 
\bse 
    X_{p^i} := \frac{\p}{\p q^i}, \qquad X_{q^i} := -\frac{\p}{\p p^i},
\ese 
and $\cL$ is the so-called Lie derivative.\footnote{For those unfamiliar, as far as we are concerned here it is just the regular derivative as we are considering scalar fields.}
\ep 

\bq 
Due to the bilinearity of the Poisson bracket it suffices to show the result for the coordinate basis (i.e. prove \Cref{eqn:PoissonCanonical})
\bse 
    \begin{split}
        \{q^i, p^j\} & = \cL_{\frac{\p}{\p q^j}} q^i = \frac{\p}{\p q^j} q^i = \del_{ij}, \\ 
        \{p^i, p^j\} & = \cL_{\frac{\p}{\p q^j}} p^i = \frac{\p}{\p q^j} p^i = 0,
    \end{split}
\ese 
and similarly for $\{q^i,q^j\}$.
\eq 

\br 
The proof that the four Poisson bracket conditions are satisfied via the above definition is given on the \href{https://en.wikipedia.org/wiki/Poisson_bracket}{Wiki page}.
\er 

\mybox{
\bp 
\label{prop:PoissonSymplecticForm}
Let $(\cM,\omega)$ be a symplectic manifold of dimension $2d$. Let $\{q^i,p^i\}$ with $i=1,...,d$ be a set of canonical coordinates with Poisson bracket relations \Cref{eqn:PoissonCanonical}. Let $\{e^n\}$ for $n=1,...,2d$ be a new set of coordinates given by a canonical transformation.\footnote{I.e. they do not mix $q$ and $p$.}  The the Poisson bracket relations for $\{e^n\}$ can be represented by a $2d\times 2d$ antisymmetric matrix
\bse 
    \{e^n,e^m\} = \omega^{nm},
\ese 
such that the symplectic form (in the dual basis formed using the new coordinates) is
\bse 
    \omega = \frac{1}{2}\omega_{nm}de^n\wedge de^m,
\ese 
where $\omega_{nm}$ is the inverse of $\omega^{nm}$.
\ep
}

 
The easiest way to see that this is true is to consider the following examples in turn.\footnote{I am going to be being fairly sloppy with indices here, but the main parts of the argument hold.} 

\bex 
First consider the trivial transformation
\bse
    e^{2i-1} := q^i, \qquad e^{2i} := p^i,
\ese 
for $i=1,...,d$. From \Cref{eqn:PoissonCanonical} we know that 
\bse 
    \{e^{2i-1},e^{2j}\} = \del^{ij}, \qquad \text{and} \qquad \{e^{2i-1},e^{2j-1}\} = 0 = \{e^{2i},e^{2j}\}. 
\ese 
Finally defining the matrix $(\omega^{nm})$ to be
\bse 
    (\omega^{nm}) := \begin{pmatrix}
    M & 0 & 0 & ... \\
    0 & M & 0 & ... \\
    \vdots & \vdots & \vdots & \vdots \\
    0 & 0 & ... & M 
    \end{pmatrix}, \qquad M := \begin{pmatrix}
    0 & 1 \\ 
    -1 & 0 
    \end{pmatrix},
\ese 
gives us $\{e^n,e^m\} = \omega^{nm}$.

Now, from the definition of the exterior derivative, we have
\bse 
    de^{2i-1} = dq^i, \qquad de^{2i} = dp^i,
\ese 
from which it follows that 
\bse 
    de^n\wedge de^m = \begin{cases}
    dp^i \wedge dp^j & \text{for } n=2i, m=2j \\
    dq^i \wedge dq^j & \text{for } n=2i-1, m=2j-1 \\
    dq^i \wedge dp^j & \text{for } n=2i-1, m=2j \\
    dp^i \wedge dq^j = - dq^j \wedge dp^i & \text{for } n=2i, m=2j-1 
    \end{cases}
\ese 
So, if we wanted to write 
\bse 
    \omega = \sum_i dq^i\wedge dp^i = \frac{1}{2}\omega_{nm}de^n\wedge de^m,
\ese 
we would need\footnote{Note that we used the convention that in $\omega^{nm}$ the $n$ tells us the row and $m$ the column, so if we define $\omega_{nm}$ to be the inverse, we then have $n$ being the column and $m$ being the row. This is seen easily using the formula $\omega^{ab}\omega_{bc} =\del^a_c$.}
\bse 
    (\omega_{nm}) := \begin{pmatrix}
    \widetilde{M} & 0 & 0 & ... \\
    0 & \widetilde{M} & 0 & ... \\
    \vdots & \vdots & \vdots & \vdots \\
    0 & 0 & ... & \widetilde{M}
    \end{pmatrix}, \qquad \widetilde{M} := \begin{pmatrix}
    0 & -1 \\ 
    1 & 0 
    \end{pmatrix}.
\ese 

To see why/check, consider: 
\begin{itemize}
    \item When both $n$ and $m$ are even or odd the matrix element vanishes so there are no $dp^i\wedge dp^j$ or $dq^i\wedge dq^j$ terms. 
    \item When $n$ is odd and $m$ is even we get $(-1)\cdot dp^i\wedge dq^i = dq^i\wedge dp^i$ terms and any $i\neq j$ terms vanish. 
    \item When $n$ is even and $m$ is odd we get $1\cdot -dp^i\wedge dq^i = dq^i\wedge dp^i$ terms and any $i\neq j$ terms vanish. 
    \item From the summation convention the two previous terms add together, but the $1/2$ in the $\omega$ definition cancels this. 
\end{itemize}
\eex 

\bex 
Showing the result for a less trivial transformation is much more tedious/super easy to make mistakes. For example consider the general canonical transformation
\bse 
    e^k = {a^k}_i p^i, \qquad \text{and} \qquad e^{\ell} = {b^{\ell}}_i q^i,
\ese 
for $i,k=1,...,d$, and $\ell=d+1,...,2d$, and the $a/b$s are constants. The Poisson relations for $p/q$ then tell us\footnote{For the final two cases recall we're using summation convention so we have two sums, the delta function removes but its important the other remains!}
\bse 
    \{e^n,e^m\} = \begin{cases}
    0 & \text{for } n,m\in\{1,...,d\} \text{ or } n,m\in\{d+1,...,2d\} \\
    \sum_{i=1}^d {a^k}_i {b^{\ell}}_i & \text{for } n=k, m=\ell \\
    -\sum_{i=1}^d {a^k}_i {b^{\ell}}_i & \text{for } n=\ell, m=k 
\end{cases}
\ese 
Therefore, we would need to define 
\bse 
    (\omega^{nm}) := \frac{1}{2}\begin{pmatrix}
    0 & M \\
    -M^T & 0 
    \end{pmatrix},
\ese 
where $M$ is a $d\times d$ matrix given by 
\bse 
    M := \sum_{i=1}^d \begin{pmatrix}
    {a^1}_i {b^{d+1}}_i & {a^1}_i {b^{d+2}}_i & ... & {a^1}_i {b^{2d}}_i \\
    {a^2}_i {b^{d+1}}_i & ... & ... & ... \\
    \vdots & \vdots & \vdots & \vdots \\
    {a^d}_i {b^{d+1}}_i & ... & ... & {a^d}_i {b^{2d}}_i
    \end{pmatrix}.
\ese 
Finding the inverse to this matrix is a nightmare!\footnote{If you can do it in a nice way and show it gives the result we need, please feel free to send me it via email and I shall write it up here and give you credit.} This nightmare is reflected in what you get when you look for $(\omega_{nm})$ using the exterior derivative. 

First define ${a_k}^i := ({a^k}_i)^{-1}$ and similarly for the $b$s. Then we have 
\bse 
    \omega = \sum_i dq^i \wedge dp^i = \sum_i {a_k}^i{b_{\ell}}^i de^{\ell} \wedge de^k,
\ese 
which is a triple sum!\footnote{Again if you have some nice way of showing these two things are equivalent, please let me know!} However, I do not wish to leave you high and dry. So let's consider a specific example. 

Let 
\bse 
    e^1 = 2q^1 + 3q^2, \quad e^2 = p^1 + 4p^2, \quad e^3 = 3q^1 - q^2, \quad e^4 = 2p^1 + 2p^2.
\ese 
Then the Poisson condition tells us 
\bse 
    \omega^{nm} = \begin{pmatrix}
    0 & 14 & 0 & 10 \\
    -14 & 0 & 1 & 0 \\
    0 & -1 & 0 & 4 \\
    -10 & 0 & -4 & 0 
    \end{pmatrix} \implies \omega_{nm} = \frac{1}{66} \begin{pmatrix}
    0 & -4 & 0 & -1 \\
    4 & 0 & -10 & 0 \\
    0 & 10 & 0 & -14 \\
    1 & 0 & 14 & 0 
    \end{pmatrix}.
\ese 
Then we also have 
\bse 
    \begin{split}
        dq^1 & = \frac{1}{11} \big( de^1 + 3de^3\big), \qquad dp^2 = \frac{1}{11} \big( 3de^1 -2 de^3\big) \\
        dp^1 & = \frac{1}{3} \big(2 de^4 - de^2\big), \qquad dp^2 = \frac{1}{6} \big( 2de^2 - de^4\big).
    \end{split}
\ese 
So we have 
\bse 
    \begin{split}
        \omega & = dq^1 \wedge dp^1 + dq^2 \wedge dp^2 \\
        & = \frac{1}{33} \big( de^1 + 3de^3\big)\wedge \big( 2 de^4 - de^2\big) + \frac{1}{66} \big( 3de^1 -2 de^3\big)\wedge \big( 2de^2 - de^4\big) \\
        & = \frac{1}{66} \big( 4 de^1\wedge de^2 + de^1\wedge de^4 + 10de^2\wedge de^3 + 14de^3\wedge de^4\big),
    \end{split}
\ese 
which gives the same matrix for $\omega_{nm}$.
\eex 

\section{Quantising the String}

Once you have the Poisson bracket relations for a classical system you can quantise it by promoting the position/momentum to operators and defining the quantum mechanical commutator relations\footnote{Note this equal sign is clearly none sense, the LHS is a quantum mechanical equation whereas the RHS is classical. However we get the idea.}

\be 
    [q^i,p_j] := i\hbar \{q^i,p_j\},
\ee 
where we have used a lower index for $p$ as this is standard notation. This is known as \textit{canonical quantisation}. 

\br 
We shall now replace $q^i \to x^i$, as this becomes consistent with the notation we've been using so far.
\er 

So our task of quantising the string consists of finding the conjugate momentum, defining the symplectic form, inverting the matrix we get to give us the Poisson bracket relations and then quantising these brackets. 

Recall that our action is of the form 
\bse 
    S = - \frac{1}{4\pi\a'} \int d\sig d\tau \p_{\a}x^{\mu} \p_{\beta} x_{\mu} g^{\a\beta},
\ese 
where the metric here is the one on the worldsheet. Then using the definition 
\bse 
    p_{\mu} := \frac{\p\cL}{\p \dot{x}^{\mu}},
\ese    
we have 
\be 
\label{eqn:ConjugateMomentum}
    p_{\mu} = \frac{1}{2\pi\a'} \dot{x}_{\mu},
\ee 
and so our symplectic form is given by\footnote{Note we get a factor of 2 for the $x^i$ terms. This comes from the fact that both the wedge and the matrix $\omega_{nm}$ are antisymmetric, so you get twice the terms.} 
\be 
\label{eqn:SymplecticFormLightconeGauge}
    \begin{split}
        \omega &= \frac{1}{2}\frac{1}{2\pi\a'}\int d\sig dx^{\mu} \wedge d\dot{x}_{\mu} \\
        &= \frac{1}{4\pi\a'} \int d\sig \bigg( - dx^+\wedge d\dot{x}^- - dx^- \wedge d\dot{x}^+ + 2\sum_{i=1}^{2d-1} d x^i \wedge d\dot{x}^i\bigg),
    \end{split}
\ee 
where we have used our lightcone gauge to go to the last line. 

Now the first thing we notice is that, because we are integrating over $\sig \in [0,2\pi)$, if one of the terms in the wedge is purely zero mode (i.e. it doesn't have any exponential terms) then we only get a contribution if the other term in the wedge is also zero mode. That is, if we mix the zero modes with the oscillator terms we end up with
\bse 
    \int d\sig A e^{\pm i\sig}
\ese 
terms which vanish. This is a fantastic result because it means that all of the horrible oscillatory behaviour of $x^-$ goes (as $x^+$ is purely zero mode by construction). Therefore, if we want to work out the contribution to the symplectic form from the oscillator modes, we only need to worry about the $x^i$s; but we have a formula for these, 
\bse 
    x^i = x^i_0 + \a p^i \tau + i\sqrt{\frac{\a'}{2}} \sum_{n\neq0} \frac{\a^i_n}{n} e^{-in\sig^+} + \frac{\widetilde{\a}^i_n}{n} e^{-in\sig^-}.
\ese 

Again we need only worry with the terms in the summation (as the others drop out) and so we get 
\bse 
    \omega = \frac{1}{4\pi\a'} (i)^2 \frac{\a'}{2} (-i) 2\sum_{i=1}^{d-2} \sum_{n\neq 0} \sum_{m\neq 0} \int d\sig \bigg( \frac{d\a^i_n}{n} e^{-in\sig^+} + \frac{d\widetilde{\a}^i_n}{n} e^{-in\sig^-}\bigg)\wedge \bigg(d\a^i_m e^{-im\sig^+} + d\widetilde{\a}^i_m e^{-im\sig^-}\bigg).
\ese 
Now, this is going to give us four terms. Lets consider them in turn
\begin{itemize}
    \item When we have a $d\a_n \wedge d\a_m$ term the exponential is $e^{-i(n+m)\sig}$. So the only terms that remain after the integral is when $m=-n$. 
    \item Similarly to above for the $d\widetilde{\a}_n\wedge d\widetilde{\a}_m$ terms. 
    \item For the cross terms (i.e. one $\a$ and one $\widetilde{\a}$) we want $n=m$. Luckily, though, we're going to get $d\a_n\wedge d\widetilde{\a}_n + d\widetilde{\a}_n\wedge d\a_n= 0$, by the antisymmetry of the wedge product. 
\end{itemize}

So what we're left with is 
\bse 
    \omega = \frac{i}{2} \sum_{i=1}^{d-2} \sum_{n\neq0} \frac{1}{n} \big( d\a^i_n\wedge d\a^i_{-n} + d\widetilde{\a}^i_n \wedge d\widetilde{\a}^i_{-n}\big).
\ese 
Now note that 
\bse 
    \frac{1}{-n} d\a_{-n} \wedge d\a_n = \frac{1}{n}d\a_n \wedge d\a_{-n},
\ese 
and the same for the $\widetilde{\a}$ term. We can then consider only positive $n$ and introduce a factor of $2$, giving us 
\bse 
    \omega = i \sum_{n=1}^{\infty} \sum_{i=1}^{d-2} \frac{1}{n} \big( d\a^i_n\wedge d\a^i_{-n} + d\widetilde{\a}^i_n \wedge d\widetilde{\a}^i_{-n}\big).
\ese
Next note that we now have a flat (i.e. diagonal) expression, and so we don't have to worry about the $1/2$ factor in \Cref{prop:PoissonSymplecticForm} and so we have 
\bse 
    \omega_{ij} = \frac{i}{n} \b1_{d-2} \implies  \omega^{ij} = -in \b1_{d-2},
\ese 
where $\b1_{d-2}$ is the $d-2$ identity matrix. In terms of Poisson bracket relations, that is 
\bse 
    \{\a^i_n, \a^j_m\} = -in \del^{ij}\del_{n+m,0} = \{\widetilde{\a}^i_n, \widetilde{\a}^j_m\},
\ese 
which finally gives us the quantum mechanical commutation relations\footnote{In units $\hbar=1$.} 
\be 
\label{eqn:AlphaCommutationRelations}
    [\a^i_n, \a^j_m] =  n \del^{ij}\del_{n+m,0} = [\widetilde{\a}^i_n, \widetilde{\a}^j_m].
\ee 

Recalling \Cref{eqn:FourierAlphaConjugate}, this result looks almost like the commutation relations for creation/annihilation operators of a quantum harmonic oscillator, the only problem is the $n$. This is easily fixed though; we define (dropping the $i$ index for a moment)

\be 
\label{eqn:CreationAnnihilationOperatos}
    a_n = \frac{\a_n}{\sqrt{n}}, \qquad a^{\dagger}_n := \frac{\a_{-n}}{\sqrt{n}},
\ee 
for $n>0$, and similarly we define $\widetilde{a}_n/\widetilde{a}^{\dagger}_n$. 

\br 
So we see we have an two infinite (as $n$ runs to infinity) towers of creation/annihilation operators. We could now ask what the energy of these states are. The classical derivation of the Hamiltonian (which we do not do here\footnote{You have to consider the Hamiltonian density in terms of the $\cL$ and then plug in the conjugate momenta etc.}) gives us 
\bse 
    H = \frac{1}{2} \sum_{n=1}^{\infty} \sum_{i=1}^{d-2} (\a_n \a_{-n} + \a_{-n}\a_n)  + \frac{1}{2} \sum_{n=1}^{\infty} \sum_{i=1}^{d-2} (\widetilde{\a}_n \widetilde{\a}_{-n} + \widetilde{\a}_{-n}\widetilde{\a}_n),
\ese
which is just the classical Hamiltonian for the harmonic oscillator. After quantising we expect something of the form 
\bse 
    H^i_n = \omega \Big[ (a^i_n)^{\dagger}a^i_n  + 1/2 \Big]  + \omega  \Big[ (\widetilde{a}^i_n)^{\dagger}\widetilde{a}^i_n + 1/2 \Big],
\ese 
which, once combined with \Cref{eqn:CreationAnnihilationOperatos}, tells us that the energy spacing between the $n$-th and the $(n+1)$-th excitation of the $i$-th component is $n$. (i.e. $\omega=n$). 
\er 

\br 
Note that all of this is just considering the oscillatory parts, we still need to do the rest. That is the topic of next lecture.
\er 
\chapter{`Poof' of $d=26$}

So far we have managed to quantise just the oscillator parts of our scalar fields. We now look to include contributions from the zero mode parts. If we consider just the zero mode parts, our symplectic form, \Cref{eqn:SymplecticFormLightconeGauge}, gives us 
\bse 
    \{X^i_0 , p_j\} = \del^{ij}, \qquad \{X^+,p^-\} = -1 = \{X^-,p^+\},
\ese
which gives us the commutation relations 
\be 
    [X^i_0,p^j] = i\del^{ij}, \qquad [X^+,p^-] = -i = [X^-,p^+].
\ee 

\section{Level Matching}
So all is done, right? Well not quite, because we still haven't fully used up our constraints 
\bse 
    \p_+X^{\mu}\p_+X_{\mu} = 0 = \p_-X^{\mu}\p_-X_{\mu},
\ese
as we have not yet accounted for the zero mode contributions. The clever way to study this constraint is to actually study the integrals 
\bse 
    \int d\sig \p_+X^{\mu}\p_+X_{\mu} = 0,
\ese 
and similarly for $\p_-$. In the lightcone gauge this simply reads 
\bse 
    \int d\sig \Big(-\p_+X^-\p_+X^+ -\p_+X^+\p_+X^- + \sum_{i=1}^{d-2}\p_+X^i\p_+X^i \Big)= 0,
\ese 
which, using again the fact that the cross terms with one zero mode and one oscillator cancel, gives us the condition
\be 
\label{eqn:LevelMatchingStep}
    \bigg(\frac{\a'}{2}\bigg)^2 p^{\mu}p_{\mu} + \frac{\a'}{2}\sum_{i=1}^{d-2} \sum_{n=1}^{\infty} \big( \a^i_n\a^i_{-n} + \a^i_{-n}\a^i_{n} \big) = 0.
\ee 
A similar expression is obtained using the $\p_-$ condition but with $\a\to\widetilde{\a}$. We now notice that the first term is (something times) the mass-squared; but this term is the same for both the $\a$ equation and the $\widetilde{\a}$ equation! Let's consider what this means for us.

Our Hilbert space is given by the tensor product of an infinite number of harmonic oscillators (which come from the $\a/\widetilde{\a}$ commutation relations) \textit{tensored with} a spatial wavefunction of $d$ variables (which comes from the fact that we have $x/p$ commutation relations),\footnote{This is not some technical notation, its just some shorthand I've made up to say what we've just written. HO means harmonic oscillator and $L^2(\R,\mu)$ is the space of square integrable functions with respect to the measure $\mu$.}
\bse 
    \cH = \cH_{HO}\otimes L^2(\R^d,\mu).
\ese 
In other words, we define our ground state to be 
\be 
\label{eqn:GroundState}
    \ket{0}\otimes \psi(x),
\ee 
where $\psi(x)$ is a square integrable function, and where
\be 
\label{eqn:VacuumState}
    \a^i_n\ket{0} = \widetilde{\a}^i_n \ket{0} = 0 
\ee 
for all $i,n$ in our ranges. We then build up the Fock space by repeated application of all the different creation operators $a^{\dagger}_n/\widetilde{a}^{\dagger}_n$. If we view this in momentum space we can write this ground state as $\ket{0;p}$ where 
\be 
\label{eqn:GroundStateP}
    \hat{p}^{\mu} \ket{0;p} = p^{\mu}\ket{0;p},
\ee 
where we have put a hat on the LHS $p^{\mu}$ to indicate that it is an operator while the RHS is an eigenvalue. 

Now things looks interesting: using \Cref{eqn:LevelMatchingStep} we get a sum $p^{\mu}p_{\mu}$, which is related to the mass of a particle, and so we see that different excitations of the harmonic oscillators in the system give rise to particles of different masses given by 
\be 
\label{eqn:MassOfExcitation}
    m^2 = \frac{2}{\a'} \sum_{i=1}^{d-2}\sum_{n=1}^{\infty} \big( \a^i_n\a^i_{-n} + \a^i_{-n}\a^i_{n} \big).
\ee 
An alternative way to see this is to consider the Klein-Gordan equation
\be 
\label{eqn:KleinGordan}
    (\p^{\mu} \p_{\mu} - m^2) \psi = 0,
\ee 
with the usual quantum prescription $p_{\mu} = -i\p_{\mu}$

Now we said above that you obtain the same result but with $\a\to\widetilde{\a}$, giving us the so-called level-matching condition
\be
\label{eqn:LevelMatching}
    \sum_{i=1}^{d-2}\sum_{n=1}^{\infty} \big( \a^i_n\a^i_{-n} + \a^i_{-n}\a^i_{n} \big) = \sum_{i=1}^{d-2}\sum_{n=1}^{\infty} \big( \widetilde{\a}^i_n\widetilde{\a}^i_{-n} + \widetilde{\a}^i_{-n}\widetilde{\a}^i_{n} \big) 
\ee 

\section{$d=26$}

As nice as the the mass expression is, it would be nicer if we could write it in some normal ordered fashion --- i.e. place all the creation operators ($\a^i_{-n}$) on the left. This is done using our commutation relation, giving 
\be 
\label{eqn:MassNormalOrder}
    m^2 = \frac{4}{\a'} \sum_{i=1}^{d-2} \sum_{n=1}^{\infty} \bigg(\a^i_{-n}\a^i_{n} + \frac{n}{2}\bigg),
\ee 
and similarly for $\widetilde{\a}$. This, in turn, changes the level matching condition to 
\be 
\label{eqn:LevelMatchingNormalOrder}
    \sum_{i=1}^{d-2} \sum_{n=1}^{\infty} \a^i_{-n}\a^i_{n} = \sum_{i=1}^{d-2} \sum_{n=1}^{\infty} \widetilde{\a}^i_{-n}\widetilde{\a}^i_{n}.
\ee 

\br
    The origin of the level matching condition can be traced all the way back to the reparameterisation invariance $\sig \to \sig + \sig_0$ for some constant $\sig_0$. When we do this, $\sig^{\pm} \to \sig^{\pm} \pm \sig_0$, and so this shifting couples left moving minus right moving. The reason it comes out here is because this result was obtained from our constraint, which stemmed from diffeomorphism invariance. Specifically we are looking at the zero mode part of the constraint, which, if we are to respect the periodicity of $\sig$, can only be given by a shift in $\sig$. 
\er 

The above all looks rather nice, however we should not celebrate too quickly. The $n/2$ term in \Cref{eqn:MassNormalOrder} is divergent!

Firstly we note, by the same argument as in the previous remark, that translations in $\tau$ are related to the addition of the left and right travelling modes. The level matching condition, just equates the two terms and we get twice the left moving (or right moving) term. Translations in time are related to the Hamiltonian, and the level matching condition comes from setting the integrals above to zero, and so it follows that the Hamiltonian on the worldsheet of the string must vanish. Indeed this must be the case, because \textit{every} element of the stress-energy tensor must vanish.\footnote{See Dr. Tong's notes from equation (1.30)-(1.33).} This just tells us that the $n/2$ terms above are just the contributions to the zero-point energy, and our problem is the fact that we are trying to equate zero (Hamiltonian) to infinity (zero point energy).

This is just the standard problem in quantum field theory for infinite zero point energy. There is one important difference here, though. In QFT this problem is often `dealt with' by chanting the mantra that only energy \textit{differences} are physically important and so who cares if we start at infinity, as long as we can still measure the differences. However, in string theory we are considering 2-dimensional \textit{gravity}, and gravity sees absolute energy!

So what do we do? We add a counter term to our action such that our physical quantities make sense and behave how we like. There is a very natural choice of how to do this: we introduce the cosmological constant to our action, 

\be 
    S = -\frac{1}{4\pi\a'} \int d\sig d\tau \sqrt{-g} \big( g^{\a\beta} \p_{\a}X^{\mu} \p_{\beta} X_{\mu} + A\Lambda^2\big),
\ee 
where $A$ is just some number. 

\br 
    Note, by introducing the cosmological constant we have destroyed our Weyl invariance we so badly needed. It turns out that actually introducing this term will help us restore the Weyl invariance into the \textit{quantum} theory. The details of this are not discussed until later in the course, but we shall just assume it works and plow on nonetheless.\footnote{It is at points like this you might begin to notice that the title of this lecture included inverted commas around "proving".} 
\er 

Now, let's insert the length of the string manually as $L$ (i.e. let's not insist it is $2\pi$ for a bit), then the contribution to the energy by introducing the $A\Lambda^2$ is just $A\Lambda^2 L$. So the idea is that any contribution to the energy we get that is proportional to the length of the string we can remove by adding the negative of the term to the action. 

It is important to note that it is \textit{only} the terms proportional to the length that we can remove.\footnote{Well without having a potentially devastating effect to all the work we've done so far.} So any other terms that arise must be interpreted as something physical (unless some other method should arises that allows us to remove them too).

\subsection{Learning To Count}

So now we return to the sum over $n$. Firstly we note that by inserting $L$ manually our sum becomes\footnote{We've also removed a factor of $2\pi$.}
\bse 
    \sum_{n=1}^{\infty} \frac{n}{L}.
\ese 
Now, as we do when dealing with divergences in QFT, we wish to regulate the sum. We introduce the regulator as a function of the physical momenta. We want it to decay for large values and be unit for small values, 

\begin{center}
    \btik 
        \draw[thick, ->] (0,-0.5) -- (0,3);
        \draw[thick, ->] (-0.5,0) -- (8,0);
        \draw (0,2) .. controls (5,2) and (5,0) .. (7.5,0);
        \node at (-0.5,2) {$f(x)$};
        \node at (7.5,-0.3) {$x$};
    \etik 
\end{center}
We use $x = p/\Lambda$, so that our summation remains unchanged when $p << \Lambda$. Now using $p=n/L$ we get the regulated sum 
\bse 
    \sum_{n=1}^{\infty} \frac{n}{L}f\bigg(\frac{n}{L\Lambda}\bigg).
\ese 

\br 
    Note it is very important that our regulator is a function of $p$ not of $n$ itself. The reason for this is that $p$ is the conjugate to \textit{physical} length, whereas $n$ is conjugate to the length scale $L$. If we are going to cut off our theory using $f$ then it would be crazy to have this cut off directly related to $L$ --- as then the cut off would depend on the \textit{relative} size of the string to the manifold it lives in, so different embeddings would give different results. This is clearly not what we want!
\er 

We now introduce the result from the Euler–Maclaurin formula\footnote{See \href{https://en.wikipedia.org/wiki/Euler–Maclaurin_formula}{wiki} for full definition. I haven't included it here just to save space.} that tells us we can write the sum as
\bse 
    \sum_{n=1}^{\infty} \frac{n}{L}f\bigg(\frac{n}{L\Lambda}\bigg) = \int_0^{\infty} \frac{n}{L}f\bigg(\frac{n}{L\Lambda}\bigg)dn -\frac{1}{12}\bigg[\frac{n}{L}f\bigg(\frac{n}{L\Lambda}\bigg)\bigg]'(0) + \frac{1}{4! 30}  \bigg[\frac{n}{L}f\bigg(\frac{n}{L\Lambda}\bigg)\bigg]'''(0) + ... 
\ese 

Let's consider these terms individually. Using the change of variable $y=n/L\Lambda$ the integral simply becomes 
\bse 
    L\Lambda^2 \int_0^{\infty} y f(y)dy = A L\Lambda^2,
\ese 
where $A$ is the result from the integral (it's just some number). 

Next consider the triple derivative term. We are evaluating at $n=0$ and so we need one of the derivatives to act on the factor $n$ or else the term vanishes. This leaves us with 
\bse 
    \frac{1}{4!30} \frac{1}{L} \bigg(\frac{1}{L\Lambda}\bigg)^2 f''\bigg(\frac{0}{L\Lambda}\bigg) = \frac{1}{4!30} \frac{1}{L} \bigg(\frac{1}{L\Lambda}\bigg)^2,
\ese 
but we are taking $|\Lambda|$ to be huge (we need it to remove an infinite contribution) and so this term is negligible. The same thing is true for the higher order terms in the expansion.

So now we just have the single derivative term. Again we need the derivative to act on the $n$, leaving us with 
\bse 
    -\frac{1}{12} \frac{1}{L} f(0) = -\frac{1}{12L}.
\ese 

So we are left with 
\bse 
    \sum_{n=1}^{\infty} \frac{n}{L}f\bigg(\frac{n}{L\Lambda}\bigg) = AL\Lambda^2 - \frac{1}{12L}.
\ese 
This is good; we can remove the first term by introducing a cosmological constant to our action. We cannot, however, remove the last term. 

\subsection{Finally... $d=26$}

Before rewriting \Cref{eqn:MassNormalOrder} let's first simplify the notation slightly

\bd 
    We define the so-called \textit{levels}
    \be
    \label{eqn:Levels}
        N := \sum_{i=1}^{d-2} \sum_{n=1}^{\infty} \a^i_{-n}\a^i_{n}, \qquad \widetilde{N} := \sum_{i=1}^{d-2} \sum_{n=1}^{\infty} \widetilde{\a}^i_{-n}\widetilde{\a}^i_{n}.
    \ee 
\ed 

\br 
    Note these are not quite the number operators of quantum mechanics, $\hat{n}= a^{\dagger}a$ (there's factors of $n$ in there remember!). They get their name from the fact that they are related to the level matching condition.\footnote{Or vice versa?}
\er 

So the situation is
\be
\label{eqn:MassSquared}
    m^2 = \frac{4}{\a'} \bigg(N - \frac{d-2}{24}\bigg) = \frac{4}{\a'} \bigg(\widetilde{N} - \frac{d-2}{24}\bigg).
\ee
This looks good, however, there is one huge problem that should be weighing on our minds at this point: we broke the Lorentz invariance of our system when we chose the lightcone gauge! This often happens in quantum field theories, and it is always very important at the end of the calculations to check whether we have (by fluke, more then anything else!) managed to retain Lorentz invariance. We shall do this by considering the different states of the system and seeing if they are indeed symmetries of the Lorentz group.

\subsubsection*{The Tachyon}

Let's first consider the ground state, $\ket{0;p}$. Normal ordering tells us that $N=\widetilde{N}=0$. This is a unique state (it's given by \Cref{eqn:VacuumState} and \Cref{eqn:GroundStateP}), and so it is clearly Lorentz invariant (there is nothing it can be rotated into).

However, it looks terrible; we have a negative mass-squared:
\be
\label{eqn:TachyonMass}
    m^2 = - \frac{d-2}{6\a'}.
\ee 
We call these particles \textit{Tachyons}. Tachyons are often claimed to travel faster then the speed of light, however this interpretation of them is nonsense. The correct way to view them is in the context of quantum field theory. 

Let's suppose we have some field propagating in spacetime whose quanta are the Tachyons. We know that the mass-squared of such a particle is related to the quadratic term in the action. In other words, if $T(x)$ is the Tachyon field, and $V(T)$ is the potential we have
\bse 
    m^2 = \frac{\p^2 V(T)}{\p T^2}\bigg|_{T=0}.
\ese 
So the negative mass is telling us that we have an maximum in the potential at $T=0$. This obviously reeks havoc for perturbation theory (as you really don't want to be perturbing a field about a maximum). The obvious question to ask is `Is the a minimum somewhere that the system will fall into?' The answer is, unfortunately, nobody knows!
\begin{center}
    \btik
        \draw[thick, ->] (0,-0.5) -- (0,3);
        \draw[thick, ->] (-0.5,0) -- (5,0);
        \draw[thick, blue] (-0.5,2) .. controls (0,2.5) .. (0.5,2);
        \draw[thick, blue, dashed] (0.5,2) .. controls (2,0.5) .. (4,2);
        \node at (-0.5,2.8) {$V(T)$};
        \node at (4.5,-0.3) {$T$};
        \node at (2.25,0.65) {Minimum?};
    \etik 
\end{center}

\subsubsection*{Wigner's Classification}

The next thing to consider is the occupied states. We see straight away that these states are not unique; say we wish to produce a system in the $n$-th excited state, we could apply one creation operator $n$ times or we could apply $n$ different creation operators once each. Of course there is a whole array of different ways in-between. Note also that the level matching condition says that if we create a $n$-th excited state with the $\a$s, we must also do so with the $\widetilde{\a}$s, giving us a multiplicative factor. 

As the states are not unique, we will need to check that they preserve the relevant rotational invariance, and this is where Wigner's classification of the representations of the Poincar\'{e} group comes into play. We will not give a hugely detailed explanation here,\footnote{A large factor being that I am relatively new to this, and so do not wish to butcher the definitions etc. If I have made some mistake in my interpretation that you notice, please feel free to contact me with your corrections.} but shall just summarise the main points that we need. 

Wigner's classification classifies the non-negative energy (and therefore mass), irreducible, unitary representations of the Poincar\'{e} group. It classifies them into two groups: massive representations and massless representations. It does so by introducing a stabiliser for the momentum.

Consider a \textit{massive} particle. If we go to the particles rest frame, $p^{\mu}=(p,0,...,0)$, we can ask how the internal indices transform under the stabiliser subgroup (or \textit{little group}) $SO(d-1)$ of spatial rotations. The relevant result to us is that massive particles must form a representation of $SO(d-1)$ if we are to preserve Lorentz invariance. That is, we are requiring that all the states of the form $\ket{n;p}$, where $p$ is always chosen to be the timelike momentum as above, to be physically the same. In other words, we have chosen a frame that fixes one component of the particles momentum. We now consider a different state of the same particle that varies only by internal indices --- that is we only allow for the generation of $n$ by the different combinations of the application of the $\a$/$\widetilde{\a}$s. 

Now consider the \textit{massless} particle. We can no longer go to the particle's rest frame, but the best we can do is to go to the frame $p^{\mu} = (p,0,...,p)$. We then proceed as above; we stabilise this momentum and require that the physically equivalent states can be obtained via the suitable stabiliser subgroup. Clearly in this case that is the spatial rotations of $SO(d-2)$.

The above classification basically says that massless particles get away with having one fewer internal degrees of freedom. For example, the photon in 4-dimensional spacetime has 2 polarisations whereas a massive spin-1 particle has 3. 

\subsubsection*{First Excited State}

So let's consider the first excited state,
\be 
\label{eqn:FirstExcitedState}
    \widetilde{a}^i_{-1}\a^j_{-1}\ket{0;p}.
\ee 
There are $(d-2)^2$ ways we can produce this state; each of the $\a^i_{-1}$ times each of the $\widetilde{\a}^i_{-1}$. Each of these terms transform in the vector representation of $SO(d-2)$. There is no way we are going to package these $(d-2)^2$ states into a representation of $SO(d-1)$, and so it follows that they cannot be massive particles. Luckily we can package these states into a representation of $SO(d-2)$, and so we therefore conclude that they \textit{must} be \textit{massless}.

So what does this tell us? Well \Cref{eqn:MassSquared} gives us ($N=1$ for the first excited state)
\bse 
    0 = 1 - \frac{d-2}{24},
\ese 
or, equivalently, 
\mybox{
\be 
\label{eqn:d26}
    d = 26.
\ee 
}

So our states transform in the $24\otimes 24$ representation of $SO(24)$.

\br 
    We can decompose the above representation into three irreducible representations, 
    \bse 
        \text{traceless symmetric} \oplus \text{antisymmetric} \oplus \text{singlet}.
    \ese
    We associate to each of these a massless field, whose quanta give the string oscillations. We denote them $G_{\mu\nu}(X)$, $B_{\mu\nu}(X)$ and $\Phi(X)$, respectively. The latter two are known as the \textit{Kalb-Ramond} field and the \textit{dilaton} field, respectively. It is actually the $G_{\mu\nu}$ term that is of most interest though. It corresponds to a massless spin-2 particle. It can be argued that \textit{any} theory of interacting massless spin-2 particles is equivalent to general relativity, and so we need to identify $G_{\mu\nu}$ as the spacetime metric. We do not present the arguments here, but a brief explanation is given on pages 43-45 of Dr. Tong's notes. 
\er 

\subsubsection*{Higher Excited States}

The above result is nice, however it could all be for nothing if we don't get massive particles with a $SO(d-1)$ representation for every other excited state. In other words, we have set $d=26$, so for $N=\widetilde{N}>1$ we get $m^2 >0$ and so the internal indices must form a $SO(d-1)$ representation or else our theory is not Lorentz invariant. 

Consider $N=\widetilde{N}=2$. For each of the left/right moving sectors we have two possible types state preparation:
\bse 
    \a^i_{-1}\a^j_{-1}\ket{0;p}, \qquad \text{and} \qquad \a^i_{-2}\ket{0;p},
\ese 
and similarly for $\widetilde{\a}$. So all together the set of states for this level is\footnote{I'm using Dr. Tong's notation here to indicate what we mean.}
\bse 
    \big(\a^i_{-1}\a^j_{-1}\oplus \a^i_{-2}\big)\otimes \big(\widetilde{\a}^i_{-1}\widetilde{\a}^j_{-1}\oplus \widetilde{\a}^i_{-2}\big)\ket{0;p}.
\ese 
The number of states in each sector is given by 
\bse 
    \frac{1}{2} (d-2)(d-1) + (d-2) = \frac{1}{2}d(d-1) -1,
\ese 
which is easily seen by thinking of a matrix whose elements are given by the different combinations of the $\a/\widetilde{\a}$s --- the first term is the number of independent components in a symmetric\footnote{If it is unclear why we say symmetric, its because the $\a$s commute and so $\a^i_{-1}\a^j_{-1} = \a^j_{-1}\a^i_{-1}$.} $(d-2)\times (d-2)$ matrix and then second is the contribution from the $\a^i_{-2}$ terms. But the right hand side is a nice representation of $SO(d-1)$, i.e. the traceless (the $-1$ term), symmetric (the first term) representation.

So we appear to be safe for $N=\widetilde{N}=2$. In fact, it turns out to be true that for all $N=\widetilde{N}>1$ that we get a representation of $SO(d-1)$, and so we have a Lorentz invariant theory, provided we set $d=26$. 

\section{Bosonic \& Fermionic String}

We have invested a considerable amount of time/effort into deriving the above results for the so-called Bosonic string. It all seems rather nice, apart from this negative mass-squared behaviour we get from the Tachyon solutions. We could rightfully ask `why have we spent all this time deriving this result if its possible that this Tachyon problem could just ruin it all?' Well firstly, its instructive to derive the results themselves, and besides that, there is a close relative to the Bosonic string known as the \textit{Fermionic} string. 

The Fermionic string theory shares all the nice results from the Bosonic string, but luckily do not share all the nasty things, namely there is no negative mass-squared terms appearing! So the next few lectures are going to be devoted to further studying the Bosonic string, in an attempt to also help with the understanding of the Fermionic string. The structure of this course will be to try and study the two strings side by side, making comparisons as we go. 

When we say we aim to better understand the Bosonic string, what we mean is to develop the theory in a much nicer way. In the previous treatment we imposed some of shady looking conditions and then later to fix the problems that arouse further restricted our results. Well what happens if even more problem crop up when we further develop the theory? For example what will happen if we try and incorporate interactions between strings? With all this in mind, we shall proceed to tackle the problem via a path integral approach. In doing this, we will not need to use the light-cone gauge (which is where we broke Lorentz invariance) and so hopefully we are less likely to run into problems further down the line. We will, however, continue to work in the conformal gauge (i.e. we will still work with a worldsheet metric that is locally flat). 

Although we now seek a separate approach to the problem, it is worth highlighting again that the above approach was not worthless, and was worth studying. Firstly it forms a relatively intuitive description of what a string is and where the idea comes from. Besides that it has given us some fantastic results that will be highly beneficial with our proceedings. For example, we have seen that the system is both Weyl and diffeomorphic invariant, and so we know that the system is conformally redundant.\footnote{By that I mean that we have a gauge given by those diffeomorphisms that can be `undone' by a Weyl transformation, giving a redundancy in the phase space.} Besides this, the result that $d=26$ is true, and it is nice to see this result early on.  
\chapter{State-Operator Map \& Ward Identities}

So far we have taken a some what shaky approach to studying first quantised string theory. This allowed us to get some nice insights to the theory, however, as we noted at the end of the last lecture, our goal now is to go back and start again. We try to develop the theory in a nicer way, which will prove beneficial when we start considering things such as \textit{interacting} strings. 

The approach we are going to take is basically conformal field theory, and so it is obviously important that we know what this is. For this reason why shall spend the next few lectures discussing CFT for a cylinder (as we obviously want to use it on the string after). We are going to study theories that are conformally invariant and then ask the question about when they are also Weyl invariant, in order to be able to touch base with the Polyakov action. 

\section{Euclidean Space}

In order to simplify the following calculations, we shall work in Euclidean space, achieved by taking a Wick rotation on our coordinates. This short section is just here to highlight the conventions we are going to use. 

We define $(\sig^1,\sig^2) = (\sig^1,i\sig^0)$ as the Euclidean coordinates on the worldsheet. In other words we consider imaginary time, also known as Euclidean time. We then introduce the complex coordinates 
\be
\label{eqn:ComplexCoordinates}
    z = \sig^1 + i\sig^2, \qquad \Bar{z} = \sig^1 - i\Bar{\sig}^2.
\ee 
This is just the light cone coordinates on the worldsheet. We can use this as a justification to refer to the holomorphic functions\footnote{A complex-valued function that is complex differentiable in a neighbourhood around every point on the function.} (i.e. $z$) as left-moving and the anti-holomorphic\footnote{Same as holomorphic but now the derivative is taken wrt the complex conjugate.} (i.e. $\Bar{z}$) as right-moving. The line element in these coordinates is simply 
\be 
\label{eqn:LineElementComplex}
    ds^2 = dz d\Bar{z},
\ee 
which is equivalent to
\be 
\label{eqn:MetricComponentsComplex}
    g_{zz} = 0 = g_{\Bar{z}\bar{z}}, \qquad g_{z\bar{z}} = \frac{1}{2}.
\ee 
It follow, then, that 
\be
\label{eqn:NablaSquaredComplex}
    \nabla^2 = g^{z\bar{z}} \p_{z}\p_{\bar{z}} + g^{\bar{z}z}\p_{\bar{z}}\p_z = 4\p_z\p_{\bar{z}},
\ee 
and our volume element is 
\be 
\label{eqn:VolumeElementComplex}
    dzd\bar{z} = 2d\sig^1d\sig^2.
\ee 
Finally our holomorphic and antiholomorphic derivatives are
\be 
\label{eqn:HolomorphicDerivatives}
    \p := \p_z = \frac{1}{2}\big(\p_1 -i\p_2\big), \qquad \text{and} \qquad \bar{\p} := \p_{\bar{z}} = \frac{1}{2}\big(\p_1 +i\p_2\big).
\ee 

\section{Path Integrals}

For completeness, and to help with the following discussion, we shall briefly discuss path integrals. We will not give a super rigorous discussion here, but just outline the ideas. 

Path integrals give us transition amplitudes. For now let's just consider a $(1+1)$-dimensional theory, in which the state of a particle is described by a single spatial coordinate and a temporal coordinate. The question we want to ask is `What is the probability of the particle going from position $x_i$ at time $t_i$ to $x_f$ at $t_f$?' The idea is to split this path into lots of discrete time slices and then ask about the probability to go between these slices subject to the constraint that it we must start at $x_i$ and end at $x_f$. We consider all of the different \textit{paths} between the two points and sum the probabilities. We then take the limit that the time separation goes to zero, giving us an integral. 

Mathematically, we break up the transition amplitude into $N$ sections by inserting the identity projectors 
\be
\label{eqn:BasesStates}
    \b1 = \int dx_i \ket{x_i}\bra{x_i},
\ee 
where $x_i$ is the position on the $i$-th slice. We then take the limit $N\to\infty$. Then, recalling that the time evolution of the system is given by the Hamiltonian, we can show that\footnote{If you aren't familiar with how to get to this result, I highly encourage you to go away and read it up before continuing.} 
\bse 
    \bra{x_f} e^{-iH}\ket{x_i} = \int_{x(t_i)=x_i}^{x(t_f)=x_f} Dx\, e^{iS[x]},
\ese 
where $S[x]$ is the action of the system. 

This idea is easily extended to higher dimensional field theories, but now instead of just specifying a single position $x_i/x_f$ we have to specify a field configuration at the times, $\phi_i/\phi_f$. The path integral simply becomes 
\bse 
    \bra{\phi_f} e^{-iH}\ket{\phi_i} = \int_{\phi(t_i)=\phi_i}^{\phi(t_f)=\phi_f} D\phi \, e^{iS[\phi]}.
\ese 

The next question to ask is `How do we insert a local operator, $\cO_1$, at some position $x_1$ at time $t_1$ into the path integral, taking into account the effect it has on the result?' The answer comes from following the same method as above but now we the operator inserted where it needs to go. This gives us the following \textit{correlation function} (in the Heisenberg picture) 
\bse 
    \bra{\phi_f,t_f} \cO_1 \ket{\phi_i,t_i} = \int_{\phi(t_i)=\phi_i}^{\phi(t_f)=\phi_f} D\phi \, e^{iS[\phi]} \cO_1(x_1).
\ese 

We can then insert more operators and repeat the same process as above. However it is important to keep the operators \textit{time ordered}.\footnote{That is put the operators that act at the latest times to the left.}

Ok, so that's the quickest crash course on path integrals ever given, so back to our scheduled programme.

\section{States}

As the title of this lecture indicates, its important that we understand what states and operators are in quantum field theory. 

Let's start with states. We saw above that the transition amplitudes are given by path integrals. We can use the path integrals to define \textit{states} of our system. 

First, consider again the $(1+1)$-dimensional problem. If the system starts off in some state $\ket{\psi_i}$, with corresponding wavefunction\footnote{Note we use wavefunctions because we inserted a basis of states above \Cref{eqn:BasesStates}, and so we are working with wavefunctions.} $\psi_i(x_i,t_i)$, then the path integral tells us that the system evolves to the state $\ket{\psi_f}$ with wavefunction\footnote{We've ignored the normalisation conditions here, but they're just numbers.}
\bse 
    \psi_f(x_f;t_f) = \int dx_i \int_{x(t_i)=x_i}^{x(t_f)=x_f} Dx \, e^{iS[x]} \psi_i(x_i;t_i).
\ese 
The extension to the field path integrals is clear 
\be 
\label{eqn:StatePathIntegral}
    \psi_f[\phi_f(\sig);t_f] = \int D\phi_i \int_{\phi(t_i)=\phi_i}^{\phi(t_f)=\phi_f} D\phi \, e^{iS[\phi]} \psi_i[\phi_i(\sig);t_i].
\ee 

\br 
Note we now have wavefunction\textit{als}, as we are considering whole field configurations, not just points in space.
\er 

\br 
Note that the initial state acts as a weighting to the initial boundary conditions. We shall use this insight shortly.
\er 

So what we have is a prescription for finding the state of the system as a functional of the final field configuration. Another way to understand what we're on about here is to view the state of the system as a \textit{cut} of the path integral at a constant time. That is, we write the state
\bse 
    \ket{\psi} = e^{-iH}\ket{\phi_i} = \int_{\phi(t_i)=\phi_i} D\phi \, e^{iS[\phi]},
\ese 
where we don't specify the upper integration limit. From here we see that the path integral for the amplitude is just given by taking two cuts and considering the bounded part of the path integral. The wavefunctional expression is just obtained by taking $\braket{\phi_2}{\psi}$ and then inserting \Cref{eqn:BasesStates}. 

\br 
By extension of the above argument, we can form the whole Hilbert space of our system by taking constant time cuts of all possible path integrals.
\er 

\section{Radial Quantisation}

So far we have a theory on a cylinder. We are now going to attempt to turn this into a theory on a plane. This is done via simple transformations. First let $\omega$ be the complex coordinate on the cylinder, i.e. 
\bse 
    \omega = \sig +i\tau,
\ese 
and then define 
\bse 
    z = e^{-i\omega} = e^{-i\sig+ \tau},
\ese 
which is a set of concentric circles on the plane, where each circle corresponds to an equal time slice on the cylinder. 
\begin{center}
    \btik
        \draw[thick] (0,0) -- (0,5);
        \draw[thick] (2,0) -- (2,5);
        \draw[thick] (1,5) ellipse (1 and 0.18);
        \draw[thick] (0,0) arc (180:360: 1 and 0.18);
        \draw[thick, dashed] (0,0) arc (180:360: 1 and -0.18); 
        \draw[thick] (6,0) -- (11,0) -- (11,5) -- (6,5) -- (6,0);
        \draw (8.5,0) -- (8.5,5);
        \draw (6,2.5) -- (11,2.5);
        \draw[dashed] (8.5,2.5) circle (0.5cm);
        \draw[dashed] (8.5,2.5) circle (1cm);
        \draw[dashed] (8.5,2.5) circle (1.5cm);
        \draw[dashed] (8.5,2.5) circle (2cm);
        \draw[dashed] (8.5,2.5) circle (2.5cm);
        \draw[ultra thick, ->] (3,2.5) -- (5,2.5);
    \etik 
\end{center}

Initially there might seem to be a problem at the origin. To see why consider $\omega = i\ln z$, giving
\bse 
    d\omega d\bar{\omega} = \frac{dzd\bar{z}}{z\bar{z}}, 
\ese 
which diverges for $z=0$. However we are considering a theory that is Weyl invariant, and so we can simply transform away the denominator, saving us from potential problems. 

On the cylider, evolution in time was given by the Hamiltonian 
\bse 
    H = \p_{\tau},
\ese 
which becomes the \textit{dilatation} operator on the plane 
\be 
\label{eqn:DilatationOperator}
    D = z\p + \bar{z}\bar{\p}.
\ee 

This tells us that time increases with the radius of the circle, and the origin corresponds to the infinite past (i.e. the bottom of the cylinder). This process is known as \textit{radial quantisation}. 

\br 
There is another way to think of the above map from the cylinder to the plane. We place the cylinder on top of the plane, and we draw a line connecting the top of the cylinder to the plane. The point on the cylinder that the line passes through corresponds to the point on the plane the line touches. From here it is easy to see that the bottom of the cylinder is the origin on the plane and that higher times (i.e. further up the cylinder) corresponds to larger circles. This is the idea of a conformal mapping. 
\er 

\section{The State-Operator Map}

Now that we have a way to turn our theory on the cylinder to a theory on the plane we are in a position to derive a surprising result in our\footnote{The result actually holds for any CFT that allows us to map the cylinder to the plane. For general dimension, it works when its possible to map $\R\times S^{d-1}$ to $\R^d$.} CFT. 

\mybox{
\bt[State-Operator Map]
The state-operator map gives us a one-to-one correspondence between states of the theory and \textbf{local} operators.
\et 
}

Before we prove this, let's first note how surprising this result is. First note that we are not saying there is a one-to-one correspondence between states and operators, but only to \textit{local} operators. However, even with this restriction, the result is still unexpected: as we explained before, states live on entire spatial slices, whereas local operators are only defined at a single point; equally we can think of a state as a $n$-column matrix (its a vector in the Hilbert space) but an operator is a $n\times n$ matrix! This all seems rather crazy, but for a hopefully we will de-crazy this now. 

\subsection{State From Operator}

We have seen that a state of the system corresponds to a constant time cut of the path integral. We have also seen that a constant time slice on the plane is a circle around the origin. So, to find a state on the plane, what we need to do is the path integral over the area inside the circle; \Cref{eqn:StatePathIntegral} becomes\footnote{We are using Euclidean time here, so $iS[\phi] \to -S[\phi]$.} 
\bse 
    \psi_f[\phi_f(\sig); r_f] = \int D\phi_i \int_{\phi(r_i)=\phi_i}^{\phi(r_f)=\phi_f} D\phi \, e^{-S[\phi]}\psi_i[\phi_i(\sig);r_i].
\ese 
Pictorially we are looking at 
\begin{center}
    \btik 
        \draw[thick] (0,0) -- (5,0) -- (5,5) -- (0,5) -- (0,0);
        \draw[dashed,pattern=north west lines, pattern color=black] (2.5,2.5) circle (2cm);
        \draw[dashed, fill=white] (2.5,2.5) circle (1cm);
        \draw[] (2.5,0) -- (2.5,5);
        \draw[] (0,2.5) -- (5,2.5);
        \node at (3,3) {$r_i$};
        \node at (4.1,4.1) {$r_f$};
    \etik 
\end{center}

Now we simply take $r_i\to 0$, which corresponds to $\tau \to -\infty$. We are now integrating over the whole disc area up to $r_f$, and the only effect from the initial state is a change of the weighting of path integral at $z=0$; this is just the definition of a local operator! So for each local operator at $z=0$ we obtain a state of the system via 
\bse 
    \psi[\phi_f;r] = \int^{\phi(r)=\phi_f} D\phi \, e^{-S[\phi]} \cO(z=0).
\ese 

This is actually a rather trivial result: all we are doing is taking a state (which is the vacuum state at the origin) and acting on it with some local operator and saying that this gives us a new state. 

\br 
Note if we want the vacuum operator at $r$ then we use the identity operator at the origin.
\er 

\subsection{Operator From State}

We now want to do something much less trivial: given a state we want to be able to uniquely define a local operator. We want this prescription to work so that if we placed the local operator produced inside a path integral we would get back the state we started with. The procedure is as follows.

First we consider a state at radius $r$. W.l.o.g. we shall take $r=1$. Now imagine we have a bunch of known operator insertions \textit{exterior} to the disc. We can take the path integral over the exterior of the disc and use the initial boundary conditions to give us our state at the boundary (i.e. weight the path integral by $\psi_i[\phi_i;1]$). This just gives us a number.\footnote{Note if we did not use the initial state to weigh the integral, we would get an infinite number of numbers, one for each possible initial field configuration.}

Now specifying a local operator amounts to a local modification of the path integral that allows you to associate to the path integral, and any number of other insertions, a number. Put another way, given that we already have a set of operators $\{\cO_1,...,\cO_N\}$, if we want to introduce a new operator, $\cO_{new}$, we need to know the set of rules for arbitrary number of insertions of $\cO_{new}$, and all other $\cO_i$s, inside the path integral. 

If we insert our $\cO_{new}$ at the origin, we're half way there given the previous subsection. However we have the problem that we are not taking just a local deformation to the path integral, but we are deforming it over the whole interior of the disc. 

\begin{center}
    \btik 
        \draw[thick, pattern=north west lines, pattern color=black] (0,0) -- (5,0) -- (5,5) -- (0,5) -- (0,0);
        \draw[] (2.5,0) -- (2.5,5);
        \draw[] (0,2.5) -- (5,2.5);
        \draw[dashed, fill=white] (2.5,2.5) circle (1cm);
        \node at (4,3) {$\cO_1$};
        \node at (2,4) {$\cO_2$};
        \node at (1,1) {$\cO_3$};
        \node at (2.5,2.5) {$\cO_{new}$};
        \node at (4,1.8) {$\psi_i[\phi_i;1]$};
    \etik 
\end{center}

The question is `can we somehow make this a local deformation?' The answer is, yes. We simply consider the circle at radius $r/2=1/2$ and we define the state at this new circle to be the one that gives the same result to the path integral, when it is now done from the smaller circle. In other words it needs to be the state that once path integrated over the annulus separating the two states gives the state we started with at $r=1$. This is just equivalent to translating the state at $r=1$ (backwards) in time to the state at radius $1/2$, which is achieved using \Cref{eqn:DilatationOperator}. So all we do is start from the inner circle and weight the path integral by both the initial state on the circle of $r=1$ \textit{and} the time dilatation operator between the two circles. But if we can do this, we can do it again and keep going until we reach the origin itself, giving us a local deformation in the path integral. 

\br 
Note in the above we have assumed that all of the operators are exterior to our initial circle. If this was not the case, we simply take a smaller starting circle such that it is. We can always do this, apart from when one of the operators also lies at the origin. If this turned out to be the case, you can simply move your new operator to a new point where there is no other operators and define that point to be the origin. 
\er 

\br 
The CFTness of the problem came in when we took the infinite dilatation operator to get the local operator. In our theory this results in a single point, the origin. However, in a more general theory going back to $t=-\infty$ wont necessarily be a point (for example for Minkowski spacetime, you get a whole 3-dimensional spatial slice) and so we can not say we have a local operator. 
\er 

\br 
We should note that the correspondance we have described here is (most likely) something you have never encountered before. You might\footnote{I didn't it's only by reading Dr. Tong's notes that it crossed my mind...} think that this is just like the idea of building a Fock space using creation operators on the vacuum. However, the creation operators are Fourier transforms of local operators, and so are about as far from local as you can get!
\er 

\subsection{Operator Product Expansion}

There is a really nice consequence of the state-operator map: given two\footnote{It need not be two, but we shall use two for arguments sake.} operators near the origin that we can circle without circling other operators, we can rewrite these as a sum over local operators at the origin. 

\begin{center}
    \btik 
        \draw[thick] (0,0) -- (5,0) -- (5,5) -- (0,5) -- (0,0);
        \draw[dashed] (2.5,2.5) circle (1.5cm);
        \node at (3,3) {$\cO_1$};
        \node at (2,1.5) {$\cO_2$};
        \node at (1,4) {$\cO_3$};
        \node at (4.5,2) {$\cO_4$};
        \node at (3,0.5) {$\cO_5$};
        \draw[ultra thick, ->] (5.5,2.5) -- (6.5,2.5);
        \draw[thick] (7,0) -- (12,0) -- (12,5) -- (7,5) -- (7,0);
        \draw[dashed] (9.5,2.5) circle (1.5cm);
        \node at (9.5,2.5) {$\sum_m\cO_m^{basis}$};
        \node at (8,4) {$\cO_3$};
        \node at (11.5,2) {$\cO_4$};
        \node at (10,0.5) {$\cO_5$};
    \etik 
\end{center}

This is not too hard to see. We know that the two inside operators are going to produce some state on the circle. We then know that this state will correspond to some local operator at the origin. However, we also know that states in a Hilbert space can always be expanded in a basis of other states, and so it follows that the local operator can be expanded in terms of some local basis operators. The coefficients of the basis elements will of course depend on the initial positions of the operators $\cO_1/\cO_2$ (i.e. they have $\sig$ dependence). This process is known as the \textit{operator product expansion} (OPE).

\br 
Its important to note that the above result is true \textit{locally}. That is, it does not depend in any way on our knowledge of the external operators $\cO_{3,4,5}$. It is therefore a statement about the operators \textit{themselves} and not about particular correlation functions.
\er 

\br 
Although in the above we have chosen to give the result at the origin, we need not do this. We can right choose to do it about the positions of one of the operators itself if we wish. 
\er 

Bearing in mind the previous remarks, we do not need to take the OPE near the origin, and we can write it generally as 

\be 
\label{eqn:OPE}
    \langle\cO_i(z,\bar{z}) \cO_j(\omega,\bar{\omega})...\rangle = \sum_{k} C^k_{ij}(z-\omega,\bar{z}-\bar{\omega}) \langle\cO_k(\omega,\bar{\omega})...\rangle,
\ee 
where we see the constants only depend on the separations. 

\br 
Note on the right-hand side of \Cref{eqn:OPE} we use $(\omega,\bar{\omega})$ and not $(z,\bar{z})$. The reason for this is that we are using a correlation function, and the operators are always time ordered, and so we know from the right-hand side that $|z|>|\omega|$ and so we expand around the inner operator.
\er 

\br 
\label{rem:OPESingular}
We will see that the OPEs have singular behaviour as the two operators approach each other, i.e. as $z\to\omega$. This singular behaviour will prove to be very interesting when considering the commutation of operators (see section \Cref{sec:CommutatorOfCharges}).
\er 

\section{Noether Currents and Ward Identities}

Recall that Noether's theorem tells us that for every classical symmetry of the system we have a conserved charge. In field theory we see this charge gives rise to a conserved current. We now want to cast this result in terms of path integrals. 

Assume we have the path integral
\bse 
    \int D\phi \, e^{-S[\phi]}.
\ese 
Now assume our system has some symmetry corresponding to 
\bse 
    \phi \to \phi' = \phi + \epsilon \del \phi,
\ese 
for some infinitesimal $\epsilon$. The fact that it is a symmetry means that both the action and the measure are left invariant\footnote{Technically we only need there combined product to be invariant, but this detail isn't too important here.}
\bse 
    S[\phi] = S[\phi'], \qquad \text{and} \qquad D\phi = D\phi'.
\ese 
We now do a clever trick\footnote{More details on this trick can be found in Dr. Tong's notes at the start of Chapter 4.} and let $\epsilon$ become a function of the coordinates, $\epsilon(\sig)$. Now, of course in general this will leave neither the action nor the measure invariant, but will result in a modification of the path integral. We do know, however, that this modification must vanish whenever $\epsilon$ is constant. We, conclude, therefore that the path integral is of the form 
\bse 
    \int D\phi' \, \exp \bigg( -S[\phi'] + \int J^{\mu} \p_{\mu}\epsilon \bigg) = \int D\phi' \, e^{-S[\phi']} \bigg( 1 + \int J^{\mu}\p_{\mu}\epsilon\bigg),
\ese 
for some $J^{\mu}$, where we have used the fact that we are considering a small change to go to the right-hand side. Now, the value of the path integral (which is the partition function) can't have changed, as all we've done is a change of variables. This must hold for all $\epsilon(\sig)$, and so (after integration by parts to shift the derivative off $\epsilon$) we conlcude 
\be 
\label{eqn:NoetherQM}
    \langle \p_{\mu}J^{\mu} \rangle = 0,
\ee 
where the $\langle (...)\rangle$ notation means $\int D\phi e^{-S[\phi]}(...)$. This is the quantum mechanical conservation of the charge. 

We can do better though. Let's now insert some operators into the system. The operators will transform as 
\be 
\label{eqn:OperatorTransform}
    \cO_i \to \cO_i + \epsilon(\sig) \del \cO_i
\ee 
First let's assume that $\epsilon(\sig)$ only has support in regions on the plane that contain no operator insertions:
\begin{center}
    \btik 
        \draw[thick] (0,0) -- (5,0) -- (5,5) -- (0,5) -- (0,0);
        \draw[fill=lightergray, dashed] (1,2.5) circle (0.7cm);
        \node at (1,2.5) {$\epsilon(\sig)$};
        \node at (3,3) {$\cO_1$};
        \node at (2,1.5) {$\cO_2$};
        \node at (1,4) {$\cO_3$};
        \node at (4.5,2) {$\cO_4$};
        \node at (3,0.5) {$\cO_5$};
    \etik 
\end{center}
and so the operators are invariant under this transformation. Then we know that
\bse 
    \langle \p_{\mu} J^{\mu} \cO_1...\cO_5 \rangle = 0,
\ese 
but because this holds for arbitrary insertions, we can think of it as being equivalent to \Cref{eqn:NoetherQM}.

Now let's assume that $\epsilon(\sig)$ is non-zero in some region that contains an operator insertion. This this operator will transform as \Cref{eqn:OperatorTransform}. Then, following as above, we get 
\be 
\label{eqn:WardIdentity}
    \int_{\epsilon} \p_{\mu}\big\langle J^{\mu} \cO_1(\sig_1)...\cO_n(\sig_n)\big\rangle = \sum_{m=1}^n \big\langle \cO_1(\sig_1)...\del\cO_m(\sig_m)...\cO_n(\sig_n)\big\rangle,
\ee 
where we have taken the derivative out of the correlation function. This is known as the \textit{Ward Identity}.

However, there appears to be a problem here. The current $J^{\mu}$ is meant to be conserved, so why when we happen to insert other operators does it suddenly appear to not be conserved? (i.e. why is the right-hand side not zero?) The answer to this question comes from recalling that path integrals \textit{time order} operator insertions, and so in our $\langle (...)\rangle$ we really have a bunch of terms with Heaviside functions inserted to give us the time ordering. When we then take the derivative back inside the correlation function, it will also act on these Heaviside functions with the time derivative. We then end up with a bunch of commutation relations between $J^0$ and the operators, and this gives us precisely the contact terms on the right-hand side. In other words, in order to take the derivative out of the correlation function, we have to compensate by taking away the terms on the right-hand side. 

\subsection{Ward Identities in 2D}

So far we haven't used the fact that we are considering a two dimensional theory. If we now talk about the $(\sig^1,\sig^2)$ plane as defined previously, we have, using Stokes' Theorem 
\bse 
    \int_{\epsilon} \p_{\a} J^{\a} = \oint_{\p\epsilon} J_{\a} \Hat{n}^{\a} = \oint_{\p\epsilon} J_1d\sig^2 - J_2d\sig^1, 
\ese 
where $\Hat{n}$ is the normal vector to the constant time circles. We can rewrite this in terms of the complex coordinates $z/\bar{z}$ as
\bse 
    \int_{\epsilon} \p_{\a} J^{\a} = \oint_{\p\epsilon} J_zdz - J_{\bar{z}} d\bar{z}.
\ese 
Putting this into the Ward identity result \Cref{eqn:WardIdentity} we get 
\be 
\label{eqn:WardIdentity2D}
    \frac{i}{2\pi}\oint_{\p\epsilon} dz \big\langle J_z(z,\bar{z}) \cO_1(\sig_1) ... \big\rangle  - \frac{i}{2\pi}\oint_{\p\epsilon} d\bar{z} \big\langle J_{\bar{z}}(z,\bar{z}) \cO_1(\sig_1) ... \big\rangle = \sum_{m=1}^n \big\langle \cO_1(\sig_1)...\del\cO_m(\sig_m)...\cO_n(\sig_n)\big\rangle,
\ee 
where we have included a normalisation constant for a reason that will become clear in the next lecture.

Note, the conservation of $J$ also tells us that 
\be 
\label{eqn:JzDerivatives}
    \p J_{\bar{z}} + \bar{\p} J_z = 0.
\ee 
This it obviously satisfied when the two terms cancel. However, the more interesting case is when the two terms vanish in themselves. In that case $J_z$ is a holomorphic function and $J_{\bar{z}}$ an antiholomorphic function. In that case, the conserved charge splits into two conserved charges, one which is holomorphic and one which is antiholomorphic. This is a very powerful result, as then \Cref{eqn:WardIdentity2D} just picks up the residue between $J_z/J_{\bar{z}}$ and the operators. For example if it is only $\cO_1$ that is within the support of $\epsilon(\sig)$, then we would have for the holomorphic current
\be 
\label{eqn:WardResidue}
    \big\langle \del\cO_1(\sig_1)...\big\rangle = \frac{i}{2\pi} \oint_{\p\epsilon} dz \big\langle J_z(z)\cO_1(\sig_1) ...\big\rangle = \text{Res}[J_z\cO_1].
\ee 
Here we mean the residue in relation to the OPE of the two operators, i.e. what we referred to in \Cref{rem:OPESingular}. If this all seems rather confusing, we shall return to this exact point next lecture (see \Cref{rem:ConformalResidue}) and show what we mean, so just bear with it. 
\chapter{Commuting Charges}

\section{The Stress-Energy Tensor}

We are now going to apply Noether's Theorem to the particular symmetry associated to translations. So what we want to consider is the transformation
\bse 
    \sig^{\a} \to \sig^{\a} + \epsilon^{\a}.
\ese
Note, that we do not require $\a=1,2$ here, as this part of the argument holds for general theories in any dimension. We shall specialise to 2D CFT soon and use $\a=1,2$. Our fields then transform as\footnote{The minus sign arises because we are considering an \textit{active} transformation. For more info on this see Dr. Tong's QFT notes.} 
\bse 
    \del \phi = - \epsilon^{\a}\p_{\a} \phi
\ese

There is a particularly nice trick in order to solve this problem. Let's suppose that we are considering a theory that we can couple to some dynamical background metric $g_{\a\beta}(\sig)$, which we can set to be the flat metric at the end if we wish. Then, we can ensure our theory remains invariant under the translation if we compensate for the above transformations by transforming the metric in the required way
\bse 
    \del g_{\a\beta} = \nabla_{\a}\epsilon_{\beta} + \nabla_{\beta}\epsilon_{\a}.
\ese 
So, if we only make the initial transformation in our theory, then we know that the change in the action must be given by the opposite of the change of the metric
\bse 
    \del S = \int d^d\sig \frac{\p S}{\p g_{\a\beta}} \del g_{\a\beta}.
\ese 
We then define the \textit{stress-energy} tensor to be 
\mybox{
\be 
\label{eqn:StressEnergyTensor}
    T^{\a\beta} := -\frac{4\pi}{\sqrt{g}} \frac{\p S}{\p g_{\a\beta}},
\ee 
}
\noindent where we have added a normalisation constant. So the change in the action is simply
\bse 
    \del S = -\frac{1}{4\pi}\int d^d\sig \sqrt{g} T^{\a\beta} \del g_{\a\beta}.
\ese 
We also know from QFT that the Noether current for translations is given by the dot of the stress tensor with the vector in the translation direction. Our aim here is to show we get the same result using the path integral formulation from last lecture. 

\subsection{Noether Charge From Path Integral}

Substituting the change in the metric above into the last equation, and using the fact that we can make the stress-energy tensor symmetric\footnote{For a nice explanation of this see Dr. Tong's QFT notes at the end of section 1.3.2}, we have 
\bse 
    \del S = \frac{1}{2\pi} \int d^d\sig T^{\a\beta}\nabla_{\a}\epsilon_{\beta},
\ese 
but then, by the discussion of how to identify the charge from the previous lecture, we have that 
\be 
    (J^{\a})^{\beta} = T^{\a\beta}. 
\ee 
The $\beta$ index here tells us which translation we are considering, and so the charge for the translation $\epsilon_{1}$, say, is just 
\bse 
    (J^{\a})^1 = T^{\a 1}\epsilon_1,
\ese 
which was the result we wanted. We can write this result in the following way 
\bse 
    J^{\a}_{(\beta)} = T^{\a\beta} \epsilon_{\beta},
\ese 
where the subscript on $J$ tells us which translation direction we're considering. 

\br 
    Note from the fact that $\p_{\a}J^{\a}_{(\beta)} =0$, i.e. it is conserved, we know that $\p_{\a}T^{\a\beta} = 0$, as the derivative on the right-hand side would have to hold for all $\epsilon_{\beta}$. This is the conservation of the stress-energy tensor!
\er 

\section{Conformal Transformations}

We now want to deal with a CFT. We know that conformal transformations are a symmetry of the system, and so we now want to ask the question of what the corresponding Noether charges are. 

Recall that a conformal transformation is the combined effort of taking a coordinate diffeomorphism and a Weyl transformation. Let's first consider the case where the fields are invariant under Weyl transformations. In this case, the transformations of the fields under conformal transformations is just a diffeomorphism. So in our 2D theory, what we want to find is the Noether currents that arise from the diffeomorphisms 
\bse 
    z' = z + v(z), \qquad \text{and} \qquad \overline{z}' = \overline{z} + \overline{v}(\overline{z}).
\ese
Let's concentrate on the holomorphic term first. Now to get the Noether current, we are meant to multiply by some small parameter of the coordinates $\epsilon(z,\overline{z})$. However, we already know that any $z$ dependence in $\epsilon$ will not contribute, as we could just absorb it into the definition of $v(z)$ and then we know this is a symmetry. So we only need to consider the $\overline{z}$ dependence of $\epsilon$. We can actually conclude from this that we expect the Noether current that arises from this to be a holomorphic function (as we will have $\overline{\p}\epsilon(z,\overline{z})$ and so the term before it will have an upper $\overline{z}$ index and a lower $z$ index). We will show this explicitly now. 

So we have
\bse 
    \del S = \frac{1}{2\pi} \int d^2\sig T^{\a\beta}\nabla_{\a} \big( v(z) \epsilon_{\beta}(z,\overline{z})\big) = \frac{1}{2\pi} \int d^2\sig T^{\a\overline{z}}\nabla_{\a} \big( v(z) \epsilon_{\overline{z}}(z,\overline{z})\big),
\ese
where we have used the fact that $\epsilon(z,\overline{z})$ only has a $\overline{z}$ coordinate. Now this gives us two terms. One with a $z$ derivative and one with a $\overline{z}$ derivative. However, we said that this part of our theory was invariant under any arbitrary change in the $z$ coordinate. That is, its only the $\overline{z}$ term that changes the path integral. We require that when this change in $\overline{z}$ is a constant, that the action doesn't change at all (i.e. when we took $\epsilon$ in the previous lecture to be a constant), and so for consistency we require the term with the $z$ derivative to always vanish. This is achieved by simply setting 
\bse 
    T^{z\overline{z}} = 0.
\ese 
This result is nothing surprising though. We required our theory by Weyl invariant, and so the transformation $\del g_{\a\beta} = \epsilon g_{\a\beta}$ must leave the action invariant. Plugging this into the equation for the change of the action we get 
\bse 
    0 = -\frac{1}{4\pi} \int d^2\sig \sqrt{g} {T^{\a}}_{\a} \epsilon,
\ese 
for all $\epsilon$, and so we conclude that the stress-energy tensor is traceless:
\be 
\label{eqn:TracelessStressEnergy}
    {T^{\a}}_{\a} = 0.
\ee 

Then from the rest of the calculation, with the identification procedure for the conserved current, we have
\be 
\label{eqn:JzConserved}
    J^{\overline{z}} = v(z)T(z)
\ee 
conserved. Following a similar process for the transformations of $\overline{z}$, we obtain the conserved current
\be
\label{eqn:JzBarConserved}
    \overline{J}^z = \overline{v}(\overline{z})\overline{T}(\overline{z}),
\ee 
where we have introduced the notation 
\be
\label{eqn:TNotation}
    T := T_{zz}, \qquad \text{and} \qquad \overline{T} := T_{\overline{z}\overline{z}}.
\ee 
Then, using $\p_{\a}T^{\a\beta}=0$ with $T^{z\overline{z}} = 0 = T^{\overline{z}z}$, we have 
\be 
\label{eqn:TTbarConserved}
    \overline{\p}T = 0, \qquad \text{and} \qquad \p \overline{T} = 0,
\ee 
and so we see that the currents are holomoprhic and antiholomorphic respectively, 
\be 
\label{eqn:JHolomorphic}
    \overline{\p} J^{\overline{z}} = 0, \qquad \text{and} \qquad \p \overline{J}^{z} = 0.
\ee 

\br 
    The above is in the notation Dr. Tong uses, where we have $J^z$ with an upper index. We can easily convert it to Dr. Minwalla's notation (where the index is lower) by using the metric. This turns a $z \leftrightarrow \overline{z}$, so we have\footnote{There's a factor of 2 floating around, but that doesn't change the holomorphisity/conservedness of the problem.}
    \bse 
        J_z^v = v(z) T(z), \qquad \text{and} \qquad \overline{J}_{\overline{z}}^{\overline{v}} = \overline{v}(\overline{z})\overline{T}(\overline{z}),
    \ese 
    and then similarly for \Cref{eqn:JHolomorphic}. We have also introduced a $v/\overline{v}$ on the $J/\overline{J}$ to remind us that its associated to the $v/\overline{v}$; it is \textit{not} a coordinate index. This notation has the nice advantage that the non-barred $J$ goes with the non-barred $z$ and vice versa. 
\er 

\br 
    We didn't need to take the transformation as we did; we could have taken a real transformation (i.e. of the $\sig$). The result of doing this will end up with one big conserved current, which can be decomposed into exactly the result we have.
\er 

\br 
\label{rem:ConformalResidue}
    Note we are now in exactly the situation we were discussing at the end of the last lecture; we are considering conserved charges for a 2D theory where the resulting currents are holomorphic/antiholomorphic. So we are working with things like \Cref{eqn:WardResidue}. We shall use this in the next section.
\er 

\section{Commutator Of Charges}
\label{sec:CommutatorOfCharges}

Suppose we have two currents $J^1$ and $J^2$, say, with associated charges 
\bse 
    Q^i = \frac{1}{2\pi i} \oint J^i(z) dz,
\ese
where the factor is included for a reason that will become clear in a moment. We now want to ask the question `what is the commutator of the two charges associated to these currents?' So we're looking to find something of the form 
\bse 
    Q^1Q^2 - Q^2Q^1,
\ese
but the obvious question is `what determines which is on the left/right?' (i.e. what makes $Q^1Q^2$ different from $Q^2Q^1$). The answer comes when we remind ourselves that the charges will appear inside correlation functions, which time order the entries, with the later times to the left. So in the radial quantisation picture, this means the larger circles to the left. Pictorially this looks like 
\begin{center}
    \btik 
        \draw[thick] (0,0) -- (5,0) -- (5,5) -- (0,5) -- (0,0);
        \draw[blue, decoration={markings, mark=at position 0.15 with {\arrow{>}}}, postaction={decorate}] (2.5,2.5) circle (2cm);
        \draw[red, decoration={markings, mark=at position 0.15 with {\arrow{>}}}, postaction={decorate}] (2.5,2.5) circle (1.5cm);
        \node at (3.3,3.3) {$z_2$};
        \node at (4.2,4.2) {$z_1$};
        \draw[ultra thick] (6,2.5) -- (6.5,2.5);
        \draw[thick] (7.5,0) -- (12.5,0) -- (12.5,5) -- (7.5,5) -- (7.5,0);
        \draw[red, decoration={markings, mark=at position 0.15 with {\arrow{>}}}, postaction={decorate}] (10,2.5) circle (2cm);
        \draw[blue, decoration={markings, mark=at position 0.15 with {\arrow{>}}}, postaction={decorate}] (10,2.5) circle (1.5cm);
        \node at (10.8,3.3) {$z_1$};
        \node at (11.7,4.2) {$z_2$};
    \etik 
\end{center}
where the blue circle is $Q^1$ and the red circle $Q^2$. We then take the limit that the circles approach each other to get an equal time relation.

So we need to do the integral
\be
\label{eqn:ContourIntegral}
    \bigg(\oint \frac{dz_1}{2\pi i}\oint \frac{dz_2}{2\pi i} - \oint \frac{dz_2}{2\pi i}\oint \frac{dz_1}{2\pi i} \bigg)J^1(z_1) J^2(z_2).
\ee 
This seems like a daunting task, but then we recall that our currents are holomorphic (or antiholomorphic), and so by Cauchy's theorem, we can distort them. The trick to solving this is to fix the position of one of them, say the $z_2$ one, in both diagrams and to just do the integrals over the other, $z_1$. 

\begin{center}
    \btik 
        \draw[thick] (0,0) -- (5,0) -- (5,5) -- (0,5) -- (0,0);
        \draw[blue, decoration={markings, mark=at position 0.15 with {\arrow{>}}}, postaction={decorate}] (2.5,2.5) circle (2cm);
        \node at (3.75,3.75) {\textcolor{red}{$\cross$}};
        \draw[blue, decoration={markings, mark=at position 0.15 with {\arrow{<}}}, postaction={decorate}] (2.5,2.5) circle (1.5cm); 
        \draw[ultra thick] (6,2.45) -- (6.5,2.45);
        \draw[ultra thick] (6,2.55) -- (6.5,2.55);
        \draw[thick] (7.5,0) -- (12.5,0) -- (12.5,5) -- (7.5,5) -- (7.5,0);
        \node at (11.25,3.75) {\textcolor{red}{$\cross$}};
        \draw[blue, decoration={markings, mark=at position 0.15 with {\arrow{>}}}, postaction={decorate}] (11.25,3.75) circle (0.5cm);
    \etik 
\end{center}

\br 
    Note in the above we have used the fact that we know $J^1$ is holomorphic at any point where there's no other insertions, but it may not be holomorphic when it meets another insertion (i.e. the Ward Identity). In this case our other insertion is $J^2$ at a particular point.
\er 

So we have a pole given by the residue of the OPE of $J^1$ and $J^2$. So our integral becomes 
\bse 
    \frac{1}{2\pi i} \oint dz_2 \,  \text{Res}[J^1J^2](z_2),
\ese 
but we know what this residue is, it comes from the OPE as discussed previously. We get a residue when in the OPE of $J^1$ and $J^2$ we have some term 
\bse 
    J^1(z_1)J^2(z_2) = ... + \frac{J^3(z_2)}{z_1-z_2} + ...,
\ese 
so if we define 
\bse 
    Q_3 = \frac{1}{2\pi i } \oint J^3(z_2) dz_2,
\ese
we get the commutation relation 
\be 
    [Q_1,Q_2] = Q_3.
\ee 

\br 
    The term in the above expression might not seem like a residue (recall it is the $\overline{\p}_{\overline{z}_1}$ we use). However one can show that 
    \bse 
        \overline{\p}_{\overline{z}} \frac{1}{z-\omega} = 2\pi \del(z-\omega,\overline{z}-\overline{\omega}),
    \ese 
    and so we really do have a pole here.
\er 

\br 
    To emphasise the point made before, what we have seen is that commutation relations between two charges that correspond to holomorphic (or antiholomorphic) currents in a 2D conformal field theory turn into a statement about the OPE of the currents. In other words, if you want to see if two charges commute or not, you can just consider the OPE of their currents and see if this OPE contains a pole. If it doesn't then the two charges commute, but if it does the commutation is given by the charge corresponding to the residue of the pole.
\er 

\br 
    We can also conclude that the commutator of a holomorphic with and antiholomorphic charge must be zero. Why? Because \Cref{eqn:OPE} tells us that the coefficients only depend on differences $z-\omega$ and $\overline{z}-\overline{\omega}$. So if your first charge only has a $z$ dependence then the coefficients must only have a $(z-\omega)$ dependence, but if the second only has a $\overline{\omega}$ dependence, the coefficients must have only a $\overline{z}-\overline{\omega}$ dependence. It can't be both, and so must vanish. 
\er 

\subsection{Action of Operators On States}

Suppose now we are interested not in two charges, but one charge and one operator. We can repeat the above procedure but now with the fixed position part really corresponding to just the position of our local operator, $\cO_1$, say. We then only have the one contour integral (as there is only one charge) and so we simply just have
\begin{center}
    \btik 
        \draw[thick] (0,0) -- (5,0) -- (5,5) -- (0,5) -- (0,0);
        \draw[] (2.5,0) -- (2.5,5);
        \draw[] (0,2.5) -- (5,2.5);
        \draw[blue, decoration={markings, mark=at position 0.15 with {\arrow{>}}}, postaction={decorate}] (4,4) circle (0.5cm);
        \node at (4.6,4.6) {\textcolor{blue}{$J$}};
        \node at (4,4) {$\cross$};
        \node at (4.3,3.7) {$\cO_1$};
    \etik 
\end{center}
This is just what we were saying back at \Cref{eqn:WardResidue}; the transformation of the local operator due to the current $J_z$ is given by the residue. 

Now let's consider what happens when we place the operator at the origin. The state-operator map tells us that a local operator at the origin corresponds to a wavefunction (and therefore a state) away from the origin. So our question about how does the operator change when a current is introduced becomes a question about how a state changes under the application of the charge operator. This is a much more familiar question! The result turns out super nicely: if the residue corresponds to come operator $\cO_2$, then we have 
\bse
    Q\ket{\cO_1} = \ket{\cO_2},
\ese 
where we have used the notation $\ket{\cO_i}$ to mean the state corresponding to the local operator $\cO_i$ at the origin. 

\br 
    We might think initially that there is a flaw in the above; when we're at the origin we can't place a smaller circle inside it. However, a bit of thought shows us that this isn't really a problem. The inner circle was just deformed away to nothing when we started considering a fixed point, so we need not even consider it when talking about operators. 
\er 

\br 
    To again emphasise the might of the OPE, we have seen that the OPE between currents and arbitrary operators tell us how:
    \ben
        \item An arbitrary local operator transforms under the symmetry that gives rise to the current, 
        \item The state generated by that local operator transforms under the symmetry, and
        \item When applied to integrals over local operators in terms of charges, charges transform under each other. 
    \een
\er 

\section{The Virasoro Algebra (Making a start)}

Now that we understand the above, we can now apply it to the concrete example of the stress-energy tensor. We first need to find the OPE of $v(z_1)T(z_1)$ with $v(z_2)T(z_2)$ (see \Cref{eqn:JzConserved}). To do this, we need to know the OPE for two stress-energy tensors in a 2D CFT. We want to calculate the OPE for $T(z_1)T(z_2)$. 

\bp 
    $T$ has dimension\footnote{If you are not familiar with what we mean by dimension $d$, basically we're working in units $c=\hbar=1$ and we say $[m]=1$ for the mass. We then work out the dimensions of other quantities from this. For example, $[z]=-1$ for distance.} $d$ in a $d$-dimensional CFT.
\ep 

\bq
    Recall that the spatial integral of $T$ will give us the energy or momentum. Both of these have dimension $1$. The integral contributes dimension $-1$ for each integration variable. So we have $1=[T]-(d-1)$, or $[T]=d$. 
\eq

We therefore require what ever appears on the right-hand side of our OPE to be a holomorphic, dimension 4 object. We are only interested in the terms in the OPE that produce poles. 

\bcl 
    In a unitary CFT there are no negative dimension operators, and the only dimension 0 operator is the identity, $\b1$. (We will show these results within the next few lectures.)
\ecl 

So, if we consider unitary CFTs, we know that the most singular term in the OPE must be proportional to $\b1/z^4$. We choose (for convenience) this proportionality constant to be $c/2$. So we have 
\bse 
    T(z_1)T(z_2) = \frac{c/2}{(z_{12})^4} + ...,
\ese 
where we have dropped the $\b1$ to lighten notation, and where we have introduced the notation $z_{12}=z_1-z_2$. The next possible term would be proportional to $1/z^3$, where we would need to insert a dimension $1$ operator in the numerator. However, we can show this term doesn't appear. 

Consider 
\bse 
    T(z_1)T(z_2) - T(z_2)T(z_1) = \bigg[\frac{c/2}{(z_{12})^4} + \frac{\cO(z_2)}{(z_{12})^3} + ... \bigg] - \bigg[\frac{c/2}{(z_{21})^4} + \frac{\cO(z_1)}{(z_{21})^3} + ... \bigg] = \frac{\cO(z_2)+\cO(z_1)}{(z_{12})^3} + ...,
\ese 
Now, the left-hand side is taken to appear inside a correlation function, and so, by time ordering, it must vanish. Now we can Taylor expand $\cO(z_2)$ about $z_1$, 
\bse 
    \cO(z_2) = \cO(z_1) - z_{12}\p\cO(z_1) + ...,
\ese 
and so we see, up to terms that are less singular (i.e. $(z_{12})^{-2}$ and $(z_{12})^{-1}$), this term must vanish. That is, because we only have the identity at a more singular term, there is no way to take something away from the $(z_{12})^{-3}$ term, and so the coefficient in front of the operator itself must vanish. 

We deal with the next two terms in one swoop via the following claim (which we verify towards the start of the next lecture): 

\bcl
\label{claim:TOTransformation}
    Let $\cO$ be a dimension $h$ holomorphic operator. Then the OPE of $T$ with $\cO$ is given by 
    \bse 
        T(z_1) \cO(z_2) = ... + h\frac{\cO(z_2)}{(z_{12})^2} + \frac{\p \cO(z_2)}{z_{12}} + ...
    \ese 
\ecl 


This finally gives us that 
\be 
\label{eqn:TTOPE}
    T(z_1) T(z_2) = \frac{c/2}{(z_{12})^4} + \frac{2T(z_2)}{(z_{12})^2} + \frac{\p T(z_2)}{z_{12}} + ... 
\ee 

\br 
    We can actually obtain the $\p T(z_2)$ term via the same method that we used to remove the cubic term before. However, there is no point in doing that here because we already have the answer.
\er 

\subsection{The Stress-Energy Currents}

What we did before was work out the OPE between the stress-energy tensors themselves. What we want to work out is the OPE between the currents that generated them, i.e. from \Cref{eqn:JzConserved}, we want the OPE

\bse 
    J_1(z_1) J_2(z_2) = v_1(z_1)T(z_1) v_2(z_2)T(z_2)
\ese 
We do this simply by using \Cref{eqn:TTOPE} and Taylor expanding one of the $v_1(z_1)$. So the pole terms are 
\bse 
    \begin{split}
        J_1(z_1) J_2(z_2) & = v_1(z_1)v_2(z_2) \bigg( \frac{c/2}{(z_{12})^4} + \frac{2T(z_2)}{(z_{12})^2} + \frac{\p T(z_2)}{z_{12}} + ... \bigg) \\
        & = v_2(z_2) \sum_{n} \frac{v_1^{(n)}(z_2)}{n!} (z_{12})^n \bigg( \frac{c/2}{(z_{12})^4} + \frac{2T(z_2)}{(z_{12})^2} + \frac{\p T(z_2)}{z_{12}} + ... \bigg) \\
        & =  \frac{v_2}{z_{12}} \bigg(v_1\p T + 2v_1'T + \frac{v_1''' c}{12}\bigg) + ...,
    \end{split}
\ese 
where we have dropped the $(z_2)$s in the last line to lighten the notation. So the commutator of the charges corresponding to the two currents is 
\bse 
    [Q_1,Q_2] = Q_3,
\ese
where 
\be 
\label{eqn:ChargeForJJ}
    \begin{split}
        Q_3 & = \oint \frac{dz_2}{2\pi i } \bigg(v_1v_2\p T + 2v_1'v_2T + \frac{v_1''' v_2}{12}c\bigg)(z_2) \\ 
        & = \oint \frac{dz_2}{2\pi i } \bigg(\big( v_1'v_2 - v_1v_2'\big)T + \frac{v_1''' v_2}{12}c\bigg)(z_2),
    \end{split}
\ee 
where we have done an integration by parts. The first term in the integrand is what you would get just by taking the commutator of the vector fields $[v_1\p_{v_1},v_2\p_{v_2}]$, however the second term is something completely unexpected. 

We can repeat all of the above for the antiholomorphic currents and arrive at the result
\be 
\label{eqn:TbarTbar}
    \overline{T}(\overline{z}_1) \overline{T}(\overline{z}_2) = \frac{\Tilde{c}/2}{(\overline{z}_{12})^4} + \frac{2\overline{T}(\overline{z}_2)}{(\overline{z}_{12})^2} + \frac{\p \overline{T}(\overline{z}_2)}{\overline{z}_{12}} + ...,
\ee 
and similarly for the other expressions. 

What we have started here is the discussion of the Virasoro algebra. We will continue to discuss this in the next lecture. 
\chapter{Virasoro Algebra \& Primary Operators}

So far we have worked out the algebra (i.e. the commutation relations) for the conformal transformations corresponding to transformations $z\to z +v(z)$ (and also for the antiholomorphic transformations). We now want to continue this discussion in order to arrive at the so called \textit{Virasoro Algebra}. 

\section{Virasoro Algebra}

Consider the case when we have $T$\footnote{Again we will just consider the holomorphic part, but obviously everything will follow for the antiholomorphic part too.} inserted within some circle in our radial quantisation picture. We can have arbitrary insertions outside the circle\footnote{By which we mean that the results hold for arbitrary correlation functions of this form.}, but the only insertion we're allowed inside is perhaps one at the origin. 

\begin{center}
    \btik 
        \draw[thick] (0,0) -- (5,0) -- (5,5) -- (0,5) -- (0,0);
        \draw[dashed] (2.5,2.5) circle (1.5cm);
        \node at (2.5,2.5) {$\cross$};
        \node at (2.85,2.3) {$\cO_0$};
        \node at (3,3.5) {$T(z)$};
        \node at (4,4.5) {$\cO_1$};
        \node at (1,4) {$\cO_2$};
        \node at (0.5,1) {$\cO_3$};
        \node at (4,1.2) {$\cO_4$};
    \etik 
\end{center}

Now recall that $T$ is holomorphic, \Cref{eqn:TTbarConserved}. Also recall that the Ward identity said this way true, provided there was no contact terms between $T$ and any other insertions. In our situation then, we know that $T$ is holomorphic everywhere inside the circle, expect perhaps at the origin. We can, therefore, expand $T(z)$ as a Laurent series 
\be
\label{eqn:TLaurent}
    T(z) = \sum_{m=-\infty}^{\infty} \frac{L_m}{z^{m+2}},
\ee 
where $L_m$ is a \textit{non-local} operator, given by inverting the above to give\footnote{If you are not familiar with how to do this, a quick Google should help.} 
\be 
\label{eqn:LaurentInversion}
    L_m = \oint \frac{dz}{2\pi i} z^{m+1} T(z).
\ee 

\br 
We have used $z^{m+2}$ above, and it is reasonable to ask `why $+2$?' The answer is, of course, its a useful convention. The reason for which is related to the fact that $[T]=2$. Note that in the integral we only have $z^{m+1}$. We can argue this simply because $[dz]=[z]$.
\er 

\br 
Note the operators $L_m$ act in the Hilbert space language, that is they act on radial slices. In other words, it is a loop on the diagram, and it acts on a circles, which correspond to states in our Hilbert space.
\er 

We now want to find the algebra of these $L_m$s, i.e. we want to find $[L_m,L_n]$. We just follow the same procedure as the previous lecture: we have a double integral, we fix one point and pick up a pole and then integrate over the remaining variable. We have 

\bse 
    \begin{split}
        [L_m,L_n] & = \bigg( \oint \frac{dz}{2\pi i} \oint \frac{d\omega}{2\pi i} - \oint \frac{d\omega}{2\pi i} \oint \frac{dz}{2\pi i}\bigg) z^{m+1}\omega^{n+1} T(z)T(\omega) \\
        & = \oint \frac{d\omega}{2\pi i} \text{Res} \bigg[ z^{m+1}\omega^{n+1} \bigg( \frac{c/2}{(z-\omega)^4} + \frac{2T(\omega)}{(z-\omega)^2} + \frac{\p T(\omega)}{z-\omega} + ... \bigg) \bigg].
    \end{split}
\ese 
Now we simply Taylor expand $z^{m+1}$ about $\omega$, 
\bse 
    z^{m+1} = \omega^{m+1} + (z-\omega)(m+1)\omega^m + (z-\omega)^2\frac{m(m+1)}{2}\omega^{m-1} + (z-\omega)^3\frac{m(m+1)(m-1)}{6}\omega^{m-2} + ...,
\ese 
giving 
\bse 
    \begin{split}
        [L_m,L_n] & = \oint \frac{d\omega}{2\pi i} \bigg(\omega^{n+m-1}m(m^2-1)\frac{c}{12} + 2\omega^{m+n+1}(m+1)T(\omega) + \omega^{m+n+2}\p T(\omega) \bigg) + ... \\
        & = \oint \frac{d\omega}{2\pi i} \bigg( \omega^{n+m-1}m(m^2-1)\frac{c}{12} + 2\omega^{m+n+1}(m+1)T(\omega) - (m+n+2)\omega^{m+n+1} T(\omega) \bigg) + ... \\
        & = \oint \frac{d\omega}{2\pi i} \bigg( \omega^{n+m-1}m(m^2-1)\frac{c}{12} + \omega^{m+n+1}(m-n)T(\omega) \bigg) + ...,
    \end{split}
\ese 
where we have used integration by parts to go to the second line. Now we note that the second term is just proportional to $L_{m+n}$. For the first term we pick up a residue at the pole (i.e. when we have a $1/\omega$ factor). We therefore have the
\mybox{
Virasoro algebra
\be 
\label{eqn:LmLnCommutator}
    \begin{split}
        [L_m,L_n]  & = (m-n)L_{m+n} + \frac{c}{12} m(m^2-1)\del_{m+n,0} \\
        [\widetilde{L}_m,\widetilde{L}_n]  & = (m-n)L_{m+n} + \frac{\widetilde{c}}{12} m(m^2-1)\del_{m+n,0} \\
        [L_m,\widetilde{L}_n] & = 0
    \end{split}
\ee 
}
where we have also included the results from the antiholomorphic part. 

\br 
We could have actually arrived at this result quite quickly given our results from the end of the last lecture. If we simply set $v_1=z^{m+1}$ and $v_2=z^{n+1}$ in \Cref{eqn:ChargeForJJ} we would get the same exact result. This makes sense, as all we have done is multiply $T$ by a holomorphic function and so, by the holomorphisity of $T$, we expect to get a new conserved charge. We have done it this way explicitly because it is standard notation and is worth seeing.
\er 

\br 
We see from the Virasoro algebra that our theory has an infinite number of conserved charges. This just reflects the fact that there is an infinite number of holomorphic functions out there by which we can multiply $T$.
\er 

We now note something interesting. If the $L_{m+n}$ term was not present, we would have something akin to \Cref{eqn:AlphaCommutationRelations}, and so we would be looking at harmonic oscillator type behaviour. We then recall the comment made after \Cref{eqn:ChargeForJJ} that this second term comes from the commutation of the classical diffeomorphisms. We see, therefore, that this harmonic oscillator behaviour arises from the Weyl transformations!

\subsection{Witt Algebra}

\bd 
A complex Witt algebra is the quotient of the Virasoro algebra by its centre,\footnote{That is $c$.} i.e. it is an algebra with basis $L_m = z^{m+1}\p_z$ with lie bracket 
\bse
    [L_m,L_n] = (m-n)L_{m+n}.
\ese 
\ed 

We see straight away from the Virasoro algebra that $m=0,\pm1$ form a closed subalgebra, and that it is a closed Witt subalgebra. The commutation relations are
\bse 
    [L_1,L_{-1}] = 2L_0, \qquad [L_1,L_0] = L_1, \qquad  [L_0,L_{-1}] = L_{-1},
\ese
and all others vanish.

\bp 
The Witt subalgebra $\{L_{-1},L_0,L_1\}$ is isomorphic to the lie algebra $\mathfrak{sl}(2,\C)$.
\ep 

\bq 
Recall that $\mathfrak{sl}(2,\C)$ has three basis elements $e,f,h$ with commutators 
\bse 
    [e,f] = h, \qquad [f,h] = 2f, \qquad [h,e] = 2e.
\ese 
Define the injective embedding
\bse 
    \iota : \mathfrak{sl}(2,\C) \hookrightarrow Witt,
\ese 
that identifies $f$ with $L_{-1}$, $e$ with $-L_{-1}$ and $h$ with $-2L_0$. This map is clearly surjective into the subalgebra of concern, and by direct calculation it preserves the bracket structure, i.e. 
\bse 
    \iota([a,b]) = [\iota(L_a),\iota(L_b)],
\ese 
for $a,b=e,f,h$ and $L_a$ obviously being the associated element in our subalgebra. 
\eq 

Note the above conditions are equivalent to requiring that the transformation is well defined everywhere. Two potentially problematic points at $z=0$ and $z\to\infty$. When $z=0$ we require $m\geq-1$, so that we don't end up dividing by $0$. For $z\to\infty$, we first take the conformal transformation
\bse 
    z = \frac{1}{\omega},
\ese 
so our condition turns into the condition $\omega=0$. Our vector field becomes 
\bse 
    z^{m+1}\p_z = -\omega^{-m+1}\p_{\omega},
\ese 
where we have used the chain rule. We therefore require $m \leq 1$. So collectively we have $-1\leq m \leq 1$, which is exactly our situation. This turns out to be related to the statement that the vacuum of the theory is annihilated by these three operators, as we shall soon see. Note the fact that they are defined everywhere in the theory means the conserved charges are \textit{global}. 

\subsection{The $\mathfrak{sl}(2,\C)$-Invariant State}

Let's now consider what happens when we insert the identity $\b1$ at the origin. Our correlation function becomes
\bse 
    \langle \b1 L_m ... \rangle = \int \oint \frac{dz}{2\pi i } z^{m+1}T(z) ... = 0,
\ese 
for all $m\geq 1$, as the contour integral vanishes in this case. Therefore we can conclude that 
\be 
    L_m\ket{0} = 0, \qquad \forall m= -1,0,1,2...,
\ee 
where $\ket{0}$ is the vacuum state. 

Now note that this is the \textit{only} state that will be annihilated by $L_{-1}$. This is because $L_{-1}=\p_z$ which is just translations, and so for any non-trivial insertion $\cO(z=0)$ just gets shifted to some other point, and, by the state-operator map, it means the $L_1\ket{\cO}\neq 0$. This makes the vacuum the \textit{only} state in the Hilbert space that is annihilated by the complete Witt subalgebra defined previously, which we shall now refer to as the $\mathfrak{sl}(2,\C)$-algebra. We, therefore, refer to this identity state as the $\mathfrak{sl}(2,\C)$-invariant state.


\section{Weight Of Operators}

\bd[Weight Of Operator]
An operator $\cO$ is said to have weight $(h,\widetilde{h})$ is under the infinitesimal conformal transformation $(z,\overline{z}) \to (z + \epsilon z, \overline{z} + \overline{\epsilon}\overline{z})$, the operator transforms as
\be 
\label{eqn:WeightOfOperator}
    \del \cO = -\epsilon(h\cO + z\p\cO) -\overline{\epsilon}(\widetilde{h}\cO + \overline{z}\overline{\p}\cO).
\ee 
\ed 

\br 
Note it is \textit{not} true that all operators have well defined weight. We see this easily when we consider the state-operator map. We can take superpositions of states that correspond to operators with different weights, and so the resulting operator will not have a well defined weight. 
\er 

In fact, this remark highlights a deeper understanding of what the weights represent: an operator only has well defined weight if the state corresponding to the operator is an eigenstate of dilatations and rotations. In other words, the $\epsilon h\cO$ term only appears in the transformation if the state is an eigenstate.\footnote{Note, that here we just have what looks like an eigenvalue equation anyway, $\del \cO = C\cO$ for some constant!} 

Now, recall that the eigenstate of the eigenvalue under rotations is the \textit{spin} of the particle. Rotations are generated by the operator 
\bse 
    L = z\p - \overline{z}\overline{\p},
\ese 
and so if we require the operator to be invariant (in the sense of its state is an eigenvector), then, by equating the holomorphic and antiholomorphic parts of \Cref{eqn:WeightOfOperator}, we get that the spin is given by

\be 
\label{eqn:SpinWeights}
    s = h - \widetilde{h}.
\ee 
Similarly, the dilatation operator is given by \Cref{eqn:DilatationOperator}, which is 
\bse 
    D = z\p + \overline{z}\overline{\p},
\ese
and so we see that the so-called \textit{scaling dimension} is given by 
\be 
\label{eqn:ScalingDimension}
    \Delta = h + \widetilde{h}.
\ee 

\br 
Note the name scaling dimension should make sense, as the dilatation operator corresponds to the Hamiltonian in the radial quantisation picture, and so corresponds to moving to higher radius rings. 
\er 

\br 
We see from the dilatation operator and the relation $L_m = z^{m+1}\p_z$ that the energy is given by $L_0 +\widetilde{L}_0$. Now also notice that 
\be 
\label{eqn:L0LmCommutator}
    [L_0,L_m] = -mL_m,
\ee 
and so, for $m>0$, $L_m$ lowers the energy of the system and, for $m<0$, $L_m$ raises the energy.
\er 

We now see an interesting result. Recalling \Cref{eqn:JzConserved} we see from the Ward identity that the OPE of $J\cO$ will determine the $1/z^2$ term in the OPE of $T\cO$. That is, when we expand $J_z = zT$ about $\omega$, we get a $(z-\omega)$ term from the $z$, which couples to the $1/z^2$ term to give a pole. Recalling also that $T$ corresponds to conformal transformations, we therefore see that for an operator of well defined weight, the $T\cO$ OPE is given by 
\be 
\label{eqn:TOOPEWeight}
    \begin{split}
        T(z)\cO(\omega,\overline{\omega}) & = ... + h\frac{\cO(\omega,\overline{\omega})}{(z-\omega)^2} + \frac{\p\cO(\omega,\overline{\omega})}{z-\omega} + ... \\
        \overline{T}(\overline{z})\cO(\omega,\overline{\omega}) & = ... + \widetilde{h}\frac{\cO(\omega,\overline{\omega})}{(\overline{z}-\overline{\omega})^2} + \frac{\p\cO(\omega,\overline{\omega})}{\overline{z}-\overline{\omega}} + ...,
    \end{split}
\ee 
where we have included the antiholomorphic expression too. 

We have, in fact, already used this result, It's \Cref{claim:TOTransformation}! In order to obtain \Cref{eqn:TTOPE}, we used the fact that $T$ has weight $(2,0)$. Why? You ask. We have already argued that the scaling dimension (before we just called it the dimension of $T$) of $T$ is 2, add to that the fact that the spin of $T$ is also 2 (as it is a symmetric 2-tensor), we are forced to conclude that $h=2$ and $\widetilde{h}=0$.

\section{Primary Operators}

\bd[Primary Operator] 
Under a conformal transformation $z\to \omega(z)$, a so-called \textit{primary operator} transforms as 
\be 
\label{eqn:PrimaryOperators}
    \widetilde{\cO}(\omega,\overline{\omega}) = \bigg(\frac{\p \omega}{\p z}\bigg)^{-h} \bigg(\frac{\p \overline{\omega}}{\p \overline{z}}\bigg)^{-\widetilde{h}} \cO(z,\overline{z}).
\ee 
\ed

Let $\omega(z) = z + f(z)$ and let's consider the case when $f(z)$ is small. Then we have to first order 
\bse 
    \bigg(\frac{\p \omega}{\p z}\bigg)^{h} = 1 + h\p f, \qquad \text{and} \qquad \cO(\omega) = \cO(z) + f\p\cO(z),
\ese 
and so 
\bse 
    \del \cO(z,\overline{z}) = -h\p f\cO(z,\overline{z}) - f\p\cO(z,\overline{z}),
\ese
and a similar expression for the antiholomorphic term. But then we know (via the Ward identity) that this gives us the residue of the OPE between $J\cO$, and so we can conclude that a primary operator's OPE with $T/\overline{T}$ must truncate at second order singularity, i.e. 
\be 
\label{eqn:TOPrimary}
    \begin{split}
        T(z)\cO(\omega,\overline{\omega}) & = h\frac{\cO(\omega,\overline{\omega})}{(z-\omega)^2} + \frac{\p\cO(\omega,\overline{\omega})}{z-\omega} + \text{non-singular}, \\
        \overline{T}(\overline{z})\cO(\omega,\overline{\omega}) & = \widetilde{h}\frac{\cO(\omega,\overline{\omega})}{(\overline{z}-\omega)^2} + \frac{\overline{\p}\cO(\omega,\overline{\omega})}{\overline{z}-\omega} + \text{non-singular}.
    \end{split}
\ee 

\br 
\Cref{eqn:TOPrimary} is often given as an alternative definition for a primary operator. 
\er 

\br 
Note that having well define weight does not mean an operator is primary. For example, we've seen that $T$ has well defined weight, but it contains a $1/z^4$ term in its OPE with itself, and therefore is not primary.
\er 

\br 
We might wonder why we are interested in primary operators at all. Well a na\"{i}ve explanation, is to go back and remind ourselves of the Polyakov action. We are taken derivatives of scalar fields (the $x^{\mu}$s), which transform as 
\bse 
    \frac{\p x^{\mu}(\omega)}{\p \omega} = \bigg(\frac{\p \omega}{\p z}\bigg)^{-1} \frac{\p x^{\mu}(z)}{\p z},
\ese 
which is a primary operator transformation. Note, however, that a second order derivative will not transform this way, and is therefore not a primary operator. 
\er 

\subsection{Primary States}

We now want to look at the action of the Virasoro algebra on states produced by inserting a primary operator at the origin. Pictorially, we are looking at the path integral 
\begin{center}
    \btik 
        \draw[thick] (0,0) -- (5,0) -- (5,5) -- (0,5) -- (0,0);
        \draw[dashed] (2.5,2.5) circle (1cm);
        \draw[blue, decoration={markings, mark=at position 0.15 with {\arrow{>}}}, postaction={decorate}] (2.5,2.5) circle (1.2cm);
        \node at (2.5,2.5) {$\cross$};
        \node at (2.7,2.3) {$\cO$}; 
        \node at (3.5,3.7) {\textcolor{blue}{$L_m$}};
    \etik 
\end{center}
where the dotted line is the state $\ket{\cO}$. We `shrink' this to the origin and then look at the OPE,

\bse 
    \begin{split}
        L_m\ket{\cO} & = \oint \frac{dz}{2\pi i} z^{m+1} T(z)\cO(\omega=0) \\
        & = \oint \frac{dz}{2\pi i} z^{m+1}\bigg( \frac{h\cO}{z^2} + \frac{\p \cO}{z} + ... \bigg),
    \end{split}
\ese 
where the ellipsis indicates non-singular terms. So let's consider the values of $m$ case by case:

\begin{itemize}
    \item If $m<-1$ then we don't know what we'll get as our residue; we'd need to know what the singular terms are in order to know that. So for a \textit{general} primary operator we don't know what to expect. 
    \item If we have $m=-1$ then we pick out the residue $\p\cO$, and so we get the state $\ket{\p\cO}$. 
    \item If $m=0$ then we pick up the residue $h\cO$ and obtain the state $h\ket{\cO}$. 
    \item If $m\geq 1$ then there are no poles and so the integral vanishes. 
\end{itemize}

We call the states corresponding to primary operators (unsurprisingly) primary states.

\br 
Note the condition for $m=-1$ helps us to understand further why the only state that is annihilated by $L_{-1}$ comes from inserting the identity at the origin. That is, it is the only one with vanishing derivative, which corresponds simply to shifting the operator off the origin. 
\er 

\br 
The condition for $m=0$ also makes sense. In this case, we are just considering scalings, and we've already seen that the derivative transforms with weight $h=1$ under such transformations. Then an operator of dimension $h$ can be thought of as $h$ derivatives, and so transforms with $h$.
\er 

\br 
Note that the conditions for $m=-1$ and $m=0$ hold for \textit{any} operator of well defined weight. That is, they need not be primary. We have seen this explicitly when considering $TT$. However, the condition for $m\geq 1$ is only true for primary operators.
\er 

\br 
There is an interesting result we can note here. Recall that $L_0 = z\p$ in a basis, and so we see that the dilatation operator is $D = L_0 + \widetilde{L}_0$ and the momentum is $L = L_0 - \widetilde{L}_0$, but both of these are symmetries of our system (they're the energy and momentum on the cylinder) and so we can always find a basis in the Hilbert space such that the states have well defined $L_0$ action. This then corresponds to the fact that we can always find a basis such that $\cO_i$ in the basis have definite $h$ and $\widetilde{h}$.
\er 

\br 
\label{rem:VermaRemark}
Recalling that $L_m$ for $m>0$ lowered the energy of the state, we see that for primary operators, we cannot lower the energy any further (as $L_m\ket{\cO} = 0$ for all $m\geq 1$). We shall return to this point next lecture when looking at the Verma module.
\er 

What we've derived above is obviously true, however we have slightly limited ourselves to placing our operators at the origin. We did this as it makes contact with the state-operator map and allows us to define primary states. However, it is also worth looking at (and we shall use it soon) the case when the operator is not at the origin. We just repeat the calculation, but now without assuming $\omega=0$. So we have 

\bse 
    \begin{split}
        L_m\cO & = \oint \frac{dz}{2\pi i} z^{m+1} T(z)\cO(\omega) \\
        & = \oint \frac{dz}{2\pi i} \big[\omega^{m+1} + (m+1)\omega^m (z-\omega) + ...\big] \bigg( \frac{h\cO(\omega)}{(z-\omega)^2} + \frac{\p\cO}{z-\omega} + ...\bigg),
    \end{split}
\ese 
and so we see that  
\mybox{
For a primary operator 
\be
\label{eqn:LmPrimaryOperator}
    \begin{split}
        L_{-1}\cO(\omega) & = \p\cO(\omega), \\
        L_{0}\cO(\omega) & = h\cO(\omega) + \omega\p\cO(\omega), \\
        L_{m}\cO(\omega) & = \omega^{m+1}\p\cO(\omega) + (m+1)\omega^m h\cO(\omega).
    \end{split}
\ee 
}

\subsection{Quasi-Primary Operators}

As we have seen above the subalgebra containing $\{L_{-1},L_0,L_1\}$ is interesting. We therefore proceed to define \textit{quasi-primary} operators, which obey \Cref{eqn:LmPrimaryOperator} apart from we need only impose the $L_1\ket{\cO} =0$, i.e. $L_m\ket{\cO}$ need not vanish for $m\geq 2$. It follows, therefore, that all primary operators are quasi-primary, but the reverse is not true. An operator which is neither primary or quasi-primary is referred to as \textit{secondary}.

\subsection{Two/Three-Point Functions for Quasi-Primary Operators}

Let's now consider the case where we have an arbitrary number of operator insertions anywhere expect at the infinite past or infinite future. So on the plane, there is no insertions at the origin or at an infinite distance from the origin. Now consider inserting the $\mathfrak{sl}(2,\C)$-algebra around these points: that is, in the plane picture, we take a closed loop around the origin and around the point that is infinity with any of $L_{-1},L_0,L_1$. We know that \textit{any} correlation function (i.e. any insertions at anywhere, subject to the constraints we just made) must vanish. The reason for this is simply that the vacuum is annihilated by all three of these operators. 

To clarify taking the contour around the infinite point, we could just do what we did at the beginning and take the transformation $z\to 1/\omega$, and take it about the origin. Note that $L_{\pm1} \to L_{\mp1}$, $L_0\to L_0$ under this transformation, and so we don't leave our Witt subalgebra. 

Mathematically, what we have is 
\bse 
    \langle L_a ... L_a \rangle = 0, \qquad \forall a=-1,0,1.
\ese 
where we can put anything in the middle. However, this is equivalent to saying the sum of the correlation functions of the poles vanishing:
\be 
    \langle \del \cO_1 \cO_2...\cO_n\rangle + \langle \cO_1 \del\cO_2 ... \cO_n \rangle + ... + \langle \cO_1\cO_2...\del\cO_n\rangle = 0.
\ee 
We can see this pictorially quite easily: simply bring the contour at $z\to\infty$ towards $z=0$, and pick up poles as you go, then use the Ward identity to turn this into a statement about how the operators transform under the current.

\begin{center}
    \btik 
        \draw[thick] (0,0) -- (0,5) -- (5,5) -- (5,0) -- (0,0);
        \draw[] (2.5,0) -- (2.5,5);
        \draw[] (0,2.5) -- (5,2.5);
        \draw[blue, decoration={markings, mark=at position 0.15 with {\arrow{>}}}, postaction={decorate}] (2.5,2.5) circle (0.2cm); 
        \draw[blue, decoration={markings, mark=at position 0.15 with {\arrow{>}}}, postaction={decorate}] (2.5,2.5) circle (2.5cm);
        \node at (4,2) {$\cross$};
        \node at (3,1) {$\cross$};
        \node at (1,4.3) {$\cross$};
        \node at (3.5,4) {$\cross$};
        \node at (1.5,2) {$\cross$};
        \draw[thick] (5.5,2.65) -- (6.5,2.65);
        \draw[thick] (5.5,2.35) -- (6.5,2.35);
        \draw[thick] (7,0) -- (7,5) -- (12,5) -- (12,0) -- (7,0);
        \draw[] (9.5,0) -- (9.5,5);
        \draw[] (7,2.5) -- (12,2.5);
        \node at (11,2) {$\cross$};
        \draw[blue, decoration={markings, mark=at position 0.15 with {\arrow{>}}}, postaction={decorate}] (11,2) circle (0.2cm);
        \node at (10,1) {$\cross$};
        \draw[blue, decoration={markings, mark=at position 0.15 with {\arrow{>}}}, postaction={decorate}] (10,1) circle (0.2cm);
        \node at (8,4.3) {$\cross$};
        \draw[blue, decoration={markings, mark=at position 0.15 with {\arrow{>}}}, postaction={decorate}] (8,4.3) circle (0.2cm);
        \node at (10.5,4) {$\cross$};
        \draw[blue, decoration={markings, mark=at position 0.15 with {\arrow{>}}}, postaction={decorate}] (10.5,4) circle (0.2cm);
        \node at (8.5,2) {$\cross$};
        \draw[blue, decoration={markings, mark=at position 0.15 with {\arrow{>}}}, postaction={decorate}] (8.5,2) circle (0.2cm);
    \etik 
\end{center}

Let's now consider the special case when all of the insertions are at least quasi-primary. We want to find the two-point function
\bse 
    G(z_1,z_2) := \langle \cO_1(z_1) \cO_2(z_2) \rangle,
\ese 
where the weights are $(h_1,0)$ and $(h_2,0)$, respectively. Note as usual we're just considering the holomorphic part, but the antiholomorphic part follows exactly the same way.

First let's consider the action of $L_{-1}$. In this case $\del\cO_i(z_i) = \p\cO_i(z_i)$ and so we have 
\bse 
    \langle \p\cO_1(z_1) \cO(z_2) \rangle + \langle \cO(z_1) \p\cO(z_2)\rangle = 0.
\ese 
Now we can take these derivatives out and turn them into derivatives on $z_1$ and $z_2$. We do not pick up any additional terms from the Heaviside functions (as we did with \Cref{eqn:WardIdentity}) as the Heaviside function is a function of the \textit{difference} $(z_1-z_2)$, and overall it cancels. We therefore have 
\bse 
    (\p_{z_1} + \p_{z_2})\langle \cO(z_1)\cO(z_2) \rangle = 0,
\ese 
but this, in turn, tells us that the two-point function is a function only of the difference, 
\bse 
    G(z_1,z_2) = G(z_1-z_2).
\ese
This is just the condition of translation invariance, which we have been using loads throughout our calculations. 

Now let's consider the $L_0$ action. In this case $\del\cO_i(z_i) = h_i\cO_i(z_i) + z_i\p\cO_i(z_i)$, and so we have
\bse 
    \big[z_1\p_{z_1} + z_2\p_{z_2} + (h_1+h_2)\big]G(z_1-z_2) = 0, 
\ese 
which tells us that $G(z_1-z_2)$ has homogeneity $-(h_1+h_2)$, and so we have 
\bse 
    G(z_1-z_2) = \frac{A}{(z_1-z_2)^{h_1+h_2}},
\ese
for some constant $A$.

Now finally act with $L_1$, here $\del\cO_i(z_i)= z^2_i\p\cO_i(z_i) + 2z_i h_i\cO(z_i)$, giving 
\bse 
    \big[z_1^2\p_{z_1} + z_2^2\p_{z_2} + 2(z_1h_1+z_2h_2)\big](z_1-z_2)^{-(h_1+h_2)} = 0.
\ese 
The left-hand side does not, in general, vanish, and so we solve it to give a condition between $h_1$ and $h_2$. We obtain
\bse 
    (z_2^2 - z_1^2)(h_1+h_2) + 2(z_1h_1 + z_2h_2)(z_1-z_2) = 0,
\ese 
which doesn't look much nicer. However we notice that when $h_1=h_2=h$ we have 
\bse 
    2h(z_2^2-z_1^2) + 2h(z_1+z_2)(z_1-z_2) = 2h(z_2^2-z_1^2) + 2h(z_1^2 -z_2^2) = 0,
\ese 
and so we have 
\be 
\label{eqn:TwoPointFunctionPrimaryOperators}
    G(z_1-z_2) = \begin{cases}
    A(z_1-z_2)^{-2h} & \text{if } h_1=h_2=h \\
    0 & \text{otherwise}
    \end{cases}
\ee 

We will see that this result is related to the fact on our Hilbert space, states have non-zero inner product only if their energies are equal. We can see already why it's related to the energy, by recalling that the Hamiltonian is given by $L_0+\widetilde{L}_0$, along with the fact that the states $\ket{\cO}$ were eigenstates of $L_0/\widetilde{L}_0$ with eigenvalues $h/\widetilde{h}$.

Following the same approach for a three point function 
\bse 
    G(z_1,z_2,z_3) := \langle \cO_1(z_1)\cO_2(z_2)\cO_3(z_3)\rangle,
\ese    
where the operators have holomorphic weights $h_1,h_2,h_3$ respectively, is given by 
\be 
\label{eqn:ThreePointFunctionPrimaryOperators}
    G(z_1,z_2,z_3) = B(z_1-z_2)^{h_3-h_1-h_2}\cdot (z_2-z_3)^{h_1-h_2-h_3}\cdot (z_3-z_1)^{h_2-h_3-h_1}
\ee 

\br 
The constant $B$ here is not independent of $A$. In fact, once you choose a specific $A$ the constants for all higher point functions are determined.
\er 
\chapter{Weyl Anomaly \& Verma Module}

\section{Weyl Anomaly}

Recall that we defined a holomorphic primary operator of weight $h$ to be one that transforms as 
\bse 
    \cO(z) = \bigg(\frac{\p \omega}{\p z}\bigg)^{h} \cO(\omega)
\ese 
under the conformal transformation $z\to \omega(z)$. Now we note the transformation of purely covariant tensors under diffeomorphisms\footnote{We're considering flat space here, so don't need to worry about covariant derivatives and Christoffel symbols.}
\bse 
   A_{\a\beta\gamma...}(z) = \bigg(\frac{\p \widetilde{\a}}{\p\a}\bigg) \bigg(\frac{\p \widetilde{\beta}}{\p \beta}\bigg) \bigg( \frac{\p\widetilde{\gamma}}{\p\gamma} ...\bigg)  A_{\widetilde{\a}\widetilde{\beta}\widetilde{\gamma}...}(\omega) 
\ese
so if we take $\a=\beta=\gamma = ... = z$, where there is a total of $h$ terms and similarly $\widetilde{\a} = ...= \omega$ we simply get 
\bse 
    A_{z...}(z) = \bigg(\frac{\p \omega}{\p z}\bigg)^h A_{\omega...}(\omega),
\ese 
which matches our primary operator transformation. It is important to note, though, that the primary operator transformation is for \textit{conformal} transformations, whereas the tensor is for \textit{diffeomorphisms}. 

\br 
As normal we have only considered the holomorphic case, but clearly there is nothing special about it over the antiholomorphic part, and so we have an analogous expression for that. 
\er 

\br 
Note also that if we took a \textit{contravariant} tensor (i.e. indices upstairs) we would get a primary operator of weight $-h$. 
\er 

Now recall the transformation of the stress-energy tensor under a conformal transformation: we take the residue of $\epsilon(z)T(z)T(\omega)$, where the OPE is given by
\bse 
    T(z)T(\omega) = \frac{c/2}{(z-\omega)^4} + \frac{2T(\omega)}{(z-\omega)^2} + \frac{\p T(\omega)}{z-\omega} + \text{non-singular}. 
\ese 
The final two terms in this are the transformation of a weight $(2,0)$ primary operator, but we also know that $T$ is a rank-2 covariant tensor, and so we conclude that both of these terms come from the diffeomorphic part of the conformal transformation. This also means that the first term \textit{must} come from the Weyl part of the conformal transformation. We have already argued this fact when we were discussing the $\del_{m+n,0}$ term present in the Virasoro algebra commutation relations --- again it was the $c$ term that was related to the Weyl transformation --- however we now have further justification for this claim.

As we did before, we see that the change in the stress-energy tensor for the Weyl transformation is
\bse 
    \del_W T(\omega) = \frac{c}{12}\epsilon'''(\omega),
\ese
where the $W$ indicates that we're only considering the Weyl part. We now want to ask the question `What Weyl factor gives rise to this transformation?' --- i.e. what is $\phi$ equal to in $g \to e^{\phi}g$.\footnote{In the original definition we gave, we used $e^{2\phi}$. However, for consistency with Dr. Minwalla's working, we shall absorb the $2$ into the definition of $\phi$ and go from there.} 

If we're considering the infinitesimal diffeomorphism transformation $z \to \omega = z +\epsilon(z)$, and similarly for $\overline{z}$, the line element transforms to
\bse 
    d\omega d\overline{\omega} = \big(1 + \p\epsilon(z)\big) \big(1 + \overline{\p}\overline{\epsilon}(\overline{z})\big) dz d\overline{z}.
\ese 
We can rewrite this as 
\bse 
    d\omega d\overline{\omega} = e^{\p\epsilon(z) + \overline{\p}\overline{\epsilon}(\overline{z})}dz d\overline{z},
\ese 
where we are only considering first order terms. We conclude, therefore, that the required Weyl factor to return to flat space is 
\bse 
    \del\phi(z,\overline{z}) = - \big[\p\epsilon(z) + \overline{\p}\overline{\epsilon}(\overline{z})\big],
\ese 
where the minus sign comes from the fact that we need to undo the effect of the diffeomorphism. This in turn gives us 
\bse 
    \del_WT(z,\overline{z}) = -\frac{c}{12} \p^2\del\phi(z,\overline{z}), \qquad \text{and} \qquad \del_W\overline{T}(z,\overline{z}) = -\frac{\widetilde{c}}{12} \overline{\p}^2\del\phi(z,\overline{z}),
\ese
where we have included the antiholomorphic term too.\footnote{Note they appear as separate expressions as $\del\phi(z,\overline{z})$ is a sum of a holomorphic and an antiholomorphic term, and so the unwanted terms vanish when we take the derivatives.}

So we've found how two particular components of $T$ transform. However, what we really want is the covariant expression (i.e. how all the components transform). Its easy to convince yourself that the general expression takes the form 
\bse 
    \del_WT_{\a\beta} = a \p_{\a}\p_{\beta} \del\phi + b\nabla^2 g_{\a\beta} \del\phi,
\ese 
for some constants $a,b$. Note, as it is a covariant expression, this does not depend on the choice of basis, and so we could have $\a,\beta = \sig^1,\sig^2$ or we could have $\a,\beta=z,\overline{z}$. If we use the $z,\overline{z}$ coordinates, it follows immediately that (recall $g_{zz} = 0 = g_{\overline{z}\overline{z}}$ in the conformally flat space)
\bse 
    -\frac{c}{12} = a = -\frac{\widetilde{c}}{12}.
\ese 

This gives us our first nice result: 
\mybox{
a gravitational anomaly
\be 
    c = \widetilde{c}.
\ee 
}

Next, recall that $T$ is conserved, i.e. $\p^{\a}T_{\a\beta} = g^{\a\gamma}\p_{\gamma}T_{\a\beta} = 0$. This gives us 
\bse 
    \begin{split}
        0 & = a g^{\a\gamma}\p_{\gamma}\p_{\a}\p_{\beta} \del\phi + b g^{\a\gamma}\p_{\gamma}\nabla^2g_{\a\beta} \del\phi \\
        & = (a + b) \nabla^2 \p_{\beta}\del\phi \\
        \implies a & = -b.
    \end{split}
\ese 

\br 
We have actually cheated slightly here because we've ignored the terms that arise from the derivatives of the metric (which has coordinate dependence through $\phi(z,\overline{z})$). However, we claim, without proof, that these cancel exactly with the Christoffel symbols we didn't include before. 
\er 

Now we consider the trace. 

\bse
    \begin{split}
        \del {T^{\a}}_{\a} & = g^{\a\beta}\del T_{\a\beta} \\
        & = -\frac{c}{12} \big[ g^{\a\beta}\p_{\a}\p_{\beta} - g^{\a\beta}\nabla^2g_{\a\beta} \big]\del\phi \\
        & = - \frac{c}{12} \big[\nabla^2 - 2\nabla^2\big]\del\phi \\
        & = \frac{c}{12}\nabla^2\del\phi,
    \end{split}
\ese 
where we have used the fact that we are considering a 2D CFT and so the trace of the metric is 2.

\bcl
The Ricci scalar for our transformation is given by 
\be 
\label{eqn:RicciScalarConformal}
    R = -\nabla^2\del\phi.
\ee 
\ecl 

We shall not prove this here,\footnote{Mainly because I did it, but ended up with an overall factor of $2$ out, and I didn't want to have to recalculate all the $\Gamma$s etc.} but just use it. We therefore have 
\bse    
    \del{T^{\a}}_{\a} = - \frac{c}{12}R.
\ese

Now, we should be careful here, what we are actually considering is the expectation value, and so we write the above properly as
\mybox{
The Weyl Anomaly 
\be 
\label{eqn:WeylAnomaly}
    \langle {T^{\a}}_{\a} \rangle = -\frac{c}{12}R
\ee 
}

\br 
A different, but similar, approach to proving this result is given by Dr. Tong in section 4.4.2.
\er 

Now a few comments are in order:
\begin{itemize}
    \item For flat space we have $\langle {T^{\a}}_{\a}\rangle = 0$. This agrees with our \textit{classical} result that the stress-energy tensor is traceless. However, when we introduce curvature into the problem, this no longer holds. 
    \item This result makes sense. We argue that the result must be the same for all states in the system. This comes from the idea that we are only considering small, local deformations of the metric, and so we are considering short distances. But at short distances, all finite energy states looks more and less the same, and so we expect the result to hold for any state. If it holds for any state, the result must depend only on the background metric and not the states themselves. So we need something which is dimension 2, a scalar and depends on the background metric... 10 points to anyone who can think of such an object that is not the Ricci scalar. 
\end{itemize}

\subsection{Liouville Field Theory}

\textcolor{red}{I am not overly happy with my understanding of what Dr. Minwalla was doing here. I follow the idea, however I do not wish to butcher the explanation and lead to confusion for anyone else. However, I feel like this discussion is not vitally important for the quantised Nambu-Goto string as we require Weyl invariance, and so for that reason I am leaving it out for now. If it turns out to be important later, I shall return and fill this in, otherwise, I shall wait until I find some time to read more properly about Liouville field theory and return to fill it in here. From a quick read up, this is related to so-called non-critical strings, which do not have $d=26$. I have written this in red to remind myself to come back. The part of the video is 43:00 - 1:10:00}

\section{Verma Module}

\subsection{Imposing Unitarity}

Let us once again recall the definition for the transformation of a primary operator under a conformal transformation 
\bse 
    \phi(\omega) = \bigg(\frac{\p z}{\p \omega} \bigg)^h \phi\big(z(\omega)\big),
\ese 
where we've used $\phi$ instead of our usual $\cO$ and reversed the roles of $\omega$ and $z$.\footnote{These are just labels so of course we can do this freely.} Now lets consider the specific case for the transformation from the cylinder to the plane, which we recall is given by 
\bse 
    \omega = \sig + i\tau, \qquad z = e^{-i\omega}.
\ese 
Note this is why we changed the roles of $\omega$ and $z$ above, because we're used to calling $z$ the coordinate on the plane. Now plugging this in simply gives 
\be 
\label{eqn:PhiTransformationCylinderPlane}
    \phi(\omega) = (-i)^h z^h \phi(z).
\ee 
If we now assume again that $\phi(z)$ is holomorphic everywhere expect possibly at the origin (as we have done many times already), we can expand it as a Laurent series, 
\bse 
    \phi(z) = \sum_{n} \frac{\phi_n}{z^{n+h}},
\ese 
where we have included a factor of $h$ in the power of $z$ as it will cancel nicely in \Cref{eqn:PhiTransformationCylinderPlane}, giving 
\bse 
    \phi(\omega) = (-i)^h \sum_n z^{-n} \phi_n = (-i)^h \sum_n e^{in\omega} \phi_n.
\ese 

Now, we want a a unitary theory, as this gives us Minkowski spacetime probabilities that are conserved. However, we are working in Euclidean space (that is we have taken the Wick rotation in defining $\omega$ with the $i$ in it), so we first go back to Lorentzian space (which is where our problem really lies, and so we should impose the condition here), 
\bse 
    \phi(\omega_L) = (-i)^h \sum_n e^{in(\sig+\tau)}\phi_n.
\ese 
We then see, up to a potential minus sign, that
\be
\label{eqn:PhinDagger}
    \phi_n^{\dagger} = \phi_{-n}.
\ee 
Note if we had taken the complex conjugate in Euclidean space, we would not have got this result (due to the extra factor of $i$). However, once we have this result, we can just impose it on the Euclidean expression.

\bp 
The $\phi_n$s have energy $-n$.
\ep 

\bq 
Consider the commutator $[L_0,\phi_n]$ for the $\phi_n$ defined above. We write both in terms of their integrals over $T(z)$ and $\phi(\omega)$, respectively, and use the method we used previously. We have 
\bse 
    \begin{split}
        [L_0,\phi_n] & = \bigg(\oint \frac{dz}{2\pi i}\oint \frac{d\omega}{2\pi i} - \oint \frac{d\omega}{2\pi i}\oint \frac{dz}{2\pi i}\bigg) zT(z) \omega^{n+h-1}\phi(\omega) \\
        & = \oint\frac{d\omega}{2\pi i} \text{Res}\bigg[ z\omega^{n+h-1}\bigg( \frac{h\phi(\omega)}{(z-\omega)^2} + \frac{\p\phi(\omega)}{z-\omega} + ...\bigg)\bigg] \\
        & = \oint \frac{d\omega}{2\pi i} \big[ \omega^{n+h}\p\phi(\omega) + h\omega^{n+h-1}\phi(\omega) \big] \\
        & = \oint \frac{d\omega}{2\pi i} (-n-h +h)\omega^{n+h-1}\phi(\omega) \\
        & = -n \phi_n.
    \end{split}
\ese 
\eq 

\br 
In the proof above we showed that the commutator gave $-n\phi_n$, so you might question why that means that the energy is $-n$. That is, what we have really shown is
\bse 
    L_0\phi_n\ket{\cO} = (-n+h)\phi_n\ket{\cO},
\ese 
where the $h$ comes from the action of $L_0$ on the primary state $\ket{\cO}$. We could obviously just consider putting the identity at the origin and then $h=0$ and so you get $L_0\ket{\phi_n} = -n\ket{\phi_n}$. But this is exactly what we're doing; we're imagining placing $\phi(\omega)$ at the origin. 
\er 

Notice that what we have just done is exactly the same as the idea of expanding $T$ in terms of the $L_m$s, and so \Cref{eqn:PhinDagger} tells us
\bse 
    L_m = L_{-m}^{\dagger}.
\ese 

We can now go back and put some weight to come of the claims we've made thus far. First consider the square of the action of $L_1$ on some quasi-primary state,
\bse 
    \begin{split}
        |L_{-1} \ket{\cO}|^2 & = \bra{\cO}L_1 L_{-1}\ket{\cO} \\
        & = \bra{\cO} [L_1,L_{-1}] \ket{\cO} + \bra{\cO} L_{-1}L_1\ket{\cO} \\
        & = 2\bra{\cO}L_0\ket{\cO} \\
        &= 2h \braket{\cO}{\cO},
    \end{split}
\ese 
where we have used the fact that $L_1\ket{\cO}=0$ and $L_0\ket{\cO} = h\ket{\cO}$ for quasi-primary states. So we see that $L_{-1}$ annihilates a state if and only if the corresponding operator has $h=0$. We can do an identical calculation for $\widetilde{L}_{-1}$ and get that the condition but with $h\to \widetilde{h}$. Now, recalling that the $L_{-1}$ operator corresponds to taking a derivative, we see that having $h=0$ is the same as being an antiholomorphic operator, and similarly for $\widetilde{h}=0$. So, if an operator has weight $(0,0)$ then it doesn't depend on either $z$ nor $\overline{z}$, and so it is completely independent of where we put it. There is only one such operator... the identity. This is just the statement that we made that the only state annihilated by $L_{-1}$ and $\widetilde{L}_{-1}$ is the vacuum. 

\subsection{Verma Module}

Now, recall \Cref{rem:VermaRemark}, which says that a primary operator's energy is at a minimum. Now, if we expect to have a physical theory, we want some minimum energy bound, at which point the state is annihilated by any attempt to further lower its energy. This is just the primary states. In other words, any state in the Hilbert space can be taken to a primary state by repeated application of the $L_m$s for $m>0$. Turning this on its head, given the set of all independent primary operators, we can generate \textit{any} state in the Hilbert space by applying the $L_m$s for $m<0$. This is the statement that the representations of the Virasoro algebra can be formed by applying the $L_m$s with $m<0$ to the primary operators. This will result in an infinite tower of states,\footnote{Note this is an infinite tower for two reasons: firstly we can apply any of the $L_m$s an arbitrary amount of times, and secondly, there is no limit to the value of $m$.} with each obtained state known as a \textit{descendant} of the starting primary state, and the complete set of states known as the \textit{Verma module} for that primary state. 

If $\ket{\phi}$ is the initial primary state, whose operator has weight $h$, then the Verma module is 
\bse 
    \begin{gathered}
        \ket{\phi} \\
        L_{-1}\ket{\phi} \\
        (L_{-1})^2\ket{\phi}, \quad L_{-2}\ket{\phi} \\
        (L_{-1})^3\ket{\phi}, \quad L_{-1}L_{-2}\ket{\phi}, \quad L_{-3}\ket{\phi},
    \end{gathered}
\ese     
et cetera. In fact that representation built is an irreducible representation. So if we know the spectrum of all the independent primary states, we can find the complete spectrum of our theory.

\br 
There is a subtle point to be noted. It is not necessarily true that all of the elements within a Verma module are independent. For example,\footnote{Thanks to Dr. Tong for giving this example and sparing me the trouble of finding one.} if 
\bse 
    c = \frac{2h(5-8h)}{(2h+1)},
\ese
then the combination
\bse 
    L_{-2}\ket{\phi} - \frac{3}{2(2h+1)} (L_{-1})^2\ket{\phi}
\ese 
has vanishing norm. These combined states are known as \textit{null states}, and, as we have just seen, depend on $c$ and $h$.
\er 

\br 
We now also see a further justification for why we even introduced primary operators in the first place; they play a vital role in the representation theory. 
\er 

Note as we go down\footnote{Or up, I guess, depending on how you write it...} the Verma module the value of $h$ increases. We can call the subset with equal $h$ a \textit{level} of the Verma module. We start the level label at $N=0$ for the primary operator itself. On each level we can define a matrix, whose elements are given by the inner products of the members of the level. At level $N=1$ we only have one element given by 
\bse 
    \bra{\phi}L_1L_{-1}\ket{\phi} = 2h\braket{\phi}{\phi},
\ese 
which, from the fact that the inner product is positive semi-definite, tells us $h\geq 0$. This is something we have already seen. 

Now consider the $N=2$ level. Define 
\bse 
    \ket{1} := (L_{-1})^2\ket{\phi}, \qquad \text{and} \qquad \ket{2} := L_{-2}\ket{\phi}.
\ese 
Then our matrix elements are given by
\bse 
    \begin{split}
        \braket{2}{2} & = \bra{\phi}L_2L_{-2}\ket{\phi} \\
        & = (4h+c/2)\braket{\phi}{\phi},
    \end{split}
\ese 

\bse 
    \begin{split}
        \braket{1}{2} & = \bra{\phi} L_1L_1L_{-2}\ket{\phi} \\
        & = 3\bra{\phi} L_1 L_{-1}\ket{\phi} \\
        & = 6h \braket{\phi}{\phi} \\
        & = \braket{2}{1},
    \end{split}
\ese 
and 
\bse 
    \begin{split}
        \braket{1}{1} & = \bra{\phi} L_1L_1L_{-1}L_{-1}\ket{\phi} \\
        & = 2\bra{\phi} L_1 L_0 L_{-1}\ket{\phi} + \bra{\phi} L_1 L_{-1} L_1 L_{-1}\ket{\phi} \\
        & = 2\bra{\phi} L_1 L_{-1}\ket{\phi} + 2\bra{\phi} L_1 L_{-1} L_0\ket{\phi} + 2\bra{\phi} L_1 L_{-1} L_0\ket{\phi} \\
        & = 4h\braket{\phi}{\phi} + 8h^2\braket{\phi}{\phi} \\
        & = 4h(1 + 2h)\braket{\phi}{\phi}. 
    \end{split}
\ese 
That is, 
\bse 
    M = \begin{pmatrix}
    4h(1+2h) & 6h \\
    6h & 4h + c/2
    \end{pmatrix}.
\ese 
Then, from the requirement at the matrix by positive semi-definite (i.e. its determinant must be non-negative), along with $h\geq 0$, we have
\bse 
    \frac{1}{2}c + (c-5)h + 8h^2 \geq 0.
\ese 

This gives us further constraints to the allowed values of $c$. Simply solving the quadratic for $h$ and imposing $h\geq 0$ you get 
\bse 
     c \leq 1, \qquad \text{or} \qquad c\geq 9.
\ese
However, we also have 
\bse 
    |L_{-n}\ket{\phi}|^2 = \bigg(2nh + \frac{c}{12}n(n^2-1)\bigg) \braket{\phi}{\phi} \geq 0.
\ese
Now if $n=1$ the $c$ term vanishes, so let's just consider $n>1$. Solving this simply gives 
\bse 
    c \geq \frac{24h}{n^2-1},
\ese 
which, along with $h\geq0$ tells us that $c\geq0$. So we finally obtain: the allowed values of the central charge for a unitary theory are
\be 
\label{eqn:cValuesUnitary}
    0 \leq c \leq 1 \qquad \text{or} \qquad c \geq 9.
\ee 
\chapter{The Free  Bosonic Scalar Field}

Although we have obviously not given a complete discussion of 2D CFT, however, we have now discussed enough to be able to get back on track and start looking at its applications in string theory. This lecture is going to look at the free bosonic scalar field, that is the $x^{\mu}$s from the start of the course. 

\br 
It appears as though we have developed a set of pretty powerful results in the last few lectures, however it turns out that these tools are not enough to be able to classify\footnote{In the sense that we can classify all continuous symmetries, as they map to Lie Groups and we have a classification of all possible Lie Groups.} all 2D CFTs. This is still an open area of research. 
\er 

\br 
\textcolor{red}{I am going to start using capital $X$s for the scalar fields. I used lower case ones at the beginning out of habit, but have now noticed that it is common notation to use capitals. I've written this remark in red to remind me to later go back and change the lower cases to capitals.}
\er 

\bcl 
If we take the  mode expansion solutions we obtained towards the start of the course, transform them to the plane, and go to Euclidean coordinates we obtain 
\be 
\label{eqn:ModeExpansionsPlane}
    \p X^{\mu} = -i \sqrt{\frac{\a'}{2}}\sum_{m=-\infty}^{\infty} \frac{\a_m^{\mu}}{z^{m+1}}, \qquad \text{and} \qquad \overline{\p} X^{\mu} = -i \sqrt{\frac{\a'}{2}}\sum_{m=-\infty}^{\infty} \frac{\widetilde{\a}_m^{\mu}}{z^{m+1}},
\ee 
with 
\be 
    \a_0 := i \sqrt{\frac{\a'}{2}}p. 
\ee 
\ecl 

\bq 
The mode expansions on the \textit{Euclidean} cylinder, whose coordinates we label by $(\omega,\overline{\omega})$, are (dropping the $\mu$ index for now)
\bse 
    X(\omega,\overline{\omega}) = x_0 + \a'p\tau + i\sqrt{\frac{\a'}{2}} \sum_{m\neq0} \frac{1}{n}\big(\a_m e^{im\omega} + \widetilde{\a}_m e^{im\overline{\omega}}\big).
\ese 
Now take the derivative w.r.t. $\omega$, 
\bse 
    \p_{\omega} X(\omega) = -\sqrt{\frac{\a'}{2}}\sum_{m=-\infty}^{\infty} \a_m e^{im\omega}, \qquad \a_0 := i\sqrt{\frac{\a'}{2}}p.
\ese 
Finally take the transformation $z = e^{-i\omega}$ and use the fact that the derivative of $X$ is a primary operator of weight $h=1$ to give 
\bse 
    \p X(z) = \big(\p_{\omega}z\big)^{-1} \p_{\omega}X(\omega) = -i\sqrt{\frac{\a'}{2}} \sum_{m=-\infty}^{\infty} \frac{\a_m}{z^{m+1}}.
\ese 
The same method holds for the $\p_{\overline{\omega}}$ derivative.
\eq 

\br 
As we just did in the proof, we shall now drop the $\mu$ index, and simply consider a single scalar field. Obviously we do this for mere notational brevity. 
\er 

Now, we want to reproduce \Cref{eqn:AlphaCommutationRelations} on the plane. We invert the expression above to give 
\be 
\label{eqn:AlphaContourIntegral}
    \a_m = i \sqrt{\frac{2}{\a'}} \oint \frac{dz}{2\pi i } z^m \p X(z).
\ee 

\br 
Note that in doing this, we see that the $\a_m$s are charges and the $\p X$s are currents, analogously to the $L_m$s and $T$. We also note that $\p X$ is a holomorphic function, and so is arbitrary $z$ derivatives of it, as well as multiplication by other holomorphic functions. We see, therefore, that for the free theory we have a huge set of a conserved currents. 
\er 

Then we just proceed as always:\footnote{Note here $\omega$ is now once again a coordinate on the plane, not the cylinder as we did in the proof above.}
\bse 
    [\a_m,\a_n]  = -\frac{2}{\a'} \bigg(\oint\frac{dz}{2\pi i} \oint \frac{d\omega}{2\pi i} - \oint\frac{d\omega}{2\pi i} \oint \frac{dz}{2\pi i}\bigg) z^mz^{n}\p X(z) \p X(\omega),
\ese 
but now we run into a problem: we don't yet know the OPE for $\p X(z)\p X(\omega)$. So we need to find this. 

Start by recalling the Polyakov action\footnote{Note we loose the overall minus sign as we are in Euclidean space.}
\bse 
    S = \frac{1}{4\pi\a'}\int d^2\sig \p_{\a}X\p^{\a}X,
\ese 
where we have used the fact that the plane has a flat metric (so $\sqrt{-g} =1$). So our partition function is 
\bse 
    Z = \int DX e^{-S} = \int DX \exp\bigg(-\frac{1}{4\pi\a'}\int d^2\sig (\p X)^2\bigg).
\ese
Now we use the fact that the path integral of a total derivative vanishes (in the same way that the regular integral of a total derivative does), and consider the following
\bse 
    \begin{split}
        0 & = \int DX \frac{\del}{\del X(\sig)}\big( X(\sig') e^{-S}\big) \\
        & = \int DX e^{-S} \big[\del(\sig -\sig') +\frac{1}{2\pi\a'}\p^2X(\sig) X(\sig') \big],
    \end{split}
\ese 
where to get to the second line we have done integration by parts in the exponential before taking the derivative. If we then divide by the partition function, we get 
\bse 
    \langle \p^2 X(\sig) X(\sig') \rangle = -2\pi\a'\del(\sig-\sig').
\ese 
Then, as we have done several times already, we move the derivatives outside the correlation brackets and solve this as a differential equation for the propagator $ G(\sig,\sig') = \langle X(\sig)X(\sig')\rangle$. 

The solution to this differential equation comes from an application of Stokes' theorem. Let $\sig = \sig_1 + i\sig_2$, and consider the integral
\bse 
    \begin{split}
        \int d^2\sig \p_2 \ln\sig^2 & = \int d^2 \sig \p_2 \ln (\sig_1^2 +\sig_2^2) \\
        & = \int d^2\sig \p^{\a} \bigg( \frac{2\sig_{\a}}{\sig_1^2 +\sig_2^2}\bigg) \\
        & = 2 \oint \frac{\sig_1d\sig^2 - \sig_2d\sig^1}{\sig_1^2 +\sig_2^2} \\
        & = 2\int \frac{r^2d\theta}{r^2} \\
        & = 4\pi = 4\pi\int d\sig \del(\sig),
    \end{split}
\ese
where we've changed to polar coordinates $\sig_1+\sig_2 = re^{i\theta}$. The final result is proportional to the right-hand side of our differential equation with $\sig'=0$. We can reinsert this and conclude 
\be 
\label{eqn:ExpectationXX}
    \langle X(\sig)X(\sig') \rangle = -\frac{\a'}{2}\ln(\sig-\sig')^2 = -\frac{\a'}{2}\ln z\overline{z}.
\ee 

\section{Conformal Normal Ordering}

So we want to find the OPE $\p X(z) \p X(\omega)$. We will see that in order to do this, we want to take the limit $z\to\omega$. However, we first need to note something important. Classically, if you took the expectation value 
\bse 
    \langle \p X(z) \p X(0) \rangle, 
\ese 
as $z\to 0$, you would just get $\langle (\p X(0))^2\rangle$, but in our theory this blows up. So what we need to do is define some well defined insertion into the path integral that corresponds to this case. For the free scalar field case, we have a very nice way of doing this.

\mybox{
\bd[Conformal Normal Ordering]
We define the \textit{conformal normal order} of two fields as
\be 
\label{eqn:ConformalNormalOrdering}
    \cl \p X \p X\cl := \lim_{z\to\omega}\big[ \p X(z) \p X(\omega) - \langle \p X(z) \p X(\omega) \rangle\big].
\ee 
\ed
}

Note that, by construction, the expectation value of the conformal normal order vanishes. This might hint to what's to come.\footnote{For those that want to know now... it's the stress-energy tensor!} 

\br 
It is important to understand that it is only the expectation value that vanishes, and not arbitrary correlation functions. If the latter was the case, we would have basically define a null insertion for our path integral, which would be basically no use whatsoever. In fact it's true that any operator of definite scaling dimension has vanishing expectation value. 
\er 

\section{The Commutation Relation}

Firstly we note that 
\bse 
    \langle \p X(z) \p X(\omega) \rangle = - \frac{\a'}{2(z-\omega)^2},
\ese
which will be the most singular contribution to our OPE. Then we note that 
\bse 
    \cl \p X(z) \p X(\omega) \cl = \cl \sum_n \frac{(z-\omega)^n}{n!} \p^{(n+1)} X(\omega) \p X(\omega) \cl, 
\ese 
which has no poles and so does not contribute to the residue. So we obtain 
\be 
\label{eqn:pXpXOPE}
    \p X(z) \p X(\omega) = - \frac{\a'}{2(z-\omega)^2} + \text{non-singular}.
\ee 

We can now plug this into our commutation relation, giving us 
\bse 
    \begin{split}
        [\a_m,\a_n] & = - \frac{2}{\a'} \bigg(\oint\frac{dz}{2\pi i}\oint\frac{d\omega}{2\pi i} - \oint\frac{d\omega}{2\pi i}\oint\frac{dz}{2\pi i}\bigg) z^m\omega^n \bigg( -\frac{\a'}{2(z-\omega)^2} + ... \bigg) \\
        & = -\frac{2}{\a'} \oint \frac{d\omega}{2\pi i} \text{Res} \bigg[ z^m \omega^n \bigg(-\frac{\a'}{2(z-\omega)^2} + ... \bigg)\bigg] \\
        & = m \oint \frac{d\omega}{2\pi i} \omega^{n+m-1} \\
        & = m \del_{n+m,0},
    \end{split}
\ese 
which is exactly the result we want. 

\section{The Stress-Energy Tensor}

Recall \Cref{eqn:StressEnergyTensor},
\bse 
    T_{\a\beta} := -\frac{4\pi}{\sqrt{g}} \frac{\p S}{\p g^{\a\beta}}.
\ese 
We can work this out for our free theory with action
\bse 
    S = \frac{1}{4\pi\a'}\int d^2\sig \p_{\a}X\p^{\a}X,
\ese
simply giving 
\bse 
    T_{\a\beta} = -\frac{1}{\a'} \Big( \p_{\a}X\p_{\beta}X - \frac{1}{2}g_{\a\beta}(\p X)^2\Big).
\ese 
Then using the fact that the metric in complex coordinates is $g_{zz}=0=g_{\overline{z}\overline{z}}$ and $g_{z\overline{z}} = 2$, we have 
\bse 
    T = -\frac{1}{\a'} \p X\p X, \qquad \overline{T} = - \frac{1}{\a'} \overline{\p}X\overline{\p}X, \qquad \text{and} \qquad T_{z\overline{z}} = 0.
\ese 
This is the classical result, we want the QM result. We might just be tempted to say `it's of the same form', however, there's a problem with that: recall our equation of motion required that the expectation value of the trace of the stress tensor vanish. Luckily, we've just discussed how to fix this problem: conformal normal ordering. So our stress-energy tensor is 
\be 
    T = -\frac{1}{\a'} \cl\p X\p X\cl, \qquad \text{and} \qquad \overline{T} = - \frac{1}{\a'} \cl\overline{\p}X\overline{\p}X\cl.
\ee 

\section{OPEs With $T$}

We are now going to look a few OPEs using $T$. In order to do these, we have to invoke Wick's theorem\footnote{A discussion of Wick's theorem is given at the start of the next lecture. See \Cref{rem:WicksTheorem} for why it's there and not here.}. It is likely that when you were first introduced to Wick's theorem, it was in the context of time ordered products and creation/annihilation operators. It might seem strange to use it here, but it turns out that it still holds here. The contractions here correspond to the propagator, i.e. 
\bse 
    \overbrace{\p X(z)\p X(\omega)} = -\frac{\a'}{2(z-\omega)^2}. 
\ese 

\subsection{$T(z)\p X(\omega)$}

The first one worth doing is $T(z)\p X(\omega)$, as this will further confirm that $\p X(\omega)$ is indeed a primary operator of weight $(1,0)$. 
\bse 
    \begin{split}
        T(z)\p X(\omega) & = -\frac{1}{\a'}\cl\p X(z)\p X(z)\cl \p X(\omega) \\
        & = -\frac{1}{\a'} \Big( 2 \p X(z)\overbrace{\p X(z)\p X(\omega)} + \cl \p X(z)\p X(z)\p X(\omega)\cl \Big) \\ 
        & = \frac{\p X(z)}{(z-\omega)^2} + ... \\
        & = \frac{\p X(\omega)}{(z-\omega)^2} + \frac{\p\p X(\omega)}{(z-\omega)} + ...,
    \end{split}
\ese 
which is the transformation of a primary operator of holomorphic weight $h=1$. If you now repeated the process with $\overline{T}(\overline{z})$ along with the fact that 
\bse 
    \overbrace{\overline{\p}X(\overline{z})\p X(\omega)} = 0,
\ese 
you get $\widetilde{h}=0$.

\subsection{$T(z)T(\omega)$}

Now let's consider the OPE of $T$ with itself. 

\bse 
    \begin{split}
        T(z)T(\omega) & = \frac{1}{(\a')^2}\cl \p X(z)\p X(z)\cl \cl \p X(\omega) \p X(\omega) \cl \\
        & = \frac{1}{(\a')^2}\bigg( 2\overbrace{\p X(z)\p X(\omega)}\overbrace{\p X(z)\p X(\omega)} + 4\cl \p X(z) \p X(\omega)\cl  \overbrace{\p X(z)\p X(\omega)}  + ...  \bigg) \\
        & = \frac{1}{(\a')^2} \bigg( \frac{2(\a')^2}{4(z-\omega)^4} - \frac{4\a'\cl \p X(z)\p X(\omega)\cl}{2(z-\omega)^2} + ... \bigg) \\
        & = \frac{1/2}{(z-\omega)^4} - \frac{2\cl \p X(\omega)\p X(\omega)\cl }{\a'(z-\omega)^2} -\frac{2\p\cl\p X(\omega)\p X(\omega) \cl}{\a'(z-\omega)} + ...\\
        & = \frac{1/2}{(z-\omega)^4} + \frac{2T(\omega)}{(z-\omega)^2} - \frac{2}{\a'}\frac{\p T(\omega)}{z-\omega} + ...,
    \end{split}
\ese 
where the $...$ on the first two lines is the total conformally ordered term 
\bse 
    \cl \p X(z)\p X(z) \p X(\omega) \p X(\omega)\cl,
\ese 
and then on the final two lines it includes some other non-singular terms from the Taylor expansion. Comparing this result to \Cref{eqn:TTOPE}, we see straight away that for a free field theory $c=1$. A similar argument will give $\widetilde{c}=1$.

\section{States of the System}

We can use the state-operator map to start working out what the states in the theory look like. 

\subsection{The Vacuum}
We can now show (more convincingly, perhaps, then the arguments we've given previously) that the state produced by inserting the identity at the origin is the vacuum. We recall that, provided there are no contact terms, $\overline{\p} \p X(z) = 0$, which tells us that $\p X(z)$ is holomorphic everywhere. So we know from \Cref{eqn:AlphaContourIntegral} that when we insert the identity at the origin that $\a_m\ket{\b1} =0$ for all $m>0$. Putting this together with the fact that we've just shown that we meet our commutation relations from the start of the course, and with \Cref{eqn:CreationAnnihilationOperatos}, we see that this is the statement at the state $\ket{\b1}$ is annihilated by \textit{all} annihilation operators. This is just the definition of the vacuum. 

\subsection{$\ket{\p X}$}

Let's now consider acting the annihilation operators on the state produced by inserting the operator $\p X$ at the origin. 

\bse 
    \begin{split}
        \a_m \ket{\p X} & = i \sqrt{\frac{2}{\a'}} \oint \frac{dz}{2\pi i} z^m \p X(z) \p X(0) \\
        & = i\sqrt{\frac{2}{\a'}} \oint \frac{dz}{2\pi i} z^m \bigg( - \frac{\a'}{2z^2} + \cl \p X(z) \p X(0)\cl \bigg) \\
        & = i \sqrt{\frac{2}{\a'}} \text{Res}\bigg[ z^m\bigg( - \frac{\a'}{2z^2} + \cl \p X(z) \p X(0)\cl \bigg)\bigg] \\
        & = i \sqrt{\frac{2}{\a'}} \text{Res}\bigg[-\frac{\a'}{2}z^{m-2} + z^m\cl \p X(z) \p X(0)\cl\bigg].
    \end{split}
\ese

Now, if $m<0$ we are considering creation operators, and this won't tell us much about how to build the state $\ket{\p X}$ from the vacuum. Therefore, we shall just consider $m>0$, and so the conformally normal ordered term doesn't contribute to the residue and we have
\be 
\label{eqn:AlphaOnpX}
    \a_1\ket{\p X} = -i\sqrt{\frac{\a'}{2}}\ket{\b1}, \qquad \text{and} \qquad  \a_m\ket{\p X} = 0 \quad \forall m\geq 2.
\ee 
It follows from this, then, that 
\bse 
    \ket{\p X} = -i\sqrt{\frac{\a'}{2}}\a_{-1}\ket{\b1}.
\ese 

We can actually generalise the above to the following claim. 

\bcl 
Let $m>0$ then, 
\be 
    \ket{\p^mX} = -i(m!) \sqrt{\frac{\a'}{2}} \a_{-m}\ket{\b1}.
\ee 
\ecl 

\bq 
Proceed as above 
\bse 
    \begin{split}
        \a_n \ket{\p^mX} & = i\sqrt{\frac{2}{\a'}} \oint \frac{dz}{2\pi i} z^n \p X(z) \p^m X(0) \\
        & = i\sqrt{\frac{2}{\a'}} \text{Res}\bigg[ z^n\p^{m-1} \bigg(\frac{-\a'}{2z^2}\bigg) + z^n\cl \p X(z)\p^mX(0)\cl \bigg] \\
        & = i\sqrt{\frac{2}{\a'}} \text{Res}\bigg[ \frac{-\a'}{2} z^n (m!) z^{-(m+1)} + z^n\cl \p X(z)\p^mX(0)\cl \bigg] \\
        & = -i \sqrt{\frac{\a'}{2}} (m!) \del_{m,n} \ket{\b1},
    \end{split}
\ese 
where again we've used the fact that we're only interested in $n>0$. Then inverting this gives the claim.
\eq 

\bcl 
Let $m_1,m_2,...m_N >0$, then 
\be 
    \ket{\cl\p^{m_1}X \p^{m_2}X ... \p^{m_N}X\cl} = A \a_{-m_1}\a_{-m_2}...\a_{-m_N}\ket{\b1},
\ee 
where $A$ is a constant determined by the values of $m_1,...,m_N$.\footnote{I tried to find a nice compact expression for it, but it wasn't so easy when you consider that some of the $m_i$s might be equal. If someone reading this can find a nice compact expression please feel free to email me it and I'll add it and dish out some credit.}
\ecl 

\br 
Note that this last claim actually covers the fact that we only considered $m>0$ above. That is, it tells us how creation operators act on the states $\ket{\p^mX}$.
\er 

So we have a prescription for creating our entire Fock space for the free theory, \textit{except} for the zero mode contributions. We will now go on to discuss these, but first note that in this non-zero mode part of the theory the only primary operators are the identity and $\p X$, and so every state is a descendant of one of these two. 

\section{The Free Particle States}

We already have some idea of what we expect the zero mode part to give us. Recall from our discussion at towards the start of the course, that the Hilbert space for our theory is given by the product of the harmonic oscillator space and a free particle state, which we can label by the momentum of the particle. So far we have only considered particles with zero momenta. How do we know this? Well, because we've only seen the identity (which obviously has no momentum) and states given by \textit{derivatives} of the scalar fields, but we expect to pick up a momentum term when shifting $X$. The derivative of a constant shift vanishes, and so we must have zero momentum. 

The natural OPE to consider is with the operator given by $e^{ikX(\omega)}$. Again, we need to take the conformal normal order. 

\bse 
    \begin{split}
        T(z) \cl e^{ikX(\omega)}\cl & = -\frac{1}{\a'} \cl \p X(z)\p X(z) \cl \cl e^{ikX(\omega)}\cl \\
        & = -\frac{1}{\a'} \cl \p X(z) \p X(z) \cl \cl \sum_{n=0}^{\infty}  \frac{(ik)^n}{n!} X^n(\omega)\cl.
    \end{split}
\ese
The question is, what is the this contraction? Well consider first the OPE 
\bse
    \begin{split}
        \p X(z) \cl \sum_{n=0}^{\infty}  \frac{(ik)^n}{n!} X^n(\omega)\cl & = \sum_{n=0}^{\infty} \frac{(ik)^n}{n!} n \cl X^{n-1}(\omega) \cl \bigg( -\frac{\a'}{2} \frac{1}{z-\omega}\bigg) + ...\\
        & = -\frac{ik\a'}{2}\frac{\cl e^{ikX(\omega)\cl}}{z-\omega} + ...,
    \end{split}
\ese 
and so we have 
\bse 
    \begin{split}
        T(z)\cl e^{ikX(\omega)}\cl & = \frac{\a' k^2}{4}\frac{\cl e^{ikX(\omega)}\cl }{(z-\omega)^2} + ik\frac{\cl \p X(z) e^{ikX(\omega)}\cl}{z-\omega} + ... \\
        & =  \frac{\a' k^2}{4}\frac{\cl e^{ikX(\omega)}\cl }{(z-\omega)^2} + \frac{\p_{\omega} \cl e^{ikX(\omega)}\cl}{z-\omega} + ...,
    \end{split}
\ese 
where to go to the last line, we have Taylor expanded the derivative term and then had it act on the conformal normal ordered exponential. We see, therefore, that $\cl e^{ikX(\omega)}\cl$ is a new primary operator of holomorphic weight 
\be 
\label{eqn:WeighteikX}
    h = \frac{\a'k^2}{4}.
\ee 

Show $\a_m\ket{\cl e^{ikX}\cl} = 0$ for all $m>0$. 

So we see that, in the harmonic oscillator part of the Hilbert space, this state is just the identity. However, we see that it has momentum $k$ in the momentum part as if we shift $X \to X + a$ then we get pick up a $e^{ika}$ term.

\bcl
The state $\ket{\cl e^{ikX}\cl}$ is purely free particle, i.e. it has identity in the harmonic oscillator sector:
\be 
\label{eqn:eikXNoHarmonic}
    \a_m\ket{\cl e^{ikX}\cl} = 0 \qquad \forall m>0.
\ee
\ecl 

\bq 
By direct calculation.
\bse 
    \begin{split}
        \a_m\ket{\cl e^{ikX}\cl} & = i \sqrt{\frac{2}{\a'}} \oint \frac{dz}{2\pi i} z^m \p X(z) \cl \sum_{n=0}^{\infty} \frac{(ik)^n}{n!} X^n(0)\cl \\
        & = i\sqrt{\frac{2}{\a'}} \oint \frac{dz}{2\pi i} z^m \bigg( - \frac{ik\a'}{2} \frac{\cl e^{ikX}\cl}{z} + ... \bigg) \\
        & = \sqrt{\frac{\a'}{2}}k \text{Res} \bigg[ z^m \frac{\cl e^{ikX}\cl}{z}\bigg] \\
        & = 0 \qquad \forall m>0.
    \end{split}
\ese 
\eq 

So we can now built up our entire Hilbert space by taking the product of the Fock space made with the creation operators acting on the vacuum and the state $\ket{\p X}$ and the momentum state $\ket{\cl e^{ikX}\cl}$.
\chapter{Wick's Theorem \& Scattering Amplitudes}

\section{Wick's Theorem}

In order to prevent potential confusion from the applications of Wick's theorem in the previous lecture when discussing conformal normal ordering, we shall include a section here explaining in more detail what Wick's theorem is. This section is not meant as an introduction to Wick's theorem, but as an explanation of where it comes from. It is therefore assumed that you are familiar with/have used Wick's theorem before in a QFT course, for example. 

\br
\label{rem:WicksTheorem}
To readers, it might seem strange to include this section \textit{after} using Wick's theorem instead of before. The reason I have written it this way, is because this is the order it was taught. In fact this discussion of Wick's theorem appears to have been an inclusion on Dr. Minwalla's behalf due to the previous lecture's content. 
\er 

Recall that, given the Gaussian integral 
\bse 
    I = \int d^nx \exp \bigg[ -\frac{\a_{ij}x^ix^j}{2} \bigg],
\ese
for some real, symmetric $\a_{ij}$, that the two point function is
\be 
\label{eqn:GaussianTwoPoint}
    \langle x^i x^i \rangle = G^{ij} := (\a^{-1})^{ij}.
\ee 
We also have the following relation for Gaussian integrals\footnote{The denominator is just included to normalise the result. It corresponds to removing the prefactor $\sqrt{\frac{(2\pi)^n}{\det{A}}}$.} 
\bse 
    \frac{\int d^nx \exp\bigg[-\frac{1}{2}\a_{ij}x^ix^j\bigg]F(x)}{\int d^nx \exp\bigg[-\frac{1}{2}\a_{ij}x^ix^j\bigg]} = \exp\bigg[\frac{1}{2}G^{ij}\frac{\p}{\p x^i}\frac{\p}{\p x^j}\bigg] F(x)\bigg|_{x=0}.
\ese
Now let's take the case $F(x) = x^a x^b$ for fixed $a,b$. Then, Taylor expanding the right-hand side, we see that the only contributing term is the first order term (note that we set $x=0$ and so the identiy piece vanishes). That is, 
\bse 
    \exp\bigg[\frac{1}{2}G^{ij}\frac{\p}{\p x^i}\frac{\p}{\p x^j}\bigg] x^ax^b\bigg|_{x=0} = G^{ab},
\ese 
where we have used the fact that $G^{ab}$ is symmetric to remove the factor of $1/2$. Now imagine we have $F(x) = x^ax^bx^cx^d$. We now have higher order contributions:
\bse 
    \begin{split}
        \exp\bigg[\frac{1}{2}G^{ij}\frac{\p}{\p x^i}\frac{\p}{\p x^j}\bigg] x^ax^bx^cx^d\bigg|_{x=0} & = G^{ab} x^c x^d + G^{ac}x^bx^d  + ... + G^{ab}G^{cd} + G^{ac}G^{bd} + ... \\
        & = \langle x^a x^b\rangle x^cx^d + \langle x^a x^c\rangle x^bx^d +  ... \langle x^ax^b\rangle \langle x^cx^d\rangle +  \langle x^ax^c\rangle \langle x^bx^d\rangle + ...,
    \end{split}
\ese 
where the ellipses indicate similar terms, and where we have used \Cref{eqn:GaussianTwoPoint} to get to the final line. 

This is just the content of Wick's theorem as we know it: take the function and cross contract in all possible ways and leave anything left over outside. It is left as an exercise to check that when $a=b$ (or similarly for the other indices) that the factors we get from Wick's theorem also arise. 

Actually we have ignored an important part here: we haven't set $x=0$ on the RHS, and so the terms that aren't completely contracted vanish. This is a problem, as these terms do not vanish when we consider normal ordering something. We shall now fix this point. 

\subsection{Normal Ordering}

We now want to look at normal ordering. We define 
\be 
\label{eqn:NormalOrdering}
    F(x) = \cl \exp\bigg[\frac{1}{2}G^{ij}\frac{\p}{\p x^i}\frac{\p}{\p x^j}\bigg] F(x)\cl,
\ee 
where we no longer require the evaluation at $x=0$. 

What we have with \Cref{eqn:NormalOrdering} is essentially a map from functions to functions, which we see clearly from the discussion in the previous section: if $F(x)$ is a quadratic function, then, as we are not setting $x=0$ anymore, we just get the normal ordering of $F(x)$ and some constant, $G^{ab}$. We therefore can ask the question of whether we can invert \Cref{eqn:NormalOrdering} to get an expression for $\cl F(x)\cl$. The answer is as we might expect: simply multiply $F(x)$ by the inverse exponential to get 
\be 
\label{eqn:NormalOrderF}
    \cl F(x) \cl = \exp \bigg[ - \frac{1}{2}G^{ij}\frac{\p}{\p x^i}\frac{\p}{\p x^j}\bigg] F(x).
\ee 

\subsubsection*{Multiple Functions}

The next thing we need to consider is the product of two normal ordered functions, $\cl F(x)\cl \cl G(x)\cl$. As we know, we want to be able to write this as the normal order of the product, $\cl F(x)G(x)\cl$. The question is: how do we get this? 

Consider the direct manipulation
\bse 
    \begin{split}
        \cl F(x) \cl \cl G(x) \cl & = \exp \bigg[ - \frac{1}{2}G^{ij}\frac{\p}{\p x^i}\frac{\p}{\p x^j}\bigg] F(x) \exp \bigg[ - \frac{1}{2}G^{ij}\frac{\p}{\p y^i}\frac{\p}{\p y^j}\bigg] G(y) \bigg|_{x=y} \\
        & = \exp\bigg[ -\frac{1}{2}G^{ij} \bigg( \frac{\p}{\p x^i} + \frac{\p}{\p y^i}\bigg) \bigg( \frac{\p}{\p x^j} + \frac{\p}{\p y^j}\bigg)\bigg] \exp\bigg[-G^{ij}\frac{\p}{\p x^i}\frac{\p}{\p y^j}\bigg]F(x)G(y)\bigg|_{x=y} \\
        &= \cl \exp\bigg[-G^{ij}\frac{\p}{\p x^i}\frac{\p}{\p y^j}\bigg]F(x)G(y)\cl \bigg|_{x=y},
    \end{split}
\ese
where we have used \Cref{eqn:NormalOrderF} and several properties of the terms (e.g. the fact that $G^{ij}$ is symmetric). This result matches what we do when we consider the product of normal ordered functions, we cross contract (which we see from the fact that we have one $x$ and one $y$ derivative) and then normal order what ever is left over. 

\subsubsection*{An Example}

As it has been written, we are considering discrete points in spacetime (we are using indices $i,j=0,...,d$). We will want to rewrite this in the context of field theory, in which case we simply replace $x^i \to X(\sig)$ and get 
\bse 
    F(X) = \cl \exp\bigg[ \int d\sig d\sig' \frac{1}{2} G(\sig-\sig') \frac{\del}{\del X(\sig)} \frac{\del}{\del X(\sig')}\bigg] F(X) \cl.
\ese 

Let's now consider the example 
\bse 
    \cl e^{ik_1X(\sig_1)}\cl \cl e^{ik_2X(\sig_2)}\cl = \cl \exp \bigg[ -\frac{\a'}{2} \int d\sig_1 d\sig_2 \ln (\sig_1-\sig_2)^2 \frac{\del}{\del X(\sig_1)} \frac{\del}{\del X(\sig_2)} \bigg] e^{ik_1X(\sig_1)}e^{ik_2X(\sig_2)} \cl,
\ese
where we have used \Cref{eqn:ExpectationXX}. Now, using the fact that the derivatives will only `click' when we get the $X(\sig_1)/X(\sig_2)$ in the exponentials, we have 
\bse 
    \cl e^{ik_1X(\sig_1)}\cl \cl e^{ik_2X(\sig_2)}\cl = \exp\bigg[\frac{\a'}{2} k_1k_2 \ln\big[ (z-z_2)(\Bar{z}_1-\Bar{z}_2)\big]\bigg] \cl e^{ik_1X(\sig_1)} e^{ik_2X(\sig_2)} \cl,
\ese 
where we have switched to the $z\Bar{z}$ notation in the log. Simplifying, we have 
\bse 
    \cl e^{ik_1X(\sig_1)}\cl \cl e^{ik_2X(\sig_2)}\cl = \big[(z_1-z_2)(\Bar{z}_1-\Bar{z_2})\big]^{\a'k_1k_2/2} \cl e^{ik_1X(\sig_1)} e^{ik_2X(\sig_2)} \cl.
\ese 
Now consider expanding $X(\sig_1)$ about $X(\sig_2)$, then to first order we have 
\bse 
    \cl e^{ik_1X(\sig_1)} \cl \cl e^{ik_2X(\sig_2)} \cl = \big[(z_1-z_2)(\Bar{z}_1-\Bar{z_2})\big]^{\a'k_1k_2/2} \cl e^{i(k_1+k_2)X(\sig_1)} \cl,
\ese 
which is consistent with what we have discussed previously. To see why, recall that for a general OPE of two holomorphic functions of weight $h_1$, $h_2$, we have 
\bse 
    \cO_1 \cO_2 = \frac{\cO_3}{(z_1-z_2)^{h_1+h_2-h_3}}.
\ese 
Then also recall that $h=\a'k^2/4$ for the exponentials and so, using $k_3=k_1k_2$, the fraction becomes 
\bse 
    \frac{1}{(z_1-z_2)^{\frac{\a'}{4} \big[k_1^2 + k_2^2 - (k_1k_2)^2\big]}} = (z_1-z_2)^{\a'k_1k_2/2},
\ese 
and where the $\Bar{z}$ term is achieved similarly. 

\br 
In the above discussion, we have assumed there is only one $X$. However, it is easy to convince yourself that the above extends to higher dimensions by simply considering the dot products everywhere. That is $k_1X(\sig_1) \to k_1\cdot X(\sig_1)$ and $k_1 k_2\to k_1\cdot k_2$.
\er 

\section{Scattering Amplitudes}

We have now studied enough 2D CFT to move on, and go back to considering our strings. The main thing we want to consider in our 2D string theory is \textit{scattering amplitudes}. In order to highlight some important points in our theory, let's first recall what we do with Feynman diagrams for particles. 

We consider the free propagation of the two interacting particles until they meet, considering all possible paths taken by these particles (i.e. we take the path integrals over them). We then consider the state after the interaction (say they become one particle), and we consider the free propagation of that. If this new particle then, say, decays into two other particles, we then just consider the free propagation of those particles. So everything is free propagation, apart from the points where the particles meet. At these points, we have something that appears highly singular on the worldline. We then develop a set of rules to tell us what is going on at these points, that is we develop the Feynman rules for those interaction vertices.

\begin{center}
    \btik 
        \draw[ thick, decoration={markings, mark=at position 0.5 with {\arrow{>}}}, postaction={decorate}] (-1,-1) -- (0,0);
        \draw[ thick, decoration={markings, mark=at position 0.5 with {\arrow{<}}}, postaction={decorate}] (1,-1) -- (0,0);
        \draw[ thick, decoration={markings, mark=at position 0.5 with {\arrow{>}}}, postaction={decorate}] (0,0) -- (0,1);
        \draw[ thick, decoration={markings, mark=at position 0.5 with {\arrow{>}}}, postaction={decorate}] (0,1) -- (-1,2);
        \draw[ thick, decoration={markings, mark=at position 0.5 with {\arrow{<}}}, postaction={decorate}] (0,1) -- (1,2);
        \draw[->] (-1,0.5) -- (-0.1,0.1);
        \draw[->] (-1,0.5) -- (-0.1,0.9);
        \node at (-1.8,0.5) {\large{non-free}};
    \etik 
\end{center}

The main point to understand above is that understanding the free propagation of a particle is not enough to be able to understand the scattering behaviour of two particles. We might think that the same thing is true for our 2D string theory. However, it amazingly turns out that this is not the case, and it is indeed enough to know about the free propagation of the strings. We can motivated why this is true by considering the diagram for our string scattering.

\begin{center}
    \btik 
        \draw[thick] (-2,-2) .. controls (0,0) .. (2,-2);
        \draw[thick, rotate around={90:(0,0.5)}] (-2,-2) .. controls (0,0) .. (2,-2);
        \draw[thick] (-2,3) .. controls (0,1) .. (2,3);
        \draw[thick, rotate around={90:(0,0.5)}] (-2,3) .. controls (0,1) .. (2,3);
        \draw[thick, rotate around={-45:(-2.25,-1.75)}] (-2.25,-1.75) ellipse (0.35 and 0.1);
        \draw[thick, rotate around={45:(-2.25,2.75)}] (-2.25,2.75) ellipse (0.35 and 0.1);
        \draw[thick, rotate around={45:(2.25,-1.75)}] (2.25,-1.75) ellipse (0.35 and 0.1);
        \draw[thick, rotate around={-45:(2.25,2.75)}] (2.25,2.75) ellipse (0.35 and 0.1);
    \etik 
\end{center}

There is no point on this surface that you can point to and say `this point is special' or `this is where the interaction happens'. To put it another way, no matter where you point, cutting the surface at that point will always give a loop, and so can be thought of as a snapshot in the propagation of a \textit{free} string. Note in order to say this, its important that our worldsheet theory is local.

Now we might say `well I could work out where the interaction happens by taking equally spaced slices of the worldsheet and waiting for the point where we go from two tubes to one.' The problem with this is that the position of this transition depends on how you choose to do the slicings. That is, slicing horizontally will give a different result to doing it vertically. 

\br
This actually highlights another, unexpected result in string theory. Recall that for the worldline description, there were different channels each with a distinct Feynman diagram. That is, we have both scattering and annihilation diagrams. For the worldsheet theory, these diagrams look identical! This is something we shall comment on again soon. \textcolor{red}{If he doesn't comment on it, come back and say something.}
\er 

Now, we still need to work something out; we require that the end tubes extend off to infinity.\footnote{For a explanation of why we require this, see the start of chapter 6 in Dr. Tong's notes.} Doing this actually allows us to make the problem look a lot nicer; we use the state-operator map. The end points are just our initial and final states, but we know that these are just represented by the insertion of a local operator. This local operator is known as a \textit{vertex operator} and is just the operator dual to the state required. So essentially we just have a sphere (which represents the middle part of the diagram above) with four insertions.

\begin{center}
    \btik 
        \draw[thick] (0,0) circle (2cm);
        \node at (1,1) {\large{$\cross$}};
        \node at (-1,1) {\large{$\cross$}};
        \node at (1,-1) {\large{$\cross$}};
        \node at (-1,-1) {\large{$\cross$}};
    \etik 
\end{center}

Now, our path integral involves considering all possible metric embeddings into the spacetime of the Polyakov action\footnote{It's written in Euclidean coordinates here.}
\bse 
    \int DX Dg \exp \bigg[ -\frac{1}{4\pi\a'} \int d^2\sig \sqrt{g} g^{\a\beta} \p_{\a}X^{\mu}\p_{\beta}X_{\mu} \bigg],
\ese
and we therefore need to also consider diagrams with holes in them. That is, things of the form 

\begin{center}
    \btik 
        \draw[thick] (-2,-2) .. controls (0,0) .. (2,-2);
        \draw[thick, rotate around={90:(0,0.5)}] (-2,-2) .. controls (0,0) .. (2,-2);
        \draw[thick] (-2,3) .. controls (0,1) .. (2,3);
        \draw[thick, rotate around={90:(0,0.5)}] (-2,3) .. controls (0,1) .. (2,3);
        \draw[thick, rotate around={-45:(-2.25,-1.75)}] (-2.25,-1.75) ellipse (0.35 and 0.1);
        \draw[thick, rotate around={45:(-2.25,2.75)}] (-2.25,2.75) ellipse (0.35 and 0.1);
        \draw[thick, rotate around={45:(2.25,-1.75)}] (2.25,-1.75) ellipse (0.35 and 0.1);
        \draw[thick, rotate around={-45:(2.25,2.75)}] (2.25,2.75) ellipse (0.35 and 0.1);
        \draw[thick] (-0.5,0.5) .. controls (-0.25,0.25) and (0.25,0.25) .. (0.5,0.5);
        \draw[thick] (-0.35,0.4) .. controls (-0.175,0.5) and (0.175,0.5) .. (0.35,0.4);
    \etik 
\end{center}
which corresponds to a final diagram of a torus with 4 vertex operators

\begin{center}
    \btik 
        \draw[thick] (0,0) circle (2cm);
        \node at (1,1) {\large{$\cross$}};
        \node at (-1,1) {\large{$\cross$}};
        \node at (1,-1) {\large{$\cross$}};
        \node at (-1,-1) {\large{$\cross$}};
        \draw[thick] (-0.5,0) .. controls (-0.25,-0.25) and (0.25,-0.25) .. (0.5,0);
        \draw[thick] (-0.35,-0.1) .. controls (-0.175,0) and (0.175,0) .. (0.35,-0.1);
    \etik 
\end{center}
Similarly we should consider multiple holes (i.e. like a figure 8 torus etc). Note that these diagrams correspond to the loop diagrams in Feynman diagrams. That is, if we slice the worldsheet across the hole, we end up with two strings in the interaction part that then join back to one. 

The next question we need to ask is `where do we put these vertex operators?' Of course, because of diffeomorphisms, we need to consider all possible positions. Another way to see this is the fact that in the initial cross type diagram we drew for the string interaction, we could have moved the legs around, which would correspond to moving the vertex insertions around on the sphere (or torus or whatever we're looking at).

So our problem is reduced to doing a path integral with four vertex insertions. The question is `how do we do it?' As we will see, there is a nice method that allows us change the integral. The general idea comes stems from the following claim. 

\bcl 
Given any metric, we can use our diffeomorphism and Weyl gauge symmetries to fix the metric to an element in an \textit{equivalence class} of metrics. The elements of these equivalence classes are related to the \textit{genus} of the manifold being considered. 
\ecl 

\br 
Note this is \textit{not} the same as saying we can make any metric into a metric that we like. It corresponds to the idea of having gauge orbits and using a gauge fixing to fix the specific metric to the representative from the orbit. 
\er 

The idea then is to replace the integral over the metric in our path integral with an integral over the parameter used to label which gauge orbit we are in. The contents of this method is the so-called \textit{Faddeev-Popov Method}, and we shall discuss it properly in the next lecture. 
\chapter{Faddeev-Popov Method}

We have brought our problem to the following path integral
\bse 
    Z = \frac{1}{\text{Vol}} \int DX Dg e^{-S_{\text{poly}}[X,g]},
\ese 
where $S_{\text{poly}}$ is of course the Polyakov action. Note that we have included a $1/$Vol term, which is used to account for the fact that we are only interested in physically distinct configurations; that is it is the volume of the space spanned by configurations that can be equated via diffeomorphisms and/or Weyl transformations. Essentially what we are talking about is integrating over a gauge fixing, which we achieve by integrating over everything and then use the Vol term to mod-out the contributions from the gauge orbits. 

\begin{center}
    \btik 
        \draw[thick, ->] (0,0) .. controls (1,2) and (2,1) .. (3,3);
        \draw[thick, ->, xshift = 1cm, yshift=-0.5cm] (0,0) .. controls (1,0.5) and (1.5,2) .. (3,3);
        \draw[thick, ->, xshift = 2cm, yshift=-1cm] (0,0) .. controls (1,2.5) and (2.5,1) .. (3,3);
        \draw[dashed] (4,-1) .. controls (3,2) and (2,1) .. (1,2.5);
        \node at (0.4,3) {Gauge};
        \node at (0.4,2.5) {Fixing};
        \node at (4.75,3) {Gauge Orbits};
    \etik 
\end{center}

As we mentioned at the end of the previous lecture, we are actually going to change the form of our path integral by using the Faddeev-Popov method. The idea is this: along a given gauge orbit, all metrics are equivalent up to a diffeo/Weyl transformation, and so we simply consider a representative for each gauge orbit (we obviously take the one that lies on our gauge fixing) and then use a delta function. More concretely, if we let $\zeta$ represent both diffeomorphisms and Weyl transformations, then we have the relation 
\be 
\label{eqn:InverseFaddeevPopov}
    \int d\tau D\zeta \, \del\big(g - g_0^{\zeta}(\tau) \big) \Delta_{FP}[g] = 1,
\ee 
where we have introduced the parameter $\tau$ which just labels which gauge orbit we are in. In case of potential confusion, let's just explain what \Cref{eqn:InverseFaddeevPopov} means again: we want to consider a specific metric, which we call $g$, and we do this by considering gauge transformations ($\zeta$) of a given representative of a gauge orbit ($g_0$) and then considering all the orbits ($\tau$) to make sure we don't just get zero.\footnote{By that we mean that we consider the gauge orbit that our $g$ lies in.} The term $\Delta_{FP}$ is the so-called \textit{Faddeev-Popov determinant} and it arises to take into account the Jacobian factor.\footnote{Its origin is analogous to the result $\int dx \del\big(f(x)\big) = 1/|f'|$.} 

To ground ourselves on why on earth we are talking about this, it follows immediately that if we chose our $g$ to be the $g$ integrated over in our path integral, the delta function will remove this integral and replace it with integrals over $\tau$ and $\zeta$. The $\zeta$ integral, though, is just the integral over the gauge orbits and so will just give exactly the Vol factor, and so that will go. We are therefore left with just the integral over $\tau$. In doing this, we essentially have inserted the Faddeev-Popov determinant into the path integral. It is important, therefore, to have an expression for it. Before doing that, it is instructive to do the following things. 

\bcl
\label{claim:InvariantMeasure}
The measure in \Cref{eqn:InverseFaddeevPopov} is the analogue to the Haar measure in Lie groups. That is it is left and right action invariant:
\be
\label{eqn:MeasureInvariantFP}
    D(\zeta\zeta') = D\zeta = D(\zeta'\zeta)
\ee 
\ecl 

\bl 
The Faddeev-Popov determinant is gauge invariant. 
\el 

\bq 
By definition we have 
\bse 
    \int d\tau D\zeta \, \del\big(g^{\zeta'} - g_0^{\zeta}(\tau) \big) \Delta_{FP}\big[g^{\zeta'}\big] = 1.
\ese 
Then using the invariance of the measure we have 
\bse 
    \begin{split}
        \int d\tau D\zeta \, \del\big(g^{\zeta'} - g_0^{\zeta}(\tau) \big) \Delta_{FP}\big[g^{\zeta'}\big] & = \int d\tau D\zeta \, \del\big(g - g_0^{(\zeta')^{-1}\zeta}(\tau) \big) \Delta_{FP}[g] \\
        & = \int d\tau D\zeta'' \, \del\big(g - g_0^{\zeta''}(\tau) \big) \Delta_{FP}[g],
    \end{split}
\ese 
where $\zeta'' := (\zeta')^{-1}\zeta$. Finally just relabelling $\zeta'' \to \zeta$ in the last line gives us 
\be 
\label{eqn:FaddeevPopovInvariant}
    \Delta_{FP}\big[g^{\zeta'}\big] = \Delta_{FP}[g].
\ee 
\eq 

So we have a formula that includes the Faddeev-Popov determinant, but we don't know how to calculate it yet. In order to do this, let's consider gauge transformations close to the identity. In this case the metric transforms as 
\bse
    \del \hat{g}_{\a\beta} = \nabla_{\a} v_{\beta} + \nabla_{\beta}v_{\a} + \phi \hat{g}_{\a\beta},
\ese 
where $v$ is the infinitesimal diffeomorphism and $\phi$ the Weyl transformation. Note we have replaced the subscript $0$ for a hat notation, this is just to make the notation a bit less cluttered. We also need to take into account the deviation of the metric as we move between gauge orbits, i.e. the derivative w.r.t. $\tau$. Now $\tau$ is what that takes us from one gauge orbit to another. There is no reason it needs to only be one factor, in the sense that multiple things could contribute to moving from one orbit to another, and so we introduce an index $\tau_i$. So the full transformation is 
\bse
    \del \hat{g}_{\a\beta} = \nabla_{\a} v_{\beta} + \nabla_{\beta}v_{\a} + \phi \hat{g}_{\a\beta} + \bigg(\frac{\p \hat{g}}{\p\tau_i}\bigg)_{\a\beta} \del\tau_i.
\ese 
Plugging this into \Cref{eqn:InverseFaddeevPopov}, we have 
\bse 
    \int d\tau_i Dv D\phi \, \del \bigg( \nabla_{\a} v_{\beta} + \nabla_{\beta}v_{\a} + \phi \hat{g}_{\a\beta} + \bigg(\frac{\p \hat{g}}{\p\tau_i}\bigg)_{\a\beta} \del\tau_i\bigg) \Delta_{FP}[g] = 1.
\ese 

Now recall that the Fourier form of a delta function is 
\bse 
    \del(x) = \int dp e^{2\pi i px}.
\ese 
We want to use something similar here to make our expression look nicer, however we need to take into account that what we have is a delta function\textit{al}; it restricts an entire function! We therefore have 
\bse 
    \int d\tau_i Dv D\phi DB \, \exp\bigg[2\pi i \int d^2\sig \sqrt{\hat{g}} B^{\a\beta} \bigg( \nabla_{\a} v_{\beta} + \nabla_{\beta}v_{\a} + \phi \hat{g}_{\a\beta} + \bigg(\frac{\p \hat{g}}{\p\tau_i}\bigg)_{\a\beta} \del\tau_i\bigg)\bigg] = \Delta_{FP}^{-1},
\ese 
where $B^{\a\beta}$ is a symmetric 2-tensor on the worldsheet of the string, and where we have used the fact that $\Delta_{FP}$ is not a function of any of the integration variables. 

So we have an expression for $\Delta_{FP}^{-1}$, but what we need is an expression for $\Delta_{FP}$ itself. So what do we do? We use our trained observation skills to notice that the exponential is a quadratic expression. To understand why that helps us we interrupt the programme for a brief discussion of Berezin and Grassmann integration.

\section{Berezin and Grassmann Integration}

This section is by no means a complete discussion on the topic, but is just focused on the specific result we want and is included for completeness of the method we need. A more complete discussion of some of the preliminaries for what is presented here can be found via Dr. David Skinner's Supersymmetry course, available via the \href{http://www.damtp.cam.ac.uk/user/dbs26/SUSY.html}{Cambridge website}.

Grassmann variables are antisymmetric under exchange, and so obey anti-commutation relations\footnote{Some texts will talk of Fermionic variables here. The important thing is just that they are antisymmetric under exchange.}. We want to be able to define integration of these Grassmann variables. First let's consider a function of a single variable, labelled $\psi$. Due to the antisymmetric nature, we know that any such function must be of the form 
\bse 
    F[\psi] = f + g\psi,
\ese
for some scalar fields $f,g$. It suffices, therefore, to just define the integrals $\int d\psi$ and $\int \psi d\psi$. 

We impose that the integral be translation invariant, 
\bse 
    \int \psi d\psi = \int (\psi + \eta) d\psi,
\ese 
which tells us that 
\be 
\label{eqn:Berezin0}
    \int 1 d\psi = 0.
\ee
Then we just choose our normalisation such that 
\be
\label{eqn:Berezin1}
    \int \psi d\psi = 1.
\ee 
\br 
Note that for Grassmann (or Fermionic) variables we have 
\bse 
    \int (A + Ba) ada = A,
\ese 
which we get from \Cref{eqn:Berezin0} and \Cref{eqn:Berezin1}. Comparing this to the symmetric (or Bosonic) expression
\bse 
    \int (A+Ba) \del(a) da = A,
\ese
we see that a Fermionic variable behaves analogously to a Bosonic delta function.
\er 

We then extend the integration definitions to a function of $n$ Grassmann variables:
\bse 
    \int \psi^{a_1}\psi^{a_2}...\psi^{a_n} d^n\psi = \int \psi^{a_1}\psi^{a_2}...\psi^{a_n} d\psi^{a_n}d\psi^{a_{n-1}}...d\psi^{a_1} = \epsilon^{a_1a_2...a_n},
\ese
where the Levi-Civita symbol is there to account for the minus signs picked up under exchange. Note that it is only when \textit{all} the variables are in the integrand that we don't get a vanishing answer. 

Now consider the case of a Gaussian integral with Grassmann variables, 
\bse 
    \begin{split}
        \int e^{-\frac{1}{2}A_{ab}\psi^a\psi^b} d^{2N}\psi & = \int \sum_{n=1}^{N} \frac{(-1)^n}{2^n n!} \big(A_{ab}\psi^a\psi^b\big)^n d^{2N}\psi \\
        & = \frac{(-1)^N}{2^NN!} \int A_{a_1a_2}A_{a_3a_4}...A_{a_{2N-1}a_{2N}} \psi^{a_1}\psi^{a_2}...\psi^{2N-1}\psi^{2N} d^{2N}\psi \\
        & = \frac{(-1)^N}{2^NN!} A_{a_1a_2}A_{a_3a_4}...A_{a_{2N-1}a_{2N}} \epsilon^{a_1a_2...a_{2N}}
    \end{split}
\ese
where we have used the fact that only the $N$ term in the sum doesn't give vanishing result along with the results above. The RHS here looks gross and so we introduce a definition
\bd 
The Pfaffian of a $2N\times 2N$ antisymmetric matrix is given by 
\bse 
    \text{Pfaff}(A) := \frac{1}{2^NN!} A_{a_1a_2}A_{a_3a_4}...A_{a_{2N-1}a_{2N}} \epsilon^{a_1a_2...a_{2N}}.
\ese 
\ed 

\bcl 
We claim without proof\footnote{Although a quick Google search should do the trick for anyone interested.} that, for Grassmann variables, the following holds 
\bse 
    \big(\text{Pfaff}(A)\big)^2 = \det A.
\ese 
\ecl 

This claim tells us, then, the for our Gaussian integral we have 
\bse 
    \int e^{-\frac{1}{2}A_{ab}\psi^a\psi^b} d^{2N}\psi = \pm \sqrt{\det A}.
\ese 
This now looks more interesting. Recall that for our Bosonic (i.e. symmetric variables) Gaussian integral, we have the following result: let $B$ be a symmetric, positive definite matrix, then 
\bse 
    \int e^{-\frac{1}{2}B_{ab}\phi^a\phi^b} d^{2N}\phi = \sqrt{\frac{(2\pi)^{2N}}{\det B}},
\ese 
and so we see, up to a factor, switching between symmetric and antisymmetric Gaussian integral is equivalent to inverting the determinant on of the matrix. This is exactly what we need for our problem, and so we return to our scheduled programme. 

\section{Ghost Fields}

So we have a quadratic exponential in terms of the variables $B,v,\phi$ and $\tau$, which are all symmetric under exchange. So if we exchange all these for Grassmann (or Fermionic) variables, then we get an expression for $\Delta_{FP}$ itself.

Before doing this substitution though, we are first going to massage our expression. This is what we have:
\bse 
    \int d\tau_i Dv D\phi DB \, \exp\bigg[2\pi i \int d^2\sig \sqrt{\hat{g}} B^{\a\beta} \bigg( \nabla_{\a} v_{\beta} + \nabla_{\beta}v_{\a} + \phi \hat{g}_{\a\beta} + \bigg(\frac{\p \hat{g}}{\p\tau_i}\bigg)_{\a\beta} \del\tau_i\bigg)\bigg] = \Delta_{FP}^{-1},
\ese 
The first thing we do is consider the $\phi$ integral. We see that this integral simply constrains $B$ to be traceless, and so our integral over $B$ is restricted to an integral over traceless, symmetric 2-tensors. Motivated by this, we can subtract the trace part of the $\nabla_{\a}v_{\beta}+\nabla_{\beta}v_{\a}$ term. We have 
\bse 
    \int d\tau_i Dv DB \, \exp\bigg[2\pi i \int d^2\sig \sqrt{\hat{g}} B^{\a\beta} \bigg( \nabla_{\a} v_{\beta} + \nabla_{\beta}v_{\a}  - \hat{g}_{\a\beta}\nabla\cdot v + \bigg(\frac{\p \hat{g}}{\p\tau_i}\bigg)_{\a\beta} \del\tau_i\bigg)\bigg] = \Delta_{FP}^{-1}.
\ese 

We now make the substitutions, changing all our symmetric (Bosonic) variables to antisymmetric (Fermionic) variables. We use the notation 
\be 
\label{eqn:BosonicFermionicChange}
    v_{\a} \to c_{\a}, \qquad B^{\a\beta} \to b^{\a\beta}, \qquad \del\tau_i \to \xi_i.
\ee 
We therefore have\footnote{Note that the derivative of $\Hat{g}$ w.r.t $\tau_i$ doesn't change to a $\xi_i$.}
\be 
    \Delta_{FP} = \int d\xi_i Dc  Db \, \exp\bigg[2\pi i \int d^2\sig \sqrt{\hat{g}} b^{\a\beta} \bigg( \nabla_{\a} c_{\beta} + \nabla_{\beta}c_{\a}  - \hat{g}_{\a\beta}\nabla\cdot c + \bigg(\frac{\p \hat{g}}{\p\tau_i}\bigg)_{\a\beta} \xi_i\bigg)\bigg].
\ee 
Then we do the integral over the $\xi_i$, using the fact that 
\bse 
    \int e^{A\xi} d\xi = A,
\ese 
which comes from \Cref{eqn:Berezin0} and \Cref{eqn:Berezin1}. We have 
\bse 
    \Delta_{FP} = A \int Dc  Db \, \exp\bigg[2\pi i \int d^2\sig \sqrt{\hat{g}} b^{\a\beta} \bigg( \nabla_{\a} c_{\beta} + \nabla_{\beta}c_{\a}  - \hat{g}_{\a\beta}\nabla\cdot c\bigg)\bigg],
\ese 
where 
\bse 
    A = \prod_{j=1}^m 2\pi i \int d^2\sig \sqrt{\Hat{g}} b^{\a\beta} \bigg(\frac{\p \hat{g}}{\p\tau_j}\bigg)_{\a\beta}.
\ese
Now we can rescale the $b$ and $c$ fields such that we get a $1/2\pi$ in the exponential, making it more comparable to our Polyakov action. In this way we define the
\mybox{
Ghost Action
\be 
\label{eqn:Sghost}
    S_{\text{ghost}} := \frac{1}{2\pi} \int d^2\sig \sqrt{\Hat{g}}b^{\a\beta} \bigg( \nabla_{\a}c_{\beta} + \nabla_{\beta} c_{\a} - \Hat{g}_{\a\beta}\nabla\cdot c\bigg).
\ee 
}
\noindent Finally, rotating back to Euclidean space (so that the factor of $i$ disappears from $\Delta_{FP}$), we have 
\mybox{
\be 
\label{eqn:PartionWithGhost}
    Z[\Hat{g}] = A \int DXDcDb \exp\Big( -S_{\text{Poly}}[X,\Hat{g}] - S_{\text{ghost}}[b,c,\Hat{g}]\Big),
\ee
}
\noindent where we have used the comment made above \Cref{claim:InvariantMeasure} about inserting \Cref{eqn:InverseFaddeevPopov} into our path integral.

\section{The Conformal Transformations}

So what we've seen is that fixing the gauge of our system has resulted in the introduction of ghost fields which appear on an equal footing to do the dynamical $X$ fields. This statement is not quite true. In fact choosing our metric up to Weyl and diffeomorphisms as we have done does not completely fix the gauge of the system. This is just the discussion we had previously, and the remaining gauge comes from the conformal transformations. There is an actual problem associated with this: the space of conformal transformations is infinite, and so, left unchecked, our integral will just give infinity. 

This problem is actually something we have just been hiding during the whole lecture. It is something we should have addressed when we first introduced the Faddeev-Popov determinant. Why? Well because without fixing it we simply get $\Delta_{FP}=0$, which is useless. This follows because in \Cref{eqn:InverseFaddeevPopov} all we have done is fix the metric, but once the metric is fixed there could still be an unfixed gauge freedom corresponding to a conformal transformation and so we actually have an infinite number of `clicks' in our $g$ delta function. 

Thankfully, this is easily fixed. We simply pick a certain number of points (which as we will see depends on the topology of the system) and fix them w.r.t diffeomorphisms. In doing this, we essentially pick a particular conformal choice and so only get one `click' in our delta function. Mathematically, we have 
\be 
    1 = \int d\tau D\zeta \, \del\big(g-\Hat{g}^{\zeta}(\tau)\big) \prod_j \del \big( v_{\a}(\Hat{\sig}_j )\big) \Delta_{FP}[g],
\ee 
where we have encapsulated the fixing of the coordinates by the fact that we require that $v_{\a}$ (the infinitesimal diffeomorphism) vanish at these points. 

Following the method of above, again replacing $v_{\a}\to c_{\a}$. So, after taking the Fourier transform, we get an insertion of the form 
\bse 
    \int d \theta e^{\theta^{\a} c_{\a}(\hat{\sig}_j) } = \prod_{\a=1}^2 c_{\a}(\hat{\sig}_j)
\ese 
where we have used the fact that we are considering Grassmann variables. We then get the corrected partition function
\bse 
    Z[\Hat{g}] = A \int DXDcDb \exp\Big( -S_{\text{Poly}}[X,\Hat{g}] - S_{\text{ghost}}[b,c,\Hat{g}]\Big)\prod_{j=1}^f \prod_{\a=1}^2 c_{\a}(\hat{\sig}_j),
\ese 
where we have used the label $f$ for the number of points we need to fix. 

\section{Inserting Vertex Operators}

We now want to return to our interacting picture. Recall that essentially what we have is the insertion of a set number of vertex operators (4 in the diagrams drawn previously) onto our sphere (or torus or whatever). We therefore need to insert these vertex operators into our path integral. For generality, let's say we are inserting $n$ vertex operators.

Now, it is important that we preserve our diffeomorphism and Weyl invariance with these insertions, as otherwise we will not be considering our Nambu-Goto action. Our vertex insertions therefore take the form 
\bse 
    \int d\sig_i \sqrt{g} V(\sig_i),
\ese 
where $V$ is a scalar under diffeomorphisms and has the opposite Weyl transformation to $g$. So if we go with the convention 
\bse 
    g_{\a\beta} \to e^{-\phi} g_{\a\beta},
\ese 
we require that $V$ transforms under Weyl transformations with weight $1$. Putting these together means that $V$ transforms under conformal transformations as a tensor plus a Weyl factor, which is a primary operator! In fact, it is a weight $(1,1)$ primary operator.

We can use the position fixings talked about in the last section to fix the position of $f$ of these $n$ vertex operators. That is, we just choose the positions of $f$ of the vertex operators to be the $\hat{\sig}_j$s. Inserting these we get an expression for the scattering amplitude path integral\footnote{We have dropped the factors of $2\pi i$ from the definition of $A$, just to slightly shorten the massive expression.}

\mybox{
\bse 
    \begin{split}
        \cA  =  \int DXDcDb \bigg[\prod_{k=1}^m \int d^2\sig \sqrt{\hat{g}} \bigg(\frac{\p \hat{g}}{\p \tau_k}\bigg)_{\a\beta} b^{\a\beta} \bigg] \bigg[ \prod_{j=1}^f & \prod_{\a=1}^2 c_{\a}(\hat{\sig}_j)  \sqrt{g} V_j(\hat{\sig}_j) \bigg] \\
        &\times  \bigg[\prod_{j=f+1}^n \int d\sig_j \sqrt{g} V_j(\sig_j)\bigg] e^{-(S_{\text{P}} + S_{\text{g}})}
    \end{split}
\ese 
}

\br 
There is a nice result in here about the number of vertex insertions compared to the number of points we fix. We will see that $f$ is a fixed number and for the sphere it is $f=3$. If we then only consider inserting $n=2$ vertex operators, we see that there is a position left unfixed by something and so we end up with an infinite result. This is just the condition that if we consider the amplitude for a propagator (which for us is clearly just two insertions) and impose the mass-shell condition (which we do here, as will become clearer later) we get an infinite result. 
\er 

\section{The $P$ Operator}

\subsection{Checking $c$ and $b$}

We need to be careful with the above expression. Remember that $b$ and $c$ are Grassman variables, and so if we don't have exactly the right number of each, our entire integral vanishes. To look at this problem, let's first go back and look at the ghost action again:
\bse 
    S_{\text{ghost}} := \frac{1}{2\pi} \int d^2\sig \sqrt{\hat{g}} b^{\a\beta} \Big( \nabla_{\a} c_{\beta} + \nabla_{\beta} c_{\a} - \hat{g}_{\a\beta} \nabla \cdot c\Big).
\ese 
We now introduce a map $P$ that maps a vector field to a symmetric, traceless 2-tensor such that 
\be 
\label{eqn:PMap}
    Pc = \nabla_{\a} c_{\beta} + \nabla_{\beta} c_{\a} - \hat{g}_{\a\beta} \nabla \cdot c,
\ee 
from which we see we can write the ghost action as\footnote{We shall drop the $1/2\pi$ factor here, mainly because Dr. Minwalla doesn't have it and it's just a number so it's not vital to our understanding.} 
\be 
\label{eqn:SGhostWithP}
    S_{\text{ghost}} =  (b,Pc),
\ee
where $(\cdot,\cdot)$ is the inner product on the worldsheet, defined by 
\bse 
    (A, B) :=  \int d^2\sig \sqrt{\hat{g}} A^{\a\beta} B_{\a\beta}.
\ese 
We want to study the map $P$ a bit more closely. 

Recall that $Pc$ is the change in the metric due to infinitesimal $c$, after you then subtract a suitable Weyl part, to remove the change to the trace. Now suppose that the result of this vanishes, that is this Weyl transformation cancels everything. We are then considering an unfixed diffeomorphism for this specific vector field $c$, i.e. a conformal transformation w.r.t. $c$. What we are saying here is that there is a one-to-one correspondance between the zero modes of $c$ and the unfixed diffeomorphisms, which is just $f$

Now note that, using integration by parts on our ghost action, we could equally have the inner product where it is the $b$ field that is acted on. As we have a good inner product this is just 
\bse 
    (b,Pc) = (P^{\dagger}b,c).
\ese 
Now, a zero mode of $P^{\dagger}$ means that for that $b$ \textit{any} $c$ field will give vanishing inner product $(b,Pc) = 0$. Which, as $Pc$ tells us about how the metric changes, means that $b$ lies in the space orthogonal to the space of metrics that can be obtained via traceless diffeomorphisms. In the gauge orbit drawing, we see this simply as the zero modes of $b$ being changes in equivalence classes, or changes in $\tau$, which is $m$

To summarise:
\begin{itemize}
    \item The number of zero modes in $c$ is $f$, and 
    \item The number of zero modes in $b$ is $m$.
\end{itemize}
These are exactly the conditions we need in order for our integral not to vanish!

\subsection{$P^{\dagger}P$}

Let's now consider the operator $P^{\dagger}P$, which maps vector fields to vector fields. Let's assume we are on some manifold and we have diagonalised this operator, with eigenvectors $c_m(\sig)$ and eigenvalues $\lambda_m$.  We use these eigenvectors as a basis, and so can express an arbitrary $c$ component-wise using these $c_m$s. 
\bse 
    c_{\a}(\sig) = \sum_m \gamma_m c_{m\a} (\sig),
\ese 
where by $c_{m\a}$ we obviously mean $(c_m)_{\a}$. 

So we have\footnote{We drop the $(\sig)$ to lighten notation.} 
\bse 
    P^{\dagger}P c_m = \lambda_m c_m.
\ese
Now if $Pc_m$ is non-vanishing we also obtain an eigenvalue equation 
\bse 
    PP^{\dagger}(Pc_m) = \lambda (Pc_m),
\ese 
which is clearly related to the $b$ fields. So we see that the spectrum of $P^{\dagger}P$ and $PP^{\dagger}$ are identical, apart from in the zero mode sector. So, outside the zero mode part, for every $c_m$ we can define a corresponding $b_m := Pc_m$. These will make a basis for the $b$s and so we have 
\bse 
    b_{\a} = \sum_m \rho_m Pc_{m\a}
\ese 
outside the zero mode sector. Plugging this into our ghost action \Cref{eqn:SGhostWithP}, we have
\bse 
    \begin{split}
        S_{\text{ghost}} & = \sum_{m,n} \rho_m \gamma_n (Pc_{m\a}, Pc_{n\beta}) \\
        & = \sum_{m,n} \rho_m \gamma_n (c_{m\a}, P^{\dagger} Pc_{n\beta}) \\
        & = \sum_{m,n} \rho_m \gamma_n \lambda_n (c_{m\a}, c_{n\beta}) \\
        & = \sum_m \rho_m \gamma_m \lambda_m \delta_{\a\beta},
    \end{split}
\ese 
where we have used the fact that the basis elements are orthonormal. So we see in this basis that the contribution to the amplitude from non-zero mode part of the ghost action is simply of the form 
\bse 
    \exp\bigg( - \sum_m \rho_m \gamma_m \lambda_m \delta_{\a\beta} \bigg).
\ese 
The zero mode contribution to the amplitude is contained completely within the products worked out before, i.e. the $m$ $b$ terms and the $f$ $c$ terms. Now, because there is no zero mode contribution to the exponential above, unless we have \textit{all} the zero modes in these product factors our integral will vanish (as there will be an integration variable that has nothing to act on, and these are Grassman variables). But each factor in the product is of the form 
\bse 
    \text{zero mode + non-zero mode},
\ese 
and so we we don't have any non-zero mode contributions in these product factors. Therefore, we can do the integral over the non-zero mode part and obtain the determinant of $P^{\dagger}P$, which we denote $\det'(P^{\dagger}P)$ where the $'$ tells us its only the non-zero mode part. 

\subsection{Zero Mode Sector}

Now what about the zero mode parts? First let's consider the $b$ part:
\bse 
    \prod_{k=1}^m \int d^2\sig \sqrt{\hat{g}} \bigg(\frac{\p\hat{g}}{\p\tau_k}\bigg)_{\a\beta} b^{\a\beta} = \prod_{k=1}^m \bigg( \frac{\p\hat{g}}{\p \tau_k} , b\bigg),
\ese 
where the RHS is the inner product. We only want to consider the zero mode parts, so if we label the eigenfunctions by $B^0_j$ and the coefficients by $\rho^0_j$ we get 
\bse 
   \int \prod_{k=1}^m d\rho^0_j \bigg( \frac{\p\hat{g}}{\p\tau_k}, B^0_j\bigg) \rho^0_j \sim \det\bigg( \frac{\p\hat{g}}{\p\tau_k}, B^0_j \bigg)
\ese 

This determinant is actually really important and is tied up in the fact that we want the result of our integral to be independent of how we choose to parameterise the $\tau_k$s. That is, it makes our integral invariant under $\tau_k \to \tau_k'$.

Now what about the $c$ insertions? The same arguments will give us 
\bse 
    \det\Big( c^0_i(\hat{\sig}_j)\Big),
\ese 
where $i$ and $j$ tell us the position in the matrix. Unlike the $b$ case, this is not some Jacobian factor, but simply a determinant result. 
\chapter{The $bc$ CFT}

Recall \Cref{rem:cproblem}, which essentially told us that, unless we can introduce something that contributes a negative central charge, our theory is doomed. Well last lecture we did introduce something new, namely the ghost action. We now want to see what effect, if any, the ghost action has on the central charge.\footnote{Spoiler: it will contribute $c_{\text{ghost}}=-26$. Note the number $26$ should look familiar...}

\section{Simplifying $S_{\text{ghost}}$}

As it stands our ghost action looks like this 
\bse 
    S_{\text{ghost}} := \frac{1}{2\pi} \int d^2\sig \, \sqrt{\hat{g}} b^{\a\beta}\Big(\nabla_{\a} c_{\beta} + \nabla_{\beta} c_{\a} - \hat{g}_{\a\beta} \nabla\cdot c\Big).
\ese 
The first thing we want to do is to use the metric to lower the indices on $b$ and raise them on the $c$. So if we define
\bse 
    \nabla^{\a} := \hat{g}^{\a\beta}\nabla_{\beta},
\ese 
our ghost action becomes 
\be 
\label{eqn:ghostblower}
    S_{\text{ghost}} := \frac{1}{2\pi} \int d^2\sig \, \sqrt{\hat{g}} b_{\a\beta}\Big(\nabla^{\a} c^{\beta} + \nabla^{\beta} c^{\a} - \hat{g}^{\a\beta} \nabla\cdot c\Big).
\ee 
We now work in conformal gauge, i.e. we set
\bse 
    \hat{g}_{\a\beta} = e^{2\phi} \gamma_{\a\beta},
\ese 
where $\gamma_{\a\beta}$\footnote{I have used the notation $\gamma$ here to avoid confusion with the delta function in the following proposition, although they are essentially the same thing.} is the Euclidean metric. As we have mentioned before, we can do this (at least locally\footnote{If it is not possible globally, we just inherit the behaviour into the transition functions. For now though, we shall ignore this.}) because of our gauge symmetries. 

\bp 
    In the conformal gauge, we can replace covariant derivatives in \Cref{eqn:ghostblower} with standard derivatives and replacing $\hat{g}\to \gamma$. 
\ep

\bq 
    We have 
    \bse 
        \nabla_{\tau}c^{\beta} = \p_{\a}c^{\beta} + {\Gamma^{\beta}}_{\sig\tau}c^{\tau}.
    \ese 
    The Christoffel symbols are given by 
    \bse
        \begin{split}
            {\Gamma^{\beta}}_{\sig\tau} & := \frac{1}{2}\hat{g}^{\beta\rho}\Big(\hat{g}_{\rho\tau,\sig} + \hat{g}_{\rho\sig,\tau} - \hat{g}_{\sig\tau,\rho}\Big) \\
            & = \frac{1}{2}e^{-2\phi}\delta^{\beta\rho} \Big( 2e^{2\phi}\gamma_{\rho\tau}\p_{\sig}\phi + 2e^{2\phi}\gamma_{\rho\sig}\p_{\tau}\phi - 2e^{2\phi}\gamma_{\sig\tau}\p_{\rho}\phi\Big) \\
            & = \delta^{\beta}_{\tau}\p_{\sig}\phi + \delta^{\beta}_{\sig}\p_{\tau}\phi - \gamma_{\sig\tau}\p^{\beta}\phi,
        \end{split}
    \ese 
    where $\p^{\beta}:=\gamma^{\beta\rho}\p_{\beta}$. We then have 
    \bse 
        \hat{g}^{\a\tau}{\Gamma^{\beta}}_{\sig\tau} = e^{-2\phi}\Big( \gamma^{\a\beta}\p_{\sig}\phi + \delta^{\beta}_{\sig} \p^{\a}\phi - \delta^{\a}_{\sig} \p^{\beta}\phi\Big).
    \ese 
    We then notice that the last two terms are antisymmetric in $\a$ and $\beta$, so we have 
    \bse 
        \nabla^{\a}c^{\beta} + \nabla^{\beta}c^{\a} = e^{-2\phi}\Big[\p^{\a}c^{\beta} + \p^{\beta}c^{\a} + 2\gamma^{\a\beta} \big(\p_{\sig}\phi\big)c^{\sig} \Big].
    \ese 
    Next we have 
    \bse 
        \nabla\cdot c := \nabla_{\a}c^{\a} = \p_{\a}c^{\a} + {\Gamma^{\a}}_{\sig\a} c^{\sig},
    \ese
    where similar calculation to the above gives 
    \bse 
        \begin{split}
            {\Gamma^{\a}}_{\sig\a} & = \gamma^{\a\tau}\Big( \gamma_{\tau\a}\p_{\sig}\phi + \gamma_{\tau\sig}\p_{\a}\phi - \gamma_{\a\sig}\p_{\tau}\phi\Big) \\
            & = 2\p_{\sig}\phi + \p_{\sig}\phi - \p_{\sig}\phi \\
            & = 2 \p_{\sig}\phi,
        \end{split}
    \ese 
    where we have used $\gamma^{\a\sig}\gamma_{\a\sig}=2$, giving us 
    \bse 
        \hat{g}^{\a\beta}\nabla\cdot c = e^{2\phi}\gamma^{\a\beta}\p_{\sig}c^{\sig} + 2\gamma^{\a\beta}\big(\p_{\sig}\phi\big) c^{\sig}
    \ese 
    Finally using $\sqrt{\hat{g}} = e^{2\phi}\sqrt{\gamma} = e^{2\phi}$, we have
    \bse 
        S_{\text{ghost}} = \frac{1}{2\pi} \int d^2\sig e^{2\phi} b_{\a\beta} e^{-2\phi}\Big( \p^{\a}c^{\beta} + \p^{\beta}c^{\a} - \gamma^{\a\beta}\p \cdot c\Big),
    \ese 
    where we have defined $\p\cdot c := \p_{\a}c^{\a}$. Then just cancelling the exponential factors gives the result. 
\eq 

Now if we transform into $(z,\overline{z})$ coordinates, and use that $b$ is traceless (so $b_{z\overline{z}}=b_{\overline{z}z}=0$), and introduce the notation
\be
    b := b_{zz}, \qquad \overline{b} := b_{\overline{z}\overline{z}}, \qquad c := c_z, \qquad \overline{c} := c_{\overline{z}},
\ee 
we have 
\be 
\label{eqn:Sghostbc}
    S_{\text{ghost}} = \frac{1}{2\pi} \int d^2z \Big( b\overline{\p}c + \overline{b}\p c\Big),
\ee 
where we have used our $\p/\overline{\p}$ notation.\footnote{Note that $\p = \p_z$ with the index down, so they are flipped for the raised indices, i.e. $\overline{\p} = \p^z$.} 

So we have just shown that the ghost action, \textit{as we have written it}, has no Weyl dependence (i.e. $\phi$ doesn't appear in \Cref{eqn:Sghostbc}). This is the statement that $b$ and $c$ are Weyl neutral. We know they transform with diffeomorphism weight 
\bse 
    \begin{split}
        b &\qquad (2,0), \\
        \overline{b} &\qquad (0,2) \\
        c &\qquad (-1,0) \\
        \overline{c} &\qquad (0,-1),
    \end{split}
\ese 
and so this is also their conformal weight. 

\br 
    As we emphasised above, the Weyl neutrality only holds for how we've written it, that is $b_{\a\beta}$ and $c^{\a}$. If we considered the raised $b^{\a\beta}$ and lowered $c_{\a}$ we would, of course, get a Weyl weight (just from the definition $c_{\a}=\hat{g}_{\a\beta}c^{\beta}$, and similarly for $b^{\a\beta}$), however their conformal weights would still turn out to be the same. To see this note that the ghost action still needs to be Weyl invariant and so $c_{\a}$ must transform with Weyl weight $(-2,0)$ in order to cancel the $(2,0)$ Weyl weight from the $\hat{g}_{\a\beta}$. Combining this with the $(1,0)$ diffeomorphism weight for $c_{\a}$ we get a total conformal weight of $(-1,0)$, as required. 
\er 

The equations of motion for \Cref{eqn:Sghostbc} give 
\bse 
    \overline{\p}b = \overline{\p}c = \p\overline{b} = \p\overline{c} = 0,
\ese 
telling us that $b$ and $c$ are holomorphic, and $\overline{b}$ and $\overline{c}$ are antiholomorphic. 

\section{The OPE}

Let's just consider the holomorphic part of the path integral, and employ the same methods we did for the bosonic fields. We have  
\bse 
    0 = \int DbDc \, \frac{\del}{\del b(\sig)}\Big[b(\sig') e^{-S_{\text{ghost}}}\Big] = \int Db Dc \, e^{-S_{\text{ghost}}} \Big[\del(\sig-\sig') + \frac{1}{2\pi} b(\sig')\overline{\p}c(\sig)\Big],
\ese
where we have used the anticommuting property to get a minus sign when moving the derivative past $b(\sig')$. We get a similar result if we differentiate w.r.t. $c(\sig)$. We therefore have 
\bse 
    \begin{split}
        \la b(z_1)\overline{\p}c(z_2) \ra  & = -\del(z_2-z_1) \\
        \la c(z_1)\overline{\p}b(z_2) \ra  & = -\del(z_2-z_1).
    \end{split}
\ese 
These should look familiar from the bosonic field discussion, and so we have 
\bse 
    \begin{split}
        \la c(z_1)b(z_2) \ra  & = \frac{1}{z_{12}} \\
        \la b(z_1)c(z_2) \ra & = \frac{1}{z_{12}}.
    \end{split}
\ese 
Note that we have used the anticommutativity and $z_{12}=-z_{21}$ to flip some signs around. The easiest way to remember the result is that $z_1$ always appears on the left and $z_2$ on the right, as above. 

\section{The Stress Tensor}
The next job is to find the stress tensor. We do actually have the specific form of the action, and so we could just vary this w.r.t. the metric and obtain the stress tensor. However, we shall derive the result here using an alternative method as it is instructive to see for similar cases but where the action is not known. The method is essentially to guess the answer and then to check it works out.\footnote{By which we mean that we will check it obeys the Virasoro algebra and that $b$ and $c$ have the correct weights.} 

We have seen that the (holomorphic part of the) stress tensor has weight $(2,0)$. We have also seen that $b$ and $c$ have weight $(2,0)$ and $(-1,0)$, so if we are to make $T$ from these two objects, we want one of each and a derivative (which increases the weight by one, recall). We take our guess to be\footnote{Of course this is not really a guess, and is done in hindsight of knowing the answer.}
\bse 
    \begin{split}
        T_{\l}(z) & = \cl\p b(z)c(z)\cl - \l\cl\p\big(b(z)c(z)\big)\cl \\
        & = (1-\l)\cl \p b(z) c(z)\cl - \l \cl b(z) \p c(z)\cl,
    \end{split}
\ese 
where $\lambda\in\R$ is some constant. It turns out that different values of $\lambda$ will correspond to different conformal field theories, and so (after we establish the required $\lambda$ for our theory) we will tend to leave it in explicitly, and only insert the value at stages for checks about our theory. 

\subsection{Primary Fields}

First let's find the OPE with $b/c$. We have 
\bse 
    \begin{split}
        T_{\l}(z)c(\omega) & = \Big[(1-\l)\cl\p b(z)c(z)\cl c(\omega)\Big] - \Big[\l \cl b(z)\p c(z)\cl c(\omega)\Big] \\
        & = \Big[(\l-1)\cl c(z)\p b(z)\cl c(\omega)\Big] + \Big[\l \cl \p c(z) b(z)\cl c(\omega)\Big] \\
        & = -\frac{(\l-1)c(z)}{(z-\omega)^2} + \frac{\l \p c(z)}{z-\omega} + ...,
    \end{split}
\ese 
from which we see that $\l=2$ if we want $c$ to be a weight $(-1,0)$ operator, i.e. we want the first coefficient to be $-1$. 

For $b$ we have 
\bse 
    \begin{split}
        T_{\l}(z)b(\omega) & = \Big[(1-\l)\cl\p b(z)c(z)\cl b(\omega)\Big] - \Big[\l \cl b(z)\p c(z)\cl b(\omega)\Big] \\
        & = \frac{(1-\l)\p b(z)}{z-\omega} + \frac{\l b(z)}{(z-\omega)^2} + ... \\
        & = \frac{\l b(z)}{(z-\omega)^2} + \frac{(1-\l)\p b(z)}{z-\omega} + ...,
    \end{split}
\ese 
again from which we see we need $\l=2$ for $b$ to be have weight $(2,0)$. 

\br 
    Note that we have actually shown that $b$ and $c$ are primary operators. 
\er 

\subsection{The Ghost Central Charge}

So let's put in $\l=2$ to simplify $T_{\l}$ a bit. We get 
\be  
\label{eqn:TGhost}
    T(z) = 2\cl\p c(z) b(z)\cl + \cl c(z)\p b(z)\cl,
\ee 
where we have used the antisymmetry behaviour to get plus signs on both terms. 

We can now try and find out what the ghost central charge for theory is, by considering the $T(z)T(\omega)$ OPE. We have 
\bse 
    \begin{split}
        T(z)T(\omega) & = 4\Big[\cl\p c(z) b(z)\cl \cl\p c(\omega) b(\omega)\cl\Big] + 2\Big[ \cl\p c(z) b(z)\cl\cl c(\omega)\p b(\omega)\cl\Big] \\
        & \qquad + 2\Big[ \cl c(z)\p b(z)\cl \cl\p c(\omega) b(\omega)\cl\Big] + \Big[ \cl c(z)\p b(z)\cl \cl c(\omega)\p b(\omega)\cl\Big].
    \end{split}
\ese 
Let's consider this term by term. Firstly, we have (ignoring non-singular terms) 
\bse 
    \begin{split}
        \cl\p c(z) b(z)\cl \cl\p c(\omega) b(\omega)\cl & = \overbrace{b(z)\p c(\omega)} \overbrace{\p c(z) b(\omega)} + \overbrace{b(z)\p c(\omega)}\cl \p c(z) b(\omega)\cl \\
        & \qquad + (-1)^2\overbrace{\p c(z)b(\omega)}\cl b(z)\p c(\omega)\cl \\
        & = \bigg(\frac{1}{(z-\omega)^2}\bigg)\bigg(\frac{-1}{(z-\omega)^2}\bigg) + \frac{\cl \p c(z)b(\omega)\cl}{(z-\omega)^2} - \frac{ \cl b(z)\p c(\omega)\cl }{(z-\omega)^2} \\
        & = \frac{-1}{(z-\omega)^4} + \frac{\cl \p c(z)b(\omega)\cl}{(z-\omega)^2} - \frac{ \cl b(z)\p c(\omega)\cl }{(z-\omega)^2},
    \end{split}
\ese 
where we have used the antisymmetry behaviour when `jumping' $c$s and $b$s over each other\footnote{I.e. $\p c(z) b(z) \p c(\omega) b(\omega) = - b(z)\p c(z)\p c(\omega) b(\omega) = (-1)^2 b(z)\p c(z) b(\omega) \p c(\omega)$.} along with the fact that the contraction
\bse
    b(z)\p c(\omega) = \p_{\omega} \bigg(\frac{1}{z-\omega}\bigg) = + \frac{1}{(z-\omega)^2}
\ese
to get the signs correct. 

Similar calculations for the other terms gives 
\bse 
    \begin{split}
        \cl\p c(z) b(z)\cl\cl c(\omega)\p b(\omega)\cl & = \frac{-2}{(z-\omega)^4} - \frac{4\cl b(z)c(\omega)}{(z-\omega)^3} + \frac{\cl \p c(z) \p b(\omega)\cl}{z-\omega}, \\
        \cl c(z) \p b(z)\cl\cl \p c(\omega) b(\omega)\cl & = \frac{-2}{(z-\omega)^4} - \frac{2\cl c(z)b(\omega)\cl}{(z-\omega)^3} + \frac{\cl \p b(z)\p c(\omega)\cl}{z-\omega}, \\
        \cl c(z) \p b(z) \cl \cl c(\omega) \p b(\omega) \cl & = \frac{-1}{(z-\omega)^4} - \frac{\cl c(z) \p b(\omega) \cl}{(z-\omega)^2} + \frac{\cl \p b(z) c(\omega) \cl}{(z-\omega)^2}.
    \end{split}
\ese 
Putting all of this together, Taylor expanding, and using the antisymmetry behaviour we get 
\bse 
    T(z)T(\omega) = \frac{-13}{(z-\omega)^4} + \frac{2T(\omega)}{(z-\omega)^2} + \frac{\p T(\omega)}{z-\omega} + ...,
\ese 
which tells us 
\mybox{
\be 
\label{eqn:cghost}
    c_{\text{ghost}} = -26.
\ee 
}

\br 
    If we hadn't put $\l=2$ into the expressions above, we would have ended up with the result fourth-power singularity being 
    \bse 
        \frac{-1 +6\l -6\l^2}{(z-\omega)^4}.
    \ese 
    Of course if we put $\l=2$ here we get \Cref{eqn:cghost} back.
\er 

\section{The Critical Dimension}

Now let's recap what we have said and done. We started this lecture by reminding ourselves of the need for something that contributes negative central charge, or else the only Weyl invariant theory we have is the trivial one (i.e. the one with no bosonic terms). This was all because we need the \textit{total} central charge to vanish. We have just shown that the ghost action contributes $-26$ to the central charge, and so we now need something(s) in our system to contribute $+26$, giving us $c_{\text{total}} = 0$. We showed previously that each bosonic term contributed $+1$ to the central charge, and so one thing we could do is simply require that we have 26 of them! In other words, we want 
\bse 
    c_{\text{boson}} = +26. 
\ese 
This is exactly the result \Cref{eqn:d26}.

\subsection{$d=26$?}

There is a important point Dr. Tong brings up that is worth noting here. What we have really shown (from lecture 4 onwards) is that each bosonic term contributes $+1$ to the central charge and that $c_{\text{ghost}}=-26$. We then got excited about reproducing \Cref{eqn:d26} and said that we conclude that there are 26 boson terms. It is this last part that we are not fully justified in doing. 

Indeed \textit{if} there was 26 boson terms we would have $c_{\text{total}}=0$, and we would have Weyl invariance, however this is not the \textit{only} way we can get such a result. For example, if we were to introduce some new CFT (that in itself was Weyl invariant and all that good stuff) that contributed $c_{\text{new}} = +22$, we would then only need $c_{\text{boson}}=+4$, and so would have $d=4$. 

We should therefore not really call it the critical \textit{dimension}, but the critical \textit{central charge}. In other words, we can think of the space that gives $c=+26$ as the space of possible solutions to string theory. 

We did get the result that $d=26$ earlier in the course though, and so we shall take this to be our solution value. This subsection is just to point out that we have not actually shown here that we \textit{need} $d=26$. 

The particular case given above with $c_{\text{new}}=+22$ is sometimes called the \textit{internal sector} and is often interpreted as the idea of having 22 `hidden' dimensions in our $d=4$ Minkowski spacetime. 

\br 
    \textcolor{red}{Dr. Tong says he will give some examples in section 7, so if Dr. Minwalla doesn't do it, have a look and type it up. Section 5.3.}
\er 

\section{State Operator Map}

\subsection{Contour Manipulation}

We now want to work out the Virasoro algebra for the $bc$ CFT. The first thing we need to do is take another look at the contour manipulation we did back in Lecture 5. There we have \textit{bosonic} (i.e. symmetric) operators, whereas now we have fermionic (i.e. antisymmetric) ones, and so we need to rethink how to get the correct contour. 

The thing we have to remember is that operators appear in correlation functions in a time ordered manner, whereas in a path integral they need not. For the bosonic case this did not matter, because the operators commuted and so we could just switch the order of the operators in the path integral to be in time-ordered fashion. However, antisymmetric charges do not commute and will pick up a minus sign for every operator switched in the path integral. The operators are already ordered in the correlation function and so we do not pick up compensating minus signs from there. We must, therefore, come up with some prescription of how we want the path integral to look \textit{before} we take the correlation function. To be more specific
\bse 
    \int D\phi e^{iS[\phi]}\cO(t_n)...\cO(t_1) = \la \cO(t_n) ... \cO(t_1)\ra,
\ese 
for $t_n>t_{n-1}>...>t_1$. Now provided we do not change this $t_n$ condition, the right-hand side will always appear like that, no matter what order the operators appear on the left-hand side. So if we were clumsy and just assumed the operators could appear in any order on the left-hand side, provided the right-hand side is time ordered, we could write something like 
\bse 
    \int D\phi e^{iS[\phi]}\cO(t_n)...\cO(t_1)\cO(t_2) = \la \cO(t_n) ... \cO(t_1)\ra,
\ese
for $t_n>...>t_1$.

We know, however, that $\cO(t_1)\cO(t_2)=-\cO(t_2)\cO(t_1)$ and so the two left-hand sides differ by a minus sign, which gives us 
\bse 
    \la \cO(t_n) ... \cO(t_1)\ra = - \la \cO(t_n) ... \cO(t_1)\ra,
\ese
and so it must vanish! 

We must therefore be more careful then we (perhaps) were with the bosonic case, and \textit{first} ensure that the left-hand side is time ordered and \textit{then} take the correlation functions. So how does this translate into our contour integrals? Well let's assume $\widetilde{Q}^1$ and $\widetilde{Q}^2$ are fermionic charges given by 
\bse 
    \widetilde{Q}^i = \frac{1}{2\pi i}\oint \widetilde{J}^idz,
\ese
for some current $\widetilde{J}^i$. We want something of the form \Cref{eqn:ContourIntegral}. We have 
\bse 
    \widetilde{Q}^1(z_1)\widetilde{Q}^2(z_2) = \oint \frac{dz_1}{2\pi i}\oint \frac{dz_2}{2\pi i} \widetilde{J}^1(z_1)\widetilde{J}^2(z_2),
\ese 
and 
\bse 
    \widetilde{Q}^2(z_2)\widetilde{Q}^1(z_1) = \oint \frac{dz_1}{2\pi i}\oint \frac{dz_2}{2\pi i} \widetilde{J}^2(z_2)\widetilde{J}^1(z_1).
\ese
Now we can use the antisymmetric behaviour to change the latter equation to
\bse 
    \widetilde{Q}^2(z_2)\widetilde{Q}^1(z_1) = - \oint \frac{dz_1}{2\pi i}\oint \frac{dz_2}{2\pi i} \widetilde{J}^1(z_1)\widetilde{J}^2(z_2).
\ese 
It is important to note that this represents \textit{the exact same} contour. We therefore see that the fermionic version of \Cref{eqn:ContourIntegral} comes from
\bse 
    \widetilde{Q}^1(z_1)\widetilde{Q}^2(z_2) + \widetilde{Q}^2(z_2)\widetilde{Q}^1(z_1) =: \big\{\widetilde{Q}^1(z_1), \widetilde{Q}^2(z_2)\big\},
\ese 
which is the anticommutator.

\subsection{The $bc$ Anticommutator}

We can use the above formula for finding the anticommutator of two fermionic operators, provided they are holomorphic (the condition needed to do the contour integrals). Both $b$ and $c$ meet these conditions and so we can try find their anticommutator. 

We start by Laurent expanding 
\bse 
    b(z) = \sum_{n=-\infty}^{\infty} \frac{b_n}{z^{n+2}}, \qand c(z) = \sum_{n=-\infty}^{\infty} \frac{c_n}{z^{n-1}},
\ese 
where the $+2$ and $-1$ come because of the weights of the operators.\footnote{See \Cref{rem:LaurentZPowers} if you need reminding why.} We then invert these to give us 
\bse 
    b_n = \oint \frac{dz}{2\pi i}z^{n+1} b(z), \qand c_n = \oint \frac{dz}{2\pi i}z^{n-2} c(z).
\ese 
We therefore have 
\bse 
    \begin{split}
        \big\{c_n(z_1), b_m(z_2)\big\} & = \bigg(\oint \frac{dz_1}{2\pi i}\oint\frac{dz_2}{2\pi i} - \oint \frac{dz_2}{2\pi i}\oint\frac{dz_1}{2\pi i}\bigg) z_1^{n-2}z_2^{m+1} c(z_1)b(z_2) \\
        & = \oint \frac{dz_2}{2\pi i} \Res\bigg[\frac{z_1^{n-2}z_2^{m+1}}{z_{12}}\bigg] \\
        & = \oint \frac{dz_2}{2\pi i} z_1^{n-2}z_2^{m+1},
    \end{split}
\ese 
which vanishes unless $(n-2)+(m+1)=-1$. Using this along with the fact that the OPE of $c$ with itself and $b$ with itself vanish, we have
\mybox{
\be 
\label{eqn:cbanticommutator}
    \begin{split}
        \{c_n,b_m\} & = \del_{n+m,0} \\
        \{c_n,c_m\} & = 0 \\
        \{b_n,b_m\} & = 0.
    \end{split}
\ee 
}

\subsection{`Spin' States}

As before $n$ tells us the energy of the state and so operators with $n>0$ correspond to lowering operators and $n<0$ correspond to raising operators. In other words, the vacuum of the theory is the state such that 
\bse 
    c_n\ket{0} = 0, \qand b_n\ket{0} = 0,
\ese 
for all $n>0$. 

The next obvious question is "what are $c_0$ and $b_0$?" Let's label the state annihilated by $b_0$ as $\ket{\downarrow}$, i.e. 
\bse 
    b_0\ket{\downarrow} = 0.
\ese 
Let's also label the state 
\bse 
    \ket{\uparrow} := c_0\ket{\downarrow}.
\ese 
Now from the anticommutator $\{c_0,b_0\}=1$, we have 
\bse 
    b_0\ket{\uparrow} = b_0c_0\ket{\downarrow} = \ket{\downarrow} - c_0b_0\ket{\downarrow} = \ket{\downarrow}.
\ese 
We also have 
\bse 
    c_0\ket{\uparrow} = c_0c_0\ket{\uparrow} = 0.
\ese 
So we see that the $n=0$ part of the algebra forms a closed system with two states. Our vacuum state, therefore, is two-fold degenerate. That is, we can have either $\ket{\uparrow}$ or $\ket{\downarrow}$ as our starting point and build on these using the creation operators $c_n/b_n$ for $n<0$. 

This looks an awful lot like the spin state system (hence the labelling using arrows). Just as we say $\ket{\uparrow}$ has spin $+1/2$ and $\ket{\downarrow}$ has spin $-1/2$ for the spin system, we associate a \textit{ghost number} to the states $\ket{\uparrow}/\ket{\downarrow}$ here. We shall say $\ket{\uparrow}$ has ghost number $+1$ and $\ket{\downarrow}$ has ghost number $-1$. 

\br 
    Note that although we have an infinite number of excitations (as $n$ can take any value from $-\infty$ to $\infty$), we do not have the `double infinity' that we had for the bosonic fields.\footnote{See footnote 6 in Lecture 7.} This is because we cannot apply the same $c_n/b_n$ more then once without annihilating the system (as $c_n^2=0=b_n^2$). That is the \textit{occupation number} for each excitation state in the ghost system is $0$ or $1$. This is of course the characteristic behaviour of a fermionic state. 
\er 

\subsection{Identity State}

Let's consider the identity state $\ket{\b1}$. The state operator map tells us that 
\bse 
    c_n\ket{\b1} = \oint\frac{dz}{2\pi i} z^{n+2} c(z),
\ese 
which vanishes for $n\geq2$. So we have 
\bse 
    c_n\ket{\b1} = 0, \qquad \forall n\geq 2.
\ese 
Similarly we have 
\bse 
    b_n\ket{\b1} = 0, \qquad \forall b\geq -1.
\ese 
So what is the state $\ket{\b1}$? Well we see that it is annihilated by $b_0$ and so it must be related to $\ket{\downarrow}$. However, this is not the end of the story. We see that it is also annihilated by $b_{-1}$ and so it must already contain a $b_{-1}$ action (as $b_{-1}^2=0$). We see therefore
\be 
\label{eqn:IdentityStateBC}
    \ket{\b1} = b_{-1}\ket{\downarrow}.
\ee 
This makes sense with the $c_n$ condition as we do not expect it to vanish for $c_1$ as $\{c_1,b_{-1}\}=1$. 

\Cref{eqn:IdentityStateBC} tells us that the identity state is \textit{not} a lowest energy state (as was the case for the bosonic theory) but is actually an excited state with occupation number $1$ in the $b_{-1}$ sector. 

\subsection{The $\ket{c}$, $\ket{b}$, $\ket{\downarrow}$ \& $\ket{\uparrow}$ States}

We now want to find out what $\ket{\downarrow}$ and $\ket{\uparrow}$ are in terms of $b$ and $c$. 

First let's consider $\ket{\downarrow}$. We know it obeys 
\bse
    \begin{split}
        c_n\ket{\downarrow} & = 0, \qquad \forall n\geq 1 \\
        b_n\ket{\downarrow} & = 0, \qquad \forall n\geq 0,
    \end{split}
\ese 
the question is, how do we make this from $c$s and $b$s? Well we know that 
\bse 
    b_n\ket{\cO} = \Res\big[ z^{n+1}\la b(z)\cO(0)\ra\big],
\ese 
and so we need something who's OPE with $b(z)$ gives a $1/z$ factor (i.e. so that we end up with $\Res[z^n]$ which vanishes for all $n\geq0$). Well we know just an operator: $c$! 

The question is does $\ket{c}$ meet the $c_n$ condition? We have 
\bse 
    c_n\ket{c} = \oint\frac{dz}{2\pi i} z^{n-2} c(z)c(0).
\ese
At first site this seems to tell us that 
\bse 
    c_n\ket{c} = 0, \qquad \forall n\geq 2,
\ese 
however, we need to note that $c(z)c(0)=-c(0)c(z)$, which tells us that $c(z)c(0) \sim z$. So we 
\bse 
    c_n\ket{c} = 0, \qquad \forall n\geq1, 
\ese 
which is exactly our condition. So we conclude 
\be 
\label{eqn:cStateBC}
    \ket{\downarrow} = \ket{c}.
\ee 

This if our first result, now what about $\ket{\uparrow}$? This obeys 
\bse 
    \begin{split}
        b_n\ket{\downarrow} & = 0, \qquad \forall n\geq 1 \\
        c_n\ket{\downarrow} & = 0, \qquad \forall n\geq 0,
    \end{split}
\ese 
which compared to the $\ket{\downarrow}$ conditions might lead us to think that 
\bse 
    \ket{\uparrow} = \ket{b}.
\ese 
However this is not true. We see this instantly by considering 
\bse 
    b_n\ket{b} = \Res\big[ z^{n+1}b(z)b(0)\big],
\ese 
which from the same argument as for the $c\ket{c}$ calculation, tells us 
\bse 
    b_n\ket{b} = 0 \qquad \forall n\geq -2.
\ese 

So it's not $\ket{b}$, what else could it be? The answer comes through some careful observation. 
\bcl 
    \be 
    \label{eqn:upStateBC}
        \ket{\uparrow} = \ket{c\p c}.
    \ee 
\ecl 

\bq 
    We have 
    \bse 
        \begin{split}
            b_n\ket{c\p c} & = \Res\big[ z^{n+1}\la b(z)c(0)\p c(0)\ra \big] \\
            & \sim \Res\Bigg[ z^{n+1}\Bigg(\frac{1}{z} + \p\bigg(\frac{1}{z}\bigg)\Bigg)\Bigg] \\
            & = \Res\big[ z^n + z^{n-1}\big],
        \end{split}
    \ese 
    from which we get 
    \bse 
        b_n\ket{c\p c} = 0, \qquad \forall n\geq 1,
    \ese 
    our required condition. 
    
    A similar calculation shows that the $c_n$ condition is met. 
\eq 

For completeness we see that 
\be
\label{eqn:bStateBC}
    \ket{b} = b_{-1}b_{-2}\ket{\downarrow},
\ee 
which requires 
\bse 
    \begin{split}
        b_n\ket{b} & = 0, \qquad \forall n\geq -2, \\
        c_n\ket{b} & = 0, \qquad \forall n\geq 3.
    \end{split}
\ese

We have already seen the $b_n$ condition and the $c_n$ condition follows easily:
\bse 
    c_n\ket{b} = \Res\big[ z^{n-2}c(z)b(0)\big] = \Res\big[ z^{n-3}\big],
\ese 
and so 
\bse 
    c_n\ket{b} = 0, \qquad \forall n\geq 3. 
\ese 
\chapter{Ghost Current \& Operator Normal Ordering}

\section{Ghost Current}

\bd 
    We define the \textbf{ghost current} as the conformal normal order
    \bse 
        J_G(z) := \cl c(z)b(z)\cl
    \ese 
\ed 

Let's look at the OPE of $J_G$ with $c$ and $b$. We have 
\bse 
    J_G(z)c(\omega) = \frac{c(z)}{z-\omega} + ... = \frac{c(\omega)}{z-\omega} + ...,
\ese 
where as normal the ellipse is the non-singular terms. Similarly, we have 
\bse 
    J_G(z)b(\omega) = -\frac{b(\omega)}{z-\omega} + ...,
\ese 
where the minus sign comes from anticommuting $c(z)$ and $b(z)$. These results tell us that 
\be 
\label{eqn:GhostChargeCommutation}
    \begin{split}
        [Q_G,c] & = c, \\
        [Q_G,b] & = -b,
    \end{split}
\ee 
where 
\bse 
    Q_G := \oint \frac{dz}{2\pi i} J_G(z)
\ese
is the \textit{ghost charge}. This tells us that $c$ carries ghost charge $+1$ and $b$ carries ghost charge $-1$.\footnote{If you don't see why, consider the action $Q_Gc\ket{\cO}$, where $\ket{\cO}$ has vanishing ghost charge, i.e. $Q_G\ket{\cO}=0$. Do the same for $b$.}

\br 
    Note that the ghost charge is a conserved quantity in our $bc$ QFT. This is easily seen from \Cref{eqn:Sghostbc} as every $c$ appears with a $b$ so the transformations\footnote{By transformations we mean that $c\to ce^+$ and $b\to be^-$, analogously to in QFT.} cancel. Of course we need our theory to be ghost charge invariant, as it is the Nambu-Goto action we are trying to study, and this has no knowledge whatsoever about ghost fields, let alone ghost currents!
\er 

Let's now look at the OPE of $J_G$ with the stress tensor. We have\footnote{Here is an example of where we leave the value of $\l$ unspecified in order to get a result valid for our set of related CFTs, i.e. the ones differing only by the value of $\l$. Also note we have switched the order of $b$ and $\p c$ in the second term of the stree tensor to remove the minus sign.} 
\bse 
    \begin{split}
        T(z)J_G(\omega) & = \big[(1-\l)\cl\p b(z)c(z)\cl + \l\cl \p c(z)b(z)\cl \big] \cl c(\omega)b(\omega) \cl \\
        & = (1-\l)\big[\cl \p b(z)c(z)\cl \cl c(\omega)b(\omega) \cl\big] + \l \big[\cl \p c(z)b(z)\cl \cl c(\omega)b(\omega)\cl\big].
    \end{split}
\ese
Let's consider each term separately. We have (only considering singular terms)
\bse 
    \begin{split}
        \cl \p b(z)c(z)\cl \cl c(\omega)b(\omega)\cl & = - \cl c(z) \p b(z)\cl \cl c(\omega)b(\omega)\cl \\ 
        & = -\Bigg[ \p_z\bigg(\frac{1}{z-\omega}\bigg)\bigg(\frac{1}{z-\omega}\bigg) + \cl c(z)b(\omega) \cl \p_z\bigg(\frac{1}{z-\omega}\bigg) + \frac{\cl \p b(z)c(\omega)}{z-\omega} \Bigg] \\
        & = \frac{1}{(z-\omega)^3} + \frac{\cl c(z)b(\omega)\cl}{(z-\omega)^2} - \frac{\cl \p b(z) c(\omega)\cl }{z-\omega} \\
        & = \frac{1}{(z-\omega)^3} + \frac{\cl c(\omega)b(\omega)\cl}{(z-\omega)^2} + \frac{\cl \p c(\omega) b(\omega)\cl - \cl \p b(\omega) c(\omega)\cl}{z-\omega} \\
        & = \frac{1}{(z-\omega)^3} + \frac{J(\omega)}{(z-\omega)^2}+ \frac{\p J(\omega)}{z-\omega},
    \end{split}
\ese 
and 
\bse 
    \begin{split}
        \cl \p c(z)b(z)\cl \cl c(\omega)b(\omega)\cl & = \p_z\bigg(\frac{1}{z-\omega}\bigg)\bigg(\frac{1}{z-\omega}\bigg)  + \cl b(z)c(\omega)\cl \p_z\bigg(\frac{1}{z-\omega}\bigg) + \frac{\cl \p c(z)b(\omega)\cl}{z-\omega} \\
        & = - \frac{1}{(z-\omega)^3} - \frac{\cl b(z)c(\omega)\cl}{(z-\omega)^2} + \frac{\cl \p c(z)b(\omega)\cl}{z-\omega} \\
        & = -\frac{1}{(z-\omega)^3} - \frac{\cl b(\omega)c(\omega)\cl}{(z-\omega)^2} + \frac{\cl \p c(\omega)b(\omega)\cl - \cl \p b(\omega)c(\omega)\cl }{z-\omega} \\
        & = -\frac{1}{(z-\omega)^3} + \frac{\cl c(\omega)b(\omega)\cl}{(z-\omega)^2} + \frac{\cl \p c(\omega)b(\omega)\cl - \cl \p b(\omega)c(\omega)\cl }{z-\omega} \\
        & = -\frac{1}{(z-\omega)^3} + \frac{J(\omega)}{(z-\omega)^2}+ \frac{\p J(\omega)}{z-\omega},
    \end{split}
\ese 
We then note that the last two terms in both these expressions are identical, and so they will not appear with a $\l$ in the OPE. We are then left with 
\be
\label{eqn:TJGOPE}
    T(z)J_G(\omega) = \frac{(1-2\l)}{(z-\omega)^3} + \frac{J(\omega)}{(z-\omega)^2} + \frac{\p J(\omega)}{z-\omega} + \text{non-singular}.
\ee 

\br 
    Note this result tells us that 
    \bse 
        L_1(z)J_G(0) = \Res\bigg[z^2\bigg(\frac{(1-2\l)}{z^3} + \frac{J(0)}{z^2} + \frac{\p J(0)}{z}\bigg)\bigg] \neq 0,
    \ese 
    (provided $\l\neq1/2$) and so we see that $J_G$ is a secondary operator. 
\er 

\section{Ghost Anomaly}

Recall that the OPE of the stress tensor with itself contained a $1/z^4$ term, and we showed/argued (at the start of Lecture 7) that this term told us that $T$ must also transform under Weyl transformations as well as diffeomorphisms. We now want to apply the same idea here and argue that the $1/z^3$ term arises from the Weyl transformations on the ghost current. 

Proceeding as we did in Lecture 7, we have 
\bse 
    \del_WJ_G(\omega) = \frac{(1-2\l)}{2}\epsilon''(\omega).
\ese 
Then recalling that $\del\phi = \epsilon'$, we have 
\bse 
    \del_WJ_G(\omega) = \frac{(1-2\l)}{2}\p(\del\phi).
\ese 
We then have 
\bse 
    \del_W(\p\cdot J_G) = \frac{(1-2\l)}{2}\nabla^2(\del\phi). 
\ese
Putting this together with the fact that $J_G$ has weight $(1,0)$, and so $\p\cdot J_G$ has weight $(2,0)$ we see that (after covariantising) 
\bse 
    \nabla \cdot J_G \sim R,
\ese
the Ricci scalar. We state (without proof\footnote{See Polchinski Vol 1, exercise 3.6, page 119 for help.}) that the proportionality is given by 
\be 
\label{eqn:GhostAnomaly}
    \nabla \cdot J_G = \frac{(1-2\l)}{4}R.
\ee 
This is known as the \textit{ghost anomaly}.

\br 
    \textcolor{red}{There is an aside here about $\l=1/2$ corresponding to free fermions and the Dirac action. Come back and add this in later. (around 30:00-40:00 in video)}
\er 

\Cref{eqn:GhostAnomaly} tells us that (for a non-flat manifold) the ghost current is not locally conserved. This doesn't sound great. However, there is an even bigger problem: it's possible that the charge isn't even conserved globally. 

Recall\footnote{Or look up if you're not familiar.} that the Euler characteristic (for a closed, orientable surface) is
\bse 
    \chi = 2(1-g),
\ese
where $g$ is the \textit{genus} of the surface. We can also write the Euler characteristic as
\bse 
    \chi = \frac{1}{4\pi}\int d^2\sig \sqrt{g}R,
\ese 
and so we see that\footnote{Annoyingly we use $g$ for both the genus and the metric, it should be clear which is which though.} 
\bse 
    \int d^2\sig \sqrt{g} \, \nabla \cdot J_G = 2\pi(1-2\l)(1-g).
\ese 

Now recall that in Lecture 4 we used Noether's theorem to derive the Ward identity. The trick we played there was to let the the infinitesimal $\epsilon$ be a function of the coordinates and then insisted that the result vanished for constant $\epsilon$. This vanishing condition was to do with the fact that $J$ was locally conserved. We no longer have that, and so we see that the action does actually change even for constant $\epsilon$! That is, for $\phi'=\phi+\epsilon\del\phi$

\bse 
    \frac{S_{\text{ghost}}[\phi']}{S_{\text{ghost}}[\phi]} = \frac{i\epsilon}{2\pi} \int d^2\sig \sqrt{g} \, \nabla \cdot J_G = i(1-2\l)(1-g).
\ese
Now we have that 
\bse 
    c(\sig) \to e^{i\epsilon} c(\sig), \qand b(\sig) \to e^{-i\epsilon}b(\sig)
\ese 
for our transformation. We then imagine inserting $f$ $c$s and $m$ $b$s into our path integral. Then for a non-vanishing path integral, the transformations from the measure (which we assume is nothing, as before), action and insertions must cancel, and so we get 
\be 
\label{eqn:cbnumbergenus}
    f-m +(1-2\l)(1-g) = 0 \qquad \implies \qquad  m-f = (2\l-1)(g-1).
\ee 
So for our ghost system (with $\l=2$), we have 
\be 
\label{eqn:cbnumbergenusghost}
    m-f = 3(g-1).
\ee 
Finally recall that in Lecture 10 we showed that unless there are the same number of insertions of $c$ as their are moduli and the same number of insertions of $b$ as there are conformal Killing vector fields, our path integral vanishes. We labelled these numbers $f$ and $m$ in Lecture 10. Now the use of $f$ and $m$ in \Cref{eqn:cbnumbergenusghost} is not poor notation, but done deliberately, as the next theorem explains. 

\bt 
    Let $f$ denote the number of moduli and $m$ the number of conformal Killing vector fields, then the their difference must obey 
    \bse 
        m-f=3(g-1),
    \ese 
    where $g$ is the genus of the surface. 
\et 

We do not prove this theorem, but instead just show that it holds for the sphere. 

\bex 
    We have already seen that for $g=0$ (i.e. the sphere) there are 3 CKVFs (namely $\mathfrak{sl}(2,\C)$). We shall present the proof here in a slightly different way. We know that the $c$s are our CKVFs (they are the unfixed diffeomorphisms that keep us in the same gauge orbit). We also know that they are weight $(1,0)$ holomorphic functions, i.e. 
    \bse 
        c = \sum_n a_nz^n,
    \ese 
    for some $n$. We obviously require that they are everywhere well defined. Our two problem areas are $z\to0$ and $z\to\infty$. The former condition clearly just requires $n\geq0$, but what about the second condition. Well recall that our notation is $c := c^z$, and so we know 
    \bse
        c^z = \frac{\p z}{\p \omega}c^{\omega},
    \ese
    for some transformation $z\to\omega(z)$. We take this transformation to be $\omega=1/z$, as this is where the $z\to\infty$ problem arises. We therefore have 
    \bse 
        c^{z} = -\omega^2\sum_n \frac{a_n}{\omega^n},
    \ese 
    and so we require $n\leq 2$. Putting this together we get 3 allowed values, $n=0,1,2$. 
    
    Repeating this calculation for $b:=b_{zz}$, and noting that the indices are down, we have
    \bse 
        b_{zz} = \bigg(\frac{\p \omega}{\p z}\bigg)^2b_{\omega\omega},
    \ese 
    which gives 
    \bse 
        b_{zz} = \frac{1}{\omega^4} \sum_n\frac{a_n}{\omega^n},
    \ese 
    which diverges unless $n\leq -4$. But we still obviously require $n\geq0$, and so there are no everywhere well defined $b$ fields. Finally just recall that the $b$s are the moduli, and you have 
    \bse 
        0-3=3(0-1),
    \ese 
    the required condition. 
\eex 

\bcl 
    The number of conformal Killing vector fields on the torus is $1$ and the number of moduli is $1$.
\ecl 

We do not prove this claim yet,\footnote{\textcolor{red}{Dr. Minwalla says we will return to this later. If he doesn't come back and try prove it.}} but just present it now as a further `proof' of the above theorem, i.e. $1-1=0=3(1-1)$.

\br 
    Note that the above theorem does not contrast with \Cref{eqn:cbnumbergenus}, in the sense that $\l$ need not be $2$. It simply says that \textit{if} $\l=2$ then the difference in the number of $c$ and $b$ insertions needs to be equal to the difference in the number of moduli and CKVFs, or our path integral must vanish. Of course this is the result we obtained in Lecture 10, just derived from a different approach.  
\er 

\br 
    \Cref{eqn:cbnumbergenus} actually gives us a constraint on the allowed values for $\l$ on a surface of well defined genus, namely 
    \bse 
        \l = \frac{1}{2}\bigg(1+ \frac{A}{B}\bigg),
    \ese 
    where $A=m-f$ and $B=g-1$ are integers.
\er 

\section{Operator Normal Ordering}

When we first encounter normal ordering it is in QFT and its done with the idea of putting all the annihilation operators to the right, so that it annihilates the vacuum. We introduced conformal normal ordering in order to get expectation values that vanish. We now want to introduce another new type of normal ordering in line with the first; we want to define a normal ordering that annihilates the ground state of our $bc$ CFT.

We have seen that $c_n/b_n$ are annihilation operators for $n>0$. On top of this we have seen that our $bc$ CFT has a doubley degenerate ground state (the states $\ket{\uparrow}$ and $\ket{\downarrow}$. We therefore want to place all the positive $n$ terms to the right. On top of this, we have seen that $b_0$ annihilates $\ket{\downarrow}$ and $c_0$ annihilates $\ket{\uparrow}$. We need to pick which ground state we want to be annihilated, and here we pick $\ket{\downarrow}$. So we want the $b_0$ term to appear on the right. 

Let's work out how this relates to $c(z_1)b(z_2)$ itself. We have the Laurent expansions 
\bse 
    c(z) = \sum_{m=-\infty}^{\infty}\frac{c_m}{z^{m-1}}, \qand b(z) = \sum_{m=-\infty}^{\infty}\frac{c_m}{z^{m+2}}.
\ese 
Then recalling $\{c_m,b_n\}=\del_{m+n,0}$ and all other anticommutators vanishing, we see that the only terms that need changing in $c(z_1)b(z_2)$ are the 
\bse 
    \sum_{m=1}^{\infty} \frac{c_mb_{-m}}{z_1^{m-1}z_2^{-m+2}} = \sum_{m=1}^{\infty} \bigg(\frac{1}{z_1^{m-1}z_2^{-m+2}} - \frac{b_{-m}c_{m}}{z_1^{m-1}z_2^{-m+2}}\bigg)
\ese 
terms. Using 
\bse 
    \begin{split}
        \sum_{m=1}^{\infty}\frac{1}{z_1^{m-1}z_2^{-m+2}} & = \frac{z_1}{z_2^2}\sum_{m=1}^{\infty}\bigg(\frac{z_2}{z_1}\bigg)^m \\
        & = \frac{z_1}{z_2^2} \bigg(\frac{z_2/z_1}{1-z_2/z_1}\bigg) \\
        & = \frac{z_1}{z_2}\bigg(\frac{1}{z_{12}}\bigg)
    \end{split}
\ese 
we have the following definition.
\bd 
    We define the \textbf{operator normal order} of two fields in our $bc$ CFT as 
    \be 
    \label{eqn:operatornormalordering}
        \tcl c(z_1)b(z_2) \tcl := c(z_1)b(z_2) - \frac{z_1}{z_2}\bigg(\frac{1}{z_{12}}\bigg)
    \ee 
\ed 

Now, the conformal normal order for our $bc$ CFT is 
\bse 
    \cl c(z_1)b(z_2) \cl = c(z_1)b(z_2) - \frac{1}{z_{12}}.
\ese 
Therefore the difference is
\bse 
    \cl c(z_1)b(z_2)\cl - \tcl c(z_1)b(z_2)\tcl = \frac{1}{z_{12}}\bigg(\frac{z_1}{z_2}-1\bigg) = \frac{1}{z_2}.
\ese 
Taking the limit $z_1\to z_2 =z$, we have 
\bse 
    J_G^{\text{conf}}(z) - J_G^{\text{op}}(z) = \frac{1}{z}.
\ese
Next, we recall that $J_G^{\text{conf}}(z)=\cl c(z)b(z)\cl$, so in the $c/b$ Laurent expansions we have 
\bse 
    L_G^{\text{conf}} = \sum_{m=-\infty}^{\infty} \frac{J_m^{\text{conf}}}{z^{m+1}},
\ese
and similarly for $J_G^{\text{op}}$. Putting this together with the result above we conclude 
\bse 
    J_0^{\text{conf}} - J_0^{\text{op}} = 1.
\ese
This result tells us that the ghost charge of the $\ket{\downarrow}$ ground state is $+1$. We can similarly show that the ghost charge of the $\ket{\uparrow}$ ground state is $+2$. 

\br 
    We actually could have got the above result very quickly at the start of this lecture. We showed that $c$ has ghost charge $+1$ at the start of this lecture and at the end of the last lecture we showed that $\ket{\downarrow}=\ket{c}$ and $\ket{\uparrow}=\ket{c\p c}$. We therefore have: ghost number for $\ket{\downarrow}=+1$, and ghost number for $\ket{\uparrow}=+1+1=+2$.
\er 

\br 
    Note that in introducing operator normal ordering, we have broken translation invariance. This is easily seen in \Cref{eqn:operatornormalordering} as $z_1/z_2$ is not translation invariant. It is therefore not a very good tool to use on the radial quantisation plane, but is just presented here to show that conformal normal ordering is indeed related to what we normally think of as normal ordering (i.e. annihilating the ground state).
\er 
\chapter{BRST Quantisation I}

Let's recall what we are trying to understand: the scattering amplitude, given by 
\bse 
    \begin{split}
        \cA  =  \int d^m\tau DXDcDb \bigg[\prod_{k=1}^m \int d^2\sig \sqrt{\hat{g}} \bigg(\frac{\p \hat{g}}{\p \tau_k}\bigg)_{\a\beta} b^{\a\beta} \bigg] & \bigg[ \prod_{j=1}^f \prod_{\a=1}^2 c_{\a}(\hat{\sig}_j)  \sqrt{g} V_j(\hat{\sig}_j) \bigg] \\
        &\times  \bigg[\prod_{j=f+1}^n \int d\sig_j \sqrt{g} V_j(\sig_j)\bigg] e^{-(S_{\text{P}} + S_{\text{g}})}
    \end{split}
\ese 
So far we have basically understood what all the terms in this expression mean, apart from the $V$s. We have argued previously that they must be diffeomorphism scalars, but must have a Weyl factor opposite to that of $\sqrt{g}$. In other words they need to be $(1,1)$-primary scalars. The obvious question to ask is "$(1,1)$ with respect to which CFT?" That is, are they purely to do with the $X^{\mu}$ CFT or purely the $bc$ CFT, or a mix of both? 

If we remember that the $V$s represent vertices, intuitively we might think `well it's not purely $bc$ as ghost things give us negative norms, and we don't want that!' A similar argument would suggest that we don't want it to be a mix of $X^{\mu}$s and $bc$. 

The next few lectures are going to be leading to answering this question by introducing and using so-called BRST\footnote{Becchi, Rouet, Stora and Tyutin.} quantisation. As the name suggests, BRST quantisation is some quantisation scheme, the question is what does it seek to quantise? The answer is field theories that possess gauge symmetries. This is exactly our case, and so we use it here. 

\section{The BRST Action}

For generality, we shall just label the classical fields by $\phi_i$. Our gauge transformation is 
\be
\label{eqn:delphiBRST}
    \del\phi_i = \epsilon^{\a}\del_{\a}\phi_i,
\ee 
for some small $\epsilon$ and where $\a$ includes the coordinate. These gauge transformation satisfy an algebra with structure constants given by 
\be
\label{eqn:BRSTGaugeAlgebra}
    [\del_{\a},\del_{\beta}] = {f^{\g}}_{\a\beta}\del_{\g}.
\ee 

So we have some action $S[\phi]$ which is invariant under the gauge transformation. But in order to actually work out the path integral, we would need to fix a gauge. We could do this by introducing some functions $F^A$ such that 
\bse 
    F^A(\phi_c)=0,
\ese 
where again $A$ includes the coordinates. We are basically repeating the Faddeev-Popov calculations from before. So we want to find 
\bse 
    Z = \frac{1}{\text{Vol}}\int D\phi e^{-S[\phi]}.
\ese 
We introduce 
\bse 
    1 = \int D\xi \, \del\Big( F^A\big(\phi_c^{\xi}\big)\Big) \Delta_{FP}\big(\phi_c\big),
\ese 
where $\phi_c$ is the needed $\phi$ for the delta function.\footnote{Recall we can do this because $\Delta_{FP}$ is invariant under the gauge transformation, and so we can pick any value in each orbit we like.} We can then consider the infinitesimal gauge transformation $\epsilon^{\a}$ and Taylor expand the $F^A$s, giving, 
\bse 
    \frac{1}{\Delta_{FP}} = \int D\epsilon^{\a} \del\Big( F^A\big(\phi_c\big) + \del_{\a}F^A\epsilon^{\a}\Big) = \int D\epsilon^{\a} \del\big( \del_{\a}F^A\epsilon^{\a}\big),
\ese 
where 
\bse 
    \del_{\a}F^A := \frac{\p F^A}{\p \phi_i}\del_{\a}\phi_i.
\ese 
Fourier tranforming gives 
\bse 
    \frac{1}{\Delta_{FP}} = \int D\epsilon^{\a}\, DB_A \, \exp\big(-iB_A\del_{\a}F^A\epsilon^{\a}\big).
\ese 
Then doing the trick of replacing the Bosonic fields with Fermionic ones and flipping the left-hand side, we get 
\bse 
    \Delta_{FP} = \int Dc^{\a} Db_A \exp\big(i b_A\del_{\a}F^Ac^{\a}\big).
\ese 
So we have (picking some $\phi$)
\bse 
    1 = \int d\xi Dc^{\a} Db_A \del\big(F^A(\phi)\big)\exp\big(i b_A\del_{\a}F^Ac^{\a}\big).
\ese 
Now we do not actually do the delta function integral here (as we did earlier in the notes), but instead just Fourier transform to give 
\bse 
    1 = \int d\xi DB_ADb_Adc^{\a} \exp\Big(iB_AF^A(\phi) +ib_A\del_{\a}F^A(\phi)c^{\a}\Big),
\ese 
and so we get (after doing the $\xi$ integral to remove the Vol factor),
\bse 
    \frac{1}{\text{Vol}} \int D\phi \, e^{-S[\phi]} = \int D\phi DB_ADb_Adc^{\a} e^{-S_1-S_2-S_3},
\ese 
where $S_1[\phi]$ is our initial action, $S_2$ is our gauge fixing condition
\bse 
    S_2 = -iB_AF^A(\phi),
\ese 
and $S_3$ is the Faddeev-Popov action
\bse 
    S_3 = b_Ac^{\a} \del_{\a}F^A(\phi),
\ese 
where $b_A$ has been rescaled to absorb the $i$ factor present above.

\section{The BRST Symmetry}

Now this seems like we've gone backwards: in trying to apply the gauge symmetry we have obtained the same action plus two other actions! So we need to check that there is indeed some symmetry. 

\bcl 
    The BRST action is invariant under the so-called BRST transformation
    \be 
    \label{eqn:BRSTTransformation}
        \begin{split}
            \del_B \phi_i & = -i\epsilon c^{\a} \del_{\a}\phi_i, \\
            \del_BB_A & = 0, \\
            \del_Bb_A & = \epsilon B_A, \\
            \del_Bc^{\a} & = \frac{i}{2} \epsilon {f^{\a}}_{\beta\g} c^{\beta}c^{\g},
        \end{split}
    \ee 
    for some small parameter $\epsilon$.
\ecl 

\br 
\label{rem:BRSTEpsilonAntisymmetric}
    Before proving the above, note the $\epsilon$ here is not the same as the bosonic $\epsilon^{\a}$ in \Cref{eqn:delphiBRST}, which has been replaced by the fermionic $c^{\a}$. In fact $\epsilon$ here needs to anticommute with $c$ in order to insure that $\del_B\phi$ remains bosonic. Similarly it must anticommute with $b_A$ so that $\del_Bb_A$ is fermionic. 
\er 

\bq 
    By assumption, $S_1$ is invariant as it only sees the $\del_B\phi_i$ and \Cref{eqn:delphiBRST} tells us its invariant under this. So we just need to check that the terms appearing from $S_2$ and $S_3$ cancel. Let's consider each separately: 
    \bse 
        \begin{split}
            \del_BS_2 & = -i \big(\del_BB_A\big)F^A(\phi) -iB_A\del_BF^{A}(\phi) \\
            & = -iB_A \frac{\p F^A}{\p \phi_i}\del_B\phi_i \\
            & = -\epsilon B_A\frac{\p F^A}{\p \phi_i} c^{\a}\del_{\a}\phi_i \\
            & =: -\epsilon B_A c^{\a}\del_{\a}F^A(\phi).
        \end{split}
    \ese 
    Next we have
    \bse 
        \begin{split}
            \del_BS_3 & = \big(\del_Bb_A\big)c^{\a} \del_{\a} F^A(\phi) + b_A\big(\del_Bc^{\a}\big)\del_{\a}F^A(\phi) + b_Ac^{\a}\del_B\big(\del_{\a}F^A(\phi)\big) \\
            & = \epsilon B_A c^{\a}\del_{\a}F^A(\phi) + \frac{i}{2}\epsilon b_A {f^{\a}}_{\beta\g}c^{\beta}c^{\g} \del_{\a}F^A(\phi) -i b_Ac^{\a}\epsilon c^{\beta}\del_{\beta}\del_{\a}F^A(\phi) \\
            & = \epsilon\Big[ B_A c^{\a}\del_{\a}F^A(\phi) + \frac{i}{2} b_A {f^{\a}}_{\beta\g}c^{\beta}c^{\g} \del_{\a}F^A(\phi) - i b_Ac^{\beta} c^{\a}\del_{\beta}\del_{\a}F^A(\phi) \Big],
        \end{split}
    \ese 
    where we have used $c^{\a}\epsilon=-\epsilon c^{\a}$ and $c^{\a}c^{\beta}=-c^{\beta}c^{\a}$ to go to the last line. We then note that $c^{\a}c^{\beta}$ is antisymmetric, and so
    \bse 
        c^{\beta}c^{\a}\del_{\beta}\del_{\a}F^A(\phi) =  \frac{1}{2}c^{\beta}c^{\a}[\del_{\beta},\del_{\a}]F^A(\phi) = \frac{1}{2}c^{\beta}c^{\a}{f^{\g}}_{\beta\a}\del_{\g}F(\phi).
    \ese 
    We see therefore that the last two terms in $\del_BS_3$ cancel and we simply get 
    \bse 
        \del_BS_3 = \epsilon B_Ac^{\a}\del_{\a} F^A(\phi).
    \ese
    Finally we see that 
    \bse 
        \del_BS_2+\del_BS_3 = -\epsilon B_Ac^{\a}\del_{\a} F^A(\phi) +\epsilon B_Ac^{\a}\del_{\a} F^A(\phi) = 0.
    \ese 
\eq 

\bp  
\label{prop:BRSTNilpotent}
    The BRST transformation is nilpotent. That is 
    \bse 
         \del_B^{\epsilon_2}\del_B^{\epsilon_1} =: \del_B^2 = 0,
    \ese 
    where the superscripts tells us that we need not take the same value of $\epsilon$ for each application of $\del_B$.
\ep  

\bq 
    We just have to show that it holds for each of the terms in \Cref{eqn:BRSTTransformation}. Clearly 
    \bse 
        \del_B^2 B_A = 0, \qand \del_B^2b_A = \epsilon_1\del_B^{\epsilon_2}B_A = 0.
    \ese 
    Now consider $\del_B^2\phi_i$:
    \bse 
        \begin{split}
            \del_B^2\phi_i & = -i\epsilon_1\del_B^{\epsilon_2}\big(c^{\a}\del_{\a}\phi_i\big) \\
            & = -i\epsilon_1\bigg[\frac{i}{2}\epsilon_2{f^{\a}}_{\beta\g}c^{\beta}c^{\g} \del_{\a}\phi_i -i c^{\beta}\epsilon_2c^{\a} \del_{\beta}\del_{\a}\phi_i\bigg] \\
            & = \frac{\epsilon_1\epsilon_2}{2}\Big[ {f^{\a}}_{\beta\g} c^{\beta}c^{\g} \del_{\a}\phi_i - c^{\a}c^{\beta}{f^{\g}}_{\a\beta}\del_{\g}\phi_i\Big] \\
            & = 0,
        \end{split}
    \ese 
    where we have used the same trick on the last term as in the proof of the above claim. 
    
    Finally we have 
    \bse 
        \begin{split}
            \del_B^2c^{\a} & = \frac{i}{2}\epsilon_1{f^{\a}}_{\beta\g}\del_B^{\epsilon_2}\Big(c^{\beta}c^{\g}\Big) \\
            & = -\frac{1}{4}\epsilon_1 {f^{\a}}_{\beta\g}\Big[\epsilon_2 {f^{\beta}}_{\sig\rho}c^{\sig}c^{\rho}c^{\g} + {f^{\g}}_{\sig\rho}c^{\beta}\epsilon_2c^{\sig}c^{\rho} \Big] \\
            & = -\frac{1}{4}\epsilon_1\epsilon_2 {f^{\a}}_{\beta\g}\Big[ {f^{\beta}}_{\sig\rho}c^{\sig}c^{\rho}c^{\g} - {f^{\g}}_{\sig\rho}c^{\beta}c^{\sig}c^{\rho} \Big] \\
            & = \frac{1}{4}\epsilon_1\epsilon_2 \Big[ {f^{\a}}_{\g\beta}{f^{\beta}}_{\sig\rho}c^{\g}c^{\sig}c^{\rho} + {f^{\a}}_{\beta\g}{f^{\g}}_{\sig\rho}c^{\beta}c^{\sig}c^{\rho} \Big] \\
            & = \frac{1}{2}\epsilon_1\epsilon_2{f^{\g}}_{\sig\rho}{f^{\a}}_{\beta\g}c^{\beta}c^{\sig}c^{\rho},
        \end{split}
    \ese 
    where we have used the fact that ${f^{\a}}_{\beta\g} = {f^{\a}}_{-\g\beta}$, as is easily seen from \Cref{eqn:BRSTGaugeAlgebra}, and then relabelled $\g\leftrightarrow\beta$ in the first term to get to the last line. This looks like a problem (it doesn't appear to be zero...), however we see it does indeed vanish by considering the Jacobi of our algebra. We have 
    \bse 
        \big[\del_{\beta},[\del_{\sig},\del_{\rho}]\big] = {f^{\g}}_{\sig\rho}{f^{\a}}_{\beta\g}\del_{\a},
    \ese 
    and so the Jacobi identity tells us 
    \bse 
        {f^{\g}}_{\sig\rho}{f^{\a}}_{\beta\g} + {f^{\g}}_{\beta\sig}{f^{\a}}_{\rho\g} + {f^{\g}}_{\rho\beta}{f^{\a}}_{\sig\g} = 0.
    \ese 
    We then use the anticommutivity of the $c$s to give us 
    \bse 
        \begin{split}
            \del_B^2 c^{\a} & = \frac{1}{6}\epsilon_1\epsilon_2{f^{\g}}_{\sig\rho}{f^{\a}}_{\beta\g}\big[ c^{\beta}c^{\sig}c^{\rho} + c^{\rho}c^{\beta}c^{\sig} + c^{\sig}c^{\rho}c^{\beta} \big] \\
            & = \frac{1}{6}\epsilon_1\epsilon_2\Big[ {f^{\g}}_{\sig\rho}{f^{\a}}_{\beta\g} + {f^{\g}}_{\beta\sig}{f^{\a}}_{\rho\g} + {f^{\g}}_{\rho\beta}{f^{\a}}_{\sig\g} \Big] c^{\beta}c^{\sig}c^{\rho} \\
            & = 0,
        \end{split}
    \ese 
    where we have done some relabelling to get the $ff$ terms.
\eq 

We can express the above result quite nicely in terms of some generator $Q_B$ of the BRST transformation, simply as 
\be 
\label{eqn:QBSquared}
    Q_B^2 = 0.
\ee 
We write the action of $Q_B$ on the fields using commutators for the Bosonic fields, e.g. $\del_B\phi_i = \epsilon[Q_B,\phi_i]$, and anticommutators for the Fermionic fields, e.g. $\del_Bb_A = i\epsilon\{Q_B,b_A\}$. Putting this together, \Cref{eqn:BRSTTransformation} can be written
\be
\label{eqn:BRSTTransformationQB}
    \begin{split}
            \big[Q_B,\phi_i\big] & = -ic^{\a} \del_{\a}\phi_i, \\
            \big[Q_B,B_A\big] & = 0, \\
            \big\{Q_B,b_A\big\} & = -iB_A, \\
            \big\{Q_B,c^{\a}\big\} & = \frac{1}{2}{f^{\a}}_{\beta\g} c^{\beta}c^{\g}. 
    \end{split}
\ee 

\br 
    Note that the anticommutator relations come with a factor of $i$. This will prove importatnt below when proving \Cref{prop:BRSTHermitian}.
\er 

\br 
    We should note two assumptions we have made above. Firstly, we have assumed that the structure constants ${f^{\a}}_{\beta\g}$ are indeed constants and not functions of the fields (if they were we would pick up some contribution from then under $\del_B$). Secondly, we have also assumed that there are no additional terms on the right-hand side of \Cref{eqn:BRSTGaugeAlgebra} that are proportional to the equations of motion. Both of these assumptions are needed to show that $Q_B$ is nilpotent. The generalisation of the results is known as the \textit{Batalin–Vilkovisky} (or BV) formalism. We will not discuss this further here.
\er 

\section{Physical States}

So we have studied our BRST transformation and its symmetry, the question now is "what does this tell us about the physical states that are allowed?" To see why the BRST symmetry is related to such a question, recall that $F^A(\phi)=0$ is our gauge fixing condition. Changing $F^A$ corresponds to changing gauge, and so should have \textit{no effect} on the physical states of the system. That is, any state that is dependent on the choice of gauge is defined to be unphysical. 

So what do we do? Firstly we notice that 
\bse 
    i\epsilon\big(S_2+S_3\big) = \del_B\big(b_AF^A\big),
\ese 
which can be checked easily. So our total action becomes 
\bse 
    S = S_1 -\frac{i}{\epsilon}\del_B\big(b_AF^A\big),
\ese
or equivalently 
\bse 
    S = S_1 + \big\{Q_B,b_AF^A\big\}.
\ese 
If we then take the transition amplitude 
\bse 
    \cT_{if}  = \bra{f} e^{-S} \ket{i},
\ese 
and consider a small gauge transformation $\del F^A$, we get 
\bse 
    \del\cT_{if} = \bra{f}\big\{Q_B,b_A\del F^A\big\}\ket{i},
\ese 
which we require to vanish for arbitrary $\del F^A$ for physical states. We therefore require that the physical states be $Q_B$ exact. That is 
\bse
    Q_B\ket{\psi} = 0 = \bra{\psi}Q_B^{\dagger}.
\ese 
for all \textit{physical} states $\ket{\phi}$.

\bp
\label{prop:BRSTHermitian}
    The BRST charge is Hermitian, i.e. 
    \be 
    \label{eqn:QBHermitian}
        Q_B = Q_B^{\dagger}.
    \ee 
\ep 

\bq 
    Firstly we note that 
    \bse 
        [A,B]^{\dagger} = -[A^{\dagger},B^{\dagger}], \qand \{A,B\}^{\dagger} = \{A^{\dagger},B^{\dagger}\}
    \ese
    Next note that $\phi_i^{\dagger}=\phi_i$, $(c^{\a})^{\dagger}=c^{\a}$, $B_A^{\dagger}=B_A$ and $b_A^{\dagger}=-b_A$.\footnote{Recall that we absorbed a factor of $i$ into $b_A$.} 
    
    Using both these results along with \Cref{eqn:BRSTTransformationQB}, we have 
    \bse 
        \begin{split}
            -\big[Q_B^{\dagger},\phi_i\big] & = +ic^{\a}\del_{\a}\phi_i, \\
            -\big[Q_B^{\dagger},B_A\big] & = 0, \\
            \big\{Q_B^{\dagger},-b_A\big\} & = +iB_A, \\
            \big\{Q_B^{\dagger}, c^{\a}\big\} & = \frac{1}{2}{f^{\a}}_{\beta\g} c^{\beta}c^{\g},
        \end{split}
    \ese
    which after some trivial rearranging of sign gives exactly the same form as \Cref{eqn:BRSTTransformationQB}. We therefore see that $Q_B$ and $Q_B^{\dagger}$ give rise to the same symmetry and therefore must be the same generator, i.e. $Q_B=Q_B^{\dagger}$. 
\eq 

Putting all of this together we get the following result.

\mybox{
The physical states of a system with a BRST symmetry are left \textit{and} right BRST exact,
\be 
\label{eqn:BRSTPhysicalStates}
    Q_B\ket{\psi} = 0 = \bra{\psi}Q_B.
\ee 
}
This is a very nice result, but it actually becomes nicer when we recall that $Q_B$ is nilpotent, \Cref{eqn:QBSquared}. This tells us that we can take \textit{any} (it need not be a physical state), state, $\ket{\xi}$ say, and if we act on it with $Q_B$ and then add it to a physical state, $\ket{\psi}$ say, then we get \textit{the same} physical state back. 
\mybox{
That is our physical states come in \textit{equivalence classes}
\be
\label{eqn:BRSTPhysicalStateEquivalenceClass}
    \begin{split}
        \ket{\psi} &\sim \ket{\psi} + Q_B\ket{\xi}, \\
        \bra{\psi} &\sim \bra{\psi} + \bra{\xi}Q_B
    \end{split}
\ee
}
The above is just the definition of a \textit{BRST cohomology},\footnote{If you are unfamiliar with cohomology, it basically means closed modulo exact. The most common example is the so-called deRham cohomology which is the cohomology of differential forms.} and so we just say that our physical states are elements (i.e. equivalence classes) of the BRST cohomology. 

\br 
    Note that if $\ket{f}$ is a closed state, then $\bra{f}Q_B\ket{\xi}=0$ for any state $\ket{\xi}$ as $\bra{f}Q_B=0$. We see therefore that 
    \bse 
        \| Q_B\ket{\xi}\|^2 = 0.
    \ese 
    So all $Q_B$ exact states are \textit{null}. This is an important result as it tells us that two physical states that differ only by a null state will have the same inner product with all other physical states, an obvious requirement. 
\er 

\subsection{Operator Insertions}

So what about inserting operators into our transition amplitudes? Following a similar process to above, it is clear that we also require that these operators commute with $Q_B$
\be 
\label{eqn:BQOperatorCommutator}
    \big[B_Q,\cO\big] = 0,
\ee 
as then we can `move' the $Q_B$ through them and end up hitting some physical state. We also see that  
\be 
\label{eqn:BRSTOperatorChange}
    \cO \to \cO' = \cO + \big[Q_B,\widetilde{\cO}\big],
\ee 
for \textit{arbitrary} $\widetilde{\cO}$ (i.e. it need not commute with $B_Q$) will also not change the transition amplitude, i.e.
\bse 
    \begin{split}
        \del\cT_{if} & = \bra{f} \cO_1 ... \big[Q_B,\widetilde{\cO}\big] ... \cO_N\ket{i} \\
        & = \bra{f} \cO_1 ... Q_B\widetilde{\cO} ... \cO_N\ket{i} - \bra{f} \cO_1 ... \widetilde{\cO}Q_B ... \cO_N\ket{i} \\
        & = 0,
    \end{split}
\ese
as the first term vanishes by moving the $Q_B$ all the way to the left and the second term vanishes by moving $Q_B$ all the way to the right. 

\Cref{eqn:BQOperatorCommutator} is the statement that $\cO$ is closed, and \Cref{eqn:BRSTOperatorChange} is the statements that $\cO$ is invariant under the addition of an exact term. Again we say that these operators are part of the BRST cohomology class (for operators).


\section{BRST Quantisation Of Point Particle}

We have outlined the process of BRST quantisation and now we want to apply it to some physics problem. We will use the point particle as an example as it will make the study of the string much easier. 

Recall \Cref{eqn:PointActionEinbein}, 
\bse 
    S = \frac{1}{2}\int d\tau \big( e^{-1}\dot{X}^2 -em^2\big), 
\ese 
where we have used our notation $\dot{X}^2 := \dot{X}^{\mu}\dot{X}_{\mu}$. The gauge symmetry here is, of course, reparameterisation invariance. Consider the small change
\bse 
    \tau \to \tau' = \tau + v(\tau).
\ese 
This gives 
\bse 
    X^{\mu}(\tau) \to X^{\mu}(\tau') \approx X^{\mu}(\tau) - \dot{X}^{\mu}(\tau)v(\tau), \qquad \implies \qquad \del X^{\mu}(\tau) = -\dot{X}^{\mu}(\tau) v(\tau),
\ese 
where the minus sign comes from the fact that we are doing an active transformation.

We also have  
\bse 
    \begin{split}
        e(\tau)d\tau & \to e(\tau')d\big(\tau+v(\tau)\big) \\
        & = \big(e(\tau)-\dot{e}(\tau)v(\tau)\big)\big(d\tau -\dot{v}(\tau)d\tau\big) \\
        \implies \del e & = -\p_{\tau}\big(e(\tau)v(\tau)\big),
    \end{split}
\ese
where we have dropped the term quadratic in $v$ and its derivative.

We need to identify what the $\a$ index is for the BRST procedure; it was meant to include the basis for the transformations, which here is shifting $\tau$ to any other value $\tau_1$. With some thought we see that 
\bse 
    \del_{\a} \tau = \del(\tau-\a), \qquad \implies \qquad v(\tau) = \int d\a \, \del(\tau-\a)v(\a),
\ese 
where the integral is done because reparameterisations are continuous.\footnote{That is, the summation convention becomes an integral.} In other words, the fields vary as 
\be 
\label{eqn:BRSTXeVariation}
    \del_{\a} X^{\mu}(\tau) = -\del(\tau-\a)\p_{\tau}X^{\mu}(\tau), \qand \del_{\a}e(\tau) = -\p_{\tau}\big(\del(\tau-\a)e(\tau)\big).
\ee
Our commutator is given by 
\bse 
    \begin{split}
        [\del_{\a},\del_{\beta}]X^{\mu}(\tau) & = \del_{\a}\big(-\del(\tau-\beta)\p_{\tau}X^{\mu}(\tau)\big) - \del_{\beta}\big(-\del(\tau-\a)\p_{\tau}X^{\mu}(\tau)\big) \\
        & = -\big(\del(\tau-\a)\p_{\tau}\del(\tau-\beta) - \del(\tau-\beta)\p_{\tau}\del(\tau-\a)\big)\p_{\tau}X^{\mu}(\tau).
    \end{split}
\ese
We then want to identify this with 
\bse 
    [\del_{\a},\del_{\beta}]X^{\mu}(\tau) = \int d\g {f^{\g}}_{\a\beta}\del_{\g}X^{\mu}(\tau),
\ese 
where again the integral is because of the continuous nature of our symmetry. We therefore see 
\be 
\label{eqn:StrucutreConstantsBRSTPointParticle}
    {f^{\g}}_{\a\beta} := \del(\g-\a)\p_{\a}\big[\del(\g-\beta)\big] - \del(\g-\beta)\p_{\beta}\big[\del(\g-\a)\big]
\ee  
does the job, as 
\bse 
    \begin{split}
        [\del_{\a},\del_{\beta}]X^{\mu}(\tau) & = \int d\g \Big(\del(\g-\a)\p_{\a}\big[\del(\g-\beta)\big] - \del(\g-\beta)\p_{\beta}\big[\del(\g-\a)\big]\Big)\del_{\g}X^{\mu}(\tau) \\
        & = \int d\g \Big(\del(\g-\a)\p_{\a}\big[\del(\g-\beta)\big] - \del(\g-\beta)\p_{\beta}\big[\del(\g-\a)\big]\Big)\big(-\del(\tau-\g)\p_{\tau}X^{\mu}(\tau)\big) \\
        & = -\big(\del(\tau-\a)\p_{\tau}\del(\tau-\beta) - \del(\tau-\beta)\p_{\tau}\del(\tau-\a)\big)\p_{\tau}X^{\mu}(\tau).
    \end{split}
\ese 

We will take out gauge fixing condition to be $e=1$ (as this will related well to the string). Therefore we have $F^A(\tau)=e(\tau)-1$ which gives us 
\bse 
    \begin{split}
        S_2 & = -iB_AF^A(\tau) \\
        & = -i\int d\tau B(\tau)\big(e(\tau)-1\big),
    \end{split}
\ese 
and 
\bse 
    \begin{split}
        S_3 & = b_Ac^{\a}\del_{\a}F^A(\tau) \\
        & = \int d\tau \, b(\tau)c(\a)\del_{\a}\big(e(\tau)-1) \\
        & = \int d\tau \, b(\tau)c(\a) \big[ -\p_{\tau}\big(\del(\tau-\a)e(\tau) \big] \\
        & = \int d\tau \, e(\tau)\dot{b}(\tau)c(\tau),
    \end{split}
\ese 
where we have used integration by parts. This gives our complete action as 
\be 
\label{eqn:ActionBRSTPointParticle}
    S = \int d\tau \bigg( \frac{1}{2}e^{-1}\dot{X}^2 + \frac{1}{2}em^2 + iB(e-1) - e\dot{b}c\bigg),
\ee 
where we have dropped the arguments for notational convenience. 
\chapter{BRST Quantisation II}

\section{Point Particle (Continued)}

So far we have obtained the overall action for the point particle, but we haven't written down the actual BRST transformation. Using $\phi_i=(X^{\mu},e)$ in \Cref{eqn:BRSTTransformation} and \Cref{eqn:BRSTXeVariation} we get 
\be
\label{eqn:BRSTTransformationPointParticle}
    \begin{split}
        \del_BX^{\mu} & = i\epsilon c\dot{X}^{\mu}, \\
        \del_Be(\tau) & = i\epsilon\p_{\tau}(ce), \\
        \del_BB & = 0, \\
        \del_Bb & = \epsilon B, \\
        \del_B c & = i\epsilon c\dot{c},
    \end{split}
\ee 
where, as with previous calculations, integrals are used and delta functions taken (note for $\del_Be$ and $\del_Bc$ we also need to use integration by parts).

We now do our gauge fixing, i.e. set $e=1$ by doing the $B$ integral. So our action becomes 
\be 
\label{eqn:ActionBRSTPointParticleGaugeFixed}
    S = \int d\tau \bigg(\frac{1}{2}\dot{X}^2 + \frac{1}{2}m^2 -\dot{b}c\bigg).
\ee
This now looks more like our string theory action, where we identify the first two terms as the matter action and the last term as the ghost action. 

Good, but what does this gauge fixing doing to \Cref{eqn:BRSTTransformationPointParticle}? Well obviously $\del_Be$ and $\del_BB$ aren't needed, but what about $\del_Bb$? The right-hand side contains a $B$, which we don't want. So we need to replace it somehow. We do this by taking the $e$ equation of motion of \Cref{eqn:ActionBRSTPointParticle}, 
\bse
    \begin{split}
        0 & = -\frac{1}{2e^2}\dot{X}^2 + \frac{1}{2}m^2 + iB - \dot{b}c  \\
        \implies B &= \frac{-i}{2}\dot{X}^2 +\frac{i}{2}m^2 -i\dot{b}c,
    \end{split}
\ese 
where in the second line we have fixed gauge (i.e. set $e=1$). So \Cref{eqn:BRSTTransformationPointParticle} simply becomes 
\be 
\label{eqn:BRSTTransformationPointParticleGaugeFixed}
    \begin{split}
        \del_BX^{\mu} & = i\epsilon c\dot{X}^{\mu} \\
        \del_Bb & = i\epsilon\bigg(-\frac{1}{2}\dot{X}^2 +\frac{1}{2}m^2 -\dot{b}c\bigg) \\
        \del_Bc & = i\epsilon c\dot{c}. 
    \end{split}
\ee 

\bcl 
    \Cref{eqn:BRSTTransformationPointParticleGaugeFixed} is a symmetry of \Cref{eqn:ActionBRSTPointParticleGaugeFixed} and it is nilpotent, provided we empose the equations of motion. 
\ecl 

\bq 
    First let's show that it is a symmetry. We have
    \bse 
        \begin{split}
            \del_BS & = \int d\tau \, \Big[ \dot{X}_{\mu}\big(\del_B\dot{X}^{\mu}\big) - \del_B\big(\dot{b}c\big) \Big] \\
            & = \int d\tau \, \Big[ \dot{X}_{\mu}\p_{\tau}\big(\del_BX^{\mu}\big) + \big(\del_Bb\big)\dot{c} - \dot{b}\big(\del_Bc\big)\Big] \\
            & = \int d\tau \, \Bigg[ i\epsilon  \dot{X}_{\mu}\p_{\tau}\big(c\dot{X}^{\mu}\big) + i\epsilon \bigg(-\frac{1}{2}\dot{X}^2 +\frac{1}{2}m^2 -\dot{b}  c \bigg)\dot{c} - \dot{b} i\epsilon c\dot{c}\Bigg] \\
            & = i\epsilon \int d\tau \, \Bigg[   \dot{X}_{\mu}\p_{\tau}\big(c\dot{X}^{\mu}\big) + \bigg(-\frac{1}{2}\dot{X}^2 +\frac{1}{2}m^2 - \dot{b}c \bigg)\dot{c} + \dot{b}c\dot{c}\Bigg] \\
            & = i\epsilon\int d\tau \bigg[ \dot{X}^2\dot{c} + \dot{X}_{\mu}\ddot{X}^{\mu}c - \frac{1}{2}\dot{X}^2\dot{c} +\frac{1}{2}m^2\dot{c} \bigg] \\
            & = \frac{i\epsilon}{2}\int d\tau \, \p_{\tau}\big( \dot{X}^2c + m^2c\big),
        \end{split}
    \ese 
    where to go from the third to the fourth line we have used $\dot{b}\epsilon = -\epsilon\dot{b}$. We know this result vanishes because the integrand is a total derivative.
    
    Now let's show that it's nilpotent. Let's consider $\del_B^2X^{\mu}$ first: 
     \bse 
        \begin{split}
            \del_B^{\epsilon_2}\del_B^{\epsilon_1} X^{\mu} & = i\epsilon_1 \del_B^{\epsilon_2}\big(c\dot{X}^{\mu}\big) \\
            & = i\epsilon_1\big( (\del_Bc)\dot{X}^{\mu} + c\del_B\dot{X}^{\mu}\big) \\
            & = i\epsilon_1\big( i\epsilon_2c\dot{c}\dot{X}^{\mu} + ci\epsilon_2\p_{\tau}(c\dot{X}^{\mu})\big) \\
            & = -\epsilon_1\epsilon_2\big( c\dot{c}\dot{X}^{\mu} - c\dot{c}\dot{X}^{\mu} - cc\ddot{X}^{\mu}\big) \\
            & = 0,
        \end{split}
    \ese 
    where we have used $c\epsilon_2 = -\epsilon_2c$ and $cc=0$.
    
    Next let's check $\del_B^2c$:
    \bse 
        \begin{split}
            \del_B^2c & = i\epsilon_1\del_B^{\epsilon_2}\big(c\dot{c}\big) \\
            & = i\epsilon_1 \Big[ \big(\del_B^{\epsilon_2}c\big)\dot{c} + c\big(\del_B^{\epsilon_2}\dot{c}\big)\Big] \\
            & = i\epsilon_1\Big[ i\epsilon_2c\dot{c}\dot{c} + c i\epsilon_2 \p_{\tau}\big(c\dot{c}\big)\Big] \\
            & = -\epsilon_1\epsilon_2 \big[c\dot{c}\dot{c} - c\dot{c}\dot{c} - cc\dot{c}\big] \\
            & = 0,
        \end{split}
    \ese 
    where again we have used $c\epsilon_2=-\epsilon_2c$ and $cc=0$.
    
    Finally we need to check $\del_B^2b$. This is the most complicated one as its the only one that is of a different form to \Cref{eqn:BRSTTransformation}. Comparing the expression for $\del_Bb$ with the integrand of \Cref{eqn:ActionBRSTPointParticleGaugeFixed}, we see that we are going to get the same calculation as the $\del_BS$ calculation at the start of the proof, \textit{apart from} now the $\dot{X}^2$ term has a negative sign, so we get 
    \bse 
        \del_Bb = -\epsilon_1\epsilon_2\bigg[-\frac{3}{2}\dot{c}\dot{X}^2 + \frac{1}{2}m^2\dot{c} -c\ddot{X}\dot{X} \bigg].
    \ese 
    There is no way to see how this is going to cancel without imposing some conditions. This is where the caveat of the claim comes: we impose the equations of motion. We have 
    \bse 
        \cL = \frac{1}{2}\dot{X}^2 + \frac{1}{2}m^2 - \dot{b}c,
    \ese 
    and so the $X^{\mu}$ and $b$ equations of motion are 
    \bse 
        \ddot{X}^{\mu} = 0, \qand \dot{c} = 0,
    \ese 
    respectively. So if we impose both of these we get 
    \bse 
        \del_B^2b=0.
    \ese
\eq 

\br 
    Note that it is only the $\del_Bb$ term that requires us to impose the equations of motion. This makes some kind of sense because the expression for $\del_Bb$ was obtained by using the equations of motion of \Cref{eqn:ActionBRSTPointParticle}, and so only holds when the equations of motion hold.
\er 

\subsection{The Current} 

So we have seen that our system has a symmetry, and now we want to find what the corresponding current is. Recall that Noether's theorem tells us to basically make $\epsilon$ a function of $\tau$ and then look for the coefficient of the $\dot{\epsilon}$ term in $\del_B S$. We repeat the calculation:
\bse 
    \begin{split}
        \del_BS & = \int d\tau \Big[ \dot{X}_{\mu}\p_{\tau}\big(\del_{B}X^{\mu}\big) + \big(\del_{B}b\big)\dot{c} - \dot{b}\big(\del_Bc\big)\Big] \\
        & = i\int d\tau \bigg[\dot{X}_{\mu}\p_{\tau}\big(\epsilon c \dot{X}^{\mu}\big) +\epsilon\bigg(-\frac{1}{2}\dot{X}^2 + \frac{1}{2}m^2 -\dot{b}c\bigg)\dot{c} - \dot{b}\epsilon c\dot{c}\bigg] \\
        & = i\int d\tau \bigg[ \dot{X}^2\dot{\epsilon}c + \dot{X}^2\epsilon\dot{c} + \dot{X}_{\mu}\ddot{X}^{\mu}\epsilon c - \frac{1}{2}\dot{X}^2\epsilon \dot{c} + \frac{1}{2}m^2\epsilon \dot{c} - \epsilon\dot{b}c\dot{c} + \epsilon\dot{b}c\dot{c}\bigg] \\
        & = i\int d\tau \bigg[ \frac{1}{2}\dot{X}^2\epsilon\dot{c} + \dot{X}_{\mu}\ddot{X}^{\mu}\epsilon c + \dot{X}^2 \dot{\epsilon}c + \frac{1}{2}m^2 \epsilon\dot{c}\bigg] \\
        & = i \int d\tau \bigg[ \p_{\tau}\bigg(\frac{1}{2}\dot{X}^2\epsilon c + \frac{1}{2}m^2 \epsilon c \bigg) + \frac{1}{2}\dot{X}^2\dot{\epsilon}c - \frac{1}{2}m^2\dot{\epsilon}c \bigg], 
    \end{split}
\ese
where in the last line we have expressed the integrand as a total derivative (so we can drop it) and compensated for it with the last two terms. So we see that 
\bse 
    \del_BS = i\int d\tau \, \dot{\epsilon}c \bigg(\frac{1}{2}\dot{X}^2 - \frac{1}{2}m^2\bigg),
\ese 
and so our current is 
\bse 
    J^0 = ic \bigg(\frac{1}{2}\dot{X}^2 - \frac{1}{2}m^2\bigg).
\ese 
This looks \textit{almost} like the Hamiltonian for the point particle in Minkowski, 
\bse 
    H = \frac{1}{2}p^2 + \frac{1}{2}m^2,
\ese 
with $p=\dot{X}$, the only problem being that pesky minus sign in front of the $m^2$ term. Well what we have to remember is that the integral done above is done in Euclidean space where $X^{\mu}_{\text{Euclidean}}= iX^{\mu}_{\text{Minkowski}}$. Putting this into the above we get (dropping the superscript $0$)
\bse 
    J = - ic\bigg(\frac{1}{2}p^2 + \frac{1}{2}m^2\bigg) = -c H.
\ese 
We can write this as the charge 
\be 
\label{eqn:BRSTChargePointParticle}
    Q = -cH.
\ee 


\br 
    In the result for the current above, Dr. Minwalla gets a $\dot{\epsilon}b\dot{c}$ term. I think that he simply made a calculation error in the lectures, but I have included this remark in case it is meant to be there and I have missed something. Fundamentally it makes no difference because the equations of motion tell us $\dot{c}=0$, and so this additional term vanishes anyway. 
\er 

\subsection{The Hilbert Space \& Cohomology}

The action for our point particle came to the form \Cref{eqn:ActionBRSTPointParticleGaugeFixed}, which looks like the point particle analogy for the string action we have seen before. It is reasonable to assume (and it is true) that the equivalent to \Cref{eqn:cbanticommutator} holds, specifically 
\bse
    \{c,b\}= 1.
\ese 
Our Hilbert space is therefore the product of the free particle states (i.e. states labelled by the momentum of the particle) and the $\ket{\uparrow}/\ket{\downarrow}$ states we saw previously. That is 
\bse 
    \ket{p_{\mu},\uparrow}, \qand \ket{p_{\mu},\downarrow}
\ese
form a basis for our Hilbert space. We want to know which states are physical, and therefore we need to compute the cohomology of the BRST operator, \Cref{eqn:BRSTChargePointParticle}.

Recalling that $c\ket{\uparrow}=0$ and $c\ket{\downarrow}=\ket{\uparrow}$, we have 
\bse 
    Q\ket{p_{\mu},\uparrow} = 0, \qand Q\ket{p_{\mu},\downarrow} = -\frac{1}{2}\big(p^2+m^2\big)\ket{p_{\mu},\uparrow}.
\ese 
The first result tells us that \textit{all} the up states are $Q$ closed and the second condition tells us that the only up states that \textit{aren't} $Q$ exact are those on shell (i.e. $p^2+m^2=0$). 

The fact that there are no $\ket{p_{\mu},\downarrow}$ states on the right-hand sides of the above means that \textit{no} down states are $Q$ exact, and the second condition tells us that the down states are only $Q$ closed when on shell. 

So we see that our cohomology has two types of physical states 
\bse 
    \ket{p_{\mu},\uparrow}, \quad p^2+m^2=0, 
\ese 
and 
\bse 
    \ket{p_{\mu},\downarrow}, \quad p^2+m^2=0.
\ese
These look like identical conditions, and so we get two copies of the same spectrum! There is a very important difference between the two though: there is nothing stopping us inserting up states that are not on shell, apart from they will vanish unless we then constrict them to being on shell; whereas we can \textit{only} insert down states that are already on shell. It follows, therefore, any quantity containing an up state must be proportional to $\del(p^2+m^2)$, whereas the down states need not. It turns out that kinematics tells us that\footnote{Apart from when $d=2$, apparently.} no such terms every arise in our QFTs. That is, although we can \textit{technically} insert any up state, we require that any amplitude obtained using them must vanish identically. This condition manifests itself as the additional requirement that physical states obey 
\bse 
    b_0\ket{\psi} = 0
\ese
for the point particle.

So our physical states are given by 
\bse 
    \ket{p_{\mu},\downarrow}, \quad p^2+m^2=0.
\ese 
This matches what we expect from previous dealings with QFT --- the physical states are on shell. 

\section{BRST Quantisation Of The String}

We now need to find the BRST transformation for the string action. We know from previous calculations what $S_1=S_{\text{Poly}}$ and $S_3=S_{\text{ghost}}$ are, but what about $S_2$? This is the gauge fixing condition, and we recall that our gauge fixing was to set the metric to the flat metric, and so we have 
\bse 
    S_2 = \frac{i}{4\pi} \int d^2\sig \, \sqrt{g} B^{ab}\big(\del_{ab}-g_{ab}\big).
\ese
If we follow the method we did for the point particle where we remove $B^{ab}$ from our BRST transformation using an equation of motion,\footnote{You use the $g_{ab}$ one.} we get the following BRST transformation
\be 
\label{eqn:BRSTTransformationString}
    \begin{split} 
        \del_BX^{\mu} &= i\epsilon\Big(c\p + \overline{c}\overline{\p} \Big)X^{\mu}, \\
        \del_Bb & = i\epsilon\Big(T^m + T^g\Big) \\
        \del_B\overline{b} & = i\epsilon\Big(\overline{T}^X + \overline{T}^g\Big) \\
        \del_Bc & = i\epsilon c\p c \\
        \del_B \overline{c} & = i \overline{c}\epsilon\overline{\p}\overline{c},
    \end{split}
\ee 
where $T^m$ and $T^g$ are the matter and ghost stress tensors respectively. It is easy to see the similarlities between this result and \Cref{eqn:BRSTTransformationPointParticleGaugeFixed}, which is why we did the point particle in the first place. 

\br 
    The equation of motion condition used to remove $B_{ab}$ from the BRST transformations makes $b_{ab}$ traceless. 
\er 

\bcl 
    \Cref{eqn:BRSTTransformationString} is a symmetry of the relavent gauge-fixed action and is nil-potent. 
\ecl 

\bq 
    We do not do the proof here as it follows analogously to the one presented for the point particle.
\eq 

We note that \Cref{eqn:BRSTTransformationString} has uncoupled holomorphic and antiholomorphic parts, and so we can expect to get a holomorphic and an antiholomorphic BRST current.\footnote{Well together they are \textit{the} BRST current, but we can split it up into two currents.} As normal, we shall just consider the holomorphic part. 


\bp 
    The BRST current for the string is given by 
    \be 
    \label{eqn:BRSTCurrentString}
        \begin{split}
            J_B & = c T^m + \frac{1}{2}\cl cT^g\cl + \frac{3}{2}\p^2c \\
            & = -\frac{1}{\a'}c\cl\p X\p X\cl   + \cl bc\p c\cl + \frac{3}{2}\p^2 c.
        \end{split}
    \ee
\ep 

\br 
\label{rem:BRSTCurrentTensorial}
    Before proving this, let's first just make a note on it. The first two terms make some sort of sense given then current for the point particle, but the $\p^2c$ term seems strange. Well we note that it is a total derivative and so will not contribute to the BRST charge, and so has no effect on the physics. The question is, then, "why include it?" The answer is simply that it gives $J_B$ a tensorial form. 
\er 

\bq 
    We could obtain \Cref{eqn:BRSTCurrentString} using Noether's theorem on our gauge fixed action. However we shall show it a different way. If $J_B$ is really the current for the BRST transformation, then its OPE with $X$, $b$ and $c$ should give the right-hand sides of \Cref{eqn:BRSTTransformationString} at the pole. So let's check that. 
    
    Firstly consider $J_BX^{\mu}$. The only contribution to the OPE will come from the $T^m$ term. Recalling 
    \bse 
        \overbrace{X(z)X(\omega)} = -\frac{\a'}{2}\ln(z-\omega),
    \ese 
    we have 
    \bse 
        \begin{split}
            J_B(z)X^{\mu}(\omega) & = -\frac{1}{\a'} c(z)\Big[2\overbrace{\p X(z)X^{\mu}(\omega)}\p X(z) + ... \Big] \\
            & = - \frac{1}{\a'} c(z)\bigg[ 2\bigg(-\frac{\a'}{2}\bigg) \frac{\p X(z)}{(z-\omega)} + ... \bigg] \\
            \implies \Res\big[J_B(z)X^{\mu}(\omega)\Big] & = c\p X^{\mu},
        \end{split}
    \ese 
    where in the last step we took the Taylor expansion to get $\omega$ as the argument. This is exactly what we need. 
    
    Next consider $J_B(z)c(\omega)$. Firstly we note that the $T^m$ term will vanish because there is no contraction between $X$ and $c$ and $c(z)c(\omega)=0$. Similarly the $\p^2c$ term will vanish. Then, using the contraction $b(z)c(\omega)=1/(z-\omega)$, we have 
    \bse 
        \begin{split}
            J_B(z)c(\omega) & =  \cl c(z)\p c(z)\cl \overbrace{b(z)c(\omega)} + ... \\
            \Res\big[J_B(z)c(\omega)\big] & = c \p c,
        \end{split}
    \ese 
    where we have used $\cl c\p c \cl = c\p c$ as $\la c \p c\ra = 0$.
    
    Finally consider $J_B(z)b(\omega)$. 
    \bse 
        \begin{split}
            J_B(z)b(\omega) & = T^m(z)\overbrace{c(z)b(\omega)} + \cl b(z)c(z)\cl \overbrace{\p c(z) b(\omega)} - \cl b(z)\p c(z)\cl \overbrace{c(z)b(\omega)} + \frac{3}{2}\overbrace{\p^2 c(z)b(\omega)} + ... \\
            & = \frac{1}{(z-\omega)}T^m(z) - \frac{\cl b(z)c(z)\cl}{(z-\omega)^2} - \frac{\cl b(z)\p c(z)\cl}{(z-\omega)} + \frac{3}{(z-\omega)^3} + ... \\
            & = \frac{T^m(\omega)}{(z-\omega)} - \frac{\cl b(\omega)c(\omega)\cl}{(z-\omega)^2} - \frac{\cl \p b(\omega) c(\omega)\cl + \cl b(\omega)\p c(\omega)\cl}{(z-\omega)} - \frac{\cl b(z)\p c(z)\cl}{(z-\omega)} + \\
            & \quad \frac{3}{(z-\omega)^3} + ... \\
            & = \frac{1}{(z-\omega)}\Big[ T^m(\omega) + 2\cl \p c(\omega)b(\omega)\cl + \cl c(\omega)\p b(\omega) \cl\Big] + \frac{\cl c(\omega)b(\omega)\cl}{(z-\omega)^2} +  \frac{3}{(z-\omega)^3} + ... \\
            & = \frac{1}{(z-\omega)} \Big[ T^m(\omega) + T^g(\omega) \Big] + \frac{J_G(\omega)}{(z-\omega)^2} + \frac{3}{(z-\omega)^3} + ... \\
            \implies \Res\big[J_B(z)b(\omega)\big] & = T^m + T^g,
        \end{split}
    \ese 
    where we have Taylor expanded the $\cl b(z)c(z)\cl$ term and used the definitions of $T^m$, $T^g$ and $J_G$.
\eq 

Let's now find the OPE between the stress tensor $T = T^m + T^g$ and $J_B$:
\bse 
    \begin{split}
        T(z)J_B(\omega) & = \Big(T^m(z) + T^g(z)\Big)\Big(c(\omega)T^m(\omega) + \frac{1}{2}\cl c(\omega)T^g(\omega) \cl + \frac{3}{2}\p^2 c(\omega)\Big) \\
        & = T^m(z)c(\omega)T^m(\omega) + T^g(z)c(\omega)T^m(\omega) + \frac{1}{2}T^g(z)\cl c(\omega)T^g(\omega)\cl + \frac{3}{2}T^g(z)\p^2c(\omega).
    \end{split}
\ese 
Let's consider this term by term.\footnote{Get ready for a loooong calculation...}. Keeping only the singular terms, firstly we have 
\bse 
    \begin{split}
        T^m(z)c(\omega)T^m(\omega) & = c(\omega)\bigg[\frac{c^m/2}{(z-\omega)^4} + \frac{2T^m(\omega)}{(z-\omega)^2} +\frac{\p T^m(\omega)}{(z-\omega)}\bigg].
    \end{split}
\ese 
Next we have 
\bse 
    \begin{split}
        T^g(z)c(\omega)T^m(\omega) & = \bigg[-\frac{c(z)}{(z-\omega)^2} + \frac{2\p c(z)}{(z-\omega)}\bigg]T^m(\omega) \\
        & = \bigg[-\frac{c(\omega)}{(z-\omega)^2} + \frac{\p c(\omega)}{(z-\omega)}\bigg]T^m(\omega).
    \end{split}
\ese 
Next, using 
\bse 
    \begin{split}
        \frac{1}{2} \cl c(\omega)T^g(\omega)\cl & = \frac{1}{2}\big(2 \cl c(\omega) \p c(\omega) b(\omega)\cl + \cl c(\omega) c(\omega) \p b(\omega) \cl\big) \\
        & = \cl c(\omega)\p c(\omega) b(\omega) \cl,
    \end{split}
\ese
we have 
\bse
    \begin{split}
        \frac{1}{2}T^g(z)\cl c(\omega) T^g(\omega) \cl & = \big( 2\cl \p c(z)b(z)\cl + \cl c(z)\p b(z)\cl \big)\cl c(\omega)\p c(\omega) b(\omega) \cl \\
        & = 2\cl \p c(z)b(z)\cl \cl c(\omega)\p c(\omega) b(\omega) \cl + \cl c(z)\p b(z)\cl \cl c(\omega)\p c(\omega) b(\omega) \cl.
    \end{split}
\ese 
Let's consider the two terms separately: 
\bse 
    \begin{split}
        2\cl \p c(z)b(z)\cl \cl c(\omega)\p c(\omega) b(\omega) \cl & = -2\overbrace{b(z)c(\omega)}\overbrace{\p c(z)b(\omega)}\p c(\omega) + 2\overbrace{b(z)\p c(\omega)}\overbrace{\p c(z)b(\omega)}c(\omega) \\ 
        & \quad + 2\overbrace{b(z)c(\omega)}\p c(z) \p c(\omega) b(\omega) -2\overbrace{b(z)\p c(\omega)}\p c(z) c(\omega) b(\omega) \\
        & \quad -2\overbrace{\p c(z)b(\omega)}b(z)c(\omega)\p c(\omega) \\
        & = \frac{2\p c(\omega)}{(z-\omega)^3} - \frac{2c(\omega)}{(z-\omega)^4} + \frac{2\cl \p c(z)\p c(\omega) b(\omega)\cl }{(z-\omega)}  \\
        & \quad - \frac{2\cl \p c(z)c(\omega)b(\omega)\cl }{(z-\omega)^2} +\frac{2\cl b(z)c(\omega)\p c(\omega)\cl }{(z-\omega)^2} \\
        & = -\frac{2c(\omega)}{(z-\omega)^4} + \frac{2\p c(\omega)}{(z-\omega)^3}+\frac{4\cl b(\omega)c(\omega)\p c(\omega)\cl }{(z-\omega)^2} \\
        & \quad + \frac{2\big(\cl b(\omega)\p c(\omega)\p c(\omega)\cl + \cl b(\omega)\p^2c (\omega) \cl + \cl \p b(\omega) c(\omega) \p c(\omega)\cl\Big)}{(z-\omega)} \\
        & = -\frac{2c(\omega)}{(z-\omega)^4} + \frac{2\p c(\omega)}{(z-\omega)^3}+\frac{4\cl b(\omega)c(\omega)\p c(\omega)\cl }{(z-\omega)^2} +\frac{2\p\cl b(\omega)c(\omega)\p c(\omega)\cl}{(z-\omega)} \\
        & = -\frac{2c(\omega)}{(z-\omega)^4} + \frac{2\p c(\omega)}{(z-\omega)^3}+\frac{2\cl c(\omega) T^g(\omega)\cl }{(z-\omega)^2} +\frac{\p\cl c(\omega) T^g(\omega)\cl}{(z-\omega)}
    \end{split}
\ese 
and\footnote{Note here that we do keep the $\cl c(z)c(\omega)b(\omega)\cl$ term (i.e. don't just use $c(z)c(\omega)=0$) as we need to take a Taylor expansion, which will give $\p c(\omega) c(\omega)$ and $\p^2c(\omega)c(\omega)$ terms. This mistake cost me about 3 hours of calculation time, so that's why this footnote is here...}
\bse 
    \begin{split}
        \cl c(z)\p b(z)\cl\cl c(\omega)\p c(\omega) b(\omega)\cl & = -\overbrace{\p b(z)c(\omega)}\overbrace{c(z)b(\omega)} \p c(\omega) + \overbrace{\p b(z)\p c(\omega)}\overbrace{c(z)b(\omega)}c(\omega) \\
        &  \quad + \overbrace{\p b(z)c(\omega)} \cl c(z)\p c(\omega) b(\omega)\cl - \overbrace{\p b(z)\p c(\omega)}\cl c(z) c(\omega) b(\omega)\cl \\
        & \quad -\overbrace{c(z)b(\omega)} \cl \p b(z) c(\omega) \p c(\omega)\cl \\
        & = \frac{\p c(\omega)}{(z-\omega)^3} - \frac{2c(\omega)}{(z-\omega)^4} - \frac{\cl c(z) \p c(\omega) b(\omega)\cl}{(z-\omega)^2} + \frac{2\cl c(z)c(\omega)b(\omega)\cl}{(z-\omega)^3} \\
        & \quad - \frac{\cl \p b(z) c(\omega) \p c(\omega)\cl}{(z-\omega)} \\
        & = -\frac{2c(\omega)}{(z-\omega)^4} + \frac{\p c(\omega)}{(z-\omega)^3} -\frac{3\cl b(\omega) c(\omega)\p c(\omega)\cl}{(z-\omega)^2} \\
        & \quad + \frac{-\cl \p b(\omega)c(\omega) \p c(\omega)\cl + \cl \p c(\omega) \p c(\omega) b(\omega)\cl + \cl \p^2 c(\omega) c(\omega) b(\omega)\cl }{(z-\omega)} \\
        & = -\frac{2c(\omega)}{(z-\omega)^4} + \frac{\p c(\omega)}{(z-\omega)^3} -\frac{3\cl c(\omega) T^g(\omega) \cl}{2(z-\omega)^2} - \frac{\p \cl c(\omega)T^g(\omega)\cl}{2(z-\omega)}.
    \end{split}
\ese 
Putting these together gives 
\bse 
    \frac{1}{2}T^g(z)\cl c(\omega)T^g(\omega)\cl = -\frac{4c(\omega)}{(z-\omega)^4} + \frac{3\p c(\omega)}{(z-\omega)^3} + \frac{\cl c(\omega)T^g(\omega)\cl}{2(z-\omega)^2} + \frac{\p \cl c(\omega)T^g(\omega)\cl}{2(z-\omega)}.
\ese 

Finally, we have 
\bse 
    \begin{split}
        \frac{3}{2}T^g(z) \p^2 c(\omega) & = \frac{3}{2}\big(2\cl \p c(z)b(z)\cl + \cl c(z)\p b(z)\cl\big) \p^2c(\omega) \\
        & = \frac{3}{2}\bigg[ 2\overbrace{b(z)\p^2c(\omega)}\p c(z) + \overbrace{\p b(z) \p^2 c(\omega)}c(z)\bigg] \\
        & = \frac{3}{2}\bigg[ \frac{4\p c(z)}{(z-\omega)^3} - \frac{6c(z)}{(z-\omega)^4} \bigg] \\
        & = \frac{3}{2}\bigg[ -\frac{6c(\omega)}{(z-\omega)^4} - \frac{2\p c(\omega)}{(z-\omega)^3} + \frac{\p^2 c(\omega)}{(z-\omega)^2} + \frac{\p^3 c(\omega)}{(z-\omega)} \bigg] \\
        & = -\frac{9c(\omega)}{(z-\omega)^4} - \frac{3\p c(\omega)}{(z-\omega)^3} + \frac{3\p^2c(\omega}{3(z-\omega)^2} + \frac{3\p\big(\p^2c(\omega)\big)}{2(z-\omega)}.
    \end{split}
\ese 

Putting everything together gives us 
\bse 
    \begin{split}
        T(z)J_B(\omega) & = \frac{c(\omega)}{(z-\omega)^4}\bigg[\frac{c^m}{2}-4-9\bigg] + \frac{3\p c(\omega) - 3\p c(\omega)}{(z-\omega)^3} \\
        & \quad + \frac{1}{(z-\omega)^2}\bigg[ 2c(\omega)T^m(\omega) - c(\omega)T^m(\omega) + \frac{1}{2}\cl c(\omega)T^g(\omega) + \frac{3}{2}\p^2c(\omega) \bigg] \\
        & \quad + \frac{1}{(z-\omega)}\bigg[ c(\omega) \p T^m(\omega) + \big(\p c(\omega)\big)T^m(\omega) + \frac{1}{2}\p \cl c(\omega) T^g(\omega)\cl + \frac{3}{2}\p\big(\p^2c(\omega)\big) \bigg] \\
        \therefore \qquad  T(z)J_B(\omega) & = \frac{c^m-26}{2(z-\omega)^4}c(\omega) + \frac{1}{(z-\omega)^2}J_B(\omega) + \frac{1}{(z-\omega)}\p J_B(\omega).
    \end{split}
\ese 
We see, therefore, that if $J_B(\omega)$ is the a true tensor we require $c^m=26$.\footnote{This result is actually somewhat backwards. Really we should use the fact that we know $c^m=26$ in order to ask the question "what do we include in the definition of $J_B$ in order for it to be a tensor?" The answer obviously turns out to be $\frac{3}{2}\p^2c$.}  This result should be haunting you by now.

The next OPE to consider is $J_B$ with itself. The calculation is equally as long (or perhaps longer) then the above, but shares much of the same steps. For this reason we shall not present the calculation here but simply quote the result: 
\bse 
    J_B(z)J_B(\omega) = - \frac{c^m-18}{2(z-\omega)^3}\cl c(\omega)\p c(\omega)\cl - \frac{c^m-18}{4(z-\omega)^2}\cl c(\omega)\p^2 c(\omega)\cl - \frac{c^m-26}{12(z-\omega)}\cl c(\omega)\p^3 c(\omega)\cl
\ese 
The single pole term tells us that the anticommutator of $Q_B$ with itself vanishes if, and only if, $c^m=26$, i.e. 
\bse 
    \big\{Q_B,Q_B\big\} = 0 \qquad \iff \qquad c^m=26.
\ese 
So our BRST charge is only nilpotent when we have a matter central charge of $26$. 
\chapter{BRST Quantisation III}

Last lecture we stated the BRST current $J_B$ and calculated some of its OPEs. We now want to continue this and calculate a few more. First we want to look at the OPE with a holomorphic, primary, matter operator of weight $h$. 

This OPE is very straight forward, as it is only the $T^m(z)$ in $J_B(z)$ that contributes to the OPE, so we get 
\begin{equation*}
    \begin{split}
        J_B(z)\cO(\omega) & = c(z)\bigg[\frac{h\cO(\omega)}{(z-\omega)^2} + \frac{\p \cO(\omega)}{(z-\omega)}\bigg] \\
        & = \frac{hc(\omega)\cO(\omega)}{(z-\omega)^2} + \frac{h\p c(\omega) \cO(\omega) + c(\omega)\p \cO(\omega)}{(z-\omega)},
    \end{split}
\end{equation*} 
where we have expanded the $c(z)$ and only kept the singular terms to get to the second line. So the commutator with the BRST charge is
\bse 
    \big[Q_B,\cO\big] = h\p c \cO + c\p \cO,
\ese
which for the special case $h=1$ just becomes 
\be
\label{eqn:BRSTChargeMatterPrimaryCommutator}
    \big[Q_B,\cO|_{h=1}\big] = \p\big(c\cO\big).
\ee 
A similar result holds for the anti-holomorphic operators.

\br 
\label{rem:BRSTPrimaryTotalDerivative}
    We shall return to this in more detail later, but as some instructive foresight, this result is important as it tells us that the vertex insertions to our scattering amplitude that appear within an integral, i.e. 
    \bse 
        \int d\sig_j \sqrt{g} V_j(\sig_j),
    \ese 
    are BRST invariant if $V$ is a primary operator of weight $(1,1)$. This is just because, under BRST transformation, we just get a total derivative within our integral, and so, provided there are no domain issues, the integral vanishes. 
\er 

In light of the above remark, the next natural thing to calculate is the OPE with $c\cO$, as this is how the rest of our vertex operators appear in the scattering amplitude. We have 
\bse 
    \begin{split}
        J_B(z)c(\omega)\cO(\omega) & = \frac{h\cl c(z)c(\omega)\cO(\omega)\cl}{(z-\omega)^2} + \frac{\cl c(z)c(\omega) \p\cO(\omega)\cl}{(z-\omega)} + \frac{\cl c(z)\p c(z)\cO(\omega)\cl}{(z-\omega)} \\
        & = \frac{h\cl \p c(\omega) c(\omega) \cO(\omega)\cl}{(z-\omega)} + \frac{\cl c(\omega)\p c(\omega) \cO(\omega)\cl}{(z-\omega)} \\
        & = \frac{\cl c(\omega) \p c(\omega) \cO(\omega)\cl}{(z-\omega)}\big(1-h\big),
    \end{split}
\ese 
where again we have Taylor expanded and also used $c(\omega)c(\omega)=0$. So the BRST charge commutator is 
\be 
\label{eqn:BRSTcPrimaryCommutator}
    \big[Q_B,c\cO\big] = c \p c\cO \big(1-h\big),
\ee 
which again obviously vanishes if $h=1$. 

\br 
    Note that, in contrast to \Cref{rem:BRSTPrimaryTotalDerivative} we now need the commutator to vanish exactly for $h=1$ as the $cV$ insertions in our scattering amplitude do not come with an integral. 
\er 

Next, we consider the Laurent series of $b(\omega)$:
\bse 
    b(\omega) = \sum_{n=-\infty}^{\infty} \frac{b_n}{\omega^{n+2}} \qquad \implies \qquad b_n = \oint \frac{d\omega}{2\pi i} \omega^{m+1}b(\omega).
\ese 
Consider the anticommutator with the BRST charge, who's (holomorphic part) can be written as 
\bse 
    Q_B = \oint \frac{dz}{2\pi i} J_B(z).
\ese 
Using the standard contour method, we have\footnote{Note we take the limit $z\to \omega$. This means we don't have to worry about Taylor expanding the $\omega^{n+1}$ term and can simply just take the residue from the OPE.}
\bse 
    \begin{split}
        \big\{Q_B,b_n\big\} & = \oint \frac{d\omega}{2\pi i} \Res\bigg[ \omega^{n+1} \bigg(\frac{T(\omega)}{(z-\omega)} + \frac{J_B(\omega)}{(z-\omega)^2} + \frac{3}{(z-\omega)^3}\bigg)\bigg] \\
        & = \oint \frac{d\omega}{2\pi i} \omega^{n+1}T(\omega) \\
        & = L_n,
    \end{split}
\ese
where $L_n$ is the Laurent expansion of the \textit{full} stress-tensor, i.e. 
\bse 
    L_n = L_n^m + L_n^g,
\ese 
where $m$ and $g$ denote matter and ghost respectively. 

The important part of this result for us is that 
\bse 
%\label{eqn:BRSTChargeb0Anticommutator}
    \big\{ Q_B, b_0\big\} = L_0.
\ese 
Why is this result important? Well, recall that our physical states\footnote{At least for the point particle, and we shall assume that the same is true for the string. We will show this is true later.} must obey 
\bse
    Q_B\ket{\psi} = 0 \qand b_0\ket{\psi} = 0,
\ese 
and so they must also obey 
\be 
\label{eqn:L0psi}
    L_0\ket{\psi} = 0.
\ee 

The next important thing to note is that 
\bse 
    L_0 J_B = J_B,
\ese 
as can easily be checked using the result at the end of the last lecture, and so $Q_B$, which is given by the integral of $J_B$, does not change the $L_0$ value of the states, i.e.
\bse
    \big[Q_B,L_0\big] = 0.
\ese 
So we can consider our cohomology in levels, labelled by the $L_0$ value. To clarify, acting on any state (not just physical ones) with $Q_B$ will not change the $L_0$ eigenvalue, and so states with the same $L_0$ value form closed subsets under action of $Q_B$. Our physical states require $L_0\ket{\psi}=0$, and so we simply just restrict ourselves to this case. 

\section{The Cohomology Classes Of Physical States}

\subsection{The Tachyon}

So let's start with the lowest state. First we could try the state that corresponds to the vacuum in both the ghost and the matter sector. The vacuum for the ghost sector is the $\ket{\downarrow} = \ket{c}$ state\footnote{Recall that we don't consider the $\ket{\uparrow}$ states as physical. We can therefore label the vacuum state simply as $\ket{0}$ without having to worry if its $\ket{\uparrow}$ or $\ket{\downarrow}$.} The vacuum for the matter sector just has no excitations. So our ground state is $\ket{c;0}$, where the semicolon is used to separate the ghost sector from the matter sector. However this cannot be a physical state, as a quick calculation shows
\bse 
    L_0^gc = - c,
\ese 
and so $c$ has weight $-1$ under $L_0$. 

We therefore need to insert some primary operator in the matter sector that has $L_0$ weight $+1$.\footnote{We clearly needed to insert some kind of matter operator, otherwise our system would be purely ghost, which we certainly do not want.} Another quick calculation shows that, for a matter primary operator of weight $h$, we have 
\bse 
    L_0^m\cO = h\cO,
\ese 
and so we need a primary operator of weight $h=1$. That is, the lowest physical state we can have is
\bse 
    \ket{\psi} = \ket{c\cO}.
\ese 

As we are ultimately interested in propagators, let's consider the matter state corresponding to $e^{ik\cdot X}$. Our $L_0$ condition means that we require 
\bse 
    \frac{\a'k^2}{4} =1,
\ese 
as the left-hand side is the weight of this operator.\footnote{Note this is the mass-shell condition, which we clearly want if the state is to be physical.} We shall denote this state by 
\bse 
    \ket{ce^{ik\cdot X}} =\ket{0;k},
\ese 
where the $0$ indicates the fact that we are in the vacuum of the ghost sector and $k$ is the momentum of the state in the matter sector.

\bp 
    Each $\ket{0;k}$ state corresponds to a cohomology class.
\ep 

\bq 
    We have already shown that this state is $Q_B$ closed (as both $L_0$ and $b_0$ annihilate it), 
    \bse 
        Q_B\ket{0;k} = 0,
    \ese 
    and so we now just need to check for non-exactness. Suppose the state was exact, then there would exist some other state $\ket{\psi}$ such that 
    \bse 
        Q_B\ket{\psi} = \ket{0;k}.
    \ese 
    However, the only part of $J_B$ (and therefore $Q_B$) that acts on the matter sector is $T^m$, which we have seen in Lecture 8 doesn't change the momentum of the state. We therefore require 
    \bse 
        \ket{\psi} \sim_m \ket{k},
    \ese 
    where $\sim_m$ is meant to indicate we just mean the matter part of the state. 
    
    We just showed above that $Q_B$ does not change the $L_0$ value, and so we require $L_0\ket{\psi} =0$. We clearly have $L_0^m\ket{\psi} = \ket{\psi}$, and so we require $L_0^g\ket{\psi}=-\ket{\psi}$. That is, we need the ghost part of the state to have weight $-1$ under $L_0$. There is only two choices for this, $\ket{c}$ and $\ket{c\p c}$, as $b$ has weight $+2$ and derivatives have weight $+1$ (i.e. $L_0\ket{\p c} = 0$). We can only use $\ket{c}$ as its the only one that is annihilated by $b_0$, and so we therefore have 
    \bse 
        \ket{\psi} \sim \ket{c;k} = \ket{0;k},
    \ese 
    where we have used our notation. This gives 
    \bse 
        Q_B\big(A\ket{0;k}\big) = \ket{0;k},
    \ese 
    for some constant $A$. However, we know $\ket{0;k}$ is closed and so we are forced to choose the trivial result $\ket{0;k}=0$. 
\eq 

\br 
\label{rem:BRSTWellDefinedk}
    The above proof actually proves another important result; because $Q_B$ does not change the value of $k$, we can further divide our $L_0$ states into states with equal $k$. That is, it is a well defined idea to consider the BRST cohomology for a given value of $k$. 
\er 

The above states are the Tachyon states; they don't have any $\a/\widetilde{\a}$ excitations and so correspond to the level $N=\widetilde{N}=0$. We see this by using the decomposition \Cref{eqn:GroundState}, i.e.\footnote{We replace the semicolon in Lecture 3's notation with a comma so that the semicolon now separates ghost from matter.}
\bse 
    \ket{0;k} = \ket{0;0,k}.
\ese 
The part after the semicolon is just the definition of the Tachyon.

\br 
    We could also see that we are dealing with Tachyons from the condition 
    \bse 
        \frac{\a' k^2}{4} = 1 \qquad \implies \qquad k^2 = \frac{4}{\a'}.
    \ese 
    Putting this together with $k^2=-m^2$, we have 
    \bse 
        m^2 = - \frac{4}{\a'} = - \frac{26-2}{6\a'},
    \ese 
    which is \Cref{eqn:TachyonMass} with $d=26$.
\er 

\subsection{First Excited State}

We now want to consider the case where we include an excitation of the ground state. This can come in three ways: we can use $c_{-1}$, $b_{-1}$ or $\a_{-1}/\widetilde{\a}_{-1}$. For simplicity of notation, we shall just write the $\a$s and drop the $\widetilde{\a}$s, but we must remember that our level matching condition requires them to come together. 

The question is, what condition do we need to place on the weight, and therefore on $k^2$, of the matter operator? We know that all $\a_{-1}$, $b_{-1}$ and $c_{-1}$ are raising operators and so will contribute $+1$ to the $L_0$ value. We still have the $-1$ contribution from $\ket{c}$, and so we now require $h=0$ for the primary operator. That is, our general state is 
\be 
\label{eqn:BRSTMasslessGeneralState}
    \ket{\psi} = \big( e_{\mu}\a^{\mu}_{-1} + \beta b_{-1} + \g c_{-1}\big)\ket{0;k}, \qquad k^2 = 0.
\ee
This is the set of massless states. 

\br 
    If you are not happy with using the fact that the operators are raising operators above, you can show it explicitly. For example, we saw towards the end of Lecture 11 that 
    \bse 
        c_n\ket{c} = \oint \frac{dz}{2\pi i} z^{n-2}c(z)c(0),
    \ese 
    which gives 
    \bse 
        c_{-1}\ket{c} = -c(0)\p^2 c(0).
    \ese 
    You can then use this to show that 
    \bse 
        T^g(z)\big(c_{-1}\ket{c}\big)(\omega) \sim \frac{2\cl c(\omega)\p c(\omega)\cl}{(z-\omega)^3} - \frac{\cl c(\omega)\p^3c(\omega)\cl + \p^2c(\omega)\p c(\omega)\cl}{(z-\omega)},
    \ese 
    which has no $1/z^2$ term and so 
    \bse 
        L_0^gc_{-1}\ket{c} = 0,
    \ese 
    and so we require $h=0$ for the primary operator.
    
    Similar calculations will give the same kind of result for the other two terms. 
\er 

Using \Cref{rem:BRSTWellDefinedk}, we can consider the cohomology for these massless states independently of any massive states (i.e. consider just $k^2=0$ states). You can show (to save space, the calculation is not done here, but it follows exactly the same ideas as all the other calculations done) that 
\be 
\label{eqn:BRSTChargealphabcminus1}
    \begin{split}
        Q_B\big(\a^{\mu}_{-1}\ket{0;k}\big) & = \sqrt{2\a'} k^{\mu} c_{-1}\ket{0;k} \\
        Q_B\big(b_{-1}\ket{0;k}\big) & = \sqrt{2\a'} k_{\mu}\a^{\mu}_{-1}\ket{0;k} \\
        Q_B\big(c_{-1}\ket{0;k}\big) & = 0 
    \end{split}
\ee

\br 
    If you plan to show the above results (and its worth doing if you haven't been doing the calculations this far yourself), note you can save yourself a bit of time by doing them in the order presented. That is, once you show the $\a^{\mu}_{-1}\ket{0;k}$ equation, the $c_{-1}\ket{0;k}$ condition follows via the nilpotency of the BRST charge.
\er 

\chapter{No Ghost Theorem}

\section{Finishing Off The String BRST States}

At the end of the last lecture we showed that for our bosonic string, the BRST charge acts as \Cref{eqn:BRSTChargealphabcminus1}. \Cref{eqn:BRSTMasslessGeneralState} tells us that we have $26+2=28$ independent states --- 26 from the $\a^{\mu}_{-1}$s and 2 from the $c_{-1}$ and $b_{-1}$. We know from our discussion towards the start of the course, that we should only have 24 states (remember we showed that our states form a $24\otimes 24$ representation of $SO(24)$), so we need to fix this somehow.

We do this by considering the cohomology. Using both \Cref{eqn:BRSTChargealphabcminus1} and \Cref{eqn:BRSTMasslessGeneralState}, our BRST closed condition for physical states becomes
\bse 
    0 = Q_B\ket{\psi} = \sqrt{2\a'}\big( e_{\mu}k^{\mu} c_{-1} + \beta k_{\mu}\a^{\mu}_{-1} \big) \ket{0;k},
\ese
which forces us to impose the conditions 
\bse 
    e\cdot k = 0, \qand \beta = 0.
\ese 
This clearly removes the $b_{-1}$ states\footnote{Well, more specifically, it gives them zero norm.} and gives us a condition on the $e_{\mu}$, which in turn gives us a relation between the 26 $\a^{\mu}_{-1}$ states. We therefore now have $25+1=26$ independent states. 

Ok, good, but we still need to remove another 2, and one of them needs to be the $c_{-1}$ state (as this wasn't present in our $SO(24)$ representation). These two states are removed by now imposing the non-exactness of our physical states. \Cref{eqn:BRSTChargealphabcminus1} shows us that the $c_{-1}$ state is exact, and so is equivalent to the zero state in the cohomology class.\footnote{So again it is a state of zero norm.} We also see that the $\a^{\mu}_{-1}$ states are exact if $e_{\mu} \propto k_{\mu}$, and so we place a further restriction on the $\a^{\mu}_{-1}$s. We now have $24+0=24$ states.\footnote{Or 24 positive norm states and 2 zero norm states.}

That is, our physical states become\footnote{The subscript $p$ is included to show we mean physical states.}
\bse 
    \ket{\psi_p} = \sqrt{2\a'} e_{\mu}\a^{\mu}_{-1} \ket{0;k}, \qquad e \cdot k = 0, \qand e_{\mu} \sim e^{\prime}_{\mu} + Ak_{\mu},
\ese 
where the second condition is an equivalence class condition, with $A$ being some constant.

\br 
    Our equivalence class condition is actually very interesting and has a nice tie in with electromagnetism. The momentum $k_{\mu}$ is given by the derivative of our $X$ scalar fields. So we can think of our equivalence class condition as saying $e_{\mu}$ is equivalent up to the derivative of a scalar field. This is exactly the gauge invariance of the electromagnetic vector potential in Lorenz gauge, i.e. $A_{\mu} \sim A_{\mu}^{\prime} +\p_{\mu}f$. In fact, if you were to approach QED using BRST quantisation, you would arrive exactly at this result. So we see that the BRST symmetry of our string theory has the BRST of QED (in 26 dimensions) built into it.  
\er 

\subsection{Inner Products}

We can see the above result nicely by considering the inner product of our space. Let's assume we use an orthogonal basis, then our general state \Cref{eqn:BRSTMasslessGeneralState} \textit{before} we place any restrictions on it has the inner product, 
\bse 
    \braket{\psi}{\psi} = e^*\cdot e + \beta^*\g + \g^*\beta,
\ese 
where we have used 
\bse
    \big[\a_m^{\mu}, \a_{n}^{\nu}\big] = m\, \eta^{\mu\nu}\del_{m+n,0}, \qquad \big\{c_1,c_{-1}\big\} = \big\{b_1,b_{-1}\big\} = 0 \qand \big\{c_1,b_{-1}\big\} = \big\{c_1,b_{-1}\big\}  = 1.
\ese
This gives us 26 positive norm states (the 26 states corresponding to the $\a^{\mu}_{-1}$s) and 2 negative norm states (the $b_{-1}$ and $c_{-1}$ states). 

If we then consider \textit{only} the $Q_B$ closed condition, we drop the $b_{-1}$ states and impose the condition $e\cdot k=0$. This latter condition can be satisfied in two ways: we can either have $e_{\mu}$ being orthogonal to $k_{\mu}$ or, as we have $k^2=0$, we can have $e_{\mu}$ being proportional to $k_{\mu}$. So our general state becomes 
\bse 
    \ket{\psi_1} = \big(e^{\perp}_{\mu}\a^{\mu}_{-1} + e^{\parallel}_{\mu} \a^{\mu}_{-1} + \g c_{-1}\big)\ket{0;k}.
\ese 
Again using an orthogonal basis, our inner product becomes 
\bse 
    \braket{\psi_1}{\psi_1} = e_{\perp}^*\cdot e_{\perp} + e_{\parallel}^*\cdot e_{\parallel} = e_{\perp}^*\cdot e_{\perp} 
\ese 
where we have used the fact that $e^{\parallel}_{\mu}\propto k_{\mu}$ and so squares to zero. It is clear that there is only 1 state that is parallel to $k_{\mu}$ (that is we decompose our 25 $\a^{\mu}_{-1}$ states into a 1+24 split), so we now have 24 positive norm states (the $e^{\perp}_{\mu}$s) and 2 zero norm states (the $e^{\parallel}$ and $c_{-1}$). 

This is better, but we don't want zero norm states. We then notice that these zero norm states are \textit{exactly} the states that are $Q_B$ exact, and so we can mod them out, giving us 24 states, all of which have positive norm. 

\section{No Ghost Theorem}

What we now want to show is that at every level the BRST cohomology has positive inner product and that it is isomorphic to the spectra in light-cone gauge. This is the content of the \textit{no ghost theorem}. We have of course just shown this for level 0 and level 1. 

So we need to find the cohomology of $Q_B$ and show it obeys the above two conditions. How do we do this? The answer is to consider it in two steps: first consider a simplified BRST operator, which we denote $Q_1$ for reasons that will be explained shortly, and find it's cohomology and show it obeys the above properties; then show that the cohomology of $Q_1$ is isomorphic to the cohomology of the full $Q_B$.

We start by defining the light-cone oscillators 
\bse 
    \a^{\pm}_m := \frac{1}{\sqrt{2}}\big(\a^0_m \pm \a^1_m\big),
\ese 
where $\{0,1\}$ are two basis directions used to define the light-cone. These satisfy the following commutation relations 
\bse 
    \big[ \a^+_m,\a^-_n \big] = -m \del_{m+n,0} = \big[\a^-_m,\a^+_n\big], \qand \big[\a^+_m,\a^+_n\big] = 0 = \big[\a^-_m,\a^-_n\big].
\ese 

Next we define the quantum number 
\be 
\label{eqn:Nlc}
    N^{lc} := \sum_{m=-\infty}^{\infty} \frac{1}{m} \tcl \a^+_{-m}\a^-_m\tcl,
\ee 
where of course $m=0$ is not included in the sum. We can rewrite the sum above using only positive $m$ as 
\bse 
    N^{lc} = \sum_{m=1}^{\infty} \frac{1}{m} \big( \a^+_{-m}\a^-_{m} - \a^-_{-m}\a^+_m \big),
\ese 
where we have used the $\tcl \tcl$ to put all the $+m$ subscripts to the right. Now image we acted this on a state with $\a^+_{-k}$, then the second term vanishes (as we can move $\a^+_m$ through $\a^+_{-k}$ and annihilate the ground state), whereas the first term gives a contribution with weight $-k$. The factor $k$ is cancelled by the $1/m$ in the sum, and so we just get 
\bse 
    N^{lc}\a^+_{-k}\ket{0} = -\a^+_{-k}\ket{0}.
\ese 
Similarly we have 
\bse 
    N^{lc}\a^-_{-k}\ket{0} = \a^-_{-k}\ket{0},
\ese 
and so we see that $N^{lc}$ counts the number of $+$ excitations minus the number of $-$ excitations.

The next question we ask is "what is the $N^{lc}$ charge of $Q_B$?" Of course we can check this explicitly, but its insightful to notice something first: $N^{lc}$ looks \textit{almost} like the boost generators, which would of course commute with $Q_B$ (as it is Lorentz invariant). The emphasis on almost is down to the fact that our sum does not include $m=0$, and therefore the centre-of-mass piece is not included. So our commutator will vanish up to the this $m=0$ piece. 

We see this more explicitly by considering an expression for $Q_B$ in terms of the Laurent expansion coefficients of $T^m/c/b$. Recall that 
\bse 
    Q_B = \oint \frac{dz}{2\pi i} J_B(z),
\ese 
and 
\bse 
    J_B = \cl cT^m\cl + \cl b c \p c \cl + \frac{3}{2}\p^2 c.
\ese 
We can use the Laurent expansions\footnote{Note the superscript $m$ on $T^m$ is there to indicate that it is the matter (i.e. not the ghost) stress tensor. It is not an index like the $n$ is.} 
\bse 
    T^m = \sum_{n} \frac{L^m_n}{z^{n+2}}, \qquad c = \sum_n \frac{c_n}{z^{n-1}}, \qand b = \sum_n \frac{b_n}{z^{n+2}}
\ese 
to obtain a Laurent expansion for $J_B$ and, by taking the $1/z$ coefficient, obtain an expression for $Q_B$ in terms of $L^m_n$, $c_n$ and $b_n$. The full calculation of this is done in the lecture video, however here we shall only do the bit we need for our discussion; we want to find the commutator with $N^{lc}$, which is expressed solely in terms of the $\a$s, and so the pure ghost part of $Q_B$ will have vanishing $N^{lc}$ charge. In other words, there are no $\a^{\pm}$s in the $\cl bc\p c\cl $ or $\p^2c$ parts of $J_B$ and so both the number of $+$ excitations and the number of $-$ excitations are zero. 

So considering just the $\cl cT^m\cl$ term, we have\footnote{Note on the last equality we dropped the conformal normal ordering colons. This is simply because $L^m$ and $c_n$ have vanishing OPE and so it is the same with or without the colons.}
\bse 
    Q_B = \oint \frac{dz}{2\pi i} \cl  \sum_{\ell,n} \frac{c_{\ell}}{z^{\ell-1}} \frac{L^m_n}{z^{n+2}} \cl = \sum_n c_n L^m_{-n}.
\ese
Then we use the result\footnote{I can't quite prove this result. If anyone reading this can give me a proper proof, I would be very grateful. I would update these notes and give you credit obviously.}
\be
\label{eqn:LmLaurent}
    L^m_n \sim \sum_{\ell} \a^-_{\ell} \a^+_{-(n+\ell)}, 
\ee
we see that there are three possible values for the $N^{lc}$ charge of $Q_B$: $+1$, $0$ and $-1$. To clarify, unless we have either (or both) $\ell=0$ or $n+\ell=0$, the sum for $L^m_n$ has the same number of $+$ excitations as it does $-$ excitations, and so the $N^{lc}$ charge vanishes. This leaves us the two cases of 
\bse 
    \a^-_{-m}\a^+_0, \qand \a^-_0\a^+_{-m},
\ese 
which give a $N^{lc}$ charge of $+1$ and $-1$ respectively. We can therefore decompose our BRST into three terms, characterised by their $N^{lc}$ charge: 
\bse 
    Q_B = Q_1 + Q_0 + Q_{-1},
\ese 
where the subscript tells us the $N^{lc}$ charge, i.e. 
\bse 
    \big[ N^{lc}, Q_j\big] = j Q_j.
\ese 
The one we will focus on is
\bse 
    Q_1 := -\sqrt{2\a'} k^+ \sum_{m\neq0} c_m\a^-_{-m}.
\ese 

\br 
    Note that each of the $Q_j$ also increase the ghost number by 1. We see this with the $Q_1$ expression straight away: $m<0$ creates a $c$ and $m>0$ destroys a $b$, both of which increase the ghost number by $1$. The same exact thing holds for $Q_0$ and $Q_{-1}$.  
\er 

\subsection{$Q_1$ Cohomology}

We said at the start of this section that we were going to consider the cohomology of a simplified BRST operator, $Q_1$, and find its cohomology. The first thing we need to ask is "does $Q_1$ have a cohomology?" That is, we need to check that it is nilpotent. Well, consider the following:
\bse 
    Q_B^2 = Q_1^2 + \{Q_1,Q_0\} + \{Q_1,Q_{-1}\} + Q_0^2 + \{Q_0,Q_{-1}\} + Q_{-1}^2.
\ese
We know this must vanish (as $Q_B$ is nilpotent), the question is what does it tell us about the terms on the right-hand side? Well we can group these into terms with the same $N^{lc}$ charge. The grouping is clearly just 
\bse 
    0 = \underbrace{Q_1^2}_{+2} + \underbrace{\{Q_1,Q_0\}}_{+1} + \underbrace{\{Q_1,Q_{-1}\} + Q_0^2}_{0} + \underbrace{\{Q_0,Q_{-1}\}}_{-1} + \underbrace{Q_{-1}^2}_{-2}.
\ese 
The equality holds identically, and so each grouped part must itself vanish, so we conclude that $Q_1$ is indeed nilpotent and so has a cohomology.  

\br 
    Note, because we have an expression for $Q_1$, we could have actually just shown the nilpotency by showing explicitly that $\{Q_1,Q_1\}=0$, which follows straight forwardly as it only contains $c$s and no $b$s. 
\er 

So we want to find the cohomology of $Q_1$. We could do this explicitly, but instead we shall follow a useful trick of mathematics. We define what is essentially the adjoint of $Q_1$, 
\be
\label{eqn:R}
    R := \frac{1}{\sqrt{2\a'}k^+} \sum_{m\neq0} b_m\a^+_{-m},
\ee 
as well as the operator
\bse 
    S := \{Q_1,R\}.
\ese 
Now note that 
\bse 
    \begin{split}
        \big[Q_1,S\big] & = Q_1\big(Q_1R+RQ_1\big) - \big(Q_1R+RQ_1\big)Q_1 \\
        & = Q_1RQ_1-Q_1RQ_1 \\
        & = 0,
    \end{split}
\ese 
and so we can look at the cohomology within the eigenspaces of the $S$ operator, with the full cohomology of $Q_1$ being given by the union of all the results. 

The useful thing to first do is to write $S$. Using the relation
\bse 
    \{A,BC\} = \{A,B\}C -B[A,C],
\ese 
and similarly for $\{AB,C\}$, we can show 
\bse 
    \begin{split}
        S & = - \sum_{m,n\neq 0} \big( -m b_nc_m + \a^-_{-m}\a^+_{-n} \big) \del_{m+n,0} \\
        & = \sum_{m,n\neq 0} \big( m b_nc_m - \a^-_{-m}\a^+_{-n} \big) \del_{m+n,0} \\
        & = \sum_{m=1}^{\infty} m b_{-m}c_m + mc_{-m}b_m - \a^-_{-m}\a^+_{m} -\a^+_{-m}\a^-_m.
    \end{split}
\ese 
If we then define $N_{bm}$ to be the number of $b_m$ oscillators and similarly for $N_{cm}$, $N^+_m$ and $N^-_m$, we get 
\be
\label{eqn:SQ1}
    S = \sum_{m=1}^{\infty} m\big( N_{bm} + N_{cm} +N^+_m + N^-_m\big),
\ee 
where we get the prefactor $-m$ for $N^{\pm}_m$ from the commutation relation between $\a^+_m$ and $\a^-_n$. 

Now suppose $\ket{\psi}$ is a physical state, it must therefore be $Q_1$ closed, and so 
\bse 
    S\ket{\psi} = Q_1R\ket{\psi}. 
\ese 
Now, we want $\ket{\psi}$ to be an eigenstate of $S$. Let it's eigenvalue be $s$, then 
\bse 
    s\ket{\psi} = Q_1R\ket{\psi},
\ese 
so \textit{for all} $s\neq 0$, $\ket{\psi}$ is a $Q_1$-exact state and so is not in the cohomology. 

What about $s=0$, is this exact? Well $Q_1$ contains a $c$ in its definition, and so carries ghost charge $+1$. Combining this with the fact that all the terms on the right-hand side of \Cref{eqn:SQ1} are non-negative, we see that it is impossible for a state that has $s=0$ to be $Q_1$-exact. In other words, if $\ket{\psi}$ was $Q_1$-exact then there would exist a $\ket{\xi}$ such that $\ket{\psi}=Q_1\ket{\xi}$. If $\ket{\psi}$ is in the kernel of $S$, we must have $S\ket{\xi}=-1$, but this is not possible. Therefore every element of the kernel of $S$ is $Q_1$-closed but is not $Q_1$-exact. 

So the kernel of $S$ is the cohomology of $Q_1$. The non-negative nature of the terms on the right-hand side of \Cref{eqn:SQ1} force us to conclude each term vanishes itself within our cohomology. We therefore conclude that our cohomology does not allow longitudinal excitations or ghost excitations. This is exactly the case we ended up at at the end of the previous section, but now our result holds for any excitation level. 

\br 
    The above calculation is completely analogous to the study of so-called \textit{harmonic forms} in differential geometry. These will not be discussed here,\footnote{Mainly because I haven't quite got to that part of Renteln's book yet.} but this remark is just included to show that the above method is more versatile then just this specific calculation. 
\er 

\subsection{$Q_B$ Cohomology}

So we have successfully shown the no ghost theorem! Well, not quite. What we have shown is that $Q_1$ satisfies the no ghost theorem, but $Q_1$ is not the full BRST charge, $Q_B$ is. We therefore need to show that the cohomologies of these two charges are isomorphic.

We define 
\bse 
    \widetilde{S} := \{Q_B,R\} = S + U, \qquad U := \{Q_0+Q_{-1},R\}.
\ese 
Following the same arguments as above, we see that the cohomology of $Q_B$ is the kernel of $\widetilde{S}$. So what is the kernel of $\widetilde{S}$? We see, from the fact that $R$ has $N^{lc}$ charge $-1$, that $U$ has $N^{lc}$ charge $-1$ (for the $Q_0$ part) or $-2$ (for the $Q_{-1}$ part). In other words, $U$ lowers the value of $N^{lc}$ by either one or two units. We therefore see, in a basis where $S$ is the diagonal of $N^{lc}$, that $U$ is lower triangular. It is a fact that the kernel of a lower triangular matrix can be no larger then the kernel of its diagonal part. In other words, the kernel of $\widetilde{S}=S+U$ can, at most, be isomorphic to the kernel of $S$. 

\bl 
    The kernel of $\widetilde{S}$ is indeed isomorphic to the kernel of $S$.
\el 

\bq 
    Let $\ket{\psi}$ be an element in the kernel of $S$. Then define 
    \be 
    \label{eqn:StateSUExpansion}
        \ket{\chi} := \big( \b1 - S^{-1}U + S^{-1}US^{-1}U - ...\big)\ket{\psi}.
    \ee 
    Firstly we note that the action of $S^{-1}$ is valid as $S(U\ket{\psi}) \in \{-1,-2\}$. This is clearly a one-to-one relationship, and so $\{\ket{\chi}\} \cong \{\ket{\psi}\}$. Finally, we have 
    \bse 
        \begin{split}
            \widetilde{S}\ket{\chi} & = \big(S+U\big)\ket{\chi} \\
            & = \big(S + U - SS^{-1}U - US^{-1}U + SS^{-1}US^{-1}U + US^{-1}US^{-1} - ...\big) \ket{\psi} \\
            & = 0,
        \end{split}
    \ese 
    where all but the first term cancel, and then we have used $S\ket{\psi}=0$. This tell us that the kernel of $S$ is isomorphic to a subset of the kernel of $\widetilde{S}$. Putting this together with the comment made before the Lemma (that the kernel of $\widetilde{S}$ is a subset of the kernel of $S$) we see conclude 
    \bse 
        \ker S \cong \ker \widetilde{S}.
    \ese
\eq 

So we have shown that the cohomology of $Q_B$ is isomorphic to the cohomology of $Q_1$, which is isomorphic to the Hilbert space in light-cone gauge. This is brilliant! We just need to check that the inner product on our cohomology is positive definite (so we don't have any ghosts kicking about). The key to showing this is to realise that the adjoint does not change $+\longleftrightarrow -$ in $\a^{\pm}_m$, but instead simply acts as
\bse 
    \big(\a^{\pm}_{m}\big)^{\dagger} = \a^{\pm}_{-m}.
\ese 
Therefore the inner product between two states is only non-vanishing if their $N^{lc}$ charges sum to zero. To further clarify, consider the example of the inner product of $\a^-_{-1}\ket{0}$ with itself:
\bse 
    \bra{0}\big(\a^-_{-1}\big)^{\dagger} \a^-_{-1}\ket{0} = \bra{0} \a^-_{1}\a^-_{-1}\ket{0} = \bra{0} \a^-_{-1}\a^-_{1}\ket{0} = 0,
\ese 
where we have used $[\a^-_m,\a^-_n]=0$. Putting this result together with the fact that $U\ket{\psi}$ has $N^{lc}$ charge $-1$ or $-2$, we see that the only non-vanishing contribution to the inner product of any state in the cohomology of $Q_B$ is the $\b1$ part, that is
\bse 
    \braket{\chi}{\chi} = \braket{\psi}{\psi}.
\ese 
The right-hand side is positive definite (as they are exactly the states of the light-cone Hilbert space) and so we conclude that the inner product for the cohomology of $Q_B$ is positive definite. 
\chapter{OCQ States \& Transition Functions}

\section{There Are Really No Ghosts}

We actually slightly misused our result above. We said that if we could show our inner product was positive definite that "we don't have any ghosts kicking about". We can not quite make this statement yet. For example, it could turn out that we do have ghosts states, however there is some rule that insists they come with matter states such that the \textit{total} inner product is positive. That is the magnitude of the inner product from the ghost part of the state is smaller then that of the matter part. 

So how do we go about showing that there really are no ghosts? We begin by giving the full expression for $Q_B$ in terms of the Laurent expansions\footnote{Again the calculation for this is done in the video for lecture 16, but to save space we don't do it here.}
\be
\label{eqn:QBLaurentExpansion}
    Q_B = \sum_{n=-\infty}^{\infty} c_n L^m_{-n} +\sum_{m,n=-\infty}^{\infty}\frac{(m-n)}{2}\tcl b_{-(m+n)}c_nc_m\tcl  - c_0.
\ee 
Note the operator normal ordering colons on the middle term. This will prove important in just a mo. Now let's imagine acting this on the state
\bse 
    \ket{\psi} = \ket{\downarrow}\otimes\ket{\phi},
\ese 
where $\ket{\phi}$ is some arbitrary matter state. As we are in the ghost vacuum, any operation with $c_{n}/b_{n}$ for $n>0$ will annihilate the system, as will $b_0$. Therefore we can just take out the bits of $Q_B$ that remain. 

For the first term this clearly just restricts the sum to $n\geq0$, and the $c_0$ term isn't effected, but what about the middle term? Well with a bit of thought, we see that every term in these summations will annihilate the state. This comes simply from the fact that the three subscripts sum to $0$ and so at at least one of them must be positive. The only exception being $m=n=0$, but in this case we have $b_0$ which annihilates the state (as the term appears within operator normal ordering).\footnote{In fact this term vanishes for two other reasons: firstly the coefficient $(m+n)=0$ and secondly because $c_0^2=0$. Take your pick of your fave.} 

So if $\ket{\psi}$ is meant to be a state in our cohomology (i.e. $Q_B$-closed), we have
\bse 
    Q_B\ket{\psi} = \sum_{n=0}^{\infty} c_{-n}\big(L^m_{n} - \del_{n,0}\big) \ket{\downarrow}\otimes\ket{\phi} = 0.
\ese 
It follows, from the fact that we have removed all the terms that annihilate the ghost vacuum, that 
\bse 
    L^m_n\ket{\phi} = 0, \quad \forall n\geq1, \qand \big(L^m_0-1\big)\ket{\phi} = 0.
\ese 
These two conditions tell us that the dual operator to $\ket{\phi}$ is, respectively, a primary operator and that it has holomorphic weight $+1$. Of course playing the same game with $\widetilde{Q}_B$ will tell us that the antiholomorphic weight is also $+1$. 

This is a step towards the result we have seen many times so far: namely that $(1,1)$ primary operators are allowed in our scattering amplitudes. We say "a step towards" because we have only shown that these states are $Q_B$-closed. We need to check when such states are $Q_B$-exact so we can mod-out such states. 

\bp 
    Let $\ket{\chi}$ be a matter state. Then the matter state $\ket{\phi}$ is $Q_B$-exact only if 
    \bse 
        \ket{\phi} = \sum_{k>0}L^m_{-k}\ket{\chi}.
    \ese 
\ep 

\bq 
    We can show this two ways. 
    \ben 
        \item Assuming $\ket{\phi}=L^m_{-k}\ket{\chi}$, then the inner product is
        \bse 
            \begin{split}
                \braket{\phi}{\phi} & = \braket{\phi}{L^m_{-k}\chi} \\
                & = \braket{L^m_k\phi}{\chi} \\
                & = 0,
            \end{split}
        \ese 
        where we have used $(L^m_{-k})^*=L^m_k$ and the fact that $\ket{\phi}$ is primary. So such states are null. We have previously shown that the states in the $Q_B$ cohomology are positive definite, so null states must be exact. 
        \item Assume $L^m_{-k}\ket{\chi}=Q_B\ket{\xi}$ for some state $\ket{\xi}$. Now if we want $\ket{\phi}=L^m_{-k}\ket{\chi}$ to hold, we require that $\ket{\chi}$ be in the ghost vacuum (as $\ket{\phi}$ is and $L^m_{-k}$ can't change that). However, every term in $Q_B$ has one more $c$ than $b$, and so in order to get $Q_B\ket{\xi}$ to be in the ghost vacuum we require that $\ket{\xi}$ have one more $b$ than $c$. We only have two options:
        \bse 
            \ket{\xi} = \sum_k b_k \ket{\rho}, \qand \ket{\xi} = \sum_{k,\ell,p} \tcl b_k b_{\ell} c_p \tcl \ket{\rho},
        \ese 
        for some purely matter state $\ket{\rho}$, as $Q_B$ contains at most 2 $c$s and 1 $b$; that is if we had 3 $b$s and 2 $c$s in $\ket{\xi}$, then the best $Q_B$ could do is reduce it to 1 $b$ and 1 $c$ both with positive index. 
        
        Now, as $\ket{\rho}$ is a purely matter state, it follows that we require all the $b$/$c$ indices to be negative (otherwise we just annihilate $\ket{\rho}$). This places then means we need all the $b$/$c$ indices in $Q_B$ to be positive. We have already seen that this cannot be done for the middle term of $Q_B$ and so we are only left with the possibility of 
        \bse 
            \ket{\xi} = \sum_{k>0} b_{-k}\ket{\rho}.
        \ese 
        Now act on this with $Q_B$, 
        \bse 
            Q_B\ket{\xi} = \sum_{n,k>0} c_nL^m_{-n}b_{-k} \ket{\rho} = \sum_{k>0}L^m_{-k} \ket{\rho}.
        \ese 
        This is just the statement we wanted, namely that $\ket{\phi} = L_{-k}\ket{\chi} = Q_B\ket{\xi}$ and so is $Q_B$-exact.
    \een 
\eq 

\br 
    The second proof highlights the "only" part of the proposition, i.e. we showed only the $\ket{\xi}=\sum_{-k}\ket{\rho}$ states will do, and all other possibilities necessarily vanish.
\er 

So what we have shown is that the $Q_B$ cohomology contains $(1,1)$ primary operators that are ghost vacuum states modulo primary descendant states. What we want to say is that \textit{every} class in the cohomology contains at least one such term, i.e. each class has a representative that contains no ghosts. We do this by considering the representatives of the form \Cref{eqn:StateSUExpansion}, and defining the quantum number 
\bse 
    N' := 2N^- + N_b + N_c.
\ese 
Now, from \Cref{eqn:SQ1} and the fact that $\ket{\psi}$ is in the kernel of $S$, we see straight away that $N'\ket{\psi} = 0$. It is trivially true that $S$ doesn't change the $N'$ value, and so it is just $U$ that we need to worry about. 

Recall that $U := \{Q_0+Q_{-1},R\}$. \Cref{eqn:R} tells us that $R$ changes the value of $N'$ by $-1$: for $m>0$ we reduce $N_c$ by one unit and for $m<0$ we reduce $N^-$ by one unit and increase $N_b$ by one unit. So now we just need to work out what $Q_0+Q_{-1}$ does. To do this consider the each term in \Cref{eqn:QBLaurentExpansion}. The easiest term is the $c_0$ term, which just increases $N_c$ by 1, and so gives $N'=+1$. Next consider the middle term; with a bit of thought we see that all that matters for the $N'$ value is the number of positive indices vs. the number of negative ones, and therefore this term can have either $N'=\pm1$. Finally consider the first term. Using \Cref{eqn:LmLaurent}, we can write it as 
\bse 
    \sum_{n=-\infty}^{\infty} \sum_{\ell} c_n \a^-_{\ell}\a^+_{-(n+\ell)}.
\ese

\bcl 
    The highest $N'$ value this can take is $+1$. 
\ecl

\bq 
    Clearly if we don't create a $-$ excitation (i.e. increase $N^-$), this term will never give $N'>1$. We therefore require $\ell<0$. If we then had $n>0$, the $c_n$ term would decrease $N_b$, giving us (at most)\footnote{As if $n<|\ell|$ then the $\a^+_{-(n+\ell)}$ term will decrease $N^-$.} $N'=+1$. So we need to take $n<0$, but this then means that the $\a^+_{-(n+\ell)}$ has a positive index and so reduces $N^-$. So we see the highest value it can take it $+1$. 
\eq 
Putting all of this together we see that the maximal $N'$ value that every term in $Q_B$, and therefore in $Q_0+Q_{-1}$, can take is $+1$. Putting this together with the $R$ result we have that the $N'$ value of $U$ is non-positive definite. However, by definition, $N'$ is non-negative definite and so we are forced to conclude that $U$ has vanishing $N'$ value. Finally we see that  every state in our representation, \Cref{eqn:StateSUExpansion}, must have vanishing $N'$ value, and more specifically must have $N_c=0=N_b$. There really are no ghosts!

This is a really important and nice result, so we repeat what we have done here one more time for clarity. We showed that the cohomology of $Q_1$ was exactly the kernel of $S$. We then showed that the cohomology of $Q_1$ and the cohomology of $Q_B$ were isomorphic, and there existed a representation in every class of the form \Cref{eqn:StateSUExpansion}. Next, we showed that the cohomology contains states purely (i.e. are ghost vacuum) dual to $(1,1)$ primary operators modulo primary descendants. Finally we showed that the representations \Cref{eqn:StateSUExpansion} were indeed of this form, telling us that \textit{every class} of our cohomology contains at least one element that has no ghost excitations, and is dual to a $(1,1)$ matter primary operator. These states are known as \textit{OCQ-type} states.

\br 
    The rest of this lecture typed here actually appears in the video for lecture 18. I have decided to just include this here because during lecture 18 Dr. Minwalla wraps up the discussion of the Bosonic string (for now) and moves on to study the Fermionic (or \textit{super-}) string. I just thought it would be nice to have the next lecture begin a new rather then do it in the middle of one. 
\er 

\section{From Metrics to Transition Functions}

Before moving on to the superstring, we want to make one last comment on our Bosonic scattering amplitudes. We have just shown (once again) that the operator insertions must be $(1,1)$ matter primary operators. The bit we haven't really talked too much about is the integral over the moduli:
\bse 
    \big(b,\hat{g}\big) = \int d\tau_k \prod_{k=1}^{m} \int d^2\sig \sqrt{\hat{g}} \bigg(\frac{\p \hat{g}}{\p \tau_k}\bigg)_{\a\beta} b^{\a\beta}.
\ese 
This section aims to show how we recast this integral over Weyl inequivalent metrics in terms of transition maps between patches on a complex manifold. This might seem at first like a very strange thing to try and do, but it is indeed common practice (particularly with mathematicians) and is an example of what are known as \textit{Beltrami Differentials}. 

First, let's quickly consider the case of just changing our metric in the above expression. From the tracelessness of $b^{\a\beta}$ it follows that any transformation of the form 
\bse 
    \del_k\hat{g}_{\a\beta} \propto \hat{g}_{\a\beta} 
\ese 
will vanish. These are the Weyl transformations, and so in what follows we can essentially forget about Weyl factors.

So how do we go about changing this to a condition on the transition functions? Well first let's clarify what it is we want to do. As it stands, we have some complex manifold which we can describe using a set of coordinate patches and their relevant transition functions. For a given point in our moduli space, $\tau_0$, we can use conformal transformations to make the metric Weyl-equivalent in each patch separately. That is, for the $m$-th patch 
\bse 
    \hat{g}_m(\tau_0) = e^{\phi(z_m,\overline{z}_m)} dz_md\overline{z}_m.
\ese 
If we are going to do this on every patch, it follows immediately that we require our transition functions to be holomorphic --- if they were not we would get $(dz_n)^2$ and $(d\overline{z}_n)^2$ terms in the metric on the $n$-th patch. We want to enforce that this Weyl-equivalence always holds and any deviation from it due to changing moduli point shall be absorbed into the transition functions.

\subsection{Transition Functions}

So, let's consider a small moduli change to the coordinates, 
\bse 
    \begin{split}
        z'_m & = z_m + \del \tau_k v_{km}^{z_m}(z_m,\overline{z}_m) \\
        \overline{z}'_m & = \overline{z}_m + \del \tau_k v_{km}^{\overline{z}_m}(z_m,\overline{z}_m),
    \end{split}
\ese 
where $v_{km}$ is a vector, as the upper index indicates, and where the $m$ reminds us that it is only defined on the $m$-th patch. This transformation clearly changes our metric condition with
\be 
\label{eqn:MetricModuliChange}
    \del_k\hat{g}_m(\tau_0 +\del\tau_k) \propto \del\tau_k\big( \p_{z_m}v^{\overline{z}_m}_{km} (dz_m)^2 + \p_{\overline{z}_m}v^{z_m}_{km} (d\overline{z}_m)^2\big).
\ee 

\br
    Note that the above result tells us that we need $v_{km}^{z_m}$ to \textit{not} be holomorphic, as otherwise the derivative just vanishes! Similarly we need $v_{km}^{\overline{z}_m}$ be not be antiholomorphic. 
\er 

These new terms in the metric are unwanted (we only want $dzd\overline{z}$ terms), and so we need to find a way to absorb them into the transition function. How does one go about doing that? Well the answer comes from noticing that these terms were generated by the changes in $z_m$ by changing the moduli. That is it is 
\bse 
    \frac{\p z_m}{\p \tau_k} = v_{km}^{z_m}, \qand \frac{\p \overline{z}_m}{\p \tau_k} = v_{km}^{\overline{z}_m}
\ese 
that are to blame. This is a nice relationship and something we can simply add in to our transition function. That is, instead of considering a transition function that goes from $(z',\overline{z}')$, we consider one going from $(z,\overline{z})$ and account for the difference in the two by adding in the above terms. 

This seems great, but we then we remember the remark made above: $v_{km}^{z_m}$ is not holomorphic, and so if we just add these terms to our transition function, we will break our required holomorphicity! This forces us to also transform the patch we are transforming to, labelled $n$. The argument follows exactly as above, but now we have to take away the additional term to go get to a Weyl-equivalent flat metric on patch $n$. So the total change to the (holomorphic part of the) transition function is 
\bse 
    \del_k f_{mn}^{z_m} = v_{km}^{z_{m}} - v_{kn}^{z_m}, \qand \del_k f_{mn}^{\overline{z}_m} = v_{km}^{\overline{z}_{m}} - v_{kn}^{\overline{z}_m},
\ese 
where we have used $z_m$ for both superscripts so that the subtraction makes sense (i.e. we're not using the $(z_n,\overline{z}_n)$ coordinates for the $v_{kn}$ term). We then just enforce the condition that this total subtraction term be holomorphic, modulo the terms where $v_{km}^{z_m}/v_{kn}^{z_m}$ are themselves holomorphic (and similarly for the antiholomorphic parts).

\subsection{Metrics}

We now just need to check that this is indeed the same thing as what we get from changing the metric itself. Well, as we've already said we don't need to worry about the Weyl factors, and so from \Cref{eqn:MetricModuliChange} we get 
\bse 
    \big(b,\hat{g}\big) = \int d^2 \sig \Big[  b^{\overline{z}_m\overline{z}_m}\p_{\overline{z}_m}v_{km}^{z_m} + b^{z_mz_m}\p_{z_m}v_{km}^{\overline{z}_m} \Big]. 
\ese 
We then integrate by parts, and use the fact that we are only considering patches and so we \textit{do} get a boundary term, to give 
\bse 
    \big(b,\hat{g}\big) = \frac{1}{2\pi i}\oint_{C_m} \Big( dz_m b_{z_mz_m} v_{km}^{z_m} - b_{\overline{z}_m\overline{z}_m} v_{km}^{\overline{z}_m}\Big),
\ese 
where we have used the metric to lower the indices on $b$, and where $C_m$ is the counterclockwise contour around patch $m$. 

Again we also have to do this for patch $n$, giving us an analogous result. Then, as both contours go round anticlockwise, in the overlap region they go against each other and so we must take their difference. So overall we get 
\bse 
    \begin{split}
        \big(b,\hat{g}\big) & = \frac{1}{2\pi i} \sum_{(mn)} \int_{C_{mn}} \Big[ dz_m b_{z_mz_m}\big(v_{km}^{z_m}-v_{kn}^{z_m}\big) - d\overline{z}_m b_{\overline{z}_m\overline{z}_m}\big(v_{km}^{\overline{z}_m}-v_{kn}^{\overline{z}_m}\big)\Big] \\
        & = \frac{1}{2\pi i} \sum_{(mn)} \int_{C_{mn}} \Big[ dz_m b_{z_mz_m} \del_kf_{mn}^{z_m} - d\overline{z}_m b_{\overline{z}_m\overline{z}_m}\del_kf_{mn}^{\overline{z}_m}\Big],
    \end{split}
\ese
where $C_{mn}$ is the counterclockwise boundary of the overlap region and where the sum is done over all overlapping regions in a symmetric way. 

\begin{figure}
    \begin{center}
        \btik 
            \draw[thick] (-4,-2.5) -- (-4,2.5) -- (4,2.5) -- (4,-2.5) -- (-4,-2.5);
            \draw[thick] (-1.5,0) circle [radius=2];
            \node at (-1.5,0) {\large{$m$}};
            \draw[thick] (1.5,0) circle [radius=2];
            \node at (1.5,0) {\large{$n$}};
            \draw[blue, decoration={markings, mark=at position 0.0 with {\arrow{>}}}, postaction={decorate}] (-1.5,0) circle (1.8cm);
            \node at (-1.5,1.5) {\large{\textcolor{blue}{$C_m$}}};
            \draw[red, decoration={markings, mark=at position 0.5 with {\arrow{>}}}, postaction={decorate}] (1.5,0) circle (1.8cm);
            \node at (1.5,1.5) {\large{\textcolor{red}{$C_m$}}};
        \etik 
    \end{center}
    \caption{A pictorial example of two overlapping patches and the contours. $C_{mn}$ is the contour inside the overlap region.}
\end{figure}

If we had three overlapping regions ($m$, $n$ and $p$, say), we would simply consider all three overlap contours and the point that they meet. With a bit of thought we can convince ourselves that this meeting point can be shifted around in the overlap region without affecting anything. 

\subsection{Vertex Insertions}

We can make a final note on the use of the above idea. We have replaced our integration over the different metrics with an integration over transition functions between patches, the question we want to ask is "can we use these patches to replace the integration over the vertex insertion positions?" The answer is "yes", and is quite simply done. We simply take our manifold with its patch structure and insert our operator wherever it's meant to go. We then introduce a new small patch centred around this vertex position and then encode this position by the new transition functions (i.e. the ones between this new patch and the original ones). As we move the insertion around, our patch also moves and so our transition functions change. 

This seems nice, but why would we do it? The answer is seen easiest by giving an example. Let's imagine the vertex is inserted in the original patch $m$ at a position $z_V$. We then introduce our new patch around $z_V$, i.e. $V$ is at $z'=0$. Clearly the coordinates in $m$ and the new patch are just given by 
\bse 
    z_m = z' + z_V.
\ese 
This looks exactly like our moduli change in coordiantes above, and so we just consider 
\bse 
    \bigg(\frac{\p z'}{\p z_V}\bigg)\bigg|_{z} = -1.
\ese 
Our ghost insertion therefore just gives a factor of $b$. 

Brilliant, apart from our scattering formula doesn't have an insertion of $b$ for the operators! Well, this is easily fixed using the identity 
\bse 
    \oint dz b(z) c(0) = 1.
\ese 
In other words we replace 
\bse 
    V(z) \to c(0) V(0),
\ese 
and then integrate over the different moduli for the transition functions. This is why we do this: now every vertex insertion in our scattering amplitude comes as a $cV$, instead of some being $cV$ and some being $V$. Equally we have put the vertex positions and the metric moduli on the same footing, so our scattering amplitude looks a lot more symmetric.  

\section{Coupling Constant}

\textcolor{red}{Dr. Minwalla talks briefly about the coupling constant for String theory, but to be honest I find his conversation a bit vague, so haven't included anything here yet. This is just a note to self to come back and add something if he doesn't discuss it in more detail later.}
\chapter{Starting The Superstring}

Everything seems quite nice with the Bosonic string, but there are two problems we need to address. The first we have seen many times, its them damn Tachyons, but the second is something we haven't yet mentioned; we don't seem to have any clear way\footnote{I say clear way, as Dr. Minwalla claims you can do it by considering D-branes in some fashion.} to introduce Fermions into our theory. As we will shall see, both of these problems can be resolved if we introduce a Fermionic field onto the worldsheet of the string.

\section{Free Fermion Theory}

The first thing we need to do, then, is study the theory of a free Fermion in $(1+1)$-dimensions. We will work in Lorentzian space and then analytically continue as before. Consider the Dirac action in 2d flat space, 
\bse 
    S_D = \frac{1}{2\pi}\int d^2\sig \, \overline{\psi} \slashed{\p} \psi.
\ese 
We shall work with real gamma matrices and set 
\bse 
    \gamma^0 = i\sig_2, \qand \gamma^1 = \sig_1,
\ese 
where $\sig_1/\sig_2$ are the Pauli matrices. Expanding the Dirac action out, we have
\bse 
    \int d^2\sig \, \psi^{\dagger}\gamma^0\big(\gamma^0\p_0 + \gamma^1\p_1\big)\psi = \int d^2\sig \psi^{\dagger}\big(-\b1 \p_0 + \sig_3\p_1\big) \psi.
\ese 
Then using 
\bse 
    \psi = \begin{pmatrix}
        \psi_+ \\
        \psi_-
    \end{pmatrix}, \qand \p_{\pm} := -\p_0 \mp \p_1,
\ese 
we have 
\bse 
    S_D = \frac{1}{2\pi}\int d^2\sig \, \big(\psi_+\p_-\psi_+ + \psi_-\p_+\psi_-\big),
\ese 
where we have used the fact that $\psi$ is real (as all the gammas are real). Analytically continuing gives us\footnote{Remembering $d^2\sig=\frac{1}{2}d^2z$}
\be 
\label{eqn:DiracAction}
    S_D = \frac{1}{4\pi}\int d^2z \big(\psi\overline{\p}\psi + \overline{\psi}\p\overline{\psi}\big),
\ee 
where we have used the $\p/\overline{\p}$ notation along with $\psi:=\psi_z$ and $\overline{\psi}:=\psi_{\overline{z}}$.

The equations of motion for \Cref{eqn:DiracAction} are simply 
\be 
\label{eqn:DiracEoM}
    \overline{\p}\psi = 0, \qand \p \overline{\psi} = 0.
\ee 

\br 
\label{rem:DiracDone}
    Note, if we had used a complex spinor, we would have obtained 
    \bse 
        S_D = \frac{1}{4\pi} \int d^2z \big(\psi^*\overline{\p}\psi + \overline{\psi^*}\p\overline{\psi}\big).
    \ese 
    Identifying $\psi/\psi^*$ with $c/b$ respectively, this action is of exactly the same form as our $bc$ ghost system. This starts what \Cref{rem:Dirac} was referring to.
\er 

We now proceed as we have done before to find the correlation function by considering the following\footnote{As normal we just consider the holomorphic part for now.}
\bse 
    0 = \frac{1}{Z}\int D\psi \frac{\del}{\del\psi(z')}\Bigg[\psi(z)\exp\bigg(-\frac{1}{2\pi}\int d^2\sig\psi\overline{\p}\psi\bigg)\Bigg] = \del(z-z') + \frac{1}{\pi} \big\la \psi(z')\overline{\p}\psi(z)\big\ra,
\ese
where the plus sign in the second term comes from two minus signs (the anticommutivitiy of $\psi$ and the minus in the exponential), and we get a factor of 2 using integration by parts. Note also that we have used $d^2\sig$, as this gives us our standard definition for the delta function with no factors of $2$ floating about. We have seen this relation before, and so we can conclude straight away that 
\be 
\label{eqn:PsiCorrelation}
    \big\la \psi(z_1)\psi(z_2)\big\ra = \frac{1}{z_{12}}.
\ee 

\br 
\label{rem:PsiComponentsOPE}
    We may think that something is wrong with the above result: the $\psi$s are meant to be Fermionic variables and so the result $(\psi(z))^2$ would imply that the right-hand side should be of order $\cO(z)$. The problem here is the fact that we've been dropping indices everywhere! The above relation should really read 
    \be 
    \label{eqn:PsiComponetsOPE}
        \big\la \psi^{\mu}(z_1)\psi^{\nu}(z_2)\big\ra = \frac{\eta^{\mu \nu}}{z_{12}}.
    \ee 
    This result is fine, because it is only the \textit{full} $\psi$ that needs to be antisymmetric, the components of $\psi$ need not antisymmetrise themselves. This is an important point and we shall return to it at the start of the Bosonisation lecture.
\er 

So we have the correlation function, the next thing to ask is "What is the stress tensor?"

\bcl 
    The stress tensor for the free (real) Fermionic is 
    \be 
    \label{eqn:FreeFermionicStressTensor}
        T := -\frac{1}{2}\cl \psi\p\psi\cl .
    \ee 
\ecl 

\bq 
    Let's compute the $T(z_1)T(z_2)$ OPE:\footnote{Dr. Minwalla uses the argument that $(\p\psi)^2=0$ in his proof. I don't see why this is the case, and have shown it without this condition. If anyone knows why that condition holds, I would appreciate an explanation.}
    \bse 
        \begin{split}
             T(z_1)T(z_2) & = \frac{1}{4} \cl \psi(z_1)\p_1\psi(z_1)\cl\cl\psi(z_2)\p_2\psi(z_2)\cl \\
             & = \frac{1}{4} \bigg( -\frac{1}{z_{12}^4} + \frac{2}{z_{12}^4} + \frac{2\cl\psi(z_1)\psi(z_2)\cl}{z_{12}^3} +  \frac{\cl\p_1\psi(z_1)\psi(z_2)\cl-\cl\psi(z_1)\p_2\psi(z_2)\cl}{z_{12}^2} \\
             & \qquad \qquad - \frac{\cl \p_1\psi(z_1)\p_2\psi(z_2)\cl}{z_{12}}\bigg) \\
             & =\frac{1}{4} \bigg( \frac{2}{z_{12}^4} + \frac{2\cl \p_2 \psi(z_2) \psi(z_2)\cl}{z_{12}^2} + \frac{\cl \p_2^2\psi(z_2)\psi(z_2)\cl}{z_{12}} + \frac{\cl \p_2\psi(z_2)\psi(z_2)\cl - \cl \psi(z_2)\p_2\psi(z_2)\cl}{z_{12}^2} \\
             & \qquad + \frac{\cl \p_2^2\psi(z_2)\psi(z_2)\cl -\cl \p_2\psi(z_2)\p_2\psi(z_2)\cl}{z_{12}} - \frac{\cl \p_2\psi(z_2)\p_2\psi(z_2)\cl}{z_{12}}\bigg) \\
             & = \frac{1}{4}\bigg( \frac{2}{z_{12}^4} - \frac{4\cl\psi(z_2)\p_2\psi(z_2)\cl}{z_{12}^2} -\frac{2\big(\cl\psi(z_2)\p_2^2\psi(z_2)\cl + \cl\p_2\psi(z_2)\p_2\psi(z_2)\cl\big)}{z_{12}}\bigg),
        \end{split}
    \ese 
    where we have expanded around $z_2$ on the third line and used the fact that $\psi(z_2)\psi(z_2)=0$. Finally plugging \Cref{eqn:FreeFermionicStressTensor} back in, we have 
    \bse 
        T(z_1)T(z_2) = \frac{1/2}{z_{12}^4} + \frac{2T(z_2)}{z_{12}^2} + \frac{\p_2 T(z_2)}{z_{12}},
    \ese
    which confirms that $T$ is a stress tensor. 
\eq 

\br 
\label{rem:DiracLambdaValue}
    Note from \Cref{rem:LambdaOPECoefficient} we have 
    \bse 
        \frac{1}{2} = -1 + 6\l - 6\l^2 \qquad \implies \l^2 -\l +\frac{1}{4} = 0,
    \ese 
    which tells us the free Fermion system can be further related to the $bc$ system with $\l=1/2$. This is the result mentioned in \Cref{rem:Dirac}.
\er 

\br 
    Dr. Minwalla gives a brief explanation at the end of the lecture why introducing Fermions on the worldsheet could give rise to Fermions in spacetime. I have chosen to leave this explanation out for now and instead focus more on showing it when it comes round to it. \textcolor{red}{Give section reference once done.}
\er 

\section{Brief Justification Of Supersymmetry}

As we will see next lecture, the way we proceed to is write down an action that contains both Bosonic variables and Fermionic ones. We will give the Fermionic variables a spacetime index (i.e. $\psi\to\psi^{\mu}$), and so have $d$ different Fermionic excitation directions. This is exactly what we did for the Bosonic string, however we argued (several times) that it is important that we only get $(d-2)$ physical excitation directions. We achieved this by using our gauge symmetries (i.e. by gauging by our conformal group). 

However, we are now all out of symmetries and so we appear to be a bit buggered for the Fermionic excitations. This is where supersymmetry comes in. We introduce supersymmetry as a method to allow for further gauge conditions so that we can reduce the Fermionic oscillations. That is, we will gauge by the so-called \textit{superconformal group}.

This all sounds very intimidating to the supersymmetry novice\footnote{As I am.}, however it really shouldn't be. In fact the above description is where supersymmetry came from! Despite this historical fact, superstring theory is often presented (as it will be here) assuming supersymmetry is a thing and using it. This is perhaps the best way to learn supersymmetry in a hands on manner, so do not be deterred. 
\chapter{Superspace \& Superfields}

\section{Superspace}

We start by introducing two new `coordinate' labels on our worldsheet, $\th$ and $\overline{\th}$. We use the inverted commas as they aren't really coordinates on the worldsheet (like the $z/\overline{z}$ are). All of our actions will appear as integrals of the form 
\bse
    \int d^2z d^2\th,
\ese 
where $\th$ are Grassman/Fermionic variables. We can therefore just integrate over the $\th/\overline{\th}$ and get back a purely Bosonic integral. A manifold that has both a ordinary complex coordinate ($z$) and one anticommuting coordinate ($\th$) is known as a \textit{supermanifold}.

We need to remember that Grassman integrals should really be thought of as a derivative, and so the integral will vanish unless a product of $\th$ and $\overline{\th}$ appears in the integrand. We need to adopt a sign convention for this integral and in these notes we shall use 
\be 
\label{eqn:ThetaIntegralConvention}
    d^2\th = d\overline{\th}d\th \qquad \implies \qquad \int d^2\th \th\overline{\th} = 1. 
\ee 

All of our fields will be variables of both $z/\overline{z}$ and $\th/\overline{\th}$, and we shall consider Taylor expansions of these fields. We will only get four kinds of terms: ones with no $\th$ or $\overline{\th}$, ones with only one $\th$, ones with only one $\overline{\th}$, and ones with one $\th$ and one $\overline{\th}$. Every other term vanishes from $\th^2=0=\overline{\th}^2$.


We define the \textit{superoperators}
\be 
\label{eqn:QSuperOperator}
    Q := \frac{\p}{\p\th} - \th \frac{\p}{\p z}, \qand \overline{Q} := \frac{\p}{\p\overline{\th}} - \overline{\th} \frac{\p}{\p \overline{z}}.
\ee
These operators mix the various Taylor expansion coefficients, and will also shift the fields. Let's just consider $Q$, and determine its square: 
\bse 
    \begin{split}
        Q^2 & = \frac{\p^2}{\p\th^2} + \th^2\frac{\p^2}{\p z^2} - \th\frac{\p^2}{\p z\p\th} - \frac{\p}{\p\th}\bigg(\th\frac{\p}{\p z}\bigg) \\
        & = - \th\frac{\p^2}{\p z\p\th} - \frac{\p}{\p z} + \th\frac{\p^2}{\p\th\p z} \\
        & = -\frac{\p}{\p z},
    \end{split}
\ese 
where we have used the fact that $\th^2=0$ to remove the first two terms (the $\p_{\th}^2$ term effectively vanishes as it would need to remove two $\th$ terms, but they never appear) and then used the antisymmetric nature of $\th$ to get the plus sign from the product rule. Similarly we get 
\bse 
    \overline{Q}^2 = -\frac{\p}{\p\overline{z}}.
\ese 
We can also easily see 
\bse 
    \big\{Q,\overline{Q}\big\} = 0.
\ese 

So we have two anticommuting operators that square to translations on the plane. We can therefore think of $Q/\overline{Q}$ as the square-root of translations on the plane. Using $[\p_z]=+1$, it follows that $[Q]=+1/2$. In other words $\theta$ has mass-dimension $[\theta]=-1/2$. 

\section{Superfields}

\bd[Superfield]
    We define a \textit{superfield} as a field whose variation under a superconformal transformation is given by $Q$ (or $\overline{Q}$ for antiholomorphic fields) acting on the field, i.e.\footnote{Both Dr. Minwalla and Polchinski use bold $\mathbf{X}$ for the superfields. I have decided to use $\Phi$ just to make the distinction between a superfield and a bosonic field clear.}
    \be 
    \label{eqn:Superfield}
        \del\Phi = Q\Phi.
    \ee 
\ed 

\bd[Supersymmetric] 
    We say a superfield is supersymmetric if 
    \bse 
        \int d^2z d^2\th \del \Phi = 0.
    \ese 
\ed 

\br 
    \textcolor{red}{I am not overly sure what the strict definition of a superfield is. From the videos it appears the above definition is what I need, but I'm sort of piecing things together. I'm going to try find out soon what it's proper definition is and change anything needed. This is just a note to self to do that.}
\er 

Let's now consider a general field in a Taylor expansion,
\be 
\label{eqn:GeneralSuperfield}
    \Phi^{\mu} = X^{\mu} + i\th\psi^{\mu} + i\overline{\th}\overline{\psi}^{\mu} + \th\overline{\th}F^{\mu},
\ee
where $X$ and $F$ are symmetric (Bosonic) variables, $\psi$ is a antisymmetric (Fermionic) one and $\mu$ is a spacetime index. Note from $[\Phi]=0$ and $[\theta]=-1/2$, it follows that $[X]=0$, $[F]=+1$, and $[\psi]=+1/2$ which is consistent with them being Bosonic and Fermionic respectively. As mentioned before, this is the most general expression we can have, as $\th^2=0=\overline{\th}^2$. We want to find the conditions under which this field is a superfield. Consider the action of $Q$ on $\Phi$ (dropping the index for notational convenience):
\be
\label{eqn:QOnGeneralSuperfield}
    Q\Phi = -\th\p_zX + i\psi - i\th\overline{\th}\p_z\overline{\psi} + \overline{\th}F,
\ee 
so, using 
\bse 
    \del\Phi = \del X + i\th\del\psi + i\overline{\th}\del\overline{\psi} + \th\overline{\th}\del F,
\ese 
if we identify 
\bse 
    \begin{split}
        \del X & = i\psi \\
        \del \psi & = i\p_zX \\
        \del \overline{\psi} & = -iF \\
        \del F & = i\p_z\overline{\psi}
    \end{split}
\ese 
then we get a  superfield, i.e. $Q\Phi=\del\Phi$.

\br 
    Note the above set of expressions show us explicitly that $Q$ rotates Bosons to Fermions and visa versa. 
\er 

\bl 
    All superfields are supersymmetric. 
\el 

\bq 
    From \Cref{eqn:GeneralSuperfield}, we see that 
    \bse 
        \int d^2z d^2\th \Phi = \int dz^2 F,
    \ese 
    as it is only the $F$ term that has both $\th$ and $\overline{\th}$. We have just shown that $\del F$ is a total derivative in $z$, and so its integral over $d^2z$ vanishes. 
\eq 

\bl 
    The product of two superfields is supersymmetric. 
\el 

\bq 
    Let
    \bse 
        \Phi_1 = X_1 + i\th\psi_1 + i\overline{\th}\overline{\psi}_2 + \th\overline{\th}F_1,
    \ese
    and similarly for $\Phi_2$. Leibniz tells us 
    \bse 
        Q(\Phi_1\Phi_2) = (Q\Phi_1)\Phi_2 + \Phi_1(Q\Phi_2).
    \ese 
    Then, using \Cref{eqn:GeneralSuperfield} and \Cref{eqn:QOnGeneralSuperfield}, we have 
    \bse 
        Q(\Phi_1\Phi_2) = \Big( -\th\p_zX_1 + i\psi_1 - i\th\overline{\th}\p_z\overline{\psi}_1 + \overline{\th}F_1 \Big)\Big( X_2 + i\th\psi_2 + i\overline{\th}\overline{\psi}_2 + \th\overline{\th}F_2 \Big) + (1\longleftrightarrow 2).
    \ese 
    As with the previous proof, we only need to consider the terms that have $\th\overline{\th}$, leaving us with 
    \bse 
        \int d^2z \Big(-i\big(\p_zX_1\big)\overline{\psi}_2 + i\psi_1F_2 - iX_1\big(\p_z\overline{\psi}_2\big) - iF_1\psi_2 + (1\longleftrightarrow 2)\Big),
    \ese 
    where for the minus sign on the $F_1\psi_2$ term comes from $\overline{\th}\th=-\th\overline{\th}$. Putting in the $(1\longleftrightarrow 2)$ terms in and cancelling, we are left with 
    \bse 
        -i \int d^2 z \Big( \big(\p_zX_1\big)\overline{\psi}_2 + X_1\big(\p_z\overline{\psi}_2\big) +  \big(\p_zX_2\big)\overline{\psi}_1 + X_2\big(\p_z\overline{\psi}_1\big) \Big) = -i \int d^2 z \p_z \big( X_1\overline{\psi}_2 + X_2\overline{\psi}_1\big),
    \ese
    which is the integral of a total derivative, so it vanishes. 
\eq 

What we want to look for are supersymmetric Lagrangians. It would obviously be very nice if these Lagrangians contained derivatives (so we can build kinetic terms), the question is `what kind of derivative?' Well it would be very nice if they anticommuted with $Q/\overline{Q}$, so we can easily move $Q$ through our derivative. The first obvious candidate is simply $\p_z/\p_{\overline{z}}$, however these won't do. Why? Well consider an action:
\bse 
    S = \int d^2z d^2\th \p \Phi \overline{\p} \Phi.
\ese
This needs to be unitless (as we're working in units $\hbar=c=1$). Being a scalar field, $[\Phi]=0$, so we don't need to worry about that. What about the measure? Well $[d^2z]=-2$, but Grassman measures, being like derivatives, have positive dimension, therefore using $[\th]=-1/2$ we have $[d^2\th]=+1$. We therefore require that $[D]=1/2$, but $[\p]=+1$, and so it won't do. 

\bp 
    The so-called \textit{superderivatives}
    \be 
    \label{eqn:Superderivative}
        D := \frac{\p}{\p\th} + \th\frac{\p}{\p z}, \qand \overline{D} := \frac{\p}{\p\overline{\th}} + \overline{\th}\frac{\p}{\p \overline{z}}
    \ee 
    will anticommute with $Q/\overline{Q}$ and have dimension $+1/2$.
\ep 

\bq 
    First let's check the dimension: $[\th]=-1/2$ and $[\p_z]=+1$ therefore $[\th\p_z]=+1/2$. Equally $[\p_{\th}]=+1/2$. Now we need to check they anticommute. Consider the holomorphic expressions:
    \bse 
        \begin{split}
            \big\{Q,D\big\} & = \big(\p_{\th} -\th\p_z\big)\big(\p_{\th} +\th\p_z\big) + \big(\p_{\th} + \th\p_z\big)\big(\p_{\th} - \th\p_z\big) \\
            & = \Big[\p_{\th}\big(\th\p_z) - \th\p_z\p_{\th}\Big] + \Big[ -\p_{\th}\big(\th\p_z) +\th\p_z\p_{\th}\Big] \\
            & = 0,
        \end{split}
    \ese 
    where again we have used $\th^2=0$ and the fact that $\p_{\th}^2$ wont contribute. The antiholomorphic expressions follow analogously. 
\eq 

Following the same calculation as for $Q/\overline{Q}$, we see that 
\bse 
    D^2 = \frac{\p}{\p z}, \qquad \overline{D}^2 = \frac{\p}{\p\overline{z}}, \qand \big\{D,\overline{D}\big\} = 0.
\ese 

\bl 
    The superderivative of a superfield is supersymmetric. 
\el 

\bq 
    Following the same idea as the previous proofs, and using $\{Q,D\}=0$, the only $\th\overline{\th}$ term that remains is $\p_zF$, which is a total derivative and so vanishes.
\eq 

\bp 
    The product $D\Phi^{\mu} \overline{D}\Phi_{\mu}$ is supersymmetric. 
\ep 

\bq 
    By direct calculation, we have (dropping the indices)
    \bse 
        \begin{split}
            D\Phi & = \th\p_z X + i\psi + i \th\overline{\th}\p_z\overline{\psi} + \overline{\th}F, \\
            \overline{D}\Phi & = \overline{\th}\p_{\overline{z}} X + i\overline{\psi} - i \th\overline{\th}\p_{\overline{z}}\psi - \th F.
        \end{split}
    \ese 
    Now, we only need to consider the terms in the product that only have a $\overline{\th}$ as the action of $Q$ will give them a $\th$ then we can do the $d^2\th$ integral. This leaves us with 
    \bse 
        \int d^2zd^2\th Q\big(D\Phi\overline{D}\Phi\big) = i\int d^2z \p_z\big(\psi\p_{\overline{z}}X - F\overline{\psi}\big),
    \ese 
    which vanishes as the right-hand side is a total derivative.
\eq 

We therefore have the supersymmetric action
\be 
\label{eqn:SupersymmetricAction}
    \frac{1}{4\pi}\int d^2zd^2\th D\Phi^{\mu} \overline{D}\Phi_{\mu} = \frac{1}{4\pi}\int dz^2 \Big( \p_z X^{\mu} \p_{\overline{z}} X_{\mu} + \psi^{\mu}\p_{\overline{z}} \psi_{\mu} + \overline{\psi}\p_{z}\overline{\psi} + F^{\mu}F_{\mu}\Big).
\ee 
The first term on the right-hand side is exactly our Bosonic action.\footnote{Really we need to associate $X^{\mu} \to \sqrt{2/a'} X^{\mu}$, but for simplicity (and to also follow the videos and Polchinski easier) we just set $\a'=2$.} The $F$ term is just some auxiliary field (it has no kinetic term and doesn't couple to anything else). It is just determined by the equation of motion (which we'll see in a second actually just sets it to zero). The middle two terms correspond exactly to the Dirac action, \Cref{eqn:DiracAction}. We can also recover all of the equations of motion from the equation of motion of \Cref{eqn:SupersymmetricAction}; 
\be
\label{eqn:SupersymmetricEoM}
    \overline{D}D\Phi^{\mu} = 0.
\ee
Expanding this out gives 
\bse 
    -\th\overline{\th}\p_{\overline{z}}\p X^{\mu} + i\overline{\th}\p_{\overline{z}}\psi^{\mu} - i\th\p_z\overline{\psi}^{\mu} + F^{\mu} = 0,
\ese 
which considering each coefficient separately gives our Bosonic and Fermionic equations of motion along with $F=0$. This final condition, along with the fact that it is auxiliary, means we can just forget about $F$.

\br 
    Note the above conditions on $F$ only apply because we only have the kinetic term in our action, i.e. $F$ doesn't couple to our fields. We can therefore think of $F$ as being related to potential terms.
\er 

\section{The Superfield OPE}

Recall the correlation functions for the Bosonic case could only be functions of the difference $z_{12}$. This comes from the fact that the correlation function must be annihilated by $\p_{z_1}+\p_{z_2}$. We now want to derive a similar result but for the correlation functions of our superfields. So we need to look for something that lies in the kernel of 
\bse 
    D_1 + D_2 := \frac{\p}{\p \th_1} + \th_1\frac{\p}{\p z_1} + \frac{\p}{\p \th_2} + \th_2\frac{\p}{\p z_2}.
\ese 
First consider a function that doesn't depend on $z_1$ or $z_2$. This just gives us an analogous story to the Bosonic case and so this function must depend only on $\th_{12}:=\th_1-\th_2$.

Now suppose the function does depend on $z_{12}$. From the antisymmetric nature of the $\th$s, it follows that we want the $\th$ dependence to be the product $\th_1\th_2$. So our function is of the form
\bse 
    f(z_1,\th_1,z_2,\th_2) = f(z_{12} -\th_1\th_2).
\ese 
For clarity, let's check explicitly, 
\bse 
    \big(D_1+D_2\big)\big(z_1-z_2-\th_1\th_2) = \th_1-\th_2+\th_2-\th_1 = 0,
\ese 
where the sign on the last term comes from anticommuting $\th_1$ with the $\th_2$ derivative.

Now we want our correlation function $\la\Phi(z_1,\th_1)\Phi(z_2,\th_2)\ra$ to reduce to the correlation function \Cref{eqn:ExpectationXX} when we set $\th_1=0=\th_2$. Given the above conditions for the form of the dependence of correlation functions, the obvious candidate is (recall footnote 2, i.e. we set $\a'=2$)
\bse 
    \big\la \Phi(z_1,\th_1)\Phi(z_2,\th_2)\big\ra = -\ln |z_{12}-\th_1\th_2|^2 = -\ln(z_{12}-\th_1\th_2) - \ln(\overline{z}_{12} - \overline{\th}_1\overline{\th}_2).
\ese 
So considering just the holomorphic part, we can take the Taylor expansion, again only keeping the leading order terms because $\th^2=0$, we have 
\bse 
    \big\la \Phi(z_1,\th_1)\Phi(z_2,\th_2)\big\ra = -\ln(z_{12}) + \frac{\th_1\th_2}{z_{12}}.
\ese 
Comparing this with\footnote{We don't need the holomorphic part for this part of the argument.}
\bse 
    \big\la \Phi(z_1,\th_1)\Phi(z_2,\th_2)\big\ra = \big\la \big(X(z_1) + i\th_2\psi(z_1)\big),\big(X(z_2) + i\th_2\psi(z_2)\big)\big\ra,
\ese 
we conclude 
\bse 
    \begin{split}
        \big\la X(z_1)X(z_2)\big\ra & = -\ln(z_{12}), \\
        \big\la \psi(z_1)\psi(z_2)\big\ra & = \frac{1}{z_{12}}
    \end{split}
\ese 
and the correlators between $X$ and $\psi$ vanishing. This is exactly what we wanted. Note we get a positive sign for the $\psi$ correlator as we have $i^2=-1$ but then $\th_2$ has to pass through $\psi_1$.

\section{The Stress Tensor}

The stress tensor played a very important role in the development of string theory, and we now want to look at a similar structure on our superspace.

Now, recall that in the Bosonic case, the stress tensor came from varying the action with respect to the metric, \Cref{eqn:StressEnergyTensor}, and gave us a spin-$2$ field. The Fermionic equivalent comes from studying supergravity and gives us a spin-$3/2$ field. We shall denote these $T_B$ and $T_F$, respectively. As we are working in superspace, we want to consider some construction of both $T_B$ and $T_F$. The obvious thing to try is 
\be
\label{eqn:SuperStressTensor}
    T := \frac{1}{2}T_F + \th T_B,
\ee 
where the $1/2$ factor is included for consistency with Polchinski. We similarly define the antiholomorphic $\overline{T}$.

\br 
\label{rem:TBTFConfusion}
    The above notation can be very confusing,\footnote{I am still trying to make sure this is correct, so take this with a pinch of salt. \textcolor{red}{Red to remind me to change this when I know.}} but what we need to pay attention to is where the different $T$s come from. By definition $T_B$ is the one and only stress tensor for our field, and despite the Bosonic `$B$' will actually contain $\psi$s (just consider varying \Cref{eqn:SupersymmetricAction}). $T_F$ is just some field introduced onto our space, and $T$ is some superspace structure made from both of them. We will refer to $T$ as the \textit{super stress tensor}.
\er 

The next question we ask is about the form of the expressions in the OPE of $T$ with itself. Being a spin-$3/2$ field, the most singular term with have $1/(z_{12}-\th_1\th_2)^3$. What we need to work out is what can appear on the numerator. The result has to be supersymmetric, so we can only really use $T$, $D_{\th}$, $\th_{12}$,  and $\p_z$.\footnote{Note we \textit{cannot} use $\p_{\th}$ as $Q$ does not commute with it, but $\p_z$ does.} The numerators need to have integer spin, so it follows that the general expression is 
\bse 
    T(z_1,\th_1)T(z_2,\th_2) \sim \frac{A}{(z_{12}-\th_1\th_2)^3} + \frac{B D_2T(z_2,\th_2)}{(z_{12}-\th_1\th_2)} + \frac{C\th_{12}T(z_2,\th_2)}{(z_{12}-\th_1\th_2)^2} + \frac{E\th_{12}\p_{z_2}T(z_2,\th_2)}{(z_{12}-\th_1\th_2)}.
\ese 

Next lecture we will expand this out and compare it to what we know, namely correlator for a stress tensor, $T_B$, to fix the factors $A,B,C$ and $E$. 
\chapter{Super Stress Tensor}

We ended the last lecture by stating that the general expression for the $TT$ OPE is
\be 
\label{eqn:SuperTTOPE}
    T(z_1,\th_1)T(z_2,\th_2) \sim \frac{A}{(z_{12}-\th_1\th_2)^3} + \frac{B D_2T(z_2,\th_2)}{(z_{12}-\th_1\th_2)} + \frac{C\th_{12}T(z_2,\th_2)}{(z_{12}-\th_1\th_2)^2} + \frac{E\th_{12}\p_{z_2}T(z_2,\th_2)}{(z_{12}-\th_1\th_2)}.
\ee 
By simply relabelling we also have 
\be 
\label{eqn:SuperTTOPEFlipped}
    T(z_2,\th_2)T(z_1,\th_1) \sim -\frac{A}{(z_{12}-\th_1\th_2)^3} - \frac{B D_1T(z_1,\th_1)}{(z_{12}-\th_1\th_2)} - \frac{C\th_{12}T(z_1,\th_1)}{(z_{12}-\th_1\th_2)^2} + \frac{E\th_{12}\p_{z_1}T(z_1,\th_1)}{(z_{12}-\th_1\th_2)},
\ee 
where $\th_2\th_1=-\th_1\th_2$ and $\th_{21}=-\th_{12}$ have been used. Combining these results with the fact that $T$ is antisymmetric, i.e.
\bse 
    T(z_1,\th_1)T(z_2,\th_2) + T(z_2,\th_2)T(z_1,\th_1) = 0,
\ese 
gives 
\bse 
    \frac{C\th_{12}\big(T(z_2,\th_2)-T(z_1,\th_1)\big)}{(z_{12}-\th_1\th_2)^2} + \frac{B\big(D_2T(z_2,\th_2)-D_1T(z_1,\th_1)\big)}{(z_{12}-\th_1\th_2)} +  \frac{E\th_{12}\big(\p_{z_2}T(z_2,\th_2)+\p_{z_1}T(z_1,\th_1)\big)}{(z_{12}-\th_1\th_2)} = 0.
\ese 

Now, we want to compare the above to something we know in order to try determine the coefficients. The obvious thing is the form of the OPE of general stress tensor with itself,\footnote{We use tildes here to avoid confusion between a normal stress tensor, $\widetilde{T}$, and our super stress tensor, $T$.}
\bse 
    \widetilde{T}(z_1)\widetilde{T}(z_2) \sim \frac{c/2}{(z_{12})^4} + \frac{2T(z_2)}{(z_{12})^2} + \frac{\p_{z_2}T(z_2)}{z_{12}}.
\ese 
We therefore want to just have $z_{12}$ in the denominator, and so we use the expansion\footnote{Again note all higher terms vanish as $\th^2=0$.}
\be
\label{eqn:SuperDenominatorExpansion}
    \frac{1}{(z_{12}-\th_1\th_2)^n} = \frac{1}{(z_{12})^n} + \frac{n\th_1\th_2}{(z_{12})^{n+1}}
\ee
along with \Cref{eqn:SuperStressTensor}. Using these, and expanding everything around $z_2$, we get
\bse 
    \begin{split}
        \frac{C\th_{12}\big(T(z_2,\th_2)-T(z_1,\th_1)\big)}{(z_{12}-\th_1\th_2)^2} & = \frac{C\th_{12}\Big[\frac{1}{2}T_F(z_2)+\th_2T_B(z_2) - \frac{1}{2}T_F(z_1) - \th_1T_B(z_1)\Big]}{(z_{12})^2} \\
        & = \frac{C\th_{12}\big[T_F(z_2)-T_F(z_1)\big]}{2(z_{12})^2} + \frac{C\th_1\th_2\big[T_B(z_2)-T_B(z_1)\big]}{(z_{12})^2} \\
        & = -\frac{C\th_{12}\p_{z_2}T_F(z_2)}{2z_{12}} - \frac{C\th_1\th_2\p_{z_2}T_B(z_2)}{z_{12}},
    \end{split}
\ese 
where care has been taken to get the signs correct for the $T_B$ term on the second line. Similarly we get 
\bse
    \begin{split}
        \frac{B\big(D_2T(z_2,\th_2)-D_1T(z_1,\th_1)\big)}{(z_{12}-\th_1\th_2)} & = -\frac{\th_{12}\p_{z_2}T_F(z_2)}{2z_{12}} - \frac{\th_1\th_2\p_{z_2}T_B(z_2)}{z_{12}}, \\
        \frac{E\th_{12}\big(\p_{z_2}T(z_2,\th_2)+\p_{z_1}T(z_1,\th_1)\big)}{(z_{12}-\th_1\th_2)} & = \frac{E\th_{12}\p_{z_2}T_F(z_2)}{z_{12}} + \frac{2E\th_1\th_2\p_{z_2}T_B(z_2)}{z_{12}}.
    \end{split}
\ese 
Comparing the $\th_{12}$ and the $\th_1\th_2$ coefficients separately both give 
\be 
\label{eqn:CBERelationSuperStress}
    C + B - 2E = 0.
\ee 

\section{OPEs}

\br
\label{rem:SuperTTOPEGeneral}
    For clarity from the outset, until we specify the exact form of our $T_B$ and $T_F$, the following OPEs are general results. That is, they are not only true for our action \Cref{eqn:SupersymmetricAction}, but hold for general supersymmetric systems.
\er 

\subsection{$T_BT_B$}

So we have a condition relating $C,B$ and $E$. What about $A$, and is there anything more we can say? The answer comes by recalling \Cref{rem:TBTFConfusion}; $T_B$ is the stress tensor for our system, and so has an OPE of the form
\bse 
    T_B(z_1)T_B(z_2) \sim \frac{c/2}{(z_{12})^4} + \frac{2T_B(z_2)}{(z_{12})^2} + \frac{\p_2T_B(z_2)}{z_{12}}.
\ese 
From \Cref{eqn:SuperStressTensor}, its the $\th_1\th_2$ term in the $TT$ OPE that needs to correspond to this. The easiest one to see is the 4-th power: simply use \Cref{eqn:SuperDenominatorExpansion} on the $A$ term 
\bse 
    \frac{A}{(z_{12}-\th_1\th_2)^3} = \frac{3\th_1\th_2A}{(z_{12})^4} + \frac{A}{(z_{12})^3},
\ese 
which tells us 
\be 
\label{eqn:AValueSuper}
    A = \frac{c}{6}
\ee 

Next, direct calculation gives
\bse 
    \begin{split}
        \frac{BD_2T(z_z,\th_2)}{(z_{12}-\th_1\th_2)} & \sim \frac{B\th_1\th_2T_B(z_2)}{(z_{12})^2}, \\
        \frac{C\th_{12}T(z_2,\th_2)}{(z_{12}-\th_1\th_2)^2} & \sim \frac{C\th_1\th_2T_B(z_2)}{(z_{12})^2}, \\
        \frac{E\th_{12}\p_{z_2}T(z_2,\th_2)}{(z_{12}-\th_1\th_2)} & \sim \frac{E\p_{z_2}T_B}{z_{12}},
    \end{split}
\ese 
which tell us 
\be
\label{eqn:EValueSuper}
    C + B = 2, \qand E = 1,
\ee 
which agrees with \Cref{eqn:CBERelationSuperStress}.

\subsection{$T_BT_F$}

Next let's try find the $T_BT_F$ OPE. $T_F$ was a field of weight $3/2$, and so, as $T_B$ is the stress tensor, we expect
\bse 
    T_B(z_1)T_F(z_2) \sim \frac{3/2 T_F(z_2)}{(z_{12})^2} + \frac{\p_{z_2}T_F(z_2)}{z_{12}}.
\ese 
The $T_BT_F$ OPE will come from the $\th_1$ term in the expansion of \Cref{eqn:SuperTTOPE}. Direct calculation gives 
\bse 
    \frac{C\th_{12}T(z_2,\th_2)}{(z_{12}-\th_1\th_2)^2} \sim \frac{\th_1CT_F(z_2)}{2(z_{12})^2}, \qand \frac{E\th_{12}\p_{z_2}T(z_2,\th_2)}{(z_{12}-\th_1\th_2)} \sim \frac{\th_1E\p_{z_2}T_F(z_2)}{2z_{12}},
\ese
which, being careful to remove the factor of $1/2$ in \Cref{eqn:SuperStressTensor}, gives us $E=1$ again and 
\be
\label{eqn:CBValuesSuper}
    C = 3/2 \qquad \implies \qquad B = 1/2.
\ee 

The $T_FT_B$ OPE just follows by antisymmetry; 
\bse 
    T_F(z_2)T_B(z_1) = - T_B(z_1)T_F(z_2).
\ese 
We can check this using \Cref{eqn:SuperTTOPEFlipped}. It's the $\th_1$ term we want, and simple calculation gives us 
\bse 
    T_F(z_2)T_B(z_1) \sim - \frac{CT_F(z_1)}{(z_{12})^2} -\frac{B\p_{z_1}T_F(z_1)}{z_{12}} + \frac{E\p_{z_1}T_F(z_1)}{z_{12}},
\ese 
which using the values of $C,B$ and $E$ and expanding around $z_2$ gives 
\bse 
    \begin{split}
        T_F(z_2)T_B(z_1) & \sim -\frac{3/2T_F(z_2)}{(z_{12})^2} - \frac{3/2\p_{z_2}T(z_2)}{z_{12}} - \frac{1/2\p_{z_2}T_F(z_2)}{z_{12}} + \frac{\p_{z_2}T_F(z_2)}{z_{12}} \\
        & = -\frac{3/2 T_F(z_2)}{(z_{12})^2} - \frac{\p_{z_2}T_F(z_2)}{z_{12}} \\
        & =- T_B(z_1)T_F(z_2).
    \end{split}
\ese 

\subsection{$T_FT_F$}

Finally let's find the $T_FT_F$ OPE. We have all the values of $A,B,C$ and $E$ and so we just get this by considering the term in \Cref{eqn:SuperTTOPE} that doesn't have any $\th$s. Being careful to include the two factors of $2$ from \Cref{eqn:SuperStressTensor}, direct calculation just gives us 
\be
\label{eqn:TFTFOPEGeneral}
    T_F(z_1)T_F(z_2) \sim \frac{4A}{(z_{12})^3} + \frac{4BT_B}{z_{12}} = \frac{2c}{3(z_{12})^3} + \frac{2T_B}{z_{12}}.
\ee 

\br 
    Note that the OPE between two $T_F$s, which are just fields on our superspace, gives rise to the stress tensor, $T_B$. This is something we have not seen so far in the course. 
\er 

\section{The Form of $T_B$ \& $T_F$}

We now want to give some explicit form for $T_B$ and $T_F$ for our action \Cref{eqn:SupersymmetricAction}. Of course we can obtain these by varying the action, but here we shall just claim the results and show the obey they OPEs above. 

\bcl 
    Reinserting the $\a'$, and leaving the relevant normal ordering colons implicit, we claim that 
    \be 
    \label{eqn:TBTFValues}
        \begin{split}
            T_B & = -\frac{1}{\a'} \p X^{\mu} \p X_{\mu} - \frac{1}{2} \psi^{\mu} \p \psi_{\mu}, \\
            T_F & = i \sqrt{\frac{2}{\a'}}\psi^{\mu}\p X_{\mu}
        \end{split}
    \ee 
    satisfy the above OPEs.
\ecl 

\bq 
    The $T_BT_B$ term is clear: we just have the Bosonic and Fermionic stress tensors added in a decoupled way, so the OPE terms just add, giving us a central charge of $c=1+1/2=3/2$. 
    
    We can also easily see the $T_BT_F$ OPE: the $\p X \p X$ part of $T_B$ just sees the weight $1$ primary field $\p X$ in $T_F$, and the $\psi\p\psi$ part of $T_B$ just sees the weight $1/2$ primary field $\psi$ in $T_F$. The product of these two gives rise to a weight $3/2$ primary field, which agrees with our OPE above.
    
    So we just need to check the $T_FT_F$ OPE. Using 
    \bse 
        \psi(z_1)\psi(z_2) \sim \frac{1}{z_{12}}, \qand \p X(z_1) \p X(z_2) \sim -\frac{\a'}{2(z_{12})^2},
    \ese
    we have 
    \bse 
        \begin{split}
            T_F(z_1)T_F(z_2) & \sim -\frac{2}{\a'}\Bigg[ \bigg(-\frac{\a'}{2}\bigg)\frac{1}{(z_{12})^2}\frac{1}{z_{12}} + \bigg(-\frac{\a'}{2}\bigg)
            \frac{\psi(z_1)\psi(z_2)}{z_{12}^2} + \frac{\p X(z_1)\p X(z_2)}{z_{12}}\Bigg] \\
            & = \frac{1}{(z_{12})^3} - \frac{\psi(z_2)\p\psi(z_2)}{z_{12}} - \frac{2}{\a'} \frac{\p X(z_2)\p X(z_2)}{z_{12}} \\
            & = \frac{1}{(z_{12})^3} - \frac{2T_B}{z_{12}},
        \end{split}
    \ese 
    where on the second line we have used $(\p\psi)\psi = -\psi\p\psi$. This agrees with our $T_FT_F$ OPE if we pick $c=3/2$, which we required from the $T_BT_B$ OPE. 
\eq 

\br 
    \textcolor{red}{Dr. Minwalla then tries to show that $T=1/2T_F + \th T_B$ is a superfield. He does this by considering $D\Phi\p\Phi$. If I follow through the calculation myself don't quite get the right answer. I get
    \bse 
        \frac{1}{2}D\Phi\p\Phi = \frac{1}{2} i\psi\p X + \frac{i\th}{2} \big(\p X\p X - i\psi\p\psi).
    \ese 
    The first term is $1/2T_F$ but the second term has the $i$ at the front, the sign on the $\p X\p X$ term is wrong and we also have an $i$ on the $\psi\p\psi$. I've probably just done something silly, and I will try fix this, but if anyone reading can show the result, I would massively appreciate the working. (credit will be given obviously)}
\er 

\section{The Ghosts}

\subsection{The Action}

So we have managed to supersymmetrise the matter part of our string theory, but that's not the only part; we need to deal with the ghosts!

Recall that $[b]=2$ and $[c]=-1$. These results came from the fact that $[b]=[g]=2$ and that $c$ was a vector field. We want some supersymmetric extension of these. We have just seen above that the extension of the metric is a dimension $3/2$ field, so we want some $\beta$ such that $[\beta]=3/2$. Similarly, vector fields will become spinor fields, and so we want some $\g$ such that $[\g]=-1/2$. 

\br 
    We can, in fact, express the above results for our 1-parameter family of CFTs, that is using $\l$. We had $[b]=\l$ and $[c]=1-\l$, so it follows that we want $[\beta]=\l-1/2$ and $[\g]=3/2-\l$. 
\er 

We now want to define superfields that are some combination of these objects, in the same way $T$ is made up of $T_F$ and $T_B$. By simple dimensional arguments, we arrive at 
\be 
    B := \beta +\th b, \qand C = c +\th\g.
\ee 
We want to construct an action that reduces to our ghost action for $\th=0$. We can find the form using dimensional arguments: we have $[B]=\l-1/2$, $[C]=1-\l$, and we clearly want one of each and a derivative (i.e. we want something of the same form as \Cref{eqn:Sghostbc}). Following the same argument as for the mass part, we can't use $\overline{\p}$ and must instead use $\overline{D}$.\footnote{Note its the barred $D$, as we want to get $b\overline{\p}c$.} We therefore arrive at the ghost-type action
\be 
\label{eqn:SupersymmetricGhostAction}
    S = \frac{1}{4\pi} \int d^2zd^2\th \Big(B\overline{D}C + \overline{B}D\overline{C}\Big),
\ee 
where the antiholomorphic terms have been included.

Let's check that this does indeed give us our ghost action, \Cref{eqn:Sghostbc}. Considering just the holomorphic part:
\bse 
    \begin{split}
        \frac{1}{4\pi} \int d^2zd^2\th \, B\overline{D}C & = \frac{1}{4\pi} \int d^2zd^2\th \Big( \big(\beta + \th b\big) \big(\p_{\overline{\th}} +\overline{\th}\p_{\overline{z}}\big) \big(c+\th\g\big) \Big) \\
        & = \frac{1}{4\pi}\int d^2zd^2\th \Big(\big(\beta+\th b\big)\big(\overline{\th}\p_{\overline{z}}c + \overline{\th}\th\p_{\overline{z}}\g\big)\Big) \\
        & = \frac{1}{4\pi}\int d^2z \, b\p_{\overline{z}}c - \frac{1}{4\pi}\int d^2z \, \beta\p_{\overline{z}}\g,
    \end{split}
\ese
where we have used the convention \Cref{eqn:ThetaIntegralConvention} to get the correct signs. The first term is exactly what we wanted. Of course the antiholomorphic part follows exactly the same. 

Now note the two terms on the right-hand side of the above expression \textit{seem} to be exactly the same, but there is one important difference; $b$ and $c$ anticommute but $\beta$ and $\g$ commute! This actually has a subtle impact on our proposed action: we just wrote $B\overline{D}C$ and powered on, but what if we'd chosen $C\overline{D}B$? Well the result would just have come out to be 
\bse 
    \frac{1}{4\pi} \int d^2zd^2\th \, C\overline{D}B = \frac{1}{4\pi}\int d^2z \, c\p_{\overline{z}}b - \frac{1}{4\pi}\int d^2z \, \g\p_{\overline{z}}\beta. 
\ese 
Now, using integration by parts and the anticommutator, we can just switch $b$ and $c$ in the first term. However $\g$ and $\beta$ commute, and so we \textit{cannot} remove the minus sign picked up from the integration by parts!

\subsection{The OPE}

Now let's find the OPE of $B$ and $C$. Well $[BC]=1/2$, and so the only thing we can have is 
\bse 
    B(z_1)C(z_2) \sim \frac{A\th_{12}}{z_{12}-\th_1\th_2} = \frac{A\th_{12}}{z_{12}},
\ese
for some constant $A$,\footnote{This is obviously not the same $A$ as the one used in $TT$ OPE.} where we used \Cref{eqn:SuperDenominatorExpansion}. We don't have any exact exchange statistics for $B$ and $C$ and so we must consider the $CB$ OPE separately. The dimensional argument still holds and so we also have 
\bse 
    C(z_1)B(z_2) \sim \frac{E\th_{12}}{z_{12}},
\ese 
for some constant $E$. Now let's expand these out and compare it to our $bc$ OPEs to find $A$ and $E$, and then use these to find other OPEs. We have
\bse 
    \begin{split}
        B(z_1)C(z_2) & = \beta(z_1) c(z_2) + \th_2\beta(z_1)\g(z_2) + \th_1b(z_1)c(z_2) + \th_1\th_2b(z_1)\g(z_2), \\
        C(z_1)B(z_2) & = c(z_1)\beta(z_2) - \th_2c(z_1)b(z_2) + \th_1\g(z_1)\beta(z_2) + \th_1\th_2\g(z_1) b(z_2).
    \end{split}
\ese 
Comparing coefficients tells us 
\bse 
    \begin{split}
        b(z_1)c(z_2) & \sim \frac{A}{z_{12}}, \qquad \qquad  c(z_1)b(z_2)  \sim \frac{E}{z_{12}}, \\
        \beta(z_1)\g(z_2) & \sim -\frac{A}{z_{12}}, \qquad \qquad \g(z_1)\beta(z_2) \sim \frac{E}{z_{12}},
    \end{split}
\ese 
and all other OPEs vanishing. The known $bc$ OPE, \Cref{eqn:bcOPE}, then tell us $A=1=E$, and so we arrive at 
\be 
\label{eqn:BetaGammaOPE}
    \beta(z_1)\g(z_2) \sim -\frac{1}{z_{12}}, \qand \g(z_1)\beta(z_2) \sim \frac{1}{z_{12}}. 
\ee 
These results agree with the symmetric nature of $\beta$ and $\g$, i.e. 
\bse 
    \beta(z_2)\g(z_1) = -\frac{1}{z_{21}} = +\frac{1}{z_{12}} = \g(z_1)\beta(z_2).
\ese

Note we could have arrived at \Cref{eqn:BetaGammaOPE} in the usual way (i.e. take a derivative of the action with an insertion). The relative minus sign between $\beta\g$ and $\g\beta$ then stems from the integration by parts to go from $\beta\overline{\p}\g$ to $\g\overline{\p}\beta$. We therefore see a perhaps more important impact of us just starting with $B\overline{D}C$ in the action and powering on. If we had instead chosen to use $C\overline{D}B$ we would have arrived at 
\bse 
    \g(z_1)\beta(z_2) \sim -\frac{1}{z_{12}}, \qand \beta(z_1)\g(z_2) \sim \frac{1}{z_{12}},
\ese 
which is the opposite to \Cref{eqn:BetaGammaOPE}. So it is important to pay attention to which convention we use in the action. 
\chapter{Ghost Super Stress Tensor}

So far we have taken our free Boson theory and formed the supersymmetric extension, with $T_B^M$ and $T_F^M$\footnote{The superscript is for `matter'.} given by \Cref{eqn:TBTFValues}. We have also written the action for the ghost system, and used it to find the OPEs between $c, b, \beta$ and $\g$, but we are yet to find the super stress tensor terms, i.e. we need to find $T_B^G$ and $T_F^G$. That is the aim of this lecture. 

\section{Ghost Super Stress Tensor}

We have already seen that $[T^M]=+3/2$, and so it follows that we need to construct $T^G=\frac{1}{2}T_F^G+\th T_B^G$ such that it has dimension $3/2$. Recalling what we did for the mass system, we are going to want to match the expression for the $T^GT^G$ OPE with the known one for the $bc$ stress tensor. For generality, we shall do this for our one-parameter family of super CFTs, that is we will leave the value of $\l$ arbitrary and compare to \Cref{eqn:GhostStressTensorLambda}.

So what do we do? The answer is, we essentially list all the possible supersymmetric weight $3/2$ terms that contain $1$ $B$ and $1$ $C$ (as otherwise it will never match \Cref{eqn:GhostStressTensorLambda}) and then expand out to get the form of $T_B^G$ and $T_F^G$. Recalling $[BC]=1/2$, $[D]=1/2$ and $[\p_z]=1$, we guess
\bse 
    T^G = pDCDB + q \big(\p_z C\big) B + r C\p_zB,
\ese 
for some, to be determined, constants $p,q$ and $r$. Of course we define $\overline{T}^G$ in a similar way. We now expand this out and look for terms with/without a $\th$ prefactor in order to identify $T_F^G/T_B^G$. Direct expansion gives us (using our notation $\p:=\p_z$)
\begin{equation}
\label{eqn:TBTFGhostUndetermined}
    \begin{split}
        T_B^G & = p\big[-b\p c + \g\p\beta\big] + q\big[b\p c + \beta\p\g\big] + r\big[-c\p b + \g\p\beta\big] \\
        & = (q-p)b\p c -rc\p b + (p+r)\g\p\beta + q\beta\p\g, \\
        \frac{1}{2}T_F^G & = p\g b + q\beta\p c + rc\p\beta,
    \end{split}
\end{equation} 
where the antisymmetry nature of $cb$ has been used carefully to get the signs right. Comparing the middle line with \Cref{eqn:GhostStressTensorLambda}, we can identify 
\be 
\label{eqn:rpqvalues}
    r = (1-\l), \qand (p-q) = \l. 
\ee 
We can use this to rewrite our expression for $T^G$ in terms of $\l$ and either $p$ or $q$. We shall use $q$, giving us 
\bse 
    T^G = (\l+q)DCDB + q\big(\p C\big)B + (1-\l)C\p B.
\ese 

This is good, but we still need to find the value of $q$ and a relation for the central charge. We can do both of these things by considering the $T_F^GT_F^G$ OPE,\footnote{Dr. Minwalla considers the full $T^GT^G$ OPE, this will of course work, but I found it takes a considerable more effort.} and comparing it to \Cref{eqn:TFTFOPEGeneral}. From \Cref{eqn:TBTFGhostUndetermined,eqn:rpqvalues} we have (using subscripts to indicate the $z$ dependence)
\begin{equation*}
    \begin{split}
        \frac{1}{4}T_F^G(z_1)T_F^G(z_2) & = q(\l+q) \Big[ \g_1b_1\beta_1\p_2c_2 + \beta_1\p_1c_1\g_2b_2\Big] + (\l+q)(1-\l)\Big[\g_1\beta_1c_2\p_2\beta_2 + c_1\p_1\beta_1\g_2b_2\Big] \\
        & = \frac{2(\l+q)(q+1-\l)}{(z_{12})^3} + q(\l+q)\bigg(\frac{\g_1\beta_2-\beta_1\gamma_2}{(z_{12})^2} + \frac{b_1\p_2c_2 - \p_1c_1b_2}{z_{12}} \bigg) \\
        & \hspace{4.5cm} + (\l+q)(1-\l) \bigg( \frac{b_1c_2 + c_1b_2}{(z_{12})^2} + \frac{\g_1\p_2\beta_2+\p_1\beta_1\g_2}{z_{12}}\bigg) \\
        & = \frac{2(\l+q)(q+1-\l)}{(z_{12})^3} + \frac{1}{z_{12}}\Big[ q(\l+q)\beta_2\p_2\g_2 + (\l+q)(2-2\l-q)\g_2\p_2\beta_2 \\
        & \hspace{4cm} + (\l+q)(\l-1+2q)b_2\p_2c_2 + (\l+q)(\l-1)c_2\p_2b_2\Big],
    \end{split}
\end{equation*} 
where we have obviously used the OPEs between $b,c,\g$ and $\beta$ to get to the second line, and then expanded around $z_2$ and cancelled on the third line. Comparing the $1/(z_{12})^3$ term to \Cref{eqn:TFTFOPEGeneral}, we see 
\bse 
    8(\l+q)(q+1-\l) = \frac{2c}{3} \qquad \implies \qquad c = 12\big(q^2 - \l^2 + q + \l\big).
\ese 
Then comparing the $\beta_2\p_2\g_2$ terms, we have 
\be
\label{eqn:qvalue}
    4q(\l+q) = 2q, \qquad \implies \qquad q = -\l + \frac{1}{2}.
\ee 
So putting this into our expression for the central charge gives us 
\be 
\label{eqn:GhostSuperCentralChargeLambda}
    c = 9-12\l. 
\ee 
Finally putting \Cref{eqn:qvalue} into \Cref{eqn:rpqvalues} and then these into \Cref{eqn:TBTFGhostUndetermined}, we get\footnote{\textcolor{red}{Polchinski gets a slightly different answer to me for $T_F^G$. I have checked mine several times and can't see where I'm going wrong. This is a note to say check again later. If any readers can get his result from what I have, please let me know!}}
\be 
    \begin{split}
        T_B^G & = (\p b)c - \l\p(bc) + \g\p\beta + \frac{1}{2}\big(1-2\l\big)\p(\g\beta), \\
        T_F^G & = \g b + (1-2\l)\beta\p c + 2(1-\l)c\p\beta,
    \end{split}
\ee 
where we have grouped the written $T_B^G$ in a similar form to the first line of \Cref{eqn:GhostStressTensorLambda}.

\section{Critical Dimension of Superstring}

We now want to consider our specific case of $\l=2$ and ask what the dimension of the whole system is? Putting $\l=2$ into \Cref{eqn:GhostSuperCentralChargeLambda} give us 
\be 
\label{eqn:GhostSuperCentralCharge}
    c_{\text{ghost}} = -15.
\ee 
Now, keeping the comments made in \Cref{sec:d26?} in mind, i.e. we are assuming that we are removing the entire ghost central charge with matter states, and the fact that we have $c=+3/2$ for our supersymmetric matter tells us that our critical dimension is 
\mybox{
\be 
\label{eqn:d10}
    d=10.
\ee 
}

\subsection{Light-Cone Procedure}

Recall that at the start of the course we arrived at the critical dimension for the free Bosonic string quite intuitively using light-cone quantisation. We now want to show that this method follows through for the superstring and will give us $d=10$, in agreement with \Cref{eqn:d10}.

First let's quickly review what we did. We arrived at a formula for the mass of the string excitations
\bse 
    m^2 = \frac{4}{\a'}\bigg( N + (d-2)\sum_{n=1}^{\infty} \frac{n}{2L}\bigg),
\ese 
where $N$ is the number operator, $n/2$ was the zero point energy, and $L$ was the embedding length of the string. We then said this sum was divergent and so we needed to regulate the momentum using some function $f(p/\Lambda)$. We did this using the Euler–Maclaurin formula and showed 
\bse 
    \sum_{n=1}^{\infty}\frac{n}{L}f\bigg(\frac{p}{\Lambda}\bigg) = AL\Lambda^2 - \frac{1}{12L},
\ese 
and argued that the first term on the right-hand side could be removed with a redefinition of the cosmological constant. The last term, however, could not be removed and so was physical. This gave us, after setting $L=1$, \Cref{eqn:LevelMatchingNormalOrder} which reads
\bse 
    m^2 = \frac{4}{\a'}\bigg(N-\frac{d-2}{24}\bigg).
\ese 
We then used Wigner's classification to show that, in order to maintain Poincar\'{e} invariance. This lead us to conclude that the first excited state, $N=1$, must be massless, and so $d=26$. 

We now want to redo this process (albeit a lot quicker now that we know what we're doing) for the superstring. So what has changed? Obviously we now have Fermions in the theory, and they contribute to the zero point energy. We need to be careful though, as we are putting these Fermions onto the string worldsheet and so we need to consider boundary conditions. For the Boson this was not a problem as we always take them to be periodic, but Fermions can be either periodic or antiperiodic! 

\bd[Ramond \& Neveu-Schwarz Sectors]
    Let $\sig\sim \sig+2\pi$ be our equivalence relation on the cylinder. Then the Ramond (R) sector is defined by 
    \be 
    \label{eqn:Ramond}
        \psi^{\mu}(\sig+2\pi) = \psi^{\mu}(\sig),
    \ee 
    and the Neveu-Schwarz (NS) sector is defined by
    \be 
    \label{eqn:NeveuSchwarz}
        \psi^{\mu}(\sig+2\pi) = -\psi^{\mu}(\sig).
    \ee 
    That is the R sector contains the periodic Fermions and the NS sector contains the antiperiodic Fermions. 
\ed 

\subsubsection{Neveu-Schwarz Sector}

First lets consider the NS sector. These have negative energy that comes in half integers. So our mass-squared formula becomes\footnote{Note we can take $n=0$ in the sum as the Bosonic term vanishes for $n=0$.} 
\bse 
    m^2 = \frac{4}{\a'} \Bigg[ N + \frac{d-2}{2}\sum_{n=0}^{\infty}\bigg( \frac{n}{L} - \frac{2n+1}{2L}\bigg)\Bigg], 
\ese 
where now $N = N_B + \frac{1}{2}N_F$, where $N_B$ and $N_F$ are the Boson and Fermion number operators respectively.\footnote{Note the half in front of $N_F$.} Of course we regularise the $n/L$ term in exactly the same way as just recapped, giving us a $-1/12$ contribution. What about the other term in the sum? Well we need the complete version of the Euler–Maclaurin formula, which tells us
\bse 
    \frac{f(0)}{2} + f(1) + f(2) + ... = \int_0^{\infty} f(x) dx - \frac{1}{12} f'(0) + ...,
\ese 
where the $f(0)/2$ term didn't appear for the Bosonic case as $f(0)=0$ there. We therefore define 
\bse 
    f(x) := xg\bigg(\frac{x}{\Lambda}\bigg),
\ese
where $g(x/\Lambda)$ is the regulator, and split the regulated Fermionic sum\footnote{We'll leave the overall minus sign out until the end.} into
\bse 
    \begin{split}
        \sum_{n=0}^{\infty} f\bigg(\frac{2n+1}{2L}\bigg) & = \frac{f(0)}{2} + \frac{f(0)}{2} + f(1) + f(2) + ... \\
        & = \frac{1}{4L} + \int_0^{\infty} \bigg(\frac{2n+1}{2L}\bigg)g\bigg(\frac{2n+1}{2L\Lambda}\bigg) dn - \frac{1}{12L},
    \end{split}
\ese 
where we have used the fact that on the derivative term if we act on $g$ we will pull out a factor of $1/\Lambda$ so in the limit $\Lambda\to\infty$ this term is negligible. The same reason has been used to ignore all higher derivatives. 

We now need to deal with the integral term. First we do the change of variable 
\bse 
    y = \frac{2n+1}{2},
\ese
giving us 
\bse 
    \int_0^{\infty} \bigg(\frac{2n+1}{2L}\bigg)g\bigg(\frac{2n+1}{2L\Lambda}\bigg) dn = \int_{1/2}^{\infty} \frac{y}{L} g\bigg(\frac{y}{L\Lambda}\bigg) dy,
\ese 
which looks very similar to the Bosonic integral we had at the start of the course, with one important difference; the lower limit is $1/2$ not $0$. We then replace the right-hand side with 
\bse 
    \int_{1/2}^{\infty} \frac{y}{L} g\bigg(\frac{y}{L\Lambda}\bigg) dy = \int_0^{\infty} \frac{y}{L} g\bigg(\frac{y}{L\Lambda}\bigg) dy - \int_0^{1/2} \frac{y}{L} g\bigg(\frac{y}{L\Lambda}\bigg) dy.
\ese 
Then we make the in the final term on the right-hand side $y/\Lambda\to0$ in both integration limits and so $g(y/L\Lambda)\to1$. Doing that integral, and then using the argument that the first term on the right-hand side is the same as in the Bosonic case, then gives us 
\bse 
    \int_0^{\infty} \bigg(\frac{2n+1}{2L}\bigg)g\bigg(\frac{2n+1}{2L\Lambda}\bigg) dn = AL^2\Lambda -\frac{1}{8L}.
\ese 
Putting this all together (and dropping the $AL^2\Lambda$ term by the same argument as for the Bosonic case) we arrive at 
\bse 
    \sum_{n=0}^{\infty} f\bigg(\frac{2n+1}{2L}\bigg) = \frac{1}{4L} - \frac{1}{12L} - \frac{1}{8L} = \frac{1}{24L}.
\ese 
So our mass-squared condition becomes
\be 
\label{eqn:MassSquaredNeveuSchwarz}
    m^2 = \frac{4}{\a'}\Bigg[ N + \frac{d-2}{2}\bigg( -\frac{1}{12} - \frac{1}{24}\bigg)\Bigg] = \frac{4}{\a'} \bigg( N - \frac{d-2}{16}\bigg).
\ee 
Finally we impose the condition that the first excited state must be massless. The first excited state has $N=1/2$, which gives us $d=10$, our desired result! It is also true that the higher excited states do indeed form representations of SO$(d-1)$, as required for them to not be massless.

\subsubsection{Ramond Sector}

Now let's look at the Ramond sector. These still have negative energy in supersymmetry, however now they have the same periodicity as the Bosons and so come in integer multiples as well. Therefore our mass condition just becomes 
\be 
\label{eqn:MassSquaredRamond}
    m^2 = \frac{4}{\a'} \bigg(N + \frac{d-2}{2L}\sum_{n=0}^{\infty}\big(n-n\big)\bigg) = \frac{4}{\a'}N,
\ee 
and so now it is $N=0$ that corresponds to the massless state. 

\section{Neveu-Schwarz vs. Ramond}

We have just seen that we can have two different periodicities for our Fermions on the worldsheet. Now recalling that our action contains the product $\psi\overline{\p}\psi$, we are left with four possibilities for our theory: we could have NS-NS, NS-R, R-NS or R-R. The obvious question is `which one do we use?' We will answer this question in much more detail over the next few lectures, but for now we give a brief discussion of what we might be thinking. 

\br 
    These different options are sometimes labelled as a doublet $(v,\widetilde{v})$ where $v=1/2$ for the NS sector and $v=0$ for the R sector. We will not do this here, but this comment is just for relation to other sources.
\er 

\subsection{The Tachyon Returns?}

The R sector seems to have one physical advantage to the NS sector: all of the states are massive and so there is no Tachyon. By contrast, subbing $N=0$ into \Cref{eqn:MassSquaredNeveuSchwarz} gives 
\be 
    m^2_{\text{Tachyon}} = -\frac{2}{\a'},
\ee 
which is half the result we get for the Bosonic string, but is still negative! 

We may therefore think that we need to restrict ourselves to only considering the R sector, however as we will see, when we consider the general theory (with both periodic and antiperiodic Fermions) and make some condition, we will project out this Tachyon in the NS sector. It will also turn out that this process will make our \textit{spacetime} supersymmetric. This will be rather surprising\footnote{Or perhaps not, now that you know!} as at no point did we supersymmetrise anything to do with the spacetime.

To clarify, what we have done is introduce objects onto the worldsheet and quantised them. This has resulted in Bosons and Fermions existing on the spacetime, but under no particular relation. It will turn out that the conditions we imply later will also have an impact on how these Bosons and Fermions appear on our spacetime, with the result being that they must appear in a supersymmetric fashion.

\br
    \textcolor{red}{I think this is the idea. I haven't got there yet, so this is just a note to remind myself to come and change anything that needs changing later on. }
\er 

\subsection{Periodicity On The Plane}

We might also think the R sector seems nicer to work with because it is periodic. However, we have to remember that almost all of our calculations are done on the plane, and, as we shall now show, it turns out that the R sector is antiperiodic on the plane and the NS sector is periodic on the plane. 

This above claim comes simply from the transformation property along with the fact that $\psi$ is a weight $(1/2,0)$ primary. Recall that the plane is related to the cylinder via radial quantisation:
\bse 
    z = \exp^{-i\omega}, \qquad \omega = \sig + i\tau.
\ese 
Now, being a weight $(1/2,0)$ primary $\psi(z)$ is related to $\psi(\omega)$ via 
\bse 
    \psi(z) = \bigg(\frac{\p \omega}{\p z}\bigg)^{1/2} \psi(\omega) = z^{-1/2} \psi(\omega).
\ese 
Therefore the Laurent expansions therefore take the form 
\be 
\label{eqn:NSRPlaneLaurentExpansion}
    \psi(z) = \sum_{r\in \Z + v} \frac{\psi_r}{z^{r+1/2}},
\ee 
where $v=0$ for the R sector and $v=1/2$ for the NS sector. We therefore see the periodicity condition flips when we transform to the plane: NS is periodic while R is antiperiodic.  
\chapter{Recap}

This lecture basically consists of Dr. Minwalla recapping a lot of the results proved earlier in the course. In order to save space (and time on my behalf), I will not rewrite the results here, but simply request that the reader goes back and reads any points they're stuck on or simply  \href{https://www.youtube.com/watch?v=FPomJFEY3B8&list=PL3PVFGnaPl_sCp2A87NVD8GT5Z8Oqw3Yr&index=22}{watch the video}. 
\chapter{Bosonisation}

So far what we have done is to introduce both Bosons and Fermions into our theory, and have seen that they take very similar forms but with a few differences. Perhaps the most notable is the OPEs,
\bse
    X^\mu(z,\overline{z})X^\nu(0) \sim -\frac{\a'}{2}\eta^{\mu\nu} \ln |z|^2, \qquad \psi^{\mu}(z)\psi^{\nu}(0) \sim \frac{\eta^{\mu\nu}}{z},
\ese
where, remembering \Cref{rem:PsiComponentsOPE}, we have been careful to include the indices.

On top of this, $X$ and $\psi$ don't have the same central charge: $c_X=1$ but $c_{\psi^{\mu}}=1/2$. The aim of this lecture is going to be to show that if we consider \textit{complex} Fermionic fields 
\be 
\label{eqn:ComplexFermions}
    \psi = \frac{1}{\sqrt{2}}\big(\psi^1 + i \psi^2\big), \qquad \psi^* = \frac{1}{\sqrt{2}}\big(\psi^1 - i \psi^2\big),
\ee 
we can make a proper connection between the two theories; indeed we will see they are actually essentially \textit{the same} theory. That is we will show that the actions 
\bse 
    S_X = \frac{1}{4\pi\a'} \int d^2z (\p X)^2, \qand S_{\psi} = \frac{1}{2\pi} \int d^2z \psi^*\p \psi
\ese
behave identically.

\br 
    \textbf{Notation warning}: In these notes I have used $\psi^*$ to indicate the complex conjugate. Polchinski uses a bar instead. The reason I am using a star here is because I have been using $\overline{\psi}$ to indicate the antiholomorphic fields. Polchinski uses a tilde for this purpose. I am (at least currently) too lazy to go back and change all my bars to tildes, so instead use the star. I am just pointing this out for two reasons: 1) to make comparisons to Polchinski easier, and 2) it is possible I will copy a formula from Polchinski with the bar notation by accident, so please be careful with what it really means. 
\er 

\br 
    Note the inspiration for the form of \Cref{eqn:ComplexFermions} could come from considering the central charge, as $c_{\psi} = c_{\psi^1}+c_{\psi^2} = 1 = c_X$.
\er 

\section{Abstract Boson Theory}

The above relation tells us essentially what we are trying to do is show that instead of considering $\psi$ as some complex Fermionic field, we can think of it as a Bosonic field. We will label an abstract Bosonic field by $H$, and normalise it such that (considering just the holomorphic part)
\be 
\label{eqn:HHOPE}
    H(z)H(0) \sim - \ln z.
\ee 
This is just equivalent to working with $\a'=2$ above. 

Now consider the operators $e^{\pm iH(z)}$. Firstly note from \Cref{eqn:SpinWeights,eqn:ScalingDimension,eqn:WeighteikX}, along with $\a'=2$, both these operators have scaling dimension and spin dimension $+1/2$. The antiholomorphic equivalents $e^{\pm i\overline{H}(\overline{z})}$ has scaling dimension $+1/2$ and spin dimension $-1/2$. This looks very nice, considering we want to relate our Fermionic fields to them! 

We can work out the OPEs easily as 
\bse
    \begin{split}
        e^{iH(z)}e^{-iH(0)} & = e^{i(-i)H(z)H(0)} \tcl e^{iH(z)-iH(0)}\tcl \\
        & = e^{-\ln z} \tcl e^{iH(z)-iH(0)}\tcl \\
        & = \frac{1}{z} + ...,
    \end{split}
\ese 
where to get to the last line we have taken a Taylor expansion of the normal ordered part, and used the fact that every term is non-singular. Similar calculations follow through for the other OPEs and we get
\be 
\label{eqn:ExpHHOPE}
    \begin{split}
        e^{iH(z)}e^{-iH(0)} & \sim \frac{1}{z} \\
        e^{iH(z)}e^{iH(0)} & \sim \cO(z) \\
        e^{-iH(z)}e^{-iH(0)} & \sim \cO(z)
    \end{split}
\ee 
We can summarise this above result as 
\be 
\label{eqn:ExpHHOPEEpsilon}
    e^{i\epsilon_iH(z_i)}e^{i\epsilon_jH(z_j)} \sim (z_{ij})^{\epsilon_i\epsilon_j},
\ee 
where $\epsilon_i \in \{-1,1\}$. 

Now let's consider the product of $2m$ such terms, with $m$ positive power and $m$ negative power terms. It follows from the scaling dimension argument above, that this product is of the form\footnote{This formula is a bit wrong, the $z$ in the denominator on the right-hand side should be of the form $z_{ij}$, but we just want to indicate the form of the expression, so abuse the notation.} 
\bse 
    \prod_{i=1}^{2m}e^{i\epsilon_iH(z_i)} = \frac{1}{z^m}\tcl e^{i(...)} \tcl, \qquad \sum_{i=1}^{2m}\epsilon_i = 0,
\ese 
where the $(...)$ in the normal ordered exponential contains all the $H$ terms. Using \Cref{eqn:ExpHHOPEEpsilon} it follows that we can write this result as 
\bse 
    \prod_{i=1}^{2m} e^{i\epsilon_iH(z_i)} = \prod_{i<j} (z_{ij})^{\epsilon_i\epsilon_j} \tcl e^{i(...)}\tcl.
\ese 
Finally, using the fact that the vacuum expectation value of $e^{iA}$ is one, we get 
\be 
\label{eqn:ExpHVacuumExpectationValue}
    \bigg\la \prod_{i=1}^{2m} e^{i\epsilon_iH(z_i)} \bigg\ra = \prod_{i<j} (z_{ij})^{\epsilon_i\epsilon_j}.
\ee 
This result can also be obtained just using \Cref{eqn:ExpHHOPEEpsilon}, and the fact that the correlation function must vanish when the operators are far away from each other.\footnote{This is just used to set the coefficient to a constant, instead of being some polynomial in $z$ which would blow up at large $z$.} This fact will be useful in just a moment. 

\section{Complex Fermion Theory}

Now let's look more closely at our complex Fermion theory, \Cref{eqn:ComplexFermions}. Using \Cref{eqn:PsiComponetsOPE}, we see that 
\be 
    \begin{split}
        \psi(z)\psi^*(0) & \sim \frac{1}{z} \\
        \psi(z)\psi(0) & \sim \cO(z) \\
        \psi^*(z)\psi^*(0) & \sim \cO(z),
    \end{split}
\ee 
where the $\cO(z)$ comes from the fact that the order $0$ term vanishes, via antisymmetry of $\psi/\psi^*$. We see, therefore, that the complex Fermions have \textit{exactly} the same OPE structure as the exponentiated abstract Bosons, \Cref{eqn:ExpHHOPE}.

So we have shown that their OPEs match, but this isn't enough to make the two theories equivalent as CFTs. The next step is to check for equivalence between local operators.

Note that local operators with integer $k_R$ and $k_L$ can be formed by repeated operator products of the $e^{\pm iH(z)}$ and $e^{\pm i\overline{H}(\overline{z})}$ operators. Equally note that any operator in Fermion theory built out of the Fermions and their derivatives can be built out of repeated products of the $\psi, \psi^*, \overline{\psi},$ and $\overline{\psi}^*$ at different $z$ values. The derivative terms come from taking Taylor expansions (see below for the stress tensor, for example). Finally, the fact that \Cref{eqn:ExpHVacuumExpectationValue} can be derived solely from the OPE structures means it must also hold for the Fermion theory. This tells us that the Bosonic operators have the same expectation value as their Fermionic counterparts, i.e. replacing $e^{iH(z)}$ with $\psi(z)$, etc. will give the same expectation value. We therefore have an equivalence of local operators. 

The last ingredient we need to check are the stress tensors. Consider the product 
\bse 
    \begin{split}
        e^{iH(z)}e^{-iH(-z)} & = \frac{1}{2z}\tcl e^{iH(z) - iH(-z)} \tcl \\
        & = \frac{1}{2z}\Big[ 1 + 2iz\p H(0) + \frac{1}{2}(2i)^2 z^2  \big(\p H(0)\big)^2 + \cO(z^3) \Big] \\
        & = \frac{1}{2z} + i\p H(0) - z \big(\p H(0)\big)^2 + \cO(z^2) \\
        & = \frac{1}{2z} + i\p H(0) + 2zT_B^H(0) + \cO(z^2)
    \end{split}
\ese  
where we have Taylor expanded around $z=0$, and used \Cref{eqn:StressTensorFreeBoson} (with $\a'=2$) to identify the stress tensor. 

Now let's consider the same thing but for the complex Fermions: 
\bse 
    \begin{split}
        \psi(z)\psi^*(-z) & = \frac{1}{2z}\tcl \psi(z)\psi^*(-z)\tcl \\
        & = \frac{1}{2z}\Big[1 + 2z\psi(0)\psi^*(0) + \frac{1}{2}z^2 \big(\p\psi(0)\psi^*(0) - \psi(0)\p\psi^*(0) \big) + \cO(z^3)\Big] \\
        & = \frac{1}{2z} + \psi(0)\psi^*(0) + 2zT_B^{\psi}(0) + \cO(z^2),
    \end{split}
\ese 
where we have used 
\bse 
    \begin{split}
        \p\psi\psi^* - \psi\p\psi^* & = \frac{1}{2} \Big[ \big(\p\psi^1 + i\p\psi^2\big)\big(\psi^1-i\psi^2\big) - \big(\psi^1+i\psi^2\big)\big(\p\psi^1-i\p\psi^2\big) \Big] \\
        & = -\psi^{\mu}\p\psi_{\mu}\\
        & = 2T_B^{\psi}
    \end{split}
\ese
where we have used the anticommutivity property $\p\psi^{\mu}\psi_{\mu} = - \psi^{\mu}\p\psi_{\mu}$.

So we see if we identify $\psi\cong e^{iH}$ and $\psi^*\cong e^{-iH}$, we also identify 
\be
\label{eqn:PsiHEquivalences}
    \psi\psi^* \cong i\p H, \qand T_B^{\psi} \cong T_B^H.
\ee 

\br 
    Note the relation $\psi\psi^*\cong i\p H$ makes sense as $\psi\psi^*$ is the Noether current of the complex Fermion system and $i\p H$ corresponds to the shift symmetry of the Bosonic system. 
\er 

\subsection{NS or R?}

One thing we haven't asked about the above procedure is `what sector do our Fermionic states belong to?' In other words, are the operators we talk about above dual to states in the NS sector or states in the R sector? The answer is the NS sector. We shall demonstrate this for a couple states. 

The first thing we have to note is that the NS sector does not contain zero-modes, i.e. $\psi_0$ does not appear on the right-hand side of \Cref{eqn:NSRPlaneLaurentExpansion}. Therefore we just define the ground state to be the one annihilated by all $\psi_r$ with $r>0$, 
\bse 
    \psi_r^*\ket{0}_{NS} = \psi_r\ket{0}_{NS} = 0, \qquad \forall r \in \{ \Z^+ + 1/2\}.
\ese 
We then define the creation operators to be $\psi_r^{\mu}$ with $r<0$. So in order to show that the operators we are talking about above live in the NS sector we need to show they are dual to states of the form $\psi_{-1/2}\ket{0}_NS$, etc. 

First let's consider the simple state $\ket{\psi}$. We use \Cref{eqn:NSRPlaneLaurentExpansion} with $v=1/2$. Inverting this gives 
\bse 
    \psi_r = \oint \frac{dz}{2\pi i} z^{r-1/2} \psi(z), \qand \psi_r^* = \oint \frac{dz}{2\pi i} z^{r-1/2} \psi^*(z).
\ese 
Consider the action of $\psi_r^*$ first: 
\bse 
    \begin{split}
        \psi_r^* \ket{\psi} & = \oint \frac{dz}{2\pi i} z^{r-1/2} \psi^*(z) \psi(0) \\
        & = \oint \frac{dz}{2\pi i} z^{r-1/2} \bigg[ \frac{1}{z} + ... \bigg] \\
        & = \begin{cases}
            0 & \forall r > 1/2 \\
            1 & r = 1/2 \\
            \neq 0 & \forall r \leq -1/2.
        \end{cases}
    \end{split}
\ese
So we see there the annihilation operator $\psi_{-1/2}^*$ doesn't annihilate the state. This suggests 
\bse 
    \ket{\psi} = \psi_{1/2}\ket{0}_{NS}.
\ese 
We can further check this by considering the action of $\psi_r$: 
\bse 
    \begin{split}
        \psi_r\ket{\psi} & = \oint \frac{dz}{2\pi i} z^{r-1/2} \psi(z)\psi(0) \\
        & = \oint \frac{dz}{2\pi i} z^{r-1/2}\big[ \cO(z) + ... \big], 
    \end{split}
\ese 
so $\psi_{-1/2}\ket{\psi} = 0$, which confirms the above result (as $(\psi_r)^2=0$). An analogous calculation will give $\ket{\psi^*} = \psi_{-1/2}^*\ket{0}_{NS}$.

For further peace of mind, let's consider state with a derivative, namely $\ket{\p\psi}$. Acting with $\psi_r^*$ we have 
\bse 
    \begin{split}
        \psi_r^*\ket{\p\psi} & = \oint \frac{dz}{2\pi i} z^{r-1/2} \psi^*(z)\p\psi(0) \\
        & = \oint \frac{dz}{2\pi i} z^{r-1/2} \bigg[ -\frac{1}{z^2} + ... \bigg] \\
        & = \begin{cases}
            0 & \forall r > 3/2 \\
            -1 & r = 3/2 \\
            0 & r = 1/2 \\
            \neq 0 & r \leq -1/2,
        \end{cases}
    \end{split}
\ese 
where the minus sign comes from the fact that the derivative acts on the second argument (i.e. consider $z_1$ and $z_2$ with $\p_2$). Note also the result for $r=1/2$, which comes from the fact that we have nothing on the numerator to expand. This suggests that 
\bse 
    \ket{\p\psi} = \psi_{-3/2}\ket{0}_{NS}.
\ese 
We can't check this result using $\psi_r$ as we need to know the actual form of the $\psi(z)\psi(0)$ OPE to at least second order. However, this was just to show that the states are related to the NS sector, so we wont bother doing all that. 

\br 
    This idea, that all the operators we can think of in the Fermionic theory appear to be in the NS sector, actually highlights the importance of Bosonisation for the R sector. That is, if we want to scatter something in the R sector, we need to operator, but we've just shown that if we consider the Fermions they are all NS, so instead we consider the Bosonised theory. The obvious question is `what form do these Boson operators take?' The answer is they are of the form $e^{\pm i kH(z)}$ where now $k \in \R\setminus\Z$, i.e. real, non-integer numbers.
\er 

\subsection{Ground States}

In nod to the remark made above, let's have a look at the ground states of for a general theory of general periodicity. That is let our fields on the cylinder obey 
\bse 
    \psi(\sig + 2\pi) = e^{2\pi i v}\psi(\sig), \qand \psi^*(\sig + 2\pi) = e^{-2\pi i v}\psi^*(\sig),
\ese 
where we now let $v\in[0,1)$. Again we take the Laurent expansion on the plane to give 
\bse 
    \psi = \sum_{r\in\Z+v} \frac{\psi_r}{z^{m+1/2}}, \qand \psi^* = \sum_{r\in\Z-v} \frac{\psi_r^*}{z^{m+1/2}}.
\ese 
Note that we take $\Z-v$ for the complex-conjugated field. Now we define the vacuum state by 
\bse 
    \psi_r\ket{0} = \psi_r^*\ket{0} = 0 \qquad \forall r>0.
\ese 
If we set $r=n\pm v$, for $n\in\Z$, then the above condition becomes 
\bse 
    \begin{split}
        \psi_{n+v}\ket{0} & = 0 \qquad \forall n\geq 0 \\
        \psi^*_{n-v}\ket{0} & = 0 \qquad \forall n\geq 1.
    \end{split}
\ese 
Similarly we would want the creation expressions to hold, these become
\bse 
    \begin{split}
        \psi_{n+v}\ket{0} & \neq 0 \qquad \forall n \leq -1 \\
        \psi_{n-v}^*\ket{0} & \neq 0 \qquad \forall n \leq 0.
    \end{split}
\ese 

So we want to find the operator dual to $\ket{0}$, which we shall denote $\cO$. To do this just consider the action of $\psi_{n+v}$ and $\psi_{n-v}^*$ on the state:
\bse 
    \begin{split}
        \psi_{n+v}\ket{0} & = \oint \frac{dz}{2\pi i} z^{n+v-1/2} \psi(z)\cO(0), \\
        \psi_{n-v}^*\ket{0} & = \oint \frac{dz}{2\pi i} z^{n-v-1/2} \psi^*(z)\cO(0).
    \end{split}
\ese 
Now use our Boson identification $\psi(z) \cong e^{iH(z)}$ and $\psi^*(z) \cong e^{-iH(z)}$ in the above formulas. If we then made $\cO$ some exponential of $H$ we could use the OPE to get factors or $1/z$. With a little thought, the answer is to choose 
\be 
\label{eqn:FermionBosonisationGroundStateOPerator}
    \cO(z) = e^{-i(v-1/2)H(z)}.
\ee 
Let's just check this works: 
\bse 
    \begin{split}
        \psi_{n+v}\ket{0} & = \oint \frac{dz}{2\pi i} z^{n+v-1/2} e^{iH(z)}e^{-i(v-1/2)H(0)} \\
        & = \oint \frac{dz}{2\pi i} z^{n+v-1/2} z^{-(v-1/2)} \cl e^{iH(z) - i(v-1/2)H(0)} \cl \\
        & = \oint \frac{dz}{2\pi i} z^n \cl ... \cl 
    \end{split}
\ese 
which vanishes for $n\geq0$ and does not vanish for $n\leq -1$. Similarly we get 
\bse 
    \psi_{n-v}^*\ket{0} = \oint \frac{dz}{2\pi i } z^{n-1} \cl ... \cl,
\ese
which vanishes for $n\geq1$ and doesn't for $n\leq0$. These are exactly the conditions we want, and so \Cref{eqn:FermionBosonisationGroundStateOPerator} is the operator dual to the ground state. 

Note this result tells us that the ground state for the NS sector (which has $v=1/2$) is the identity operator and the ground state for the R sector (which has $v=0$) is 
\be 
\label{eqn:OperatorGroundStateRSector}
    \cO_R = e^{iH/2}.
\ee 

Now the energy of an operator of the form \Cref{eqn:FermionBosonisationGroundStateOPerator} is the scaling dimension (plus some shift). The scaling dimension is
\bse 
    \Delta = \frac{(v-1/2)^2}{2}.
\ese 
This tells us that the vacuum of the NS sector has zero energy (as the identity does). However it is more interesting for the R sector as  
\bse 
    \cO_R = e^{iH/2}, \qand \psi^*\cO_R = e^{-iH/2}
\ese 
have the same energy! So our ground state energy becomes degenerate. This does not happen for any other value of $v$, and is in fact a consequence of the zero-modes of the R sector. 

\br 
    \textcolor{red}{I need to understand the R sector ground states a little better and expand on the point above. This is just a note to self to do that.}
\er 


% -------------------------------------------------------------------
% Bibliography/Further Readings
% -------------------------------------------------------------------

\chapter*{Useful Texts \& Further Readings}

\section*{String Theory}
\begin{itemize}
    \item Polchinski, J., \textit{String theory: Volumes 1\& 2, an introduction to the bosonic string}, Cambridge university press, (1998).
    \item Green, M. B., Schwarz, J. H., \& Witten, E., \textit{Superstring theory: volume 2, loop amplitudes, anomalies and phenomenology}, Cambridge university press, (2012).
    \item Zwiebach, B., \textit{A first course in string theory}, Cambridge university press, (2004).
    \item Kiritsis, E., \textit{String theory in a nutshell} (Vol. 9), Princeton University Press, (2011).
    \item Ried-Edwards, R. A., \textit{String Theory Lectures}, taught at Cambridge (2019), available via \href{http://www.damtp.cam.ac.uk/people/rar31/LectureNotes.pdf}{http://www.damtp.cam.ac.uk/people/rar31/LectureNotes.pdf}.
\end{itemize}

\section*{Conformal Field Theory}
\begin{itemize}
    \item Blumenhagen, R., \& Plauschinn, E., \textit{Introduction to conformal field theory: with applications to string theory} (Vol. 779), Springer Science \& Business Media, (2009).
    \item Francesco, P., Mathieu, P., \& Sénéchal, D., \textit{Conformal field theory}, Springer Science \& Business Media, (2012).
    \item Qualls, J. D., \textit{Lectures On Confromal Field Theory}, (2016), available via \\
    \href{https://arxiv.org/pdf/1511.04074.pdf}{https://arxiv.org/pdf/1511.04074.pdf}.
\end{itemize}

\section*{Quantum Field Theory}
\begin{itemize}
    \item  Zee A., \textit{Quantum field theory in a nutshell}, Second Edition, Princeton university press, 2010.
    \item Peskin M. E. \& Schroeder D. V., \textit{An introduction to quantum field theory}, CRC Press
\end{itemize}

\section*{Miscellaneous}
\begin{itemize}
    \item Goddard, P., \& Olive, D., \textit{Kac-Moody and Virasoro algebras in relation to quantum physics}, International Journal of Modern Physics A, 1(02), 303-414, (1986).
\end{itemize}


%\bibliographystyle{agsm} 
%\bibliography{mybibliography} 
%\printbibliography[heading=bibintoc]


% -------------------------------------------------------------------
% Appendices
% -------------------------------------------------------------------

%\begin{appendices}
%\input{sections/appendixA.tex}
%\end{appendices}

\end{document}
