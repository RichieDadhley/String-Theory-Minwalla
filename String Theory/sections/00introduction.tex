\chapter{Introduction}

I shall update the introduction once I have completed more of the course and have a better understanding of what to write here... For now I shall leave you with a nice quote from Dr. David Tong's String Theory course:

\begin{center}
    \textit{
        "Our current understanding of physics, embodied in the standard model, is valid up to energy scales of 103 GeV. This is 15 orders of magnitude away from the Planck scale. Why do we think the time is now ripe to tackle quantum gravity? Surely we are like the ancient Greeks arguing about atomism. Why on earth do we believe that we've developed the right tools to even address the question?\\
        The honest answer, I think, is hubris."
    }
\end{center}

% Lecture 10 at 1:38:00 ish for comment on why he teaches worldsheet string theory

\textcolor{red}{Disclaimer: It turns out Polchinski uses the convention $d^2\sig = \frac{1}{2}d^2z$. I have missed the fraction in some places. This obviously doesn't effect any of the physics but is worth noting. In fact in general the factors of $2$ and $\pi$ are likely to be wrong here and there. I shall try fix this at some point, but it will involve rereading the entire set of notes with a keen eye so might take some time. If anyone notices any, I would appreciate them being pointed out.}

% Maybe put a comment about not stressing when stuff is on the worldsheet enough. The worldsheet is not physical so just because something is on there, doesn't mean its on the spacetime (target space). When we quantise worldsheet we get spacetime objects. 