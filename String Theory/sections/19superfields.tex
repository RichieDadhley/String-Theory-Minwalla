\chapter{Superspace \& Superfields}

\section{Superspace}

We start by introducing two new `coordinate' labels on our worldsheet, $\th$ and $\overline{\th}$. We use the inverted commas as they aren't really coordinates on the worldsheet (like the $z/\overline{z}$ are). All of our actions will appear as integrals of the form 
\bse
    \int d^2z d^2\th,
\ese 
where $\th$ are Grassman/Fermionic variables. We can therefore just integrate over the $\th/\overline{\th}$ and get back a purely Bosonic integral. A manifold that has both a ordinary complex coordinate ($z$) and one anticommuting coordinate ($\th$) is known as a \textit{supermanifold}.

We need to remember that Grassman integrals should really be thought of as a derivative, and so the integral will vanish unless a product of $\th$ and $\overline{\th}$ appears in the integrand. We need to adopt a sign convention for this integral and in these notes we shall use 
\be 
\label{eqn:ThetaIntegralConvention}
    d^2\th = d\overline{\th}d\th \qquad \implies \qquad \int d^2\th \th\overline{\th} = 1. 
\ee 

All of our fields will be variables of both $z/\overline{z}$ and $\th/\overline{\th}$, and we shall consider Taylor expansions of these fields. We will only get four kinds of terms: ones with no $\th$ or $\overline{\th}$, ones with only one $\th$, ones with only one $\overline{\th}$, and ones with one $\th$ and one $\overline{\th}$. Every other term vanishes from $\th^2=0=\overline{\th}^2$.


We define the \textit{superoperators}
\be 
\label{eqn:QSuperOperator}
    Q := \frac{\p}{\p\th} - \th \frac{\p}{\p z}, \qand \overline{Q} := \frac{\p}{\p\overline{\th}} - \overline{\th} \frac{\p}{\p \overline{z}}.
\ee
These operators mix the various Taylor expansion coefficients, and will also shift the fields. Let's just consider $Q$, and determine its square: 
\bse 
    \begin{split}
        Q^2 & = \frac{\p^2}{\p\th^2} + \th^2\frac{\p^2}{\p z^2} - \th\frac{\p^2}{\p z\p\th} - \frac{\p}{\p\th}\bigg(\th\frac{\p}{\p z}\bigg) \\
        & = - \th\frac{\p^2}{\p z\p\th} - \frac{\p}{\p z} + \th\frac{\p^2}{\p\th\p z} \\
        & = -\frac{\p}{\p z},
    \end{split}
\ese 
where we have used the fact that $\th^2=0$ to remove the first two terms (the $\p_{\th}^2$ term effectively vanishes as it would need to remove two $\th$ terms, but they never appear) and then used the antisymmetric nature of $\th$ to get the plus sign from the product rule. Similarly we get 
\bse 
    \overline{Q}^2 = -\frac{\p}{\p\overline{z}}.
\ese 
We can also easily see 
\bse 
    \big\{Q,\overline{Q}\big\} = 0.
\ese 

So we have two anticommuting operators that square to translations on the plane. We can therefore think of $Q/\overline{Q}$ as the square-root of translations on the plane. 

\section{Superfields}

\bd[Superfield]
    We define a \textit{superfield} as a field whose variation under a superconformal transformation is given by $Q$ (or $\overline{Q}$ for antiholomorphic fields) acting on the field, i.e.\footnote{Both Dr. Minwalla and Polchinski use $\mathbf{X}$ for the superfields. I have decided to use $\Phi$ just to make the distinction between a superfield and a bosonic field clear.}
    \be 
    \label{eqn:Superfield}
        \del\Phi = Q\phi.
    \ee 
\ed 

\bd[Supersymmetric] 
    We say a superfield is supersymmetric if 
    \bse 
        \int d^2z d^2\th \del \Phi = 0.
    \ese 
\ed 

\br 
    \textcolor{red}{I am not overly sure what the strict definition of a superfield is. From the videos it appears the above definition is what I need, but I'm sort of piecing things together. I'm going to try find out soon what it's proper definition is and change anything needed. This is just a note to self to do that.}
\er 

Let's now consider a general field in a Taylor expansion,
\be 
\label{eqn:GeneralSuperfield}
    \Phi^{\mu} = X^{\mu} + i\th\psi^{\mu} + i\overline{\th}\overline{\psi}^{\mu} + \th\overline{\th}F^{\mu},
\ee
where $X$ and $F$ are symmetric (Bosonic) variables, $\psi$ is a antisymmetric (Fermionic) one and $\mu$ is a spacetime index. As mentioned before, this is the most general expression we can have, as $\th^2=0=\overline{\th}^2$. We want to find the conditions under which this field is a superfield. Consider the action of $Q$ on $\Phi$ (dropping the index for notational convenience):
\be
\label{eqn:QOnGeneralSuperfield}
    Q\Phi = -\th\p_zX + i\psi - i\th\overline{\th}\p_z\overline{\psi} + \overline{\th}F,
\ee 
so, using 
\bse 
    \del\Phi = \del X + i\th\del\psi + i\overline{\th}\del\overline{\psi} + \th\overline{\th}\del F,
\ese 
if we identify 
\bse 
    \begin{split}
        \del X & = i\psi \\
        \del \psi & = i\p_zX \\
        \del \overline{\psi} & = -iF \\
        \del F & = i\p_z\overline{\psi}
    \end{split}
\ese 
then we get a  superfield, i.e. $Q\Phi=\del\Phi$.

\br 
    Note the above set of expressions show us explicitly that $Q$ rotates Bosons to Fermions and visa versa. 
\er 

\bl 
    All superfields are supersymmetric. 
\el 

\bq 
    From \Cref{eqn:GeneralSuperfield}, we see that 
    \bse 
        \int d^2z d^2\th \Phi = \int dz^2 F,
    \ese 
    as it is only the $F$ term that has both $\th$ and $\overline{\th}$. We have just shown that $\del F$ is a total derivative in $z$, and so its integral over $d^2z$ vanishes. 
\eq 

\bl 
    The product of two superfields is supersymmetric. 
\el 

\bq 
    Let
    \bse 
        \Phi_1 = X_1 + i\th\psi_1 + i\overline{\th}\overline{\psi}_2 + \th\overline{\th}F_1,
    \ese
    and similarly for $\Phi_2$. Leibniz tells us 
    \bse 
        Q(\Phi_1\Phi_2) = (Q\Phi_1)\Phi_2 + \Phi_1(Q\Phi_2).
    \ese 
    Then, using \Cref{eqn:GeneralSuperfield} and \Cref{eqn:QOnGeneralSuperfield}, we have 
    \bse 
        Q(\Phi_1\Phi_2) = \Big( -\th\p_zX_1 + i\psi_1 - i\th\overline{\th}\p_z\overline{\psi}_1 + \overline{\th}F_1 \Big)\Big( X_2 + i\th\psi_2 + i\overline{\th}\overline{\psi}_2 + \th\overline{\th}F_2 \Big) + (1\longleftrightarrow 2).
    \ese 
    As with the previous proof, we only need to consider the terms that have $\th\overline{\th}$, leaving us with 
    \bse 
        \int d^2z \Big(-i\big(\p_zX_1\big)\overline{\psi}_2 + i\psi_1F_2 - iX_1\big(\p_z\overline{\psi}_2\big) - iF_1\psi_2 + (1\longleftrightarrow 2)\Big),
    \ese 
    where for the minus sign on the $F_1\psi_2$ term comes from $\overline{\th}\th=-\th\overline{\th}$. Putting in the $(1\longleftrightarrow 2)$ terms in and cancelling, we are left with 
    \bse 
        -i \int d^2 z \Big( \big(\p_zX_1\big)\overline{\psi}_2 + X_1\big(\p_z\overline{\psi}_2\big) +  \big(\p_zX_2\big)\overline{\psi}_1 + X_2\big(\p_z\overline{\psi}_1\big) \Big) = -i \int d^2 z \p_z \big( X_1\overline{\psi}_2 + X_2\overline{\psi}_1\big),
    \ese
    which is the integral of a total derivative, so it vanishes. 
\eq 

What we want to look for supersymmetric Lagrangians. It would obviously be very nice if these Lagrangians contained derivatives (so we can build kinetic terms), the question is `what kind of derivative?' Well it would be very nice if they anticommuted with $Q/\overline{Q}$, so we can easily move $Q$ through our derivative. The first obvious candidate is simply $\p_z/\p_{\overline{z}}$, however these won't do. Why? Well consider an action:
\bse 
    S = \int d^2z d^2\th \p \Phi \overline{\p} \Phi.
\ese
This needs to be unitless (as we're working in units $\hbar=c=1$). Being a scalar field, $[\Phi]=0$, so we don't need to worry about that. What about the measure? Well $[d^2z]=-2$, but Grassman measures, being like derivatives, have positive dimension, therefore using $[\th]=-1/2$\footnote{Can see this from the $Q$ equation.} we have $[d^2\th]=+1$. We therefore require that $[D]=1/2$, but $[\p]=+1$, and so it won't do. 

\bp 
    The so-called \textit{superderivatives}
    \be 
    \label{eqn:Superderivative}
        D := \frac{\p}{\p\th} + \th\frac{\p}{\p z}, \qand \overline{D} := \frac{\p}{\p\overline{\th}} + \overline{\th}\frac{\p}{\p \overline{z}}
    \ee 
    will anticommute with $Q/\overline{Q}$ and have dimension $+1/2$.
\ep 

\bq 
    First let's check the dimension: $[\th]=-1/2$ and $[\p_z]=+1$ therefore $[\th\p_z]=+1/2$. Equally $[\p_{\th}]=+1/2$. Now we need to check they anticommute. Consider the holomorphic expressions:
    \bse 
        \begin{split}
            \big\{Q,D\big\} & = \big(\p_{\th} -\th\p_z\big)\big(\p_{\th} +\th\p_z\big) + \big(\p_{\th} + \th\p_z\big)\big(\p_{\th} - \th\p_z\big) \\
            & = \Big[\p_{\th}\big(\th\p_z) - \th\p_z\p_{\th}\Big] + \Big[ -\p_{\th}\big(\th\p_z) +\th\p_z\p_{\th}\Big] \\
            & = 0,
        \end{split}
    \ese 
    where again we have used $\th^2=0$ and the fact that $\p_{\th}^2$ wont contribute. The antiholomorphic expressions follow analogously. 
\eq 

Following the same calculation as for $Q/\overline{Q}$, we see that 
\bse 
    D^2 = \frac{\p}{\p z}, \qquad \overline{D}^2 = \frac{\p}{\p\overline{z}}, \qand \big\{D,\overline{D}\big\} = 0.
\ese 

\bl 
    The superderivative of a superfield is supersymmetric. 
\el 

\bq 
    Following the same idea as the previous proofs, and using $\{Q,D\}=0$, the only $\th\overline{\th}$ term that remains is $\p_zF$, which is a total derivative and so vanishes.
\eq 

\bp 
    The product $D\Phi^{\mu} \overline{D}\Phi_{\mu}$ is supersymmetric. 
\ep 

\bq 
    By direct calculation, we have (dropping the indices)
    \bse 
        \begin{split}
            D\Phi & = \th\p_z X + i\psi + i \th\overline{\th}\p_z\overline{\psi} + \overline{\th}F, \\
            \overline{D}\Phi & = \overline{\th}\p_{\overline{z}} X + i\overline{\psi} - i \th\overline{\th}\p_{\overline{z}}\psi - \th F.
        \end{split}
    \ese 
    Now, we only need to consider the terms in the product that only have a $\overline{\th}$ as the action of $Q$ will give them a $\th$ then we can do the $d^2\th$ integral. This leaves us with 
    \bse 
        \int d^2zd^2\th Q\big(D\Phi\overline{D}\Phi\big) = i\int d^2z \p\big(\psi\p_{\overline{z}}X - F\overline{\psi}\big),
    \ese 
    which vanishes as the right-hand side is a total derivative.
\eq 

We therefore have the supersymmetric action
\be 
\label{eqn:SupersymmetricAction}
    \frac{1}{4\pi}\int d^2zd^2\th D\Phi^{\mu} \overline{D}\Phi_{\mu} = \frac{1}{4\pi}\int dz^2 \Big( \p_z X^{\mu} \p_{\overline{z}} X_{\mu} + \psi^{\mu}\p_{\overline{z}} \psi_{\mu} + \overline{\psi}\p_{z}\overline{\psi} + F^{\mu}F_{\mu}\Big).
\ee 
The first term on the right-hand side is exactly our Bosonic action.\footnote{Really we need to associate $X^{\mu} \to \sqrt{2/a'} X^{\mu}$, but for simplicity (and to also follow the videos and Polchinski easier) we just set $\a'=2$.} The $F$ term is just some auxiliary field (it has no kinetic term and doesn't couple to anything else). It is just determined by the equation of motion (which we'll see in a second actually just sets it to zero). The middle two terms correspond exactly to the Dirac action, \Cref{eqn:DiracAction}. We can also recover all of the equations of motion from the equation of motion of \Cref{eqn:SupersymmetricAction}; 
\be
\label{eqn:SupersymmetricEoM}
    \overline{D}D\Phi^{\mu} = 0.
\ee
Expanding this out gives 
\bse 
    -\th\overline{\th}\p_{\overline{z}}\p X^{\mu} + i\overline{\th}\p_{\overline{z}}\psi^{\mu} - i\th\p_z\overline{\psi}^{\mu} + F^{\mu} = 0,
\ese 
which considering each coefficient separately gives our Bosonic and Fermionic equations of motion along with $F=0$. This final condition, along with the fact that it is auxiliary, means we can just forget about $F$.

\br 
    Note the above conditions on $F$ only apply because we only have the kinetic term in our action. We can therefore think of $F$ as being related to potential terms.
\er 

\section{The Superfield OPE}

Recall the correlation functions for the Bosonic case could only be functions of the difference $z_{12}$. This comes from the fact that the correlation function must be annihilated by $\p_{z_1}+\p_{z_2}$. We now want to derive a similar result but for the correlation functions of our super fields. So we need to look for something that lies in the kernel of 
\bse 
    D_1 + D_2 := \frac{\p}{\p \th_1} + \th_1\frac{\p}{\p z_1} + \frac{\p}{\p \th_2} + \th_2\frac{\p}{\p z_2}.
\ese 
First consider a function that doesn't depend on $z_1$ or $z_2$. This just gives us an analogous story to the Bosonic case and so this function must depend only on $\th_{12}:=\th_1-\th_2$.

Now suppose the function does depend on $z_{12}$. From the antisymmetric nature of the $\th$s, it follows that we want the $\th$ dependence to be the product $\th_1\th_2$. So our function is of the form
\bse 
    f(z_1,\th_1,z_2,\th_2) = f(z_{12} -\th_1\th_2).
\ese 
For clarity, let's check explicitly, 
\bse 
    \big(D_1+D_2\big)\big(z_1-z_2-\th_1\th_2) = \th_1-\th_2+\th_2-\th_1 = 0,
\ese 
where the sign on the last term comes from anticommuting $\th_1$ with the $\th_2$ derivative.

Now we want our correlation function $\la\Phi(z_1,\th_1)\Phi(z_2,\th_2)\ra$ to reduce to the correlation function \Cref{eqn:ExpectationXX} when we set $\th_1=0=\th_2$. Given the above conditions for the form of the dependence of correlation functions, the obvious candidate is (recall footnote 3, i.e. we set $\a'=2$)
\bse 
    \big\la \Phi(z_1,\th_1)\Phi(z_2,\th_2)\big\ra = -\ln |z_{12}-\th_1\th_2|^2 = -\ln(z_{12}-\th_1\th_2) - \ln(\overline{z}_{12} - \overline{\th}_1\overline{\th}_2).
\ese 
So considering just the holomorphic part, we can take the Taylor expansion, again only keeping the leading order terms because $\th^2=0$, we have 
\bse 
    \big\la \Phi(z_1,\th_1)\Phi(z_2,\th_2)\big\ra = -\ln(z_{12}) + \frac{\th_1\th_2}{z_{12}}.
\ese 
Comparing this with\footnote{We don't need the holomorphic part for this part of the argument.}
\bse 
    \big\la \Phi(z_1,\th_1)\Phi(z_2,\th_2)\big\ra = \big\la \big(X(z_1) + i\th_2\psi(z_1)\big)\big(X(z_2) + i\th_2\psi(z_2)\big)\ra,
\ese 
we conclude 
\bse 
    \begin{split}
        \big\la X(z_1)X(z_2)\big\ra & = -\ln(z_{12}), \\
        \big\la \psi(z_1)\psi(z_2)\big\ra & = \frac{1}{z_{12}}
    \end{split}
\ese 
and the correlators between $X$ and $\psi$ vanishing. This is exactly what we wanted. Note we get a positive sign for the $\psi$ correlator as we have $i^2=-1$ but then $\th_2$ has to pass through $\psi_1$.

\section{The Stress Tensor}

The stress tensor played a very important role in the development of string theory, and we now want to look at a similar structure on our superspace.

Now, recall that the Bosonic stress tensor came from varying the action with respect to the metric, \Cref{eqn:StressEnergyTensor}, and gave us a spin-$2$ field. The Fermionic equivalent comes from studying supergravity and gives us a spin-$3/2$ field. We shall denote these $T_B$ and $T_F$, respectively. As we are working in superspace, we want to consider some construction of both $T_B$ and $T_F$. The obvious thing to try is 
\be
\label{eqn:SuperStressTensor}
    T := \frac{1}{2}T_F + \th T_B,
\ee 
where the $1/2$ factor is included for consistency with Polchinski. We similarly define the antiholomorphic $\overline{T}$.

\br 
\label{rem:TBTFConfusion}
    The above notation can be very confusing,\footnote{I am still trying to make sure this is correct, so take this with a pinch of salt. \textcolor{red}{Red to remind me to change this when I know.}} but what we need to pay attention to is where the different $T$s come from. By definition $T_B$ is the one and only stress tensor for our field, and despite the Bosonic $`B'$ will actually contain $\psi$s (just consider varying \Cref{eqn:SupersymmetricAction}). $T_F$ is just some field introduced onto our space, and $T$ is some superspace structure made from both of them. 
\er 

The next question we ask is about the form of the expressions in the OPE of $T$ with itself. Being a spin-$3/2$ field, the most singular term with have $1/(z_{12}-\th_1\th_2)^3$. What we need to work out is what can appear on the numerator. The result has to be supersymmetric, so we can only really use $T$, $D_{\th}$, $\th_{12}$,  and $\p_z$.\footnote{Note we \textit{cannot} use $\p_{\th}$ as $Q$ does not commute with it, but it does commute with $\p_z$.} The numerators need to have integer spin, so it follows that the general expression is 
\bse 
    T(z_1,\th_1)T(z_2,\th_2) \sim \frac{A}{(z_{12}-\th_1\th_2)^3} + \frac{B D_2T(z_2,\th_2)}{(z_{12}-\th_1\th_2)} + \frac{C\th_{12}T(z_2,\th_2)}{(z_{12}-\th_1\th_2)^2} + \frac{E\th_{12}\p_{z_2}T(z_2,\th_2)}{(z_{12}-\th_1\th_2)}.
\ese 

Next lecture we will expand this out and compare it to what we know, namely correlator for a stress tensor, to fix the factors $A,B,C$ and $E$. 